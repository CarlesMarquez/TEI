
\section[{Documentation Elements}]{Documentation Elements}\label{TD}\par
This chapter describes a module which may be used for the documentation of the XML elements and element classes which make up any markup scheme, in particular that described by the TEI Guidelines, and also for the automatic generation of schemas or DTDs conforming to that documentation. It should be used also by those wishing to customize or modify these Guidelines in a conformant manner, as further described in chapters \textit{\hyperref[MD]{23.3.\ Customization}} and \textit{\hyperref[CF]{23.4.\ Conformance}} and may also be useful in the documentation of any other comparable encoding scheme, even though it contains some aspects which are specific to the TEI and may not be generally applicable.\par
An overview of the kind of processing environment envisaged for the module described by this chapter may be helpful. In the remainder of this chapter we refer to software which provides such a processing environment as an \textit{ODD processor}.\footnote{ODD is short for ‘One Document Does it all’, and was the name invented by the original TEI Editors for the predecessor of the system currently used for this purpose. See further \cite{Burnard1995b} and \cite{TD-BIBL-01}.} Like any other piece of XML software, an ODD processor may be instantiated in many ways: the current system uses a number of XSLT stylesheets which are freely available from the TEI, but this specification makes no particular assumptions about the tools which will be used to provide an ODD processing environment.\par
As the name suggests, an ODD processor uses a single XML document to generate multiple outputs. These outputs will include: \begin{itemize}
\item formal reference documentation for elements, attributes, element classes, patterns, etc. such as those provided in \textit{\hyperref[REF-ELEMENTS]{Appendix C\ Elements}} below;
\item detailed descriptive documentation, embedding some parts of the formal reference documentation, such as the tag description lists provided in this and other chapters of these Guidelines;
\item declarative code for one or more XML schema languages, such as RELAX NG, W3C Schema, ISO Schematron, or DTD.
\end{itemize} \par
The input required to generate these outputs consists of running prose, and special purpose elements documenting the components (elements, classes, etc.) which are to be declared in the chosen schema language. All of this input is encoded in XML using elements defined in this chapter. In order to support more than one schema language, these elements constitute a comparatively high-level model which can then be mapped by an ODD processor to the specific constructs appropriate for the schema language in use. Although some modern schema languages such as RELAX NG or W3C Schema natively support self-documentary features of this kind, we have chosen to retain the ODD model, if only for reasons of compatibility with earlier versions of these Guidelines. For reasons of backwards compatibility, the ISO standard XML schema language RELAX NG (\url{http://www.relaxng.org}) may be used as a means of declaring content models and datatypes, but it is also possible to express content models using native TEI XML constructs. We also use the ISO Schematron language to define additional constraints beyond those expressed in the content model, as further discussed in \textit{\hyperref[TDTAGCONS]{22.5.2.\ Additional Constraints}} below.\par
In the TEI system, a \textit{schema} is built by combining element and attribute declarations, more or less as required. Each element is documented by an appropriate \textit{specification element} and has an identifier unique across the whole TEI scheme. For convenience, these specifications are grouped into a number of discrete \textit{modules}, which can also be combined more or less as required. Each major chapter of these Guidelines defines a distinct module. Each module declares a number of \textit{elements} specific to that module, and may also populate particular \textit{classes}. All classes are available globally, irrespective of the module in which they are declared; particular modules extend the meaning of a class by adding elements or attributes to it. Wherever possible, element content models are defined in terms of classes rather than in terms of specific elements. Modules can also declare particular \textit{patterns}, which act as short-cuts for commonly used content models or class references.\par
In the present chapter, we discuss the components needed to support this system. In addition, section \textit{\hyperref[TDphrase]{22.1.\ Phrase Level Documentary Elements}} discusses some general purpose elements which may be useful in any kind of technical documentation, wherever there is need to talk about technical features of an XML encoding such as element names and attributes. Section \textit{\hyperref[TDmodules]{22.2.\ Modules and Schemas}} discusses the elements which are used to document XML \textit{modules} and their high-level components. Section \textit{\hyperref[TDcrystals]{22.3.\ Specification Elements}} discusses the elements which document XML elements and their attributes, element classes, and generic patterns or macros. Finally, section \textit{\hyperref[TDformal]{22.9.\ Module for Documentation Elements}} provides a summary overview of the elements provided by this module.
\subsection[{Phrase Level Documentary Elements}]{Phrase Level Documentary Elements}\label{TDphrase}
\subsubsection[{Phrase Level Terms}]{Phrase Level Terms}\label{TDphraseTE}\par
In any kind of technical documentation, the following phrase-level elements may be found useful for marking up strings of text which need to be distinguished from the running text because they come from some formal language: 
\begin{sansreflist}
  
\item [\textbf{<code>}] contains literal code from some formal language such as a programming language.\hfil\\[-10pt]\begin{sansreflist}
    \item[@{\itshape lang}]
  (formal language) a name identifying the formal language in which the code is expressed
\end{sansreflist}  
\item [\textbf{<ident>}] (identifier) contains an identifier or name for an object of some kind in a formal language. \hyperref[TEI.ident]{<ident>} is used for tokens such as variable names, class names, type names, function names etc. in formal programming languages.
\end{sansreflist}
 Like other phrase-level elements used to indicate the semantics of a typographically distinct string, these are members of the \textsf{model.emphLike} class. They are available anywhere that running prose is permitted when the module defined by this chapter is included in a schema.\par
The \hyperref[TEI.code]{<code>} and \hyperref[TEI.ident]{<ident>} elements are intended for use when citing brief passages in some formal language such as a programming language, as in the following example: \par\bgroup\index{p=<p>|exampleindex}\index{ident=<ident>|exampleindex}\index{code=<code>|exampleindex}\exampleFont \begin{shaded}\noindent\mbox{}{<\textbf{p}>}If the variable {<\textbf{ident}>}z{</\textbf{ident}>} has a value of zero, a statement such as {<\textbf{code}>}x=y/z{</\textbf{code}>} will\mbox{}\newline 
 usually cause a fatal error.{</\textbf{p}>}\end{shaded}\egroup\par \par
If the cited phrase is a mathematical or chemical formula, the more specific \hyperref[TEI.formula]{<formula>} element defined by the \textsf{figures} module (\textit{\hyperref[FTFOR]{14.2.\ Formulæ and Mathematical Expressions}}) may be more appropriate.\par
A further group of similar phrase-level elements is also defined for the special case of representing parts of an XML document: 
\begin{sansreflist}
  
\item [\textbf{<att>}] (attribute) contains the name of an attribute appearing within running text.
\item [\textbf{<gi>}] (element name) contains the name (generic identifier) of an element.
\item [\textbf{<tag>}] (tag) contains text of a complete start- or end-tag, possibly including attribute specifications, but excluding the opening and closing markup delimiter characters.
\item [\textbf{<val>}] (value) contains a single attribute value.
\end{sansreflist}
 These elements constitute the \textsf{model.phrase.xml} class, which is also a subclass of \textsf{model.phrase}. They are also available anywhere that running prose is permitted when the module defined by this chapter is included in a schema.\par
As an example of the recommended use of these elements, we quote from an imaginary TEI working paper: \par\bgroup\index{p=<p>|exampleindex}\index{gi=<gi>|exampleindex}\index{gi=<gi>|exampleindex}\index{gi=<gi>|exampleindex}\index{tag=<tag>|exampleindex}\index{att=<att>|exampleindex}\index{val=<val>|exampleindex}\exampleFont \begin{shaded}\noindent\mbox{}{<\textbf{p}>}The {<\textbf{gi}>}gi{</\textbf{gi}>} element is used to tag element\mbox{}\newline 
 names when they appear in the text; the {<\textbf{gi}>}tag{</\textbf{gi}>} element however is used to show how a tag as\mbox{}\newline 
 such might appear. So one might talk of an occurrence of the {<\textbf{gi}>}blort{</\textbf{gi}>} element which had been\mbox{}\newline 
 tagged {<\textbf{tag}>}blort type='runcible'{</\textbf{tag}>}. The {<\textbf{att}>}type{</\textbf{att}>} attribute may take any name token as\mbox{}\newline 
 value; the default value is {<\textbf{val}>}spqr{</\textbf{val}>}, in memory of its creator.{</\textbf{p}>}\end{shaded}\egroup\par \par
Within technical documentation, it is also often necessary to provide more extended examples of usage or to present passages of markup for discussion. The following special elements are provided for these purposes: 
\begin{sansreflist}
  
\item [\textbf{<eg>}] (example) contains any kind of illustrative example.
\item [\textbf{<egXML>}] (example of XML) a single XML fragment demonstrating the use of some XML, such as elements, attributes, or processing instructions, etc., in which the \hyperref[TEI.egXML]{<egXML>} element functions as the root element.
\end{sansreflist}
\par
Like the \hyperref[TEI.code]{<code>} element, the \hyperref[TEI.egXML]{<egXML>} element is used to mark strings of formal code, or passages of XML markup. The \hyperref[TEI.eg]{<eg>} element may be used to enclose any kind of example, which will typically be rendered as a distinct block, possibly using particular formatting conventions, when the document is processed. It is a specialized form of the more general \hyperref[TEI.q]{<q>} element provided by the TEI core module. In documents containing examples of XML markup, the \hyperref[TEI.egXML]{<egXML>} element should be used for preference, as further discussed below in \textit{\hyperref[TDeg]{22.4.2.\ Exemplification of Components}}, since the content of this element can be checked for well-formedness.\par
These elements are added to the class \textsf{model.egLike} when this module is included in a schema. That class is a part of the general \textsf{model.inter} class, thus permitting \hyperref[TEI.eg]{<eg>} or \hyperref[TEI.egXML]{<egXML>} elements to appear either within or between paragraph-like elements.
\subsubsection[{Element and Attribute Descriptions}]{Element and Attribute Descriptions}\label{TDphraseEA}\par
Within the body of a document using this module, the following elements may be used to reference parts of the specification elements discussed in section \textit{\hyperref[TDcrystals]{22.3.\ Specification Elements}}, in particular the brief prose descriptions these provide for elements and attributes. 
\begin{sansreflist}
  
\item [\textbf{<specList>}] (specification list) marks where a list of descriptions is to be inserted into the prose documentation.
\item [\textbf{<specDesc>}] (specification description) indicates that a description of the specified element, class, or macro should be included at this point within a document.\hfil\\[-10pt]\begin{sansreflist}
    \item[@{\itshape atts}]
  (attributes) supplies attribute names for which descriptions should additionally be obtained.
\end{sansreflist}  
\end{sansreflist}
\par
TEI practice recommends that a \hyperref[TEI.specList]{<specList>} listing the elements under discussion introduce each subsection of a module's documentation. The source for the present section, for example, begins as follows: \par\bgroup\index{div=<div>|exampleindex}\index{head=<head>|exampleindex}\index{p=<p>|exampleindex}\index{specList=<specList>|exampleindex}\index{specDesc=<specDesc>|exampleindex}\index{key=@key!<specDesc>|exampleindex}\index{specDesc=<specDesc>|exampleindex}\index{key=@key!<specDesc>|exampleindex}\index{atts=@atts!<specDesc>|exampleindex}\index{p=<p>|exampleindex}\index{gi=<gi>|exampleindex}\exampleFont \begin{shaded}\noindent\mbox{}{<\textbf{div}>}\mbox{}\newline 
\hspace*{1em}{<\textbf{head}>}Element and Attribute Descriptions{</\textbf{head}>}\mbox{}\newline 
\hspace*{1em}{<\textbf{p}>}Within the body of a document using this module, the … the brief prose descriptions these provide for elements and attributes.\mbox{}\newline 
\hspace*{1em}{<\textbf{specList}>}\mbox{}\newline 
\hspace*{1em}\hspace*{1em}\hspace*{1em}{<\textbf{specDesc}\hspace*{1em}{key}="{specList}"/>}\mbox{}\newline 
\hspace*{1em}\hspace*{1em}\hspace*{1em}{<\textbf{specDesc}\hspace*{1em}{key}="{specDesc}"\hspace*{1em}{atts}="{atts}"/>}\mbox{}\newline 
\hspace*{1em}\hspace*{1em}{</\textbf{specList}>}\mbox{}\newline 
\hspace*{1em}{</\textbf{p}>}\mbox{}\newline 
\hspace*{1em}{<\textbf{p}>}TEI practice recommends that a {<\textbf{gi}>}specList{</\textbf{gi}>} listing the elements under … {</\textbf{p}>}\mbox{}\newline 
\textit{<!-- ... -->}\mbox{}\newline 
{</\textbf{div}>}\end{shaded}\egroup\par \par
When formatting the \hyperref[TEI.ptr]{<ptr>} element in this example, an ODD processor might simply generate the section number and title of the section referred to, perhaps additionally inserting a link to the section. In a similar way, when processing the \hyperref[TEI.specDesc]{<specDesc>} elements, an ODD processor may recover relevant details of the elements being specified (\hyperref[TEI.specList]{<specList>} and \hyperref[TEI.specDesc]{<specDesc>} in this case) from their associated declaration elements: typically, the details recovered will include a brief description of the element and its attributes. These, and other data, will be stored in a specification element elsewhere within the current document, or they may be supplied by the ODD processor in some other way, for example from a database. For this reason, the link to the required specification element is always made using a TEI-defined key rather than an XML IDREF value. The ODD processor uses this key as a means of accessing the specification element required. There is no requirement that this be performed using the XML ID/IDREF mechanism, but there is an assumption that the identifier be unique.\par
A \hyperref[TEI.specDesc]{<specDesc>} generates in the documentation the identifier, and also the contents of the \hyperref[TEI.desc]{<desc>} child of whatever specification element is indicated by its {\itshape key} attribute, as in the example above. Documentation for any attributes specified by the {\itshape atts} attribute will also be generated as an associated attribute list.
\subsection[{Modules and Schemas}]{Modules and Schemas}\label{TDmodules}\par
As mentioned above, the primary purpose of this module is to facilitate the documentation and creation of an XML schema derived from the TEI Guidelines. The following elements are provided for this purpose: 
\begin{sansreflist}
  
\item [\textbf{<schemaSpec>}] (schema specification) generates a TEI-conformant schema and documentation for it.
\item [\textbf{<moduleSpec>}] (module specification) documents the structure, content, and purpose of a single module, i.e. a named and externally visible group of declarations.
\item [\textbf{<moduleRef>}] (module reference) references a module which is to be incorporated into a schema.\hfil\\[-10pt]\begin{sansreflist}
    \item[@{\itshape include}]
  supplies a list of the elements which are to be copied from the specified module into the schema being defined.
    \item[@{\itshape except}]
  supplies a list of the elements which are not to be copied from the specified module into the schema being defined.
\end{sansreflist}  
\item [\textbf{<specGrp>}] (specification group) contains any convenient grouping of specifications for use within the current module.
\item [\textbf{<specGrpRef>}] (reference to a specification group) indicates that the declarations contained by the \hyperref[TEI.specGrp]{<specGrp>} referenced should be inserted at this point.
\item [\textbf{<attRef>}] (attribute pointer) points to the definition of an attribute or group of attributes.
\item [\textbf{<elementRef>}] points to the specification for some element which is to be included in a schema
\end{sansreflist}
  A \textit{module} is a convenient way of grouping together element and other declarations, and of associating an externally-visible name with the resulting group. A \textit{specification group} performs essentially the same function, but the resulting group is not accessible outside the scope of the ODD document in which it is defined, whereas a module can be accessed by name from any TEI schema specification. Elements, and their attributes, element classes, and patterns are all individually documented using further elements described in section \textit{\hyperref[TDcrystals]{22.3.\ Specification Elements}} below; part of that specification includes the name of the module to which the component belongs.\par
An ODD processor generating XML DTD or schema fragments from a document marked up according to the recommendations of this chapter will generate such fragments for each \hyperref[TEI.moduleSpec]{<moduleSpec>} element found. For example, the chapter documenting the TEI module for names and dates contains a module specification like the following: \par\bgroup\index{moduleSpec=<moduleSpec>|exampleindex}\index{ident=@ident!<moduleSpec>|exampleindex}\index{altIdent=<altIdent>|exampleindex}\index{type=@type!<altIdent>|exampleindex}\index{desc=<desc>|exampleindex}\exampleFont \begin{shaded}\noindent\mbox{}{<\textbf{moduleSpec}\hspace*{1em}{ident}="{namesdates}">}\mbox{}\newline 
\hspace*{1em}{<\textbf{altIdent}\hspace*{1em}{type}="{FPI}">}Names and Dates{</\textbf{altIdent}>}\mbox{}\newline 
\hspace*{1em}{<\textbf{desc}>}Additional elements for names and dates{</\textbf{desc}>}\mbox{}\newline 
{</\textbf{moduleSpec}>}\end{shaded}\egroup\par \noindent  together with specifications for all the elements, classes, and patterns which make up that module, expressed using \hyperref[TEI.elementSpec]{<elementSpec>}, \hyperref[TEI.classSpec]{<classSpec>}, or \hyperref[TEI.macroSpec]{<macroSpec>} elements as appropriate. (These elements are discussed in section \textit{\hyperref[TDcrystals]{22.3.\ Specification Elements}} below.) Each of those specifications carries a {\itshape module} attribute, the value of which is \texttt{namesdates}. An ODD processor encountering the \hyperref[TEI.moduleSpec]{<moduleSpec>} element above can thus generate a schema fragment for the TEI \textsf{namesdates} module that includes declarations for all the elements (etc.) which reference it.\par
In most realistic applications, it will be desirable to combine more than one module together to form a complete \textit{schema}. A schema consists of references to one or more modules or specification groups, and may also contain explicit declarations or redeclarations of elements (see further \textit{\hyperref[TDbuild]{22.8.1.\ TEI customizations}}). Any combination of modules can be used to create a schema \footnote{The distinction between base and additional tagsets in earlier versions of the TEI scheme has not been carried forward into P5.}\par
A schema can combine references to TEI modules with references to other (non-TEI) modules using different namespaces, for example to include mathematical markup expressed using MathML in a TEI document. By default, the effect of combining modules is to allow all of the components declared by the constituent modules to coexist (where this is syntactically possible: where it is not—for example, because of name clashes—a schema cannot be generated). It is also possible to over-ride declarations contained by a module, as further discussed in section \textit{\hyperref[TDbuild]{22.8.1.\ TEI customizations}}\par
It is often convenient to describe and operate on sets of declarations smaller than the whole, and to document them in a specific order: such collections are called \textit{specGrps} (specification groups). Individual \hyperref[TEI.specGrp]{<specGrp>} elements are identified using the global {\itshape xml:id} attribute, and may then be referenced from any point in an ODD document using the \hyperref[TEI.specGrpRef]{<specGrpRef>} element. This is useful if, for example, it is desired to describe particular groups of elements in a specific sequence. Note however that the order in which element declarations appear within the schema code generated from an ODD file element is not in general affected by the order of declarations within a \hyperref[TEI.specGrp]{<specGrp>}.\par
An ODD processor will generate a piece of schema code corresponding with the declarations contained by a \hyperref[TEI.specGrp]{<specGrp>} element in the documentation being output, and a cross-reference to such a piece of schema code when processing a \hyperref[TEI.specGrpRef]{<specGrpRef>}. For example, if the input text reads \par\bgroup\index{p=<p>|exampleindex}\index{specGrp=<specGrp>|exampleindex}\index{elementSpec=<elementSpec>|exampleindex}\index{ident=@ident!<elementSpec>|exampleindex}\index{elementSpec=<elementSpec>|exampleindex}\index{ident=@ident!<elementSpec>|exampleindex}\index{elementSpec=<elementSpec>|exampleindex}\index{ident=@ident!<elementSpec>|exampleindex}\index{specGrp=<specGrp>|exampleindex}\index{elementSpec=<elementSpec>|exampleindex}\index{ident=@ident!<elementSpec>|exampleindex}\index{elementSpec=<elementSpec>|exampleindex}\index{ident=@ident!<elementSpec>|exampleindex}\exampleFont \begin{shaded}\noindent\mbox{}{<\textbf{p}>}This module contains three red elements: {<\textbf{specGrp}\hspace*{1em}{xml:id}="{RED}">}\mbox{}\newline 
\hspace*{1em}\hspace*{1em}{<\textbf{elementSpec}\hspace*{1em}{ident}="{beetroot}">}\mbox{}\newline 
\textit{<!-- ... -->}\mbox{}\newline 
\hspace*{1em}\hspace*{1em}{</\textbf{elementSpec}>}\mbox{}\newline 
\hspace*{1em}\hspace*{1em}{<\textbf{elementSpec}\hspace*{1em}{ident}="{east}">}\mbox{}\newline 
\textit{<!-- ... -->}\mbox{}\newline 
\hspace*{1em}\hspace*{1em}{</\textbf{elementSpec}>}\mbox{}\newline 
\hspace*{1em}\hspace*{1em}{<\textbf{elementSpec}\hspace*{1em}{ident}="{rose}">}\mbox{}\newline 
\textit{<!-- ... -->}\mbox{}\newline 
\hspace*{1em}\hspace*{1em}{</\textbf{elementSpec}>}\mbox{}\newline 
\hspace*{1em}{</\textbf{specGrp}>} and two blue ones: {<\textbf{specGrp}\hspace*{1em}{xml:id}="{BLUE}">}\mbox{}\newline 
\hspace*{1em}\hspace*{1em}{<\textbf{elementSpec}\hspace*{1em}{ident}="{sky}">}\mbox{}\newline 
\textit{<!-- ... -->}\mbox{}\newline 
\hspace*{1em}\hspace*{1em}{</\textbf{elementSpec}>}\mbox{}\newline 
\hspace*{1em}\hspace*{1em}{<\textbf{elementSpec}\hspace*{1em}{ident}="{bayou}">}\mbox{}\newline 
\textit{<!-- ... -->}\mbox{}\newline 
\hspace*{1em}\hspace*{1em}{</\textbf{elementSpec}>}\mbox{}\newline 
\hspace*{1em}{</\textbf{specGrp}>}\mbox{}\newline 
{</\textbf{p}>}\end{shaded}\egroup\par \noindent  then the output documentation will replace the two \hyperref[TEI.specGrp]{<specGrp>} elements above with a representation of the schema code declaring the elements \texttt{<beetroot>}, \texttt{<east>}, and \texttt{<rose>} and that declaring the elements \texttt{<sky>} and \texttt{<bayou>} respectively. Similarly, if the input text contains elsewhere a passage such as \par\bgroup\index{div=<div>|exampleindex}\index{head=<head>|exampleindex}\index{p=<p>|exampleindex}\index{specGrpRef=<specGrpRef>|exampleindex}\index{target=@target!<specGrpRef>|exampleindex}\index{specGrpRef=<specGrpRef>|exampleindex}\index{target=@target!<specGrpRef>|exampleindex}\exampleFont \begin{shaded}\noindent\mbox{}{<\textbf{div}>}\mbox{}\newline 
\hspace*{1em}{<\textbf{head}>}An overview of the imaginary module{</\textbf{head}>}\mbox{}\newline 
\hspace*{1em}{<\textbf{p}>}The imaginary module contains declarations for coloured things: {<\textbf{specGrpRef}\hspace*{1em}{target}="{\#RED}"/>}\mbox{}\newline 
\hspace*{1em}\hspace*{1em}{<\textbf{specGrpRef}\hspace*{1em}{target}="{\#BLUE}"/>}\mbox{}\newline 
\hspace*{1em}{</\textbf{p}>}\mbox{}\newline 
{</\textbf{div}>}\end{shaded}\egroup\par \noindent  then the \hyperref[TEI.specGrpRef]{<specGrpRef>} elements may be replaced by an appropriate piece of reference text such as ‘The RED elements were declared in section 4.2 above’, or even by a copy of the relevant declarations. As stated above, the order of declarations within the imaginary module described above will not be affected in any way. Indeed, it is possible that the imaginary module will contain declarations not present in any specification group, or that the specification groups will refer to elements that come from different modules. Specification groups are always local to the document in which they are defined, and cannot be referenced externally (unlike modules).
\subsection[{Specification Elements}]{Specification Elements}\label{TDcrystals}\par
The following elements are used to specify elements, classes, patterns, and datatypes: 
\begin{sansreflist}
  
\item [\textbf{<elementSpec>}] (element specification) documents the structure, content, and purpose of a single element type.
\item [\textbf{<classSpec>}] (class specification) contains reference information for a TEI element class; that is a group of elements which appear together in content models, or which share some common attribute, or both.\hfil\\[-10pt]\begin{sansreflist}
    \item[@{\itshape generate}]
  indicates which alternation and sequence instantiations of a model class may be referenced. By default, all variations are permitted.
\end{sansreflist}  
\item [\textbf{<macroSpec>}] (macro specification) documents the function and implementation of a pattern.
\item [\textbf{<dataSpec>}] (datatype specification) documents a datatype.
\end{sansreflist}
\par
Unlike most elements in the TEI scheme, each of these ‘specification elements’ has a fairly rigid internal structure consisting of a large number of child elements which are always presented in the same order. Furthermore, since these elements all describe markup objects in broadly similar ways, they have several child elements in common. In the remainder of this chapter, we discuss first the elements which are common to all the specification elements, and then those which are specific to a particular type.\par
Specification elements may appear at any point in an ODD document, both between and within paragraphs as well as inside a \hyperref[TEI.specGrp]{<specGrp>} element, but the specification element for any particular component may only appear once (except in the case where a modification is being defined; see further \textit{\hyperref[TDbuild]{22.8.1.\ TEI customizations}}). The order in which they appear will not affect the order in which they are presented within any schema module generated from the document. In documentation mode, however, an ODD processor will output the schema declarations corresponding with a specification element at the point in the text where they are encountered, provided that they are contained by a \hyperref[TEI.specGrp]{<specGrp>} element,  as discussed in the previous section. An ODD processor will also associate all declarations found with the nominated module, thus including them within the schema code generated for that module, and it will also generate a full reference description for the object concerned in a catalogue of markup objects. These latter two actions always occur irrespective of whether or not the declaration is included in a \hyperref[TEI.specGrp]{<specGrp>}.
\subsection[{Common Elements}]{Common Elements}\label{TDcrystalsCE}\par
This section discusses the child elements common to all of the specification elements; some of these are defined in the core module (\textit{\hyperref[COHTG]{3.4.1.\ Terms and Glosses}}). These child elements are used to specify the naming, description, exemplification, and classification of the specification elements.
\subsubsection[{Description of Components}]{Description of Components}\label{TDcrystalsCEdc}\par

\begin{sansreflist}
  
\item [\textbf{<gloss>}] (gloss) identifies a phrase or word used to provide a gloss or definition for some other word or phrase.
\item [\textbf{<desc>}] (description) contains a short description of the purpose, function, or use of its parent element, or when the parent is a documentation element, describes or defines the object being documented. 
\item [\textbf{<equiv>}] (equivalent) specifies a component which is considered equivalent to the parent element, either by co-reference, or by external link.\hfil\\[-10pt]\begin{sansreflist}
    \item[@{\itshape uri}]
  (uniform resource identifier) references the underlying concept of which the parent is a representation by means of some external identifier
    \item[@{\itshape filter}]
  references an external script which contains a method to transform instances of this element to canonical TEI
    \item[@{\itshape name}]
  a single word which follows the rules defining a legal XML name (see \url{http://www.w3.org/TR/REC-xml/\#dt-name}), naming the underlying concept of which the parent is a representation.
    \item[@{\itshape predicate [att.predicate]}]
  the condition under which the element bearing this attribute applies, given as an XPath predicate expression.
\end{sansreflist}  
\item [\textbf{<altIdent>}] (alternate identifier) supplies the recommended XML name for an element, class, attribute, etc. in some language.
\item [\textbf{<listRef>}] (list of references) supplies a list of significant references to places where this element is discussed, in the current document or elsewhere.
\item [\textbf{<remarks>}] (remarks) contains any commentary or discussion about the usage of an element, attribute, class, or entity not otherwise documented within the containing element.
\end{sansreflist}
\par
The \hyperref[TEI.gloss]{<gloss>} element may be used to provide a brief explanation for the name of the object if this is not self-explanatory. For example, the specification for the element \hyperref[TEI.ab]{<ab>} used to mark arbitrary blocks of text begins as follows: \par\bgroup\index{elementSpec=<elementSpec>|exampleindex}\index{module=@module!<elementSpec>|exampleindex}\index{ident=@ident!<elementSpec>|exampleindex}\index{gloss=<gloss>|exampleindex}\exampleFont \begin{shaded}\noindent\mbox{}{<\textbf{elementSpec}\hspace*{1em}{module}="{linking}"\hspace*{1em}{ident}="{ab}">}\mbox{}\newline 
\hspace*{1em}{<\textbf{gloss}>}anonymous block{</\textbf{gloss}>}\mbox{}\newline 
\textit{<!--... -->}\mbox{}\newline 
{</\textbf{elementSpec}>}\end{shaded}\egroup\par \noindent  A \hyperref[TEI.gloss]{<gloss>} may also be supplied for an attribute name or an attribute value in similar circumstances: \par\bgroup\index{valList=<valList>|exampleindex}\index{type=@type!<valList>|exampleindex}\index{valItem=<valItem>|exampleindex}\index{ident=@ident!<valItem>|exampleindex}\index{gloss=<gloss>|exampleindex}\index{desc=<desc>|exampleindex}\index{valItem=<valItem>|exampleindex}\index{ident=@ident!<valItem>|exampleindex}\index{gloss=<gloss>|exampleindex}\index{desc=<desc>|exampleindex}\exampleFont \begin{shaded}\noindent\mbox{}{<\textbf{valList}\hspace*{1em}{type}="{open}">}\mbox{}\newline 
\hspace*{1em}{<\textbf{valItem}\hspace*{1em}{ident}="{susp}">}\mbox{}\newline 
\hspace*{1em}\hspace*{1em}{<\textbf{gloss}>}suspension{</\textbf{gloss}>}\mbox{}\newline 
\hspace*{1em}\hspace*{1em}{<\textbf{desc}>}the abbreviation provides the first letter(s) of the word or phrase, omitting the\mbox{}\newline 
\hspace*{1em}\hspace*{1em}\hspace*{1em}\hspace*{1em} remainder.{</\textbf{desc}>}\mbox{}\newline 
\hspace*{1em}{</\textbf{valItem}>}\mbox{}\newline 
\hspace*{1em}{<\textbf{valItem}\hspace*{1em}{ident}="{contr}">}\mbox{}\newline 
\hspace*{1em}\hspace*{1em}{<\textbf{gloss}>}contraction{</\textbf{gloss}>}\mbox{}\newline 
\hspace*{1em}\hspace*{1em}{<\textbf{desc}>}the abbreviation omits some letter(s) in the middle.{</\textbf{desc}>}\mbox{}\newline 
\hspace*{1em}{</\textbf{valItem}>}\mbox{}\newline 
\textit{<!--...-->}\mbox{}\newline 
{</\textbf{valList}>}\end{shaded}\egroup\par \par
Note that the \hyperref[TEI.gloss]{<gloss>} element is needed to explain the significance of the identifier for an item only when this is not apparent, for example because it is abbreviated, as in the above example. It should not be used to provide a full description of the intended meaning (this is the function of the \hyperref[TEI.desc]{<desc>} element), nor to comment on equivalent values in other schemes (this is the purpose of the \hyperref[TEI.equiv]{<equiv>} element), nor to provide alternative versions of the {\itshape ident} attribute value in other languages (this is the purpose of the \hyperref[TEI.altIdent]{<altIdent>} element).\par
The contents of the \hyperref[TEI.desc]{<desc>} element provide a brief characterization of the intended function of the object being documented in a form that permits its quotation out of context, as in the following example: \par\bgroup\index{elementSpec=<elementSpec>|exampleindex}\index{module=@module!<elementSpec>|exampleindex}\index{ident=@ident!<elementSpec>|exampleindex}\index{desc=<desc>|exampleindex}\index{versionDate=@versionDate!<desc>|exampleindex}\exampleFont \begin{shaded}\noindent\mbox{}{<\textbf{elementSpec}\hspace*{1em}{module}="{core}"\hspace*{1em}{ident}="{foreign}">}\mbox{}\newline 
\textit{<!--... -->}\mbox{}\newline 
\hspace*{1em}{<\textbf{desc}\hspace*{1em}{xml:lang}="{en}"\mbox{}\newline 
\hspace*{1em}\hspace*{1em}{versionDate}="{2007-07-21}">}identifies a word or phrase as belonging to some\mbox{}\newline 
\hspace*{1em}\hspace*{1em} language other than that of the surrounding text. {</\textbf{desc}>}\mbox{}\newline 
\textit{<!--... -->}\mbox{}\newline 
{</\textbf{elementSpec}>}\end{shaded}\egroup\par \noindent  By convention, a \hyperref[TEI.desc]{<desc>} element begins with a verb such as \textit{contains}, \textit{indicates}, \textit{specifies}, etc. and contains a single clause.\par
Both the \hyperref[TEI.gloss]{<gloss>} and \hyperref[TEI.desc]{<desc>} elements (in addition to \hyperref[TEI.exemplum]{<exemplum>}, \hyperref[TEI.remarks]{<remarks>}, and \hyperref[TEI.valDesc]{<valDesc>}) are members of \textsf{att.translatable}, and thus carry the {\itshape versionDate} attributre. Where specifications are supplied in multiple languages, these elements may be repeated as often as needed. Each such element should carry both an {\itshape xml:lang} and a {\itshape versionDate} attribute to indicate the language used and the date on which the translated text was last checked against its source.\par
The \hyperref[TEI.equiv]{<equiv>} element is used to document equivalencies between the concept represented by this object and the same concept as described in other schemes or ontologies. The {\itshape uri} attribute is used to supply a pointer to some location where such external concepts are defined. For example, to indicate that the TEI \hyperref[TEI.death]{<death>} element corresponds to the concept defined by the CIDOC CRM category E69, the declaration for the former might begin as follows: \par\bgroup\index{elementSpec=<elementSpec>|exampleindex}\index{module=@module!<elementSpec>|exampleindex}\index{ident=@ident!<elementSpec>|exampleindex}\index{equiv=<equiv>|exampleindex}\index{name=@name!<equiv>|exampleindex}\index{uri=@uri!<equiv>|exampleindex}\index{desc=<desc>|exampleindex}\exampleFont \begin{shaded}\noindent\mbox{}{<\textbf{elementSpec}\hspace*{1em}{module}="{namesdates}"\mbox{}\newline 
\hspace*{1em}{ident}="{death}">}\mbox{}\newline 
\hspace*{1em}{<\textbf{equiv}\hspace*{1em}{name}="{E69}"\mbox{}\newline 
\hspace*{1em}\hspace*{1em}{uri}="{http://cidoc.ics.forth.gr/}"/>}\mbox{}\newline 
\hspace*{1em}{<\textbf{desc}>}\mbox{}\newline 
\textit{<!--... -->}\mbox{}\newline 
\hspace*{1em}{</\textbf{desc}>}\mbox{}\newline 
{</\textbf{elementSpec}>}\end{shaded}\egroup\par \par
The \hyperref[TEI.equiv]{<equiv>} element may also be used to map newly-defined elements onto existing constructs in the TEI, using the {\itshape filter} and {\itshape name} attributes to point to an implementation of the mapping. This is useful when a TEI customization (see \textit{\hyperref[MD]{23.3.\ Customization}}) defines ‘shortcuts’ for convenience of data entry or markup readability. For example, suppose that in some TEI customization an element \texttt{<bo>} has been defined which is conceptually equivalent to the standard markup construct <hi rend='bold'>. The following declarations would additionally indicate that instances of the \texttt{<bo>} element can be converted to canonical TEI by obtaining a filter from the URI specified, and running the procedure with the name \textsf{bold}. The {\itshape mimeType} attribute specifies the language (in this case XSL) in which the filter is written: \par\bgroup\index{elementSpec=<elementSpec>|exampleindex}\index{ident=@ident!<elementSpec>|exampleindex}\index{ns=@ns!<elementSpec>|exampleindex}\index{equiv=<equiv>|exampleindex}\index{filter=@filter!<equiv>|exampleindex}\index{mimeType=@mimeType!<equiv>|exampleindex}\index{name=@name!<equiv>|exampleindex}\index{gloss=<gloss>|exampleindex}\index{desc=<desc>|exampleindex}\exampleFont \begin{shaded}\noindent\mbox{}{<\textbf{elementSpec}\hspace*{1em}{ident}="{bo}"\mbox{}\newline 
\hspace*{1em}{ns}="{http://www.example.com/ns/nonTEI}">}\mbox{}\newline 
\hspace*{1em}{<\textbf{equiv}\hspace*{1em}{filter}="{http://www.example.com/equiv-filter.xsl}"\mbox{}\newline 
\hspace*{1em}\hspace*{1em}{mimeType}="{text/xsl}"\hspace*{1em}{name}="{bold}"/>}\mbox{}\newline 
\hspace*{1em}{<\textbf{gloss}>}bold{</\textbf{gloss}>}\mbox{}\newline 
\hspace*{1em}{<\textbf{desc}>}contains a sequence of characters rendered in a bold face.{</\textbf{desc}>}\mbox{}\newline 
\textit{<!-- ... -->}\mbox{}\newline 
{</\textbf{elementSpec}>}\end{shaded}\egroup\par \par
The \hyperref[TEI.altIdent]{<altIdent>} element is used to provide an alternative name for an object, for example using a different natural language. Thus, the following might be used to indicate that the \hyperref[TEI.abbr]{<abbr>} element should be identified using the German word \textit{Abkürzung}: \par\bgroup\index{elementSpec=<elementSpec>|exampleindex}\index{ident=@ident!<elementSpec>|exampleindex}\index{mode=@mode!<elementSpec>|exampleindex}\index{altIdent=<altIdent>|exampleindex}\exampleFont \begin{shaded}\noindent\mbox{}{<\textbf{elementSpec}\hspace*{1em}{ident}="{abbr}"\hspace*{1em}{mode}="{change}">}\mbox{}\newline 
\hspace*{1em}{<\textbf{altIdent}\hspace*{1em}{xml:lang}="{de}">}Abkürzung{</\textbf{altIdent}>}\mbox{}\newline 
\textit{<!--...-->}\mbox{}\newline 
{</\textbf{elementSpec}>}\end{shaded}\egroup\par \noindent  In the same way, the following specification for the \hyperref[TEI.graphic]{<graphic>} element indicates that the attribute {\itshape url} may also be referred to using the alternate identifier \textsf{href}: \par\bgroup\index{elementSpec=<elementSpec>|exampleindex}\index{ident=@ident!<elementSpec>|exampleindex}\index{mode=@mode!<elementSpec>|exampleindex}\index{attList=<attList>|exampleindex}\index{attDef=<attDef>|exampleindex}\index{mode=@mode!<attDef>|exampleindex}\index{ident=@ident!<attDef>|exampleindex}\index{altIdent=<altIdent>|exampleindex}\exampleFont \begin{shaded}\noindent\mbox{}{<\textbf{elementSpec}\hspace*{1em}{ident}="{graphic}"\hspace*{1em}{mode}="{change}">}\mbox{}\newline 
\hspace*{1em}{<\textbf{attList}>}\mbox{}\newline 
\hspace*{1em}\hspace*{1em}{<\textbf{attDef}\hspace*{1em}{mode}="{change}"\hspace*{1em}{ident}="{url}">}\mbox{}\newline 
\hspace*{1em}\hspace*{1em}\hspace*{1em}{<\textbf{altIdent}>}href{</\textbf{altIdent}>}\mbox{}\newline 
\hspace*{1em}\hspace*{1em}{</\textbf{attDef}>}\mbox{}\newline 
\textit{<!-- ... -->}\mbox{}\newline 
\hspace*{1em}{</\textbf{attList}>}\mbox{}\newline 
{</\textbf{elementSpec}>}\end{shaded}\egroup\par \par
By default, the \hyperref[TEI.altIdent]{<altIdent>} of a component is identical to the value of its {\itshape ident} attribute.\par
The \hyperref[TEI.remarks]{<remarks>} element contains any additional commentary about how the item concerned may be used, details of implementation-related issues, suggestions for other ways of treating related information etc., as in the following example: \par\bgroup\index{elementSpec=<elementSpec>|exampleindex}\index{module=@module!<elementSpec>|exampleindex}\index{ident=@ident!<elementSpec>|exampleindex}\index{remarks=<remarks>|exampleindex}\index{p=<p>|exampleindex}\index{att=<att>|exampleindex}\index{p=<p>|exampleindex}\index{gi=<gi>|exampleindex}\exampleFont \begin{shaded}\noindent\mbox{}{<\textbf{elementSpec}\hspace*{1em}{module}="{core}"\hspace*{1em}{ident}="{foreign}">}\mbox{}\newline 
\textit{<!--... -->}\mbox{}\newline 
\hspace*{1em}{<\textbf{remarks}>}\mbox{}\newline 
\hspace*{1em}\hspace*{1em}{<\textbf{p}>}This element is intended for use only where no other element is available to mark the phrase\mbox{}\newline 
\hspace*{1em}\hspace*{1em}\hspace*{1em}\hspace*{1em} or words concerned. The global {<\textbf{att}>}xml:lang{</\textbf{att}>} attribute should be used in preference to\mbox{}\newline 
\hspace*{1em}\hspace*{1em}\hspace*{1em}\hspace*{1em} this element where it is intended to mark the language of the whole of some text element.{</\textbf{p}>}\mbox{}\newline 
\hspace*{1em}\hspace*{1em}{<\textbf{p}>}The {<\textbf{gi}>}distinct{</\textbf{gi}>} element may be used to identify phrases belonging to sublanguages or\mbox{}\newline 
\hspace*{1em}\hspace*{1em}\hspace*{1em}\hspace*{1em} registers not generally regarded as true languages.{</\textbf{p}>}\mbox{}\newline 
\hspace*{1em}{</\textbf{remarks}>}\mbox{}\newline 
\textit{<!--... -->}\mbox{}\newline 
{</\textbf{elementSpec}>}\end{shaded}\egroup\par \par
A specification element will usually conclude with a list of references, each tagged using the standard \hyperref[TEI.ptr]{<ptr>} element, and grouped together into a \hyperref[TEI.listRef]{<listRef>} element: in the case of the \hyperref[TEI.foreign]{<foreign>} element discussed above, the list is as follows: \par\bgroup\index{listRef=<listRef>|exampleindex}\index{ptr=<ptr>|exampleindex}\index{target=@target!<ptr>|exampleindex}\exampleFont \begin{shaded}\noindent\mbox{}{<\textbf{listRef}>}\mbox{}\newline 
\hspace*{1em}{<\textbf{ptr}\hspace*{1em}{target}="{\#COHQHF}"/>}\mbox{}\newline 
{</\textbf{listRef}>}\end{shaded}\egroup\par \noindent  where the value COHQF is the identifier of the section in these Guidelines where this element is fully documented.
\subsubsection[{Exemplification of Components}]{Exemplification of Components}\label{TDeg}\par

\begin{sansreflist}
  
\item [\textbf{<exemplum>}] (exemplum) groups an example demonstrating the use of an element along with optional paragraphs of commentary.
\item [\textbf{<eg>}] (example) contains any kind of illustrative example.
\item [\textbf{<egXML>}] (example of XML) a single XML fragment demonstrating the use of some XML, such as elements, attributes, or processing instructions, etc., in which the \hyperref[TEI.egXML]{<egXML>} element functions as the root element.\hfil\\[-10pt]\begin{sansreflist}
    \item[@{\itshape valid}]
  indicates the intended validity of the example with respect to a schema.
    \item[@{\itshape source [att.global.source]}]
  specifies the source from which some aspect of this element is drawn.
\end{sansreflist}  
\end{sansreflist}
\par
The \hyperref[TEI.exemplum]{<exemplum>} element is used to combine a single illustrative example with an optional paragraph of commentary following or preceding it. The illustrative example itself may be marked up using either the \hyperref[TEI.eg]{<eg>} or the \hyperref[TEI.egXML]{<egXML>} element.\par
The {\itshape source} attribute may be used on either element to indicate the source from which an example is taken, typically by means of a pointer to an entry in an associated bibliography, as in the following example: \par\bgroup\exampleFont \begin{shaded}\noindent\mbox{}\newline
  <exemplum versionDate="2008-04-06" xml:lang="fr">\newline
  <p>L'element <gi>foreign</gi> s'applique également aux termes considerés étrangers.</p>\newline
    <egXML xmlns="http://www.tei-c.org/ns/Examples" source="\#fr-ex-Queneau\textunderscore Journ">\newline
      <p>Pendant ce temps-là, dans le bureau du rez- de-chaussée, les secrétaires faisaient du\newline
        <foreign xml:lang="en">hulla-hoop</foreign>.</p>\newline
    </egXML>\newline
  </exemplum>\newline
\end{shaded}\egroup\par \par
When, as here, an example contains valid XML markup, the \hyperref[TEI.egXML]{<egXML>} element should be used. In such a case, it will clearly be necessary to distinguish the markup within the example from the markup of the document itself. In an XML environment, this is easily done by using a different name space for the content of the \hyperref[TEI.egXML]{<egXML>} element. For example: \par\hfill\bgroup\exampleFont\vskip 10pt\begin{shaded}
\obeyspaces <p>The <gi>term</gi> element may be used \newline
to mark any technical term, thus:\newline
<egXML xmlns="http://www.tei-c.org/ns/Examples">\newline
  This <term>recursion</term> is \newline
  giving me a headache.</egXML></p>\end{shaded}
\par\egroup 
\par
Alternatively, the XML tagging within an example may be ‘escaped’, either by using entity references to represent the opening angle bracket, or by wrapping the whole example in a CDATA marked section: \par\hfill\bgroup\exampleFont\vskip 10pt\begin{shaded}
\obeyspaces <p>The <gi>term</gi> element may be used \newline
to mark any technical term, thus:\newline
<egXML xmlns="http://www.tei-c.org/ns/Examples">\newline
  This \&lt;term\&gt;recursion\&lt;/term\&gt; is \newline
  giving me a headache.</egXML></p>\end{shaded}
\par\egroup 
 or, equivalently: \par\hfill\bgroup\exampleFont\vskip 10pt\begin{shaded}
\obeyspaces <p>The <gi>term</gi> element may be used \newline
to mark any technical term, thus:\newline
<egXML xmlns="http://www.tei-c.org/ns/Examples"><![CDATA[\newline
  This <term>recursion</term> is \newline
  giving me a headache.]]></egXML></p>\end{shaded}
\par\egroup 
 However, escaping the markup in this way will make it impossible to validate, and should therefore generally be avoided.\par
If the XML contained in an example is not well-formed then it must either be enclosed in a CDATA marked section, or ‘escaped’ as above: this applies whether the \hyperref[TEI.eg]{<eg>} or \hyperref[TEI.egXML]{<egXML>} is used. The {\itshape valid} attribute on \hyperref[TEI.egXML]{<egXML>} may be used to indicate the XML validity of the example with respect to some schema, as being valid, invalid, or feasibly valid.\par
The \hyperref[TEI.egXML]{<egXML>} element should not be used to tag non-XML examples: the general purpose \hyperref[TEI.eg]{<eg>} or \hyperref[TEI.q]{<q>} elements should be used for such purposes.
\subsubsection[{Classification of Components}]{Classification of Components}\label{TDcrystalsCEcl}\par
In the TEI scheme elements are assigned to one or more \textit{classes}, which may themselves have subclasses. The following elements are used to indicate class membership: 
\begin{sansreflist}
  
\item [\textbf{<classes>}] (classes) specifies all the classes of which the documented element or class is a member or subclass.
\item [\textbf{<memberOf>}] specifies class membership of the documented element or class.\hfil\\[-10pt]\begin{sansreflist}
    \item[@{\itshape key}]
  specifies the identifier for a class of which the documented element or class is a member or subclass
\end{sansreflist}  
\end{sansreflist}
\par
The \hyperref[TEI.classes]{<classes>} element appears within either the \hyperref[TEI.elementSpec]{<elementSpec>} or \hyperref[TEI.classSpec]{<classSpec>} element. It specifies the classes of which the element or class concerned is a member by means of one or more \hyperref[TEI.memberOf]{<memberOf>} child elements. Each such element references a class by means of its {\itshape key} attribute. Classes themselves are defined by the \hyperref[TEI.classSpec]{<classSpec>} element described in section \textit{\hyperref[TDCLA]{22.6.\ Class Specifications}} below.\par
For example, to show that the element \hyperref[TEI.gi]{<gi>} is a member of the class \textsf{model.phrase.xml}, the \hyperref[TEI.elementSpec]{<elementSpec>} which documents this element contains the following \hyperref[TEI.classes]{<classes>} element: \par\bgroup\index{classes=<classes>|exampleindex}\index{memberOf=<memberOf>|exampleindex}\index{key=@key!<memberOf>|exampleindex}\exampleFont \begin{shaded}\noindent\mbox{}{<\textbf{classes}>}\mbox{}\newline 
\hspace*{1em}{<\textbf{memberOf}\hspace*{1em}{key}="{model.phrase.xml}"/>}\mbox{}\newline 
{</\textbf{classes}>}\end{shaded}\egroup\par 
\subsection[{Element Specifications}]{Element Specifications}\label{TDTAG}\par
The \hyperref[TEI.elementSpec]{<elementSpec>} element is used to document an element type, together with its associated attributes. In addition to the elements listed above, it may contain the following subcomponents: 
\begin{sansreflist}
  
\item [\textbf{<content>}] (content model) contains a declaration of the intended content model for the element (or other construct) being specified.\hfil\\[-10pt]\begin{sansreflist}
    \item[@{\itshape autoPrefix}]
  controls whether or not pattern names generated in the corresponding RELAX NG schema source are automatically prefixed to avoid potential nameclashes.
\end{sansreflist}  
\item [\textbf{<constraintSpec>}] (constraint on schema) contains a formal constraint, typically expressed in a rule-based schema language, to which a construct must conform in order to be considered valid\hfil\\[-10pt]\begin{sansreflist}
    \item[@{\itshape scheme}]
  supplies the name of the language in which the constraints are defined
\end{sansreflist}  
\item [\textbf{<attList>}] (attribute list) contains documentation for all the attributes associated with this element, as a series of \hyperref[TEI.attDef]{<attDef>} elements.\hfil\\[-10pt]\begin{sansreflist}
    \item[@{\itshape org}]
  (organization) specifies whether all the attributes in the list are available (org="group") or only one of them (org="choice")
\end{sansreflist}  
\item [\textbf{<model>}] describes the processing intended for a specified element.\hfil\\[-10pt]\begin{sansreflist}
    \item[@{\itshape behaviour}]
  names the process or function which this processing model uses in order to produce output.
\end{sansreflist}  
\end{sansreflist}
\par
These subcomponents are discussed in the following sections.
\subsubsection[{Defining Content Models}]{Defining Content Models}\label{DEFCON}\par
As described in \textit{\hyperref[SG132]{v.3.2\ Content Models: an Example}} and \textit{\hyperref[SG143]{Content Model}}, the \textit{content} of the element being defined — that is, what elements are allowed inside it, and in what order they are permitted — is described by its \textit{content model}. The content model is defined by the \hyperref[TEI.content]{<content>} child of \hyperref[TEI.elementSpec]{<elementSpec>}. There are three distinctly different ways of specifying a content model: \begin{itemize}
\item The content model can be described using TEI elements defined by this chapter, as discussed in \textit{\hyperref[DEFCONTEI]{22.5.1.1.\ Defining Content Models: TEI}} immediately below. Two such TEI elements that may be used to define a content model are \hyperref[TEI.dataRef]{<dataRef>} and \hyperref[TEI.valList]{<valList>}. But because these are most often used to define attribute values, they are discussed separately near the beginning and towards the end of \textit{\hyperref[TDATTvs]{22.5.3.2.\ Value Specification}}, respectively.
\item Alternatively, and primarily for backwards compatibility, the content model may be expressed using a RELAX NG pattern. This is discussed in \textit{\hyperref[TDTAGCONT]{22.5.1.2.\ Defining Content Models: RELAX NG}}, below.
\item Lastly, content models may be expressed using a schema language other than TEI or RELAX NG, but no further recommendations on doing so are provided by these Guidelines.
\end{itemize} 
\paragraph[{Defining Content Models: TEI}]{Defining Content Models: TEI}\label{DEFCONTEI}\par
In the simplest case, the content model of an element may be expressed using a single \hyperref[TEI.empty]{<empty>} element as the only child of \hyperref[TEI.content]{<content>}. This describes the element being defined as \textit{empty}, meaning a valid instance of said element can not have any content.\footnote{It would still be allowed to contain comments or processing instructions, as these are not considered part of the content model.} 
\begin{sansreflist}
  
\item [\textbf{<empty>}] indicates the presence of an empty node within a content model
\end{sansreflist}
\par
More commonly, one or more of the following elements are used to define a content model: 
\begin{sansreflist}
  
\item [\textbf{<elementRef>}] points to the specification for some element which is to be included in a schema
\item [\textbf{<anyElement>}] indicates the presence of any elements in a content model
\item [\textbf{<classRef>}] points to the specification for an attribute or model class which is to be included in a schema
\item [\textbf{<macroRef>}] points to the specification for some pattern which is to be included in a schema
\end{sansreflist}
\par
An \hyperref[TEI.elementRef]{<elementRef>} provides the name of an element which may appear at a certain point in a content model. An \hyperref[TEI.anyElement]{<anyElement>} also asserts that an element may appear at a certain point in a content model, but rather than providing the name of a particular element type that may appear, any element regardless of its name may appear (and may have any attributes). A \hyperref[TEI.classRef]{<classRef>} provides the name of a model class, members of which may appear at a certain point in content model.\footnote{The \hyperref[TEI.classRef]{<classRef>} element may be used to refer to attribute classes, but this should not be done within a \hyperref[TEI.content]{<content>}.} A \hyperref[TEI.macroRef]{<macroRef>} provides the name of a predefined macro, the expansion of which is to be inserted at a certain point in a content model.\par
These three elements are all members of an attribute class which provides attributes that further modify their significance as follows: 
\begin{sansreflist}
  
\item [\textbf{att.repeatable}] supplies attributes for the elements which define component parts of a content model.\hfil\\[-10pt]\begin{sansreflist}
    \item[@{\itshape minOccurs}]
  (minimum number of occurences) indicates the smallest number of times this component may occur.
    \item[@{\itshape maxOccurs}]
  (maximum number of occurences) indicates the largest number of times this component may occur.
\end{sansreflist}  
\end{sansreflist}
\par
Additionally, two wrapper elements are provided to indicate whether the components listed as their children form a sequence or an alternation: 
\begin{sansreflist}
  
\item [\textbf{<sequence>}] indicates that the constructs referenced by its children form a sequence
\item [\textbf{<alternate>}] indicates that the constructs referenced by its children form an alternation
\end{sansreflist}
 These two wrapper elements are also members of \textsf{att.repeatable}. References listed as children of \hyperref[TEI.sequence]{<sequence>} must appear in the order and cardinality specified. Only one of the references listed as children of \hyperref[TEI.alternate]{<alternate>} may appear, although the cardinality of the \hyperref[TEI.alternate]{<alternate>} itself applies. Thus the following fanciful content model permits either any number of \hyperref[TEI.ptr]{<ptr>} elements (except zero) or any number of \hyperref[TEI.ref]{<ref>} elements (except zero); at least one element must be present, but having both a \hyperref[TEI.ptr]{<ptr>} and a \hyperref[TEI.ref]{<ref>} would be invalid. \par\bgroup\index{content=<content>|exampleindex}\index{alternate=<alternate>|exampleindex}\index{elementRef=<elementRef>|exampleindex}\index{key=@key!<elementRef>|exampleindex}\index{minOccurs=@minOccurs!<elementRef>|exampleindex}\index{maxOccurs=@maxOccurs!<elementRef>|exampleindex}\index{elementRef=<elementRef>|exampleindex}\index{key=@key!<elementRef>|exampleindex}\index{minOccurs=@minOccurs!<elementRef>|exampleindex}\index{maxOccurs=@maxOccurs!<elementRef>|exampleindex}\exampleFont \begin{shaded}\noindent\mbox{}{<\textbf{content}>}\mbox{}\newline 
\hspace*{1em}{<\textbf{alternate}>}\mbox{}\newline 
\hspace*{1em}\hspace*{1em}{<\textbf{elementRef}\hspace*{1em}{key}="{ptr}"\hspace*{1em}{minOccurs}="{1}"\mbox{}\newline 
\hspace*{1em}\hspace*{1em}\hspace*{1em}{maxOccurs}="{unbounded}"/>}\mbox{}\newline 
\hspace*{1em}\hspace*{1em}{<\textbf{elementRef}\hspace*{1em}{key}="{ref}"\hspace*{1em}{minOccurs}="{1}"\mbox{}\newline 
\hspace*{1em}\hspace*{1em}\hspace*{1em}{maxOccurs}="{unbounded}"/>}\mbox{}\newline 
\hspace*{1em}{</\textbf{alternate}>}\mbox{}\newline 
{</\textbf{content}>}\end{shaded}\egroup\par \noindent  However, the following content model permits any number of either \hyperref[TEI.ptr]{<ptr>} or \hyperref[TEI.ref]{<ref>} elements (except zero); one element must be present, and having both \hyperref[TEI.ptr]{<ptr>} elements and \hyperref[TEI.ref]{<ref>} elements (even intermixed) would be valid. \par\bgroup\index{content=<content>|exampleindex}\index{alternate=<alternate>|exampleindex}\index{minOccurs=@minOccurs!<alternate>|exampleindex}\index{maxOccurs=@maxOccurs!<alternate>|exampleindex}\index{elementRef=<elementRef>|exampleindex}\index{key=@key!<elementRef>|exampleindex}\index{elementRef=<elementRef>|exampleindex}\index{key=@key!<elementRef>|exampleindex}\exampleFont \begin{shaded}\noindent\mbox{}{<\textbf{content}>}\mbox{}\newline 
\hspace*{1em}{<\textbf{alternate}\hspace*{1em}{minOccurs}="{1}"\mbox{}\newline 
\hspace*{1em}\hspace*{1em}{maxOccurs}="{unbounded}">}\mbox{}\newline 
\hspace*{1em}\hspace*{1em}{<\textbf{elementRef}\hspace*{1em}{key}="{ptr}"/>}\mbox{}\newline 
\hspace*{1em}\hspace*{1em}{<\textbf{elementRef}\hspace*{1em}{key}="{ref}"/>}\mbox{}\newline 
\hspace*{1em}{</\textbf{alternate}>}\mbox{}\newline 
{</\textbf{content}>}\end{shaded}\egroup\par \par
The \hyperref[TEI.sequence]{<sequence>} and \hyperref[TEI.alternate]{<alternate>} elements may be used in combination with great expressive power. For example, in the following example, which might be imagined as a clean replacement for the content of the \hyperref[TEI.choice]{<choice>} element, one and only one of the element pairs \hyperref[TEI.sic]{<sic>} and \hyperref[TEI.corr]{<corr>}, \hyperref[TEI.orig]{<orig>} and \hyperref[TEI.reg]{<reg>}, or \hyperref[TEI.abbr]{<abbr>} and \hyperref[TEI.expan]{<expan>} is allowed. \par\bgroup\index{content=<content>|exampleindex}\index{alternate=<alternate>|exampleindex}\index{sequence=<sequence>|exampleindex}\index{elementRef=<elementRef>|exampleindex}\index{key=@key!<elementRef>|exampleindex}\index{elementRef=<elementRef>|exampleindex}\index{key=@key!<elementRef>|exampleindex}\index{sequence=<sequence>|exampleindex}\index{elementRef=<elementRef>|exampleindex}\index{key=@key!<elementRef>|exampleindex}\index{elementRef=<elementRef>|exampleindex}\index{key=@key!<elementRef>|exampleindex}\index{sequence=<sequence>|exampleindex}\index{elementRef=<elementRef>|exampleindex}\index{key=@key!<elementRef>|exampleindex}\index{elementRef=<elementRef>|exampleindex}\index{key=@key!<elementRef>|exampleindex}\exampleFont \begin{shaded}\noindent\mbox{}{<\textbf{content}>}\mbox{}\newline 
\hspace*{1em}{<\textbf{alternate}>}\mbox{}\newline 
\hspace*{1em}\hspace*{1em}{<\textbf{sequence}>}\mbox{}\newline 
\hspace*{1em}\hspace*{1em}\hspace*{1em}{<\textbf{elementRef}\hspace*{1em}{key}="{sic}"/>}\mbox{}\newline 
\hspace*{1em}\hspace*{1em}\hspace*{1em}{<\textbf{elementRef}\hspace*{1em}{key}="{corr}"/>}\mbox{}\newline 
\hspace*{1em}\hspace*{1em}{</\textbf{sequence}>}\mbox{}\newline 
\hspace*{1em}\hspace*{1em}{<\textbf{sequence}>}\mbox{}\newline 
\hspace*{1em}\hspace*{1em}\hspace*{1em}{<\textbf{elementRef}\hspace*{1em}{key}="{orig}"/>}\mbox{}\newline 
\hspace*{1em}\hspace*{1em}\hspace*{1em}{<\textbf{elementRef}\hspace*{1em}{key}="{reg}"/>}\mbox{}\newline 
\hspace*{1em}\hspace*{1em}{</\textbf{sequence}>}\mbox{}\newline 
\hspace*{1em}\hspace*{1em}{<\textbf{sequence}>}\mbox{}\newline 
\hspace*{1em}\hspace*{1em}\hspace*{1em}{<\textbf{elementRef}\hspace*{1em}{key}="{abbr}"/>}\mbox{}\newline 
\hspace*{1em}\hspace*{1em}\hspace*{1em}{<\textbf{elementRef}\hspace*{1em}{key}="{expan}"/>}\mbox{}\newline 
\hspace*{1em}\hspace*{1em}{</\textbf{sequence}>}\mbox{}\newline 
\hspace*{1em}{</\textbf{alternate}>}\mbox{}\newline 
{</\textbf{content}>}\end{shaded}\egroup\par \noindent  In the following example, which might be imagined as a clean replacement for the content of the \hyperref[TEI.address]{<address>} element, the encoder is given a choice of either: \begin{itemize}
\item a single \hyperref[TEI.street]{<street>} followed by a single \hyperref[TEI.placeName]{<placeName>} followed by a single \hyperref[TEI.postCode]{<postCode>} followed by an optional \hyperref[TEI.country]{<country>}, or
\item 2, 3, or 4 \hyperref[TEI.addrLine]{<addrLine>} elements.
\end{itemize}  \par\bgroup\index{content=<content>|exampleindex}\index{alternate=<alternate>|exampleindex}\index{sequence=<sequence>|exampleindex}\index{elementRef=<elementRef>|exampleindex}\index{key=@key!<elementRef>|exampleindex}\index{elementRef=<elementRef>|exampleindex}\index{key=@key!<elementRef>|exampleindex}\index{elementRef=<elementRef>|exampleindex}\index{key=@key!<elementRef>|exampleindex}\index{elementRef=<elementRef>|exampleindex}\index{key=@key!<elementRef>|exampleindex}\index{minOccurs=@minOccurs!<elementRef>|exampleindex}\index{maxOccurs=@maxOccurs!<elementRef>|exampleindex}\index{elementRef=<elementRef>|exampleindex}\index{key=@key!<elementRef>|exampleindex}\index{minOccurs=@minOccurs!<elementRef>|exampleindex}\index{maxOccurs=@maxOccurs!<elementRef>|exampleindex}\exampleFont \begin{shaded}\noindent\mbox{}{<\textbf{content}>}\mbox{}\newline 
\hspace*{1em}{<\textbf{alternate}>}\mbox{}\newline 
\hspace*{1em}\hspace*{1em}{<\textbf{sequence}>}\mbox{}\newline 
\hspace*{1em}\hspace*{1em}\hspace*{1em}{<\textbf{elementRef}\hspace*{1em}{key}="{street}"/>}\mbox{}\newline 
\hspace*{1em}\hspace*{1em}\hspace*{1em}{<\textbf{elementRef}\hspace*{1em}{key}="{placeName}"/>}\mbox{}\newline 
\hspace*{1em}\hspace*{1em}\hspace*{1em}{<\textbf{elementRef}\hspace*{1em}{key}="{postCode}"/>}\mbox{}\newline 
\hspace*{1em}\hspace*{1em}\hspace*{1em}{<\textbf{elementRef}\hspace*{1em}{key}="{country}"\hspace*{1em}{minOccurs}="{0}"\mbox{}\newline 
\hspace*{1em}\hspace*{1em}\hspace*{1em}\hspace*{1em}{maxOccurs}="{1}"/>}\mbox{}\newline 
\hspace*{1em}\hspace*{1em}{</\textbf{sequence}>}\mbox{}\newline 
\hspace*{1em}\hspace*{1em}{<\textbf{elementRef}\hspace*{1em}{key}="{addrLine}"\hspace*{1em}{minOccurs}="{2}"\mbox{}\newline 
\hspace*{1em}\hspace*{1em}\hspace*{1em}{maxOccurs}="{4}"/>}\mbox{}\newline 
\hspace*{1em}{</\textbf{alternate}>}\mbox{}\newline 
{</\textbf{content}>}\end{shaded}\egroup\par \par
In addition to expressing where certain elements, members of a class of elements, or constructs matching a predefined macro may occur inside an element, a content model may permit a string of zero or more Unicode characters to occur at a certain point in the content model. This is indicated by supplying the element \hyperref[TEI.textNode]{<textNode>} within the \hyperref[TEI.content]{<content>} element. 
\begin{sansreflist}
  
\item [\textbf{<textNode>}] indicates the presence of a text node in a content model
\end{sansreflist}
 If nothing but a \hyperref[TEI.textNode]{<textNode>} element is present inside a \hyperref[TEI.content]{<content>} element, valid instances of the element being defined may contain a sequence of zero or more Unicode characters, but may not contain any elements.\footnote{This content model is not used very often in the TEI scheme. Because only Unicode characters are permitted, there is no way to record characters that are not (yet) represented in Unicode. Thus in TEI instead of \hyperref[TEI.textNode]{<textNode>} we often use a reference to \textsf{macro.xText} which permits both Unicode characters and the \hyperref[TEI.g]{<g>} element.}
\paragraph[{Defining Content Models: RELAX NG}]{Defining Content Models: RELAX NG}\label{TDTAGCONT}\par
Element content models may also be defined using RELAX NG patterns. Here is a very simple example \par\bgroup\index{content=<content>|exampleindex}\exampleFont \begin{shaded}\noindent\mbox{}{<\textbf{content}>}\mbox{}\newline 
\hspace*{1em}{<\textbf{rng:text}/>}\mbox{}\newline 
{</\textbf{content}>}\end{shaded}\egroup\par \noindent  The element within whose specification element this \hyperref[TEI.content]{<content>} element appears will have a content model which is expressed in RELAX NG as \texttt{text}, using the RELAX NG namespace. This model will be copied unchanged to the output when RELAX NG schemas are being generated. When an XML DTD is being generated, an equivalent declaration (in this case \texttt{(\#PCDATA)}) will be output.\par
Here is a more complex example: \par\bgroup\index{content=<content>|exampleindex}\exampleFont \begin{shaded}\noindent\mbox{}{<\textbf{content}>}\mbox{}\newline 
\hspace*{1em}{<\textbf{rng:group}>}\mbox{}\newline 
\hspace*{1em}\hspace*{1em}{<\textbf{rng:ref}\hspace*{1em}{name}="{fileDesc}"/>}\mbox{}\newline 
\hspace*{1em}\hspace*{1em}{<\textbf{rng:zeroOrMore}>}\mbox{}\newline 
\hspace*{1em}\hspace*{1em}\hspace*{1em}{<\textbf{rng:ref}\hspace*{1em}{name}="{model.teiHeaderPart}"/>}\mbox{}\newline 
\hspace*{1em}\hspace*{1em}{</\textbf{rng:zeroOrMore}>}\mbox{}\newline 
\hspace*{1em}\hspace*{1em}{<\textbf{rng:optional}>}\mbox{}\newline 
\hspace*{1em}\hspace*{1em}\hspace*{1em}{<\textbf{rng:ref}\hspace*{1em}{name}="{revisionDesc}"/>}\mbox{}\newline 
\hspace*{1em}\hspace*{1em}{</\textbf{rng:optional}>}\mbox{}\newline 
\hspace*{1em}{</\textbf{rng:group}>}\mbox{}\newline 
{</\textbf{content}>}\end{shaded}\egroup\par \noindent  This is the content model for the \hyperref[TEI.teiHeader]{<teiHeader>} element, expressed in the RELAX NG syntax, which again is copied unchanged to the output during schema generation. The equivalent DTD notation generated from this is \texttt{(fileDesc, (\%model.teiHeaderPart;)*, revisionDesc?)}.\par
The RELAX NG language does not formally distinguish element names, attribute names, class names, or macro names: all names are patterns which are handled in the same way, as the above example shows. Within the TEI scheme, however, different naming conventions are used to distinguish amongst the objects being named. Unqualified names (\texttt{fileDesc}, \texttt{revisionDesc}) are always element names. Names prefixed with \texttt{model.} or \texttt{att.} (e.g. \textsf{model.teiHeaderPart} and \textsf{att.typed}) are always class names. In DTD language, classes are represented by parameter entities (\texttt{\%model.teiHeaderPart;} in the above example); see further \textit{\hyperref[ST]{1.\ The TEI Infrastructure}}.\par
The RELAX NG pattern names generated by an ODD processor by default include a special prefix, the default value for which is set using the {\itshape prefix} attribute on \hyperref[TEI.schemaSpec]{<schemaSpec>}.  The purpose of this is to ensure that the pattern name generated is uniquely identified as belonging to a particular schema, and thus avoid name clashes. For example, in a RELAX NG schema combining the TEI element \hyperref[TEI.ident]{<ident>} with another element called \hyperref[TEI.ident]{<ident>} from some other vocabulary, the former will be defined by a pattern called \texttt{TEI\textunderscore ident} rather than simply \texttt{ident}. Most of the time, this behaviour is entirely transparent to the user; the one occasion when it is not will be where a content model (expressed using RELAX NG syntax) needs explicitly to reference either the TEI \hyperref[TEI.ident]{<ident>} or the other one. In such a situation, the {\itshape autoPrefix} attribute on \hyperref[TEI.content]{<content>} may be used. For example, suppose that we wish to define a content model for \hyperref[TEI.term]{<term>} which permits either a TEI \hyperref[TEI.ident]{<ident>} or the \hyperref[TEI.ident]{<ident>} defined by some other vocabulary. A suitable content model would be generated from the following \hyperref[TEI.content]{<content>} element: \par\bgroup\index{content=<content>|exampleindex}\index{autoPrefix=@autoPrefix!<content>|exampleindex}\exampleFont \begin{shaded}\noindent\mbox{}{<\textbf{content}\hspace*{1em}{autoPrefix}="{false}">}\mbox{}\newline 
\hspace*{1em}{<\textbf{rng:choice}>}\mbox{}\newline 
\hspace*{1em}\hspace*{1em}{<\textbf{rng:ref}\hspace*{1em}{name}="{TEI\textunderscore ident}"/>}\mbox{}\newline 
\hspace*{1em}\hspace*{1em}{<\textbf{rng:ref}\hspace*{1em}{name}="{ident}"/>}\mbox{}\newline 
\hspace*{1em}{</\textbf{rng:choice}>}\mbox{}\newline 
{</\textbf{content}>}\end{shaded}\egroup\par 
\subsubsection[{Additional Constraints}]{Additional Constraints}\label{TDTAGCONS}\par
In addition to the \hyperref[TEI.content]{<content>} element, a set of general \hyperref[TEI.constraintSpec]{<constraintSpec>} elements can be used to express rules about the validity of an element. Like some other specification elements, they are identifiable (using the {\itshape ident} attribute) in order that a TEI customization may override, delete or change them individually. Each \hyperref[TEI.constraintSpec]{<constraintSpec>} can be expressed in any notation which is found useful; the notation used must be recorded using the {\itshape scheme} attribute.\par
Schematron is an ISO standard (\hyperref[ISO-19757-3]{ISO/IEC 19757-3:2006}) which defines a simple XML vocabulary for an ‘assertion language’, together with a RELAX NG schema to validate it. The Schematron assertion language provides a powerful way of expressing constraints on the content of any XML document in addition to those provided by other schema languages. Such constraints can be embedded within a TEI schema specification using the methods exemplified in this chapter. An ODD processor will typically process any \hyperref[TEI.constraintSpec]{<constraintSpec>} elements in a TEI specification whose {\itshape scheme} attribute indicates that they are expressed in Schematron to create an ISO-conformant Schematron schema which may be used to validate document instances.\par
The TEI Guidelines include some additional constraints which are expressed using the ISO Schematron language. A conformant TEI document should respect these constraints, although automatic validation of them may not be possible for all processors. A TEI customization may likewise specify additional constraints using this mechanism. Some examples of what is possible using the Schematron language are given below. \par
Constraints are generally used to model local rules which may be outside the scope of the target schema language. For example, in earlier versions of these Guidelines several constraints on the usage of the attributes of the TEI element \hyperref[TEI.relation]{<relation>} were expressed informally as follows: ‘only one of the attributes {\itshape active} and {\itshape mutual} may be supplied; the attribute {\itshape passive} may be supplied only if the attribute {\itshape active} is supplied.’. In the current version of the Guidelines, constraint specifications expressed as Schematron rules have been added, as follows: \par\bgroup\index{constraintSpec=<constraintSpec>|exampleindex}\index{ident=@ident!<constraintSpec>|exampleindex}\index{scheme=@scheme!<constraintSpec>|exampleindex}\index{constraint=<constraint>|exampleindex}\index{constraintSpec=<constraintSpec>|exampleindex}\index{ident=@ident!<constraintSpec>|exampleindex}\index{scheme=@scheme!<constraintSpec>|exampleindex}\index{constraint=<constraint>|exampleindex}\index{constraintSpec=<constraintSpec>|exampleindex}\index{ident=@ident!<constraintSpec>|exampleindex}\index{scheme=@scheme!<constraintSpec>|exampleindex}\index{constraint=<constraint>|exampleindex}\exampleFont \begin{shaded}\noindent\mbox{}{<\textbf{constraintSpec}\hspace*{1em}{ident}="{reforkeyorname}"\mbox{}\newline 
\hspace*{1em}{scheme}="{schematron}">}\mbox{}\newline 
\hspace*{1em}{<\textbf{constraint}>}\mbox{}\newline 
\hspace*{1em}\hspace*{1em}{<\textbf{s:assert}\hspace*{1em}{test}="{@ref or @key or @name}">}One of the\mbox{}\newline 
\hspace*{1em}\hspace*{1em}\hspace*{1em}\hspace*{1em} attributes 'name', 'ref' or 'key' must be supplied{</\textbf{s:assert}>}\mbox{}\newline 
\hspace*{1em}{</\textbf{constraint}>}\mbox{}\newline 
{</\textbf{constraintSpec}>}\mbox{}\newline 
{<\textbf{constraintSpec}\hspace*{1em}{ident}="{activemutual}"\mbox{}\newline 
\hspace*{1em}{scheme}="{schematron}">}\mbox{}\newline 
\hspace*{1em}{<\textbf{constraint}>}\mbox{}\newline 
\hspace*{1em}\hspace*{1em}{<\textbf{s:report}\hspace*{1em}{test}="{@active and @mutual}">}Only one of the\mbox{}\newline 
\hspace*{1em}\hspace*{1em}\hspace*{1em}\hspace*{1em} attributes @active and @mutual may be supplied{</\textbf{s:report}>}\mbox{}\newline 
\hspace*{1em}{</\textbf{constraint}>}\mbox{}\newline 
{</\textbf{constraintSpec}>}\mbox{}\newline 
{<\textbf{constraintSpec}\hspace*{1em}{ident}="{activepassive}"\mbox{}\newline 
\hspace*{1em}{scheme}="{schematron}">}\mbox{}\newline 
\hspace*{1em}{<\textbf{constraint}>}\mbox{}\newline 
\hspace*{1em}\hspace*{1em}{<\textbf{s:report}\hspace*{1em}{test}="{@passive and not(@active)}">}the\mbox{}\newline 
\hspace*{1em}\hspace*{1em}\hspace*{1em}\hspace*{1em} attribute 'passive' may be supplied only if the attribute 'active' is supplied{</\textbf{s:report}>}\mbox{}\newline 
\hspace*{1em}{</\textbf{constraint}>}\mbox{}\newline 
{</\textbf{constraintSpec}>}\end{shaded}\egroup\par \par
The constraints in the preceding example all related to attributes in the empty namespace, and the Schematron rules did not therefore need to define a TEI namespace prefix. The Schematron language \texttt{<ns>} element should be used to do this when a constraint needs to refer to a TEI element, as in the following example, which models the constraint that a TEI \hyperref[TEI.div]{<div>} must contain either no subdivisions or at least two of them: \par\bgroup\index{constraintSpec=<constraintSpec>|exampleindex}\index{ident=@ident!<constraintSpec>|exampleindex}\index{scheme=@scheme!<constraintSpec>|exampleindex}\index{constraint=<constraint>|exampleindex}\exampleFont \begin{shaded}\noindent\mbox{}{<\textbf{constraintSpec}\hspace*{1em}{ident}="{subclauses}"\mbox{}\newline 
\hspace*{1em}{scheme}="{schematron}">}\mbox{}\newline 
\hspace*{1em}{<\textbf{constraint}>}\mbox{}\newline 
\hspace*{1em}\hspace*{1em}{<\textbf{s:ns}\hspace*{1em}{prefix}="{tei}"\mbox{}\newline 
\hspace*{1em}\hspace*{1em}\hspace*{1em}{uri}="{http://www.tei-c.org/ns/1.0}"/>}\mbox{}\newline 
\hspace*{1em}\hspace*{1em}{<\textbf{s:rule}\hspace*{1em}{context}="{tei:div}">}\mbox{}\newline 
\hspace*{1em}\hspace*{1em}\hspace*{1em}{<\textbf{s:report}\hspace*{1em}{test}="{count( tei:div ) eq 1}">}if it contains any subdivisions, a\mbox{}\newline 
\hspace*{1em}\hspace*{1em}\hspace*{1em}\hspace*{1em}\hspace*{1em}\hspace*{1em} division must contain at least two of them{</\textbf{s:report}>}\mbox{}\newline 
\hspace*{1em}\hspace*{1em}{</\textbf{s:rule}>}\mbox{}\newline 
\hspace*{1em}{</\textbf{constraint}>}\mbox{}\newline 
{</\textbf{constraintSpec}>}\end{shaded}\egroup\par \noindent  Schematron rules are also useful where an application needs to enforce rules on attribute values, as in the following examples which check that various types of \hyperref[TEI.title]{<title>} are provided: \par\bgroup\index{constraintSpec=<constraintSpec>|exampleindex}\index{ident=@ident!<constraintSpec>|exampleindex}\index{scheme=@scheme!<constraintSpec>|exampleindex}\index{constraint=<constraint>|exampleindex}\index{constraintSpec=<constraintSpec>|exampleindex}\index{ident=@ident!<constraintSpec>|exampleindex}\index{scheme=@scheme!<constraintSpec>|exampleindex}\index{constraint=<constraint>|exampleindex}\exampleFont \begin{shaded}\noindent\mbox{}{<\textbf{constraintSpec}\hspace*{1em}{ident}="{introtitle}"\mbox{}\newline 
\hspace*{1em}{scheme}="{schematron}">}\mbox{}\newline 
\hspace*{1em}{<\textbf{constraint}>}\mbox{}\newline 
\hspace*{1em}\hspace*{1em}{<\textbf{s:assert}\hspace*{1em}{test}="{tei:fileDesc/tei:titleStmt/tei:title[@type='introductory']}">} an introductory component\mbox{}\newline 
\hspace*{1em}\hspace*{1em}\hspace*{1em}\hspace*{1em} of the title is expected {</\textbf{s:assert}>}\mbox{}\newline 
\hspace*{1em}{</\textbf{constraint}>}\mbox{}\newline 
{</\textbf{constraintSpec}>}\mbox{}\newline 
{<\textbf{constraintSpec}\hspace*{1em}{ident}="{maintitle}"\mbox{}\newline 
\hspace*{1em}{scheme}="{schematron}">}\mbox{}\newline 
\hspace*{1em}{<\textbf{constraint}>}\mbox{}\newline 
\hspace*{1em}\hspace*{1em}{<\textbf{s:assert}\hspace*{1em}{test}="{tei:fileDesc/tei:titleStmt/tei:title[@type='main']}">} a main title must be supplied\mbox{}\newline 
\hspace*{1em}\hspace*{1em}{</\textbf{s:assert}>}\mbox{}\newline 
\hspace*{1em}{</\textbf{constraint}>}\mbox{}\newline 
{</\textbf{constraintSpec}>}\end{shaded}\egroup\par \par
As a further example, Schematron may be used to enforce rules applicable to a TEI document which is going to be rendered into accessible HTML, for example to check that some sort of content is available from which the {\itshape alt} attribute of an HTML \texttt{<img>} can be created: \par\bgroup\index{constraintSpec=<constraintSpec>|exampleindex}\index{ident=@ident!<constraintSpec>|exampleindex}\index{scheme=@scheme!<constraintSpec>|exampleindex}\index{constraint=<constraint>|exampleindex}\exampleFont \begin{shaded}\noindent\mbox{}{<\textbf{constraintSpec}\hspace*{1em}{ident}="{alt}"\mbox{}\newline 
\hspace*{1em}{scheme}="{schematron}">}\mbox{}\newline 
\hspace*{1em}{<\textbf{constraint}>}\mbox{}\newline 
\hspace*{1em}\hspace*{1em}{<\textbf{s:ns}\hspace*{1em}{prefix}="{tei}"\mbox{}\newline 
\hspace*{1em}\hspace*{1em}\hspace*{1em}{uri}="{http://www.tei-c.org/ns/1.0}"/>}\mbox{}\newline 
\hspace*{1em}\hspace*{1em}{<\textbf{s:pattern}\hspace*{1em}{id}="{altTags}">}\mbox{}\newline 
\hspace*{1em}\hspace*{1em}\hspace*{1em}{<\textbf{s:rule}\hspace*{1em}{context}="{tei:figure}">}\mbox{}\newline 
\hspace*{1em}\hspace*{1em}\hspace*{1em}\hspace*{1em}{<\textbf{s:report}\hspace*{1em}{test}="{not(tei:figDesc or tei:head)}">} You should provide information in a figure from\mbox{}\newline 
\hspace*{1em}\hspace*{1em}\hspace*{1em}\hspace*{1em}\hspace*{1em}\hspace*{1em}\hspace*{1em}\hspace*{1em} which we can construct an alt attribute in HTML {</\textbf{s:report}>}\mbox{}\newline 
\hspace*{1em}\hspace*{1em}\hspace*{1em}{</\textbf{s:rule}>}\mbox{}\newline 
\hspace*{1em}\hspace*{1em}{</\textbf{s:pattern}>}\mbox{}\newline 
\hspace*{1em}{</\textbf{constraint}>}\mbox{}\newline 
{</\textbf{constraintSpec}>}\end{shaded}\egroup\par \noindent  Schematron rules can also be used to enforce other HTML accessibility rules about tables; note here the use of a report and an assertion within one pattern: \par\bgroup\index{constraintSpec=<constraintSpec>|exampleindex}\index{ident=@ident!<constraintSpec>|exampleindex}\index{scheme=@scheme!<constraintSpec>|exampleindex}\index{constraint=<constraint>|exampleindex}\exampleFont \begin{shaded}\noindent\mbox{}{<\textbf{constraintSpec}\hspace*{1em}{ident}="{tables}"\mbox{}\newline 
\hspace*{1em}{scheme}="{schematron}">}\mbox{}\newline 
\hspace*{1em}{<\textbf{constraint}>}\mbox{}\newline 
\hspace*{1em}\hspace*{1em}{<\textbf{s:ns}\hspace*{1em}{prefix}="{tei}"\mbox{}\newline 
\hspace*{1em}\hspace*{1em}\hspace*{1em}{uri}="{http://www.tei-c.org/ns/1.0}"/>}\mbox{}\newline 
\hspace*{1em}\hspace*{1em}{<\textbf{s:pattern}\hspace*{1em}{id}="{Tables}">}\mbox{}\newline 
\hspace*{1em}\hspace*{1em}\hspace*{1em}{<\textbf{s:rule}\hspace*{1em}{context}="{tei:table}">}\mbox{}\newline 
\hspace*{1em}\hspace*{1em}\hspace*{1em}\hspace*{1em}{<\textbf{s:assert}\hspace*{1em}{test}="{tei:head}">}A <table> should have a caption, using a <head>\mbox{}\newline 
\hspace*{1em}\hspace*{1em}\hspace*{1em}\hspace*{1em}\hspace*{1em}\hspace*{1em}\hspace*{1em}\hspace*{1em} element{</\textbf{s:assert}>}\mbox{}\newline 
\hspace*{1em}\hspace*{1em}\hspace*{1em}\hspace*{1em}{<\textbf{s:report}\hspace*{1em}{test}="{parent::tei:body}">}Do not use tables to lay out the document body{</\textbf{s:report}>}\mbox{}\newline 
\hspace*{1em}\hspace*{1em}\hspace*{1em}{</\textbf{s:rule}>}\mbox{}\newline 
\hspace*{1em}\hspace*{1em}{</\textbf{s:pattern}>}\mbox{}\newline 
\hspace*{1em}{</\textbf{constraint}>}\mbox{}\newline 
{</\textbf{constraintSpec}>}\end{shaded}\egroup\par \par
Constraints can be expressed using any convenient language. The following example uses a pattern matching language called SPITBOL to express the requirement that title and author should be different. \par\bgroup\index{constraintSpec=<constraintSpec>|exampleindex}\index{ident=@ident!<constraintSpec>|exampleindex}\index{scheme=@scheme!<constraintSpec>|exampleindex}\index{constraint=<constraint>|exampleindex}\exampleFont \begin{shaded}\noindent\mbox{}{<\textbf{constraintSpec}\hspace*{1em}{ident}="{local}"\mbox{}\newline 
\hspace*{1em}{scheme}="{SPITBOL}">}\mbox{}\newline 
\hspace*{1em}{<\textbf{constraint}>} (output = leq(title,author) "title and author cannot be the same") {</\textbf{constraint}>}\mbox{}\newline 
{</\textbf{constraintSpec}>}\end{shaded}\egroup\par \noindent  Note that the value of {\itshape scheme} is SPITBOL. In order to properly constrain and document the values of {\itshape scheme} used in their customization file, a project may wish to create a customization that (among other things) adds and explains this value for use in validating their customization file. Thus using schemes other than those provided for by the TEI (currently schematron and isoschematron) may require somewhat more effort when creating a customization file. Such private schemes will generally be even more problematic on implementation of the constraints themselves, as it may require siginficant programming work. The TEI only provides this capability for the suggested values.
\subsubsection[{Attribute List Specification}]{Attribute List Specification}\label{TDATT}\par
The \hyperref[TEI.attList]{<attList>} element is used to document information about a collection of attributes, either within an \hyperref[TEI.elementSpec]{<elementSpec>}, or within a \hyperref[TEI.classSpec]{<classSpec>}. An attribute list can be organized either as a group of attribute definitions, all of which are understood to be available, or as a choice of attribute definitions, of which only one is understood to be available. An attribute list may thus contain nested attribute lists.\par
The attribute {\itshape org} is used to indicate whether its child \hyperref[TEI.attDef]{<attDef>} elements are all to be made available, or whether only one of them may be used. For example, the attribute list for the element \hyperref[TEI.moduleRef]{<moduleRef>} contains a nested attribute list to indicate that either the {\itshape include} or the {\itshape except} attribute may be supplied, but not both: \par\bgroup\index{attList=<attList>|exampleindex}\index{attList=<attList>|exampleindex}\index{org=@org!<attList>|exampleindex}\index{attDef=<attDef>|exampleindex}\index{ident=@ident!<attDef>|exampleindex}\index{attDef=<attDef>|exampleindex}\index{ident=@ident!<attDef>|exampleindex}\exampleFont \begin{shaded}\noindent\mbox{}{<\textbf{attList}>}\mbox{}\newline 
\textit{<!-- other attribute definitions here -->}\mbox{}\newline 
\hspace*{1em}{<\textbf{attList}\hspace*{1em}{org}="{choice}">}\mbox{}\newline 
\hspace*{1em}\hspace*{1em}{<\textbf{attDef}\hspace*{1em}{ident}="{include}">}\mbox{}\newline 
\textit{<!-- definition for the include attribute -->}\mbox{}\newline 
\hspace*{1em}\hspace*{1em}{</\textbf{attDef}>}\mbox{}\newline 
\hspace*{1em}\hspace*{1em}{<\textbf{attDef}\hspace*{1em}{ident}="{except}">}\mbox{}\newline 
\textit{<!-- definition for the except attribute -->}\mbox{}\newline 
\hspace*{1em}\hspace*{1em}{</\textbf{attDef}>}\mbox{}\newline 
\hspace*{1em}{</\textbf{attList}>}\mbox{}\newline 
{</\textbf{attList}>}\end{shaded}\egroup\par \par
The \hyperref[TEI.attDef]{<attDef>} element is used to document a single attribute, using an appropriate selection from the common elements already mentioned and the following : 
\begin{sansreflist}
  
\item [\textbf{<attDef>}] (attribute definition) contains the definition of a single attribute.\hfil\\[-10pt]\begin{sansreflist}
    \item[@{\itshape usage}]
  specifies the optionality of the attribute.
\end{sansreflist}  
\item [\textbf{<datatype>}] (datatype) specifies the declared value for an attribute, by referring to any datatype defined by the chosen schema language.\hfil\\[-10pt]\begin{sansreflist}
    \item[@{\itshape minOccurs}]
  (minimum number of occurences) indicates the minimum number of times this datatype may occur in an instance of the attribute being defined
    \item[@{\itshape maxOccurs}]
  (maximum number of occurences) indicates the maximum number of times this datatype may occur in an instance of the attribute being defined
\end{sansreflist}  
\item [\textbf{<dataRef>}] identifies the datatype of an attribute value, either by referencing an item in an externally defined datatype library, or by pointing to a TEI-defined data specification
\item [\textbf{<defaultVal>}] (default value) specifies the default declared value for an attribute.
\item [\textbf{<valDesc>}] (value description) specifies any semantic or syntactic constraint on the value that an attribute may take, additional to the information carried by the \hyperref[TEI.datatype]{<datatype>} element.
\item [\textbf{<valList>}] (value list) contains one or more \hyperref[TEI.valItem]{<valItem>} elements defining possible values.
\item [\textbf{<valItem>}] documents a single value in a predefined list of values.
\end{sansreflist}
\par
The \hyperref[TEI.attList]{<attList>} within an \hyperref[TEI.elementSpec]{<elementSpec>} is used to specify only the attributes which are specific to that particular element. Instances of the element may carry other attributes which are declared by the classes of which the element is a member. These extra attributes, which are shared by other elements, or by all elements, are specified by an \hyperref[TEI.attList]{<attList>} contained within a \hyperref[TEI.classSpec]{<classSpec>} element, as described in section \textit{\hyperref[TDCLA]{22.6.\ Class Specifications}} below.
\paragraph[{Datatypes}]{Datatypes}\label{TD-datatypes}\par
The ‘datatype’ (i.e. the kind of value) for an attribute may be specified using the elements \hyperref[TEI.datatype]{<datatype>} and \hyperref[TEI.dataRef]{<dataRef>}. A datatype may be defined in any of the following three ways: \begin{itemize}
\item by reference to an existing TEI datatype definition; 
\item by use of its name in the widely used schema datatype library maintained by the W3C as part of the definition of its schema language; 
\item by referencing its URI within some other datatype library.
\end{itemize} The TEI defines a number of datatypes, each with an identifier beginning \texttt{teidata.}, which are used in preference to the datatypes available natively from a target schema such as RELAX NG or W3C Schema since the facilities provided by different schema languages vary so widely. The TEI datatypes available are described in section \textit{\hyperref[DTYPES]{1.4.2.\ Datatype Specifications}} above. Note that each is, of necessity, mapped eventually to an externally defined datatype such as W3C Schema's \textsf{text} or \textsf{name}, possibly combined to give more expressivity, or constrained to a particular defined usage.\par
It is possible to reference a W3C schema datatype directly using {\itshape name}. In this case, the child \hyperref[TEI.dataFacet]{<dataFacet>} can be used instead of {\itshape restriction} to set W3C schema compliant restrictions on the datatype. A \hyperref[TEI.dataFacet]{<dataFacet>} is particularly useful for restrictions that can be difficult to impose and to read as a regular expression pattern. \par\bgroup\index{dataRef=<dataRef>|exampleindex}\index{name=@name!<dataRef>|exampleindex}\index{dataFacet=<dataFacet>|exampleindex}\index{name=@name!<dataFacet>|exampleindex}\index{value=@value!<dataFacet>|exampleindex}\index{dataFacet=<dataFacet>|exampleindex}\index{name=@name!<dataFacet>|exampleindex}\index{value=@value!<dataFacet>|exampleindex}\exampleFont \begin{shaded}\noindent\mbox{}{<\textbf{dataRef}\hspace*{1em}{name}="{decimal}">}\mbox{}\newline 
\hspace*{1em}{<\textbf{dataFacet}\hspace*{1em}{name}="{maxInclusive}"\mbox{}\newline 
\hspace*{1em}\hspace*{1em}{value}="{360.0}"/>}\mbox{}\newline 
\hspace*{1em}{<\textbf{dataFacet}\hspace*{1em}{name}="{minInclusive}"\mbox{}\newline 
\hspace*{1em}\hspace*{1em}{value}="{-360.0}"/>}\mbox{}\newline 
{</\textbf{dataRef}>}\end{shaded}\egroup\par \noindent  Note that restrictions are either expressed with {\itshape restriction} or \hyperref[TEI.dataFacet]{<dataFacet>}, never both.\par
Attributes {\itshape minOccurs} and {\itshape maxOccurs} are available for the case where an attribute may take more than one value of the type specified. For example, the {\itshape target} attribute provided by the \textsf{att.pointing} class has the following declaration: \par\bgroup\index{attDef=<attDef>|exampleindex}\index{ident=@ident!<attDef>|exampleindex}\index{desc=<desc>|exampleindex}\index{versionDate=@versionDate!<desc>|exampleindex}\index{datatype=<datatype>|exampleindex}\index{minOccurs=@minOccurs!<datatype>|exampleindex}\index{maxOccurs=@maxOccurs!<datatype>|exampleindex}\index{dataRef=<dataRef>|exampleindex}\index{key=@key!<dataRef>|exampleindex}\exampleFont \begin{shaded}\noindent\mbox{}{<\textbf{attDef}\hspace*{1em}{ident}="{target}">}\mbox{}\newline 
\hspace*{1em}{<\textbf{desc}\hspace*{1em}{versionDate}="{2010-05-02}"\mbox{}\newline 
\hspace*{1em}\hspace*{1em}{xml:lang}="{en}">}specifies the destination of the reference by\mbox{}\newline 
\hspace*{1em}\hspace*{1em} supplying one or more URI References{</\textbf{desc}>}\mbox{}\newline 
\hspace*{1em}{<\textbf{datatype}\hspace*{1em}{minOccurs}="{1}"\mbox{}\newline 
\hspace*{1em}\hspace*{1em}{maxOccurs}="{unbounded}">}\mbox{}\newline 
\hspace*{1em}\hspace*{1em}{<\textbf{dataRef}\hspace*{1em}{key}="{teidata.pointer}"/>}\mbox{}\newline 
\hspace*{1em}{</\textbf{datatype}>}\mbox{}\newline 
{</\textbf{attDef}>}\end{shaded}\egroup\par \noindent  indicating that the {\itshape target} attribute may take any number of values, each being of the same datatype, namely the TEI data specification \textsf{teidata.pointer}. As is usual in XML, multiple values for a single attribute are separated by one or more white space characters. Hence, values such as \texttt{\#a \#b \#c} or \texttt{http://example.org http://www.tei-c.org/index.xml} may be supplied. 
\paragraph[{Value Specification}]{Value Specification}\label{TDATTvs}\par
The \hyperref[TEI.valDesc]{<valDesc>} element may be used to describe constraints on data content in an informal way: for example \par\bgroup\index{valDesc=<valDesc>|exampleindex}\index{gi=<gi>|exampleindex}\exampleFont \begin{shaded}\noindent\mbox{}{<\textbf{valDesc}>}must point to another {<\textbf{gi}>}align{</\textbf{gi}>}\mbox{}\newline 
 element logically preceding this one.{</\textbf{valDesc}>}\end{shaded}\egroup\par \noindent  \par\bgroup\index{valDesc=<valDesc>|exampleindex}\exampleFont \begin{shaded}\noindent\mbox{}{<\textbf{valDesc}>}Values should be Library of Congress\mbox{}\newline 
 subject headings.{</\textbf{valDesc}>}\end{shaded}\egroup\par \noindent  \par\bgroup\index{valDesc=<valDesc>|exampleindex}\index{title=<title>|exampleindex}\exampleFont \begin{shaded}\noindent\mbox{}{<\textbf{valDesc}>}A bookseller's surname, taken from the list\mbox{}\newline 
 in {<\textbf{title}>}Pollard and Redgrave{</\textbf{title}>}\mbox{}\newline 
{</\textbf{valDesc}>}\end{shaded}\egroup\par \par
Constraints expressed in this way are purely documentary; to enforce them, the \hyperref[TEI.constraintSpec]{<constraintSpec>} element described in section \textit{\hyperref[TDTAGCONS]{22.5.2.\ Additional Constraints}} must be used. For example, to specify that an imaginary attribute {\itshape ageAtDeath} must take positive integer values less than 150, the datatype \textsf{teidata.numeric} might be used in combination with a \hyperref[TEI.constraintSpec]{<constraintSpec>} such as the following: \par\bgroup\index{attDef=<attDef>|exampleindex}\index{ident=@ident!<attDef>|exampleindex}\index{desc=<desc>|exampleindex}\index{datatype=<datatype>|exampleindex}\index{dataRef=<dataRef>|exampleindex}\index{key=@key!<dataRef>|exampleindex}\index{constraintSpec=<constraintSpec>|exampleindex}\index{ident=@ident!<constraintSpec>|exampleindex}\index{scheme=@scheme!<constraintSpec>|exampleindex}\index{constraint=<constraint>|exampleindex}\exampleFont \begin{shaded}\noindent\mbox{}{<\textbf{attDef}\hspace*{1em}{ident}="{ageAtDeath}">}\mbox{}\newline 
\hspace*{1em}{<\textbf{desc}>}age in years at death{</\textbf{desc}>}\mbox{}\newline 
\hspace*{1em}{<\textbf{datatype}>}\mbox{}\newline 
\hspace*{1em}\hspace*{1em}{<\textbf{dataRef}\hspace*{1em}{key}="{teidata.count}"/>}\mbox{}\newline 
\hspace*{1em}{</\textbf{datatype}>}\mbox{}\newline 
\hspace*{1em}{<\textbf{constraintSpec}\hspace*{1em}{ident}="{lessThan150}"\mbox{}\newline 
\hspace*{1em}\hspace*{1em}{scheme}="{schematron}">}\mbox{}\newline 
\hspace*{1em}\hspace*{1em}{<\textbf{constraint}>}\mbox{}\newline 
\hspace*{1em}\hspace*{1em}\hspace*{1em}{<\textbf{s:report}\hspace*{1em}{test}="{. >= 150}">} age at death must be an integer less than 150 {</\textbf{s:report}>}\mbox{}\newline 
\hspace*{1em}\hspace*{1em}{</\textbf{constraint}>}\mbox{}\newline 
\hspace*{1em}{</\textbf{constraintSpec}>}\mbox{}\newline 
{</\textbf{attDef}>}\end{shaded}\egroup\par \par
The elements \hyperref[TEI.altIdent]{<altIdent>}, \hyperref[TEI.equiv]{<equiv>}, \hyperref[TEI.gloss]{<gloss>} and \hyperref[TEI.desc]{<desc>} may all be used in the same way as they are elsewhere to describe fully the meaning of a coded value, as in the following example: \par\bgroup\index{valItem=<valItem>|exampleindex}\index{ident=@ident!<valItem>|exampleindex}\index{altIdent=<altIdent>|exampleindex}\index{equiv=<equiv>|exampleindex}\index{name=@name!<equiv>|exampleindex}\index{gloss=<gloss>|exampleindex}\index{desc=<desc>|exampleindex}\exampleFont \begin{shaded}\noindent\mbox{}{<\textbf{valItem}\hspace*{1em}{ident}="{dub}">}\mbox{}\newline 
\hspace*{1em}{<\textbf{altIdent}\hspace*{1em}{xml:lang}="{fr}">}dou{</\textbf{altIdent}>}\mbox{}\newline 
\hspace*{1em}{<\textbf{equiv}\hspace*{1em}{name}="{unknown}"/>}\mbox{}\newline 
\hspace*{1em}{<\textbf{gloss}>}dubious{</\textbf{gloss}>}\mbox{}\newline 
\hspace*{1em}{<\textbf{desc}>}used when the application of this element is doubtful or uncertain{</\textbf{desc}>}\mbox{}\newline 
{</\textbf{valItem}>}\end{shaded}\egroup\par \par
Where all the possible values for an attribute can be enumerated, the datatype \textsf{teidata.enumerated} should be used, together with a \hyperref[TEI.valList]{<valList>} element specifying the values and their significance, as in the following example: \par\bgroup\index{valList=<valList>|exampleindex}\index{type=@type!<valList>|exampleindex}\index{valItem=<valItem>|exampleindex}\index{ident=@ident!<valItem>|exampleindex}\index{gloss=<gloss>|exampleindex}\index{valItem=<valItem>|exampleindex}\index{ident=@ident!<valItem>|exampleindex}\index{gloss=<gloss>|exampleindex}\index{valItem=<valItem>|exampleindex}\index{ident=@ident!<valItem>|exampleindex}\index{gloss=<gloss>|exampleindex}\exampleFont \begin{shaded}\noindent\mbox{}{<\textbf{valList}\hspace*{1em}{type}="{closed}">}\mbox{}\newline 
\hspace*{1em}{<\textbf{valItem}\hspace*{1em}{ident}="{req}">}\mbox{}\newline 
\hspace*{1em}\hspace*{1em}{<\textbf{gloss}>}required{</\textbf{gloss}>}\mbox{}\newline 
\hspace*{1em}{</\textbf{valItem}>}\mbox{}\newline 
\hspace*{1em}{<\textbf{valItem}\hspace*{1em}{ident}="{rec}">}\mbox{}\newline 
\hspace*{1em}\hspace*{1em}{<\textbf{gloss}>}recommended{</\textbf{gloss}>}\mbox{}\newline 
\hspace*{1em}{</\textbf{valItem}>}\mbox{}\newline 
\hspace*{1em}{<\textbf{valItem}\hspace*{1em}{ident}="{opt}">}\mbox{}\newline 
\hspace*{1em}\hspace*{1em}{<\textbf{gloss}>}optional{</\textbf{gloss}>}\mbox{}\newline 
\hspace*{1em}{</\textbf{valItem}>}\mbox{}\newline 
{</\textbf{valList}>}\end{shaded}\egroup\par \noindent  Note the use of the \hyperref[TEI.gloss]{<gloss>} element here to explain the otherwise less than obvious meaning of the codes used for these values. Since this value list specifies that it is of type closed, only the values enumerated are legal, and an ODD processor will typically enforce these constraints in the schema fragment generated.\par
The \hyperref[TEI.valList]{<valList>} element can also be used to provide illustrative examples of the kinds of values expected without listing all of them. In such cases the {\itshape type} attribute will have the value open, as in the following example: \par\bgroup\index{attDef=<attDef>|exampleindex}\index{ident=@ident!<attDef>|exampleindex}\index{usage=@usage!<attDef>|exampleindex}\index{desc=<desc>|exampleindex}\index{versionDate=@versionDate!<desc>|exampleindex}\index{desc=<desc>|exampleindex}\index{versionDate=@versionDate!<desc>|exampleindex}\index{datatype=<datatype>|exampleindex}\index{dataRef=<dataRef>|exampleindex}\index{key=@key!<dataRef>|exampleindex}\index{valList=<valList>|exampleindex}\index{type=@type!<valList>|exampleindex}\index{valItem=<valItem>|exampleindex}\index{ident=@ident!<valItem>|exampleindex}\index{desc=<desc>|exampleindex}\index{versionDate=@versionDate!<desc>|exampleindex}\index{desc=<desc>|exampleindex}\index{versionDate=@versionDate!<desc>|exampleindex}\index{valItem=<valItem>|exampleindex}\index{ident=@ident!<valItem>|exampleindex}\index{desc=<desc>|exampleindex}\index{versionDate=@versionDate!<desc>|exampleindex}\index{desc=<desc>|exampleindex}\index{versionDate=@versionDate!<desc>|exampleindex}\index{valItem=<valItem>|exampleindex}\index{ident=@ident!<valItem>|exampleindex}\index{desc=<desc>|exampleindex}\index{versionDate=@versionDate!<desc>|exampleindex}\index{desc=<desc>|exampleindex}\index{versionDate=@versionDate!<desc>|exampleindex}\exampleFont \begin{shaded}\noindent\mbox{}{<\textbf{attDef}\hspace*{1em}{ident}="{type}"\hspace*{1em}{usage}="{opt}">}\mbox{}\newline 
\hspace*{1em}{<\textbf{desc}\hspace*{1em}{versionDate}="{2005-01-14}"\mbox{}\newline 
\hspace*{1em}\hspace*{1em}{xml:lang}="{en}">}characterizes the movement, for example as an\mbox{}\newline 
\hspace*{1em}\hspace*{1em} entrance or exit.{</\textbf{desc}>}\mbox{}\newline 
\hspace*{1em}{<\textbf{desc}\hspace*{1em}{versionDate}="{2007-12-20}"\mbox{}\newline 
\hspace*{1em}\hspace*{1em}{xml:lang}="{ko}">}{\textKorean 예를 들어 입장 또는 퇴장과 같은, 이동의 특성을 기술한다.}{</\textbf{desc}>}\mbox{}\newline 
\hspace*{1em}{<\textbf{datatype}>}\mbox{}\newline 
\hspace*{1em}\hspace*{1em}{<\textbf{dataRef}\hspace*{1em}{key}="{teidata.enumerated}"/>}\mbox{}\newline 
\hspace*{1em}{</\textbf{datatype}>}\mbox{}\newline 
\hspace*{1em}{<\textbf{valList}\hspace*{1em}{type}="{open}">}\mbox{}\newline 
\hspace*{1em}\hspace*{1em}{<\textbf{valItem}\hspace*{1em}{ident}="{entrance}">}\mbox{}\newline 
\hspace*{1em}\hspace*{1em}\hspace*{1em}{<\textbf{desc}\hspace*{1em}{versionDate}="{2007-06-27}"\mbox{}\newline 
\hspace*{1em}\hspace*{1em}\hspace*{1em}\hspace*{1em}{xml:lang}="{en}">}character is entering the stage.{</\textbf{desc}>}\mbox{}\newline 
\hspace*{1em}\hspace*{1em}\hspace*{1em}{<\textbf{desc}\hspace*{1em}{versionDate}="{2007-12-20}"\mbox{}\newline 
\hspace*{1em}\hspace*{1em}\hspace*{1em}\hspace*{1em}{xml:lang}="{ko}">}{\textKorean 등장인물이 무대에 등장하고 있다.}{</\textbf{desc}>}\mbox{}\newline 
\hspace*{1em}\hspace*{1em}{</\textbf{valItem}>}\mbox{}\newline 
\hspace*{1em}\hspace*{1em}{<\textbf{valItem}\hspace*{1em}{ident}="{exit}">}\mbox{}\newline 
\hspace*{1em}\hspace*{1em}\hspace*{1em}{<\textbf{desc}\hspace*{1em}{versionDate}="{2007-06-27}"\mbox{}\newline 
\hspace*{1em}\hspace*{1em}\hspace*{1em}\hspace*{1em}{xml:lang}="{en}">}character is exiting the stage.{</\textbf{desc}>}\mbox{}\newline 
\hspace*{1em}\hspace*{1em}\hspace*{1em}{<\textbf{desc}\hspace*{1em}{versionDate}="{2007-12-20}"\mbox{}\newline 
\hspace*{1em}\hspace*{1em}\hspace*{1em}\hspace*{1em}{xml:lang}="{ko}">}{\textKorean 등장인물이 무대에서 퇴장하고 있다.}{</\textbf{desc}>}\mbox{}\newline 
\hspace*{1em}\hspace*{1em}{</\textbf{valItem}>}\mbox{}\newline 
\hspace*{1em}\hspace*{1em}{<\textbf{valItem}\hspace*{1em}{ident}="{onStage}">}\mbox{}\newline 
\hspace*{1em}\hspace*{1em}\hspace*{1em}{<\textbf{desc}\hspace*{1em}{versionDate}="{2007-07-04}"\mbox{}\newline 
\hspace*{1em}\hspace*{1em}\hspace*{1em}\hspace*{1em}{xml:lang}="{en}">}character moves on stage{</\textbf{desc}>}\mbox{}\newline 
\hspace*{1em}\hspace*{1em}\hspace*{1em}{<\textbf{desc}\hspace*{1em}{versionDate}="{2007-12-20}"\mbox{}\newline 
\hspace*{1em}\hspace*{1em}\hspace*{1em}\hspace*{1em}{xml:lang}="{ko}">}{\textKorean 등장인물이 무대에서 이동한다.}{</\textbf{desc}>}\mbox{}\newline 
\hspace*{1em}\hspace*{1em}{</\textbf{valItem}>}\mbox{}\newline 
\hspace*{1em}{</\textbf{valList}>}\mbox{}\newline 
{</\textbf{attDef}>}\end{shaded}\egroup\par \noindent  The datatype will be \textsf{teidata.enumerated} in either case.\par
The \hyperref[TEI.valList]{<valList>} or \hyperref[TEI.dataRef]{<dataRef>} elements may also be used (as a child of the \hyperref[TEI.content]{<content>} element) to put constraints on the permitted content of an element, as noted at \textit{\hyperref[TDTAGCONT]{22.5.1.2.\ Defining Content Models: RELAX NG}}. This use is not however supported by all schema languages, and is therefore not recommended if support for non-RELAX NG systems is a consideration.
\subsubsection[{Processing Models}]{Processing Models}\label{TDPM}\par
As far as possible, the TEI defines elements and their attributes in a way which is entirely independent of their subsequent processing, since its intention is to maximize the reusability of encoded documents and their use in multiple contexts. Nevertheless, it can be very useful to specify one or more possible models for such processing, both to clarify the intentions of the encoder, and to provide default behaviours for a software engineer to implement when documents conforming to a particular TEI customization are processed. To that end, the following elements may be used to document one or more \textit{processing models} for a given element. 
\begin{sansreflist}
  
\item [\textbf{<model>}] describes the processing intended for a specified element.
\item [\textbf{<modelGrp>}] any grouping of \hyperref[TEI.model]{<model>} or \hyperref[TEI.modelSequence]{<modelSequence>} elements with a common output method
\item [\textbf{<modelSequence>}] any sequence of model or \hyperref[TEI.modelSequence]{<modelSequence>} elements which is to be processed as a single set of actions
\end{sansreflist}
\par
One or more of these elements may appear directly within an element specification to define the processing anticipated for that element, more specifically how it should be processed to produce the kind of output indicated by the {\itshape output} attribute. Where multiple such elements appear directly within an \hyperref[TEI.elementSpec]{<elementSpec>}, they are understood to document mutually exclusive processing models, possibly for different outputs or applicable in different contexts. Alternatively, the \hyperref[TEI.modelGrp]{<modelGrp>} element may be used to group alternative \hyperref[TEI.model]{<model>} elements intended for a single kind of output. The \hyperref[TEI.modelSequence]{<modelSequence>} element is provided for the case where a sequence of models is to be processed, functioning as a single unit.\par
A processing model suggests how a given element may be transformed to produce one or more outputs. The model is expressed in terms of \textit{behaviours} and \textit{parameters}, using high-level formatting concepts familiar to software engineers and web designers, such as ‘block’ or ‘inline’. As such, it has a different purpose from existing TEI mechanisms for documenting the appearance of source materials, such as the global attributes {\itshape rend}, {\itshape rendition} and {\itshape style}, described in sections \textit{\hyperref[HD57-1]{2.3.4.1.\ Rendition}} and \textit{\hyperref[COHQW]{3.3.1.\ What Is Highlighting?}}. It does not necessarily describe anything present in the original source, nor does it necessarily represent its original structure or semantics. A processing model is a template description, which may be used to simplify the task of producing or customizing the stylesheets needed by a formatting engine or any other form of processor.
\paragraph[{The TEI processing model}]{The TEI processing model}\label{TDPMPM}\par
The \hyperref[TEI.model]{<model>} element is used to document the processing model intended for a particular element in an abstract manner, independently of its implementation in whatever processing language is chosen. This is achieved by means of the following attributes and elements: 
\begin{sansreflist}
  
\item [\textbf{<model>}] describes the processing intended for a specified element.\hfil\\[-10pt]\begin{sansreflist}
    \item[@{\itshape predicate [att.predicate]}]
  the condition under which the element bearing this attribute applies, given as an XPath predicate expression.
    \item[@{\itshape behaviour}]
  names the process or function which this processing model uses in order to produce output.
    \item[@{\itshape output}]
  the intended output.
    \item[@{\itshape useSourceRendition}]
  whether to obey any rendition attribute which is present.
    \item[@{\itshape cssClass}]
  the name of a CSS class which should be associated with this element
\end{sansreflist}  
\item [\textbf{<outputRendition>}] describes the rendering or appearance intended for all occurrences of an element in a specified context for a specified type of output.\hfil\\[-10pt]\begin{sansreflist}
    \item[@{\itshape scope}]
  provides a way of defining ‘pseudo-elements’, that is, styling rules applicable to specific sub-portions of an element.
\end{sansreflist}  
\end{sansreflist}
 The mandatory {\itshape behaviour} attribute defines in broad terms how an element should be processed, for example as a block or as an inline element. The optional {\itshape predicate} attribute may be used to specify a subset of contexts in which this model should be applicable: for example, an element might be treated as a block element in some contexts, but not in others. The {\itshape output} attribute supplies a name for the output for which this model is intended, for example for screen display, for a printed reading copy, for a scholarly publication, etc. The way in which an element should be rendered is declared independently of its behaviour, using either the attribute {\itshape useSourceRendition} or the element \hyperref[TEI.outputRendition]{<outputRendition>}. These Guidelines recommend that \hyperref[TEI.outputRendition]{<outputRendition>} be expressed using the W3C Cascading Stylesheet Language (CSS), but other possibilities are not excluded. The particular language used may be documented by means of the \hyperref[TEI.styleDefDecl]{<styleDefDecl>} element described in \textit{\hyperref[HD57-1a]{2.3.5.\ The Default Style Definition Language Declaration}}. 
\paragraph[{Output Rendition }]{Output Rendition }\label{TDPMOR}\par
Here is a simple example of a processing model which might be included in the specification for an element such as \hyperref[TEI.hi]{<hi>} or \hyperref[TEI.foreign]{<foreign>}. The intent is that these elements should be presented inline using an italic font. \par\bgroup\index{model=<model>|exampleindex}\index{behaviour=@behaviour!<model>|exampleindex}\index{outputRendition=<outputRendition>|exampleindex}\exampleFont \begin{shaded}\noindent\mbox{}{<\textbf{model}\hspace*{1em}{behaviour}="{inline}">}\mbox{}\newline 
\hspace*{1em}{<\textbf{outputRendition}>}font-style: italic;{</\textbf{outputRendition}>}\mbox{}\newline 
{</\textbf{model}>}\end{shaded}\egroup\par \noindent  If the \hyperref[TEI.rendition]{<rendition>} element, or the attributes {\itshape style}, {\itshape rend}, or {\itshape rendition} have already been used in the source document to indicate elements that were originally rendered in italic, and we wish simply to follow this in our processing, then there is no need to include an \hyperref[TEI.outputRendition]{<outputRendition>} element, and the attribute {\itshape useSourceRendition} could be used as follows: \par\bgroup\index{model=<model>|exampleindex}\index{behaviour=@behaviour!<model>|exampleindex}\index{useSourceRendition=@useSourceRendition!<model>|exampleindex}\exampleFont \begin{shaded}\noindent\mbox{}{<\textbf{model}\hspace*{1em}{behaviour}="{inline}"\mbox{}\newline 
\hspace*{1em}{useSourceRendition}="{true}"/>}\end{shaded}\egroup\par \noindent  Any rendition information present in the source document will be ignored unless the {\itshape useSourceRendition} attribute has the value true. If that is the case, then such information will be combined with any rendition information supplied by the \hyperref[TEI.outputRendition]{<outputRendition>} element. For example, using CSS, an element which appears in the source as follows \par\bgroup\index{hi=<hi>|exampleindex}\index{style=@style!<hi>|exampleindex}\exampleFont \begin{shaded}\noindent\mbox{}{<\textbf{hi}\hspace*{1em}{style}="{font-weight:bold;}">}this is in bold{</\textbf{hi}>}\end{shaded}\egroup\par \noindent  would appear in bold and italic if processed by the following model \par\bgroup\index{model=<model>|exampleindex}\index{behaviour=@behaviour!<model>|exampleindex}\index{useSourceRendition=@useSourceRendition!<model>|exampleindex}\index{outputRendition=<outputRendition>|exampleindex}\exampleFont \begin{shaded}\noindent\mbox{}{<\textbf{model}\hspace*{1em}{behaviour}="{inline}"\mbox{}\newline 
\hspace*{1em}{useSourceRendition}="{true}">}\mbox{}\newline 
\hspace*{1em}{<\textbf{outputRendition}>}font-style: italic;{</\textbf{outputRendition}>}\mbox{}\newline 
{</\textbf{model}>}\end{shaded}\egroup\par 
\paragraph[{CSS Class}]{CSS Class}\label{TDPMCC}\par
In a typical workflow processing TEI documents for display on the web, a system designer will often wish to use an externally defined CSS stylesheet. The {\itshape cssClass} attribute simplifies the task of maintaining compatibility amongst the possibly many applications using such a stylesheet and also enables a TEI application to specify the names of classes to be used for particular processing models. For example, supposing that the associated CSS stylesheet includes a CSS class called \textsf{labeled-list}, the following processing model might be used to request it be used for \hyperref[TEI.list]{<list>} elements containing a child \hyperref[TEI.label]{<label>} element: \par\bgroup\index{elementSpec=<elementSpec>|exampleindex}\index{ident=@ident!<elementSpec>|exampleindex}\index{mode=@mode!<elementSpec>|exampleindex}\index{model=<model>|exampleindex}\index{predicate=@predicate!<model>|exampleindex}\index{behaviour=@behaviour!<model>|exampleindex}\index{cssClass=@cssClass!<model>|exampleindex}\exampleFont \begin{shaded}\noindent\mbox{}{<\textbf{elementSpec}\hspace*{1em}{ident}="{list}"\hspace*{1em}{mode}="{change}">}\mbox{}\newline 
\hspace*{1em}{<\textbf{model}\hspace*{1em}{predicate}="{label}"\hspace*{1em}{behaviour}="{list}"\mbox{}\newline 
\hspace*{1em}\hspace*{1em}{cssClass}="{labeled-list}">}\mbox{}\newline 
\textit{<!-- ... -->}\mbox{}\newline 
\hspace*{1em}{</\textbf{model}>}\mbox{}\newline 
{</\textbf{elementSpec}>}\end{shaded}\egroup\par \par
In the following example, a table will be formatted using renditional information provided in the source if that is available, or by an external stylesheet, using one of the CSS classes specified, if it is not: \par\bgroup\index{elementSpec=<elementSpec>|exampleindex}\index{mode=@mode!<elementSpec>|exampleindex}\index{ident=@ident!<elementSpec>|exampleindex}\index{model=<model>|exampleindex}\index{predicate=@predicate!<model>|exampleindex}\index{behaviour=@behaviour!<model>|exampleindex}\index{useSourceRendition=@useSourceRendition!<model>|exampleindex}\index{model=<model>|exampleindex}\index{behaviour=@behaviour!<model>|exampleindex}\index{useSourceRendition=@useSourceRendition!<model>|exampleindex}\index{cssClass=@cssClass!<model>|exampleindex}\exampleFont \begin{shaded}\noindent\mbox{}{<\textbf{elementSpec}\hspace*{1em}{mode}="{change}"\hspace*{1em}{ident}="{table}">}\mbox{}\newline 
\textit{<!-- Preserve original rendition for tables which contain @rendition hints -->}\mbox{}\newline 
\hspace*{1em}{<\textbf{model}\hspace*{1em}{predicate}="{.//row/@rendition or .//cell/@rendition}"\mbox{}\newline 
\hspace*{1em}\hspace*{1em}{behaviour}="{table}"\hspace*{1em}{useSourceRendition}="{true}"/>}\mbox{}\newline 
\textit{<!-- Use bootstrap for default table styling -->}\mbox{}\newline 
\hspace*{1em}{<\textbf{model}\hspace*{1em}{behaviour}="{table}"\mbox{}\newline 
\hspace*{1em}\hspace*{1em}{useSourceRendition}="{true}"\mbox{}\newline 
\hspace*{1em}\hspace*{1em}{cssClass}="{table table-hover table-bordered}"/>}\mbox{}\newline 
{</\textbf{elementSpec}>}\end{shaded}\egroup\par \par
As discussed further below, the input data available to a processing model is by default the content of the element being processed, together with its child nodes. 
\paragraph[{Model Contexts and Outputs}]{Model Contexts and Outputs}\label{TDPMMC}\par
Sometimes different processing models are required for the same element in different contexts. For example, we may wish to process the \hyperref[TEI.quote]{<quote>} element as an inline italic element when it appears inside a \hyperref[TEI.p]{<p>} element, but as an indented block when it appears elsewhere. To achieve this, we need to change the specification for the \hyperref[TEI.quote]{<quote>} element to include two \hyperref[TEI.model]{<model>} elements as follows: \par\bgroup\index{elementSpec=<elementSpec>|exampleindex}\index{ident=@ident!<elementSpec>|exampleindex}\index{mode=@mode!<elementSpec>|exampleindex}\index{model=<model>|exampleindex}\index{predicate=@predicate!<model>|exampleindex}\index{behaviour=@behaviour!<model>|exampleindex}\index{outputRendition=<outputRendition>|exampleindex}\index{model=<model>|exampleindex}\index{behaviour=@behaviour!<model>|exampleindex}\index{outputRendition=<outputRendition>|exampleindex}\exampleFont \begin{shaded}\noindent\mbox{}{<\textbf{elementSpec}\hspace*{1em}{ident}="{quote}"\hspace*{1em}{mode}="{change}">}\mbox{}\newline 
\hspace*{1em}{<\textbf{model}\hspace*{1em}{predicate}="{ancestor::p}"\mbox{}\newline 
\hspace*{1em}\hspace*{1em}{behaviour}="{inline}">}\mbox{}\newline 
\hspace*{1em}\hspace*{1em}{<\textbf{outputRendition}>}font-style: italic;{</\textbf{outputRendition}>}\mbox{}\newline 
\hspace*{1em}{</\textbf{model}>}\mbox{}\newline 
\hspace*{1em}{<\textbf{model}\hspace*{1em}{behaviour}="{block}">}\mbox{}\newline 
\hspace*{1em}\hspace*{1em}{<\textbf{outputRendition}>}left-margin: 2em;{</\textbf{outputRendition}>}\mbox{}\newline 
\hspace*{1em}{</\textbf{model}>}\mbox{}\newline 
{</\textbf{elementSpec}>}\end{shaded}\egroup\par \noindent  As noted above, these two models are mutually exclusive. The first processing model will be used only for \hyperref[TEI.quote]{<quote>} elements which match the XPath expression given as value for the {\itshape predicate} attribute. All other element occurrences will use the second processing model.\par
When, as here, multiple behaviours are required for the same element, it will often be the case that the appropriate processing will depend on the context. It may however be the case that the choice of an appropriate model will be made on the basis of the intended output. For example, we might wish to define quite different behaviours when a document is to be displayed on a mobile device and when it is to be displayed on a desktop screen. Different behaviours again might be specified for a print version intended for the general reader, and for a print version aimed at the technical specialist.\par
The \hyperref[TEI.modelGrp]{<modelGrp>} element can be used to group together all the processing models which have in common a particular intended output, as in the following example: \par\bgroup\index{modelGrp=<modelGrp>|exampleindex}\index{output=@output!<modelGrp>|exampleindex}\index{model=<model>|exampleindex}\index{behaviour=@behaviour!<model>|exampleindex}\index{predicate=@predicate!<model>|exampleindex}\index{outputRendition=<outputRendition>|exampleindex}\index{model=<model>|exampleindex}\index{behaviour=@behaviour!<model>|exampleindex}\index{predicate=@predicate!<model>|exampleindex}\index{outputRendition=<outputRendition>|exampleindex}\index{modelGrp=<modelGrp>|exampleindex}\index{output=@output!<modelGrp>|exampleindex}\index{model=<model>|exampleindex}\index{behaviour=@behaviour!<model>|exampleindex}\index{predicate=@predicate!<model>|exampleindex}\index{outputRendition=<outputRendition>|exampleindex}\index{model=<model>|exampleindex}\index{behaviour=@behaviour!<model>|exampleindex}\index{useSourceRendition=@useSourceRendition!<model>|exampleindex}\index{predicate=@predicate!<model>|exampleindex}\index{outputRendition=<outputRendition>|exampleindex}\exampleFont \begin{shaded}\noindent\mbox{}{<\textbf{modelGrp}\hspace*{1em}{output}="{mobile}">}\mbox{}\newline 
\hspace*{1em}{<\textbf{model}\hspace*{1em}{behaviour}="{inline}"\mbox{}\newline 
\hspace*{1em}\hspace*{1em}{predicate}="{@rend='inline'}">}\mbox{}\newline 
\hspace*{1em}\hspace*{1em}{<\textbf{outputRendition}>}font-size: 7pt;{</\textbf{outputRendition}>}\mbox{}\newline 
\hspace*{1em}{</\textbf{model}>}\mbox{}\newline 
\hspace*{1em}{<\textbf{model}\hspace*{1em}{behaviour}="{block}"\mbox{}\newline 
\hspace*{1em}\hspace*{1em}{predicate}="{@rend='block'}">}\mbox{}\newline 
\hspace*{1em}\hspace*{1em}{<\textbf{outputRendition}>}text-color: red;{</\textbf{outputRendition}>}\mbox{}\newline 
\hspace*{1em}{</\textbf{model}>}\mbox{}\newline 
{</\textbf{modelGrp}>}\mbox{}\newline 
{<\textbf{modelGrp}\hspace*{1em}{output}="{print}">}\mbox{}\newline 
\hspace*{1em}{<\textbf{model}\hspace*{1em}{behaviour}="{inline}"\mbox{}\newline 
\hspace*{1em}\hspace*{1em}{predicate}="{@rend='inline'}">}\mbox{}\newline 
\hspace*{1em}\hspace*{1em}{<\textbf{outputRendition}>}font-size: 12pt;{</\textbf{outputRendition}>}\mbox{}\newline 
\hspace*{1em}{</\textbf{model}>}\mbox{}\newline 
\hspace*{1em}{<\textbf{model}\hspace*{1em}{behaviour}="{block}"\mbox{}\newline 
\hspace*{1em}\hspace*{1em}{useSourceRendition}="{true}"\hspace*{1em}{predicate}="{@rend='block'}">}\mbox{}\newline 
\hspace*{1em}\hspace*{1em}{<\textbf{outputRendition}>}text-align: center;{</\textbf{outputRendition}>}\mbox{}\newline 
\hspace*{1em}{</\textbf{model}>}\mbox{}\newline 
{</\textbf{modelGrp}>}\end{shaded}\egroup\par 
\paragraph[{Behaviours and their parameters}]{Behaviours and their parameters}\label{TDPMMB}\par
In the examples above we have used without explanation or definition two simple \textit{behaviours}: \textsf{inline} and \textsf{block}, but many other behaviours are possible. A list of recommended behaviour names forms part of the specification for the element \hyperref[TEI.model]{<model>}. A processing model can specify any named behaviour, some of which have additional \textit{parameters}. The \textsf{parameters} of a \textsf{behaviour} resemble the arguments of a function in many programming languages: they provide names which can be used to distinguish particular parts of the input data available to the process used to implement the behaviour in question.\par
The following elements are used to represent and to define parameters: 
\begin{sansreflist}
  
\item [\textbf{<param>}] provides a parameter for a model behaviour by supplying its name and an XPath expression identifying the location of its content.\hfil\\[-10pt]\begin{sansreflist}
    \item[@{\itshape name}]
  a name for the parameter being supplied
\end{sansreflist}  
\item [\textbf{<paramList>}] list of parameter specifications
\item [\textbf{<paramSpec>}] supplies specification for one parameter of a model behaviour
\end{sansreflist}
\par
By default, a processor implementing the TEI processing model for a particular element has available to it as input data the content of the element itself, and all of its children. One or more \hyperref[TEI.param]{<param>} elements may be supplied within a \hyperref[TEI.model]{<model>} element to specify \textit{parameters} which modify this, either by selecting particular parts of the default input data, or by selecting data which would not otherwise be available. In either case, the value supplied for the parameter is given as an XPath expression, evaluated with respect to the element node being processed. An arbitrary name (defined in the corresponding \hyperref[TEI.paramSpec]{<paramSpec>}) is also supplied to a processor to identify each parameter.\par
For example, an element such as the TEI \hyperref[TEI.ref]{<ref>} element will probably be associated with a processing model which treats it as a hyperlink. But a hyperlink (in most implementations) often has two associated pieces of information: the address indicated, and some textual content serving to label the link . In HTML, the former is provided as value of the {\itshape href} element, and the latter by the content of an \texttt{<a>} element. In the following processing model we define a behaviour called \textsf{link}, which will use whatever is indicated by the parameter called \textsf{uri}  to provide the former, while the latter is provided by the content of the \hyperref[TEI.ref]{<ref>} element itself: \par\bgroup\index{elementSpec=<elementSpec>|exampleindex}\index{ident=@ident!<elementSpec>|exampleindex}\index{mode=@mode!<elementSpec>|exampleindex}\index{model=<model>|exampleindex}\index{behaviour=@behaviour!<model>|exampleindex}\index{param=<param>|exampleindex}\index{name=@name!<param>|exampleindex}\index{value=@value!<param>|exampleindex}\index{param=<param>|exampleindex}\index{name=@name!<param>|exampleindex}\index{value=@value!<param>|exampleindex}\exampleFont \begin{shaded}\noindent\mbox{}{<\textbf{elementSpec}\hspace*{1em}{ident}="{ref}"\hspace*{1em}{mode}="{add}">}\mbox{}\newline 
\hspace*{1em}{<\textbf{model}\hspace*{1em}{behaviour}="{link}">}\mbox{}\newline 
\hspace*{1em}\hspace*{1em}{<\textbf{param}\hspace*{1em}{name}="{uri}"\hspace*{1em}{value}="{@target}"/>}\mbox{}\newline 
\hspace*{1em}\hspace*{1em}{<\textbf{param}\hspace*{1em}{name}="{content}"\hspace*{1em}{value}="{.}"/>}\mbox{}\newline 
\hspace*{1em}{</\textbf{model}>}\mbox{}\newline 
{</\textbf{elementSpec}>}\end{shaded}\egroup\par \noindent  The {\itshape value} attribute of a \hyperref[TEI.param]{<param>} element supplies an XPath expression that indicates where the required value may be found. The context for this XPath is the element which is being processed; hence in this example, the \textsf{uri} parameter takes the value of the {\itshape target} attribute on the \hyperref[TEI.ref]{<ref>} element being processed. The \textsf{content} parameter indicates that the content of that \hyperref[TEI.ref]{<ref>} element should be provided as its value. (This parameter is not strictly necessary, since by default the whole content of the element being processed is always available to a processor, but supplying it in this way makes the procedure more explicit).\par
All the parameters available for a given behaviour are defined as a part of the definition of the behaviour itself, as further discussed in section \textit{\hyperref[TDPMDEF]{22.5.4.8.\ Defining a processing model}} below.\par
As a further example, the TEI \hyperref[TEI.choice]{<choice>} element requires a different behaviour for which the name \textsf{alternate} is proposed as in the following example: \par\bgroup\index{elementSpec=<elementSpec>|exampleindex}\index{ident=@ident!<elementSpec>|exampleindex}\index{mode=@mode!<elementSpec>|exampleindex}\index{model=<model>|exampleindex}\index{predicate=@predicate!<model>|exampleindex}\index{behaviour=@behaviour!<model>|exampleindex}\index{param=<param>|exampleindex}\index{name=@name!<param>|exampleindex}\index{value=@value!<param>|exampleindex}\index{param=<param>|exampleindex}\index{name=@name!<param>|exampleindex}\index{value=@value!<param>|exampleindex}\exampleFont \begin{shaded}\noindent\mbox{}{<\textbf{elementSpec}\hspace*{1em}{ident}="{choice}"\hspace*{1em}{mode}="{change}">}\mbox{}\newline 
\hspace*{1em}{<\textbf{model}\hspace*{1em}{predicate}="{sic and corr}"\mbox{}\newline 
\hspace*{1em}\hspace*{1em}{behaviour}="{alternate}">}\mbox{}\newline 
\hspace*{1em}\hspace*{1em}{<\textbf{param}\hspace*{1em}{name}="{default}"\hspace*{1em}{value}="{corr}"/>}\mbox{}\newline 
\hspace*{1em}\hspace*{1em}{<\textbf{param}\hspace*{1em}{name}="{alternate}"\hspace*{1em}{value}="{sic}"/>}\mbox{}\newline 
\hspace*{1em}{</\textbf{model}>}\mbox{}\newline 
{</\textbf{elementSpec}>}\end{shaded}\egroup\par \noindent  The processing model shown here will be selected for processing a \hyperref[TEI.choice]{<choice>} element which has both \hyperref[TEI.sic]{<sic>} and \hyperref[TEI.corr]{<corr>} child elements. The names \textsf{default} and \textsf{alternate} here are provided for convenience. The \textsf{default} parameter provides the value of the child \hyperref[TEI.corr]{<corr>} element, and the \textsf{alternate} parameter will provide that of the child \hyperref[TEI.sic]{<sic>} elements. If neither \hyperref[TEI.param]{<param>} element was supplied, both elements would still be available to an application, but the application would need to distinguish them for itself.\par
A \hyperref[TEI.choice]{<choice>} element might contain multiple corrections, each with differing values for their {\itshape cert} attribute. In the following processing model, we will accept as value of the \textsf{default} attribute only those child \hyperref[TEI.corr]{<corr>} elements which have a value high for that attribute: \par\bgroup\index{elementSpec=<elementSpec>|exampleindex}\index{ident=@ident!<elementSpec>|exampleindex}\index{mode=@mode!<elementSpec>|exampleindex}\index{model=<model>|exampleindex}\index{predicate=@predicate!<model>|exampleindex}\index{behaviour=@behaviour!<model>|exampleindex}\index{param=<param>|exampleindex}\index{name=@name!<param>|exampleindex}\index{value=@value!<param>|exampleindex}\index{param=<param>|exampleindex}\index{name=@name!<param>|exampleindex}\index{value=@value!<param>|exampleindex}\exampleFont \begin{shaded}\noindent\mbox{}{<\textbf{elementSpec}\hspace*{1em}{ident}="{choice}"\hspace*{1em}{mode}="{change}">}\mbox{}\newline 
\hspace*{1em}{<\textbf{model}\hspace*{1em}{predicate}="{sic and corr}"\mbox{}\newline 
\hspace*{1em}\hspace*{1em}{behaviour}="{alternate}">}\mbox{}\newline 
\hspace*{1em}\hspace*{1em}{<\textbf{param}\hspace*{1em}{name}="{default}"\mbox{}\newline 
\hspace*{1em}\hspace*{1em}\hspace*{1em}{value}="{corr[@cert='high']}"/>}\mbox{}\newline 
\hspace*{1em}\hspace*{1em}{<\textbf{param}\hspace*{1em}{name}="{alternate}"\hspace*{1em}{value}="{sic}"/>}\mbox{}\newline 
\hspace*{1em}{</\textbf{model}>}\mbox{}\newline 
{</\textbf{elementSpec}>}\end{shaded}\egroup\par \par
A \hyperref[TEI.choice]{<choice>} element might contain several different pairs of alternate elements (\hyperref[TEI.abbr]{<abbr>} and \hyperref[TEI.expan]{<expan>}, \hyperref[TEI.orig]{<orig>} and \hyperref[TEI.reg]{<reg>}, etc.) We might wish to group together a set of processing models for these, for example to determine which of the possible alternatives is displayed by default whenever a \hyperref[TEI.choice]{<choice>} element is processed for output to the web: \par\bgroup\index{elementSpec=<elementSpec>|exampleindex}\index{ident=@ident!<elementSpec>|exampleindex}\index{mode=@mode!<elementSpec>|exampleindex}\index{modelGrp=<modelGrp>|exampleindex}\index{output=@output!<modelGrp>|exampleindex}\index{model=<model>|exampleindex}\index{predicate=@predicate!<model>|exampleindex}\index{behaviour=@behaviour!<model>|exampleindex}\index{param=<param>|exampleindex}\index{name=@name!<param>|exampleindex}\index{value=@value!<param>|exampleindex}\index{param=<param>|exampleindex}\index{name=@name!<param>|exampleindex}\index{value=@value!<param>|exampleindex}\index{model=<model>|exampleindex}\index{predicate=@predicate!<model>|exampleindex}\index{behaviour=@behaviour!<model>|exampleindex}\index{param=<param>|exampleindex}\index{name=@name!<param>|exampleindex}\index{value=@value!<param>|exampleindex}\index{param=<param>|exampleindex}\index{name=@name!<param>|exampleindex}\index{value=@value!<param>|exampleindex}\index{model=<model>|exampleindex}\index{predicate=@predicate!<model>|exampleindex}\index{behaviour=@behaviour!<model>|exampleindex}\index{param=<param>|exampleindex}\index{name=@name!<param>|exampleindex}\index{value=@value!<param>|exampleindex}\index{param=<param>|exampleindex}\index{name=@name!<param>|exampleindex}\index{value=@value!<param>|exampleindex}\exampleFont \begin{shaded}\noindent\mbox{}{<\textbf{elementSpec}\hspace*{1em}{ident}="{choice}"\hspace*{1em}{mode}="{change}">}\mbox{}\newline 
\hspace*{1em}{<\textbf{modelGrp}\hspace*{1em}{output}="{web}">}\mbox{}\newline 
\hspace*{1em}\hspace*{1em}{<\textbf{model}\hspace*{1em}{predicate}="{sic and corr}"\mbox{}\newline 
\hspace*{1em}\hspace*{1em}\hspace*{1em}{behaviour}="{alternate}">}\mbox{}\newline 
\hspace*{1em}\hspace*{1em}\hspace*{1em}{<\textbf{param}\hspace*{1em}{name}="{default}"\mbox{}\newline 
\hspace*{1em}\hspace*{1em}\hspace*{1em}\hspace*{1em}{value}="{corr[@cert='high']}"/>}\mbox{}\newline 
\hspace*{1em}\hspace*{1em}\hspace*{1em}{<\textbf{param}\hspace*{1em}{name}="{alternate}"\hspace*{1em}{value}="{sic}"/>}\mbox{}\newline 
\hspace*{1em}\hspace*{1em}{</\textbf{model}>}\mbox{}\newline 
\hspace*{1em}\hspace*{1em}{<\textbf{model}\hspace*{1em}{predicate}="{abbr and expan}"\mbox{}\newline 
\hspace*{1em}\hspace*{1em}\hspace*{1em}{behaviour}="{alternate}">}\mbox{}\newline 
\hspace*{1em}\hspace*{1em}\hspace*{1em}{<\textbf{param}\hspace*{1em}{name}="{default}"\hspace*{1em}{value}="{expan[1]}"/>}\mbox{}\newline 
\hspace*{1em}\hspace*{1em}\hspace*{1em}{<\textbf{param}\hspace*{1em}{name}="{alternate}"\hspace*{1em}{value}="{abbr}"/>}\mbox{}\newline 
\hspace*{1em}\hspace*{1em}{</\textbf{model}>}\mbox{}\newline 
\hspace*{1em}\hspace*{1em}{<\textbf{model}\hspace*{1em}{predicate}="{orig and reg}"\mbox{}\newline 
\hspace*{1em}\hspace*{1em}\hspace*{1em}{behaviour}="{alternate}">}\mbox{}\newline 
\hspace*{1em}\hspace*{1em}\hspace*{1em}{<\textbf{param}\hspace*{1em}{name}="{default}"\hspace*{1em}{value}="{reg}"/>}\mbox{}\newline 
\hspace*{1em}\hspace*{1em}\hspace*{1em}{<\textbf{param}\hspace*{1em}{name}="{alternate}"\hspace*{1em}{value}="{orig}"/>}\mbox{}\newline 
\hspace*{1em}\hspace*{1em}{</\textbf{model}>}\mbox{}\newline 
\hspace*{1em}{</\textbf{modelGrp}>}\mbox{}\newline 
{</\textbf{elementSpec}>}\end{shaded}\egroup\par \par
If nothing matches the XPath defining the value of a particular parameter (e.g. if in the above example there is no correction with {\itshape cert}=high) then the \textsf{default} parameter has no value. It is left to implementors to determine how null-valued parameters should be processed.
\paragraph[{Outputs}]{Outputs}\label{TDPMOU}\par
As noted above, the {\itshape output} attribute is used to associate particular processing models with a specific type of output. The following example documents a range of processing intentions for the \hyperref[TEI.date]{<date>} element, intended to cope with at least the following three situations:\begin{enumerate}
\item there is text inside the element, and the output is print; 
\item there is no text inside the element but there is a {\itshape when} attribute, and the output is print;
\item there is a {\itshape when} attribute, there is text inside the element, and the output is web
\end{enumerate}\par\bgroup\index{elementSpec=<elementSpec>|exampleindex}\index{ident=@ident!<elementSpec>|exampleindex}\index{mode=@mode!<elementSpec>|exampleindex}\index{modelGrp=<modelGrp>|exampleindex}\index{output=@output!<modelGrp>|exampleindex}\index{model=<model>|exampleindex}\index{predicate=@predicate!<model>|exampleindex}\index{behaviour=@behaviour!<model>|exampleindex}\index{model=<model>|exampleindex}\index{predicate=@predicate!<model>|exampleindex}\index{behaviour=@behaviour!<model>|exampleindex}\index{param=<param>|exampleindex}\index{name=@name!<param>|exampleindex}\index{value=@value!<param>|exampleindex}\index{model=<model>|exampleindex}\index{output=@output!<model>|exampleindex}\index{predicate=@predicate!<model>|exampleindex}\index{behaviour=@behaviour!<model>|exampleindex}\index{param=<param>|exampleindex}\index{name=@name!<param>|exampleindex}\index{value=@value!<param>|exampleindex}\index{param=<param>|exampleindex}\index{name=@name!<param>|exampleindex}\index{value=@value!<param>|exampleindex}\exampleFont \begin{shaded}\noindent\mbox{}{<\textbf{elementSpec}\hspace*{1em}{ident}="{date}"\hspace*{1em}{mode}="{change}">}\mbox{}\newline 
\hspace*{1em}{<\textbf{modelGrp}\hspace*{1em}{output}="{print}">}\mbox{}\newline 
\hspace*{1em}\hspace*{1em}{<\textbf{model}\hspace*{1em}{predicate}="{text()}"\mbox{}\newline 
\hspace*{1em}\hspace*{1em}\hspace*{1em}{behaviour}="{inline}"/>}\mbox{}\newline 
\hspace*{1em}\hspace*{1em}{<\textbf{model}\hspace*{1em}{predicate}="{@when and not(text())}"\mbox{}\newline 
\hspace*{1em}\hspace*{1em}\hspace*{1em}{behaviour}="{inline}">}\mbox{}\newline 
\hspace*{1em}\hspace*{1em}\hspace*{1em}{<\textbf{param}\hspace*{1em}{name}="{content}"\hspace*{1em}{value}="{@when}"/>}\mbox{}\newline 
\hspace*{1em}\hspace*{1em}{</\textbf{model}>}\mbox{}\newline 
\hspace*{1em}{</\textbf{modelGrp}>}\mbox{}\newline 
\hspace*{1em}{<\textbf{model}\hspace*{1em}{output}="{web}"\hspace*{1em}{predicate}="{@when}"\mbox{}\newline 
\hspace*{1em}\hspace*{1em}{behaviour}="{alternate}">}\mbox{}\newline 
\hspace*{1em}\hspace*{1em}{<\textbf{param}\hspace*{1em}{name}="{default}"\hspace*{1em}{value}="{.}"/>}\mbox{}\newline 
\hspace*{1em}\hspace*{1em}{<\textbf{param}\hspace*{1em}{name}="{alternate}"\hspace*{1em}{value}="{@when}"/>}\mbox{}\newline 
\hspace*{1em}{</\textbf{model}>}\mbox{}\newline 
{</\textbf{elementSpec}>}\end{shaded}\egroup\par \par
For output to print we supply two processing models, one for the simplest case where the content of the \hyperref[TEI.date]{<date>} is to be treated as an inline element, and the other for the case where there is no content and the value of the {\itshape when} attribute is to be used in its place. This is specified by a parameter, called \textsf{content} in this example. For output to web, we use the \textsf{alternate} behaviour discussed in the previous section to indicate that by default the content of the element will be used, while retaining access to the value of the {\itshape when} attribute, this time via a parameter called \textsf{alternate}.
\paragraph[{Model sequence}]{Model sequence}\label{TDPMMS}\par
As well as being combined to form model groups, several models may be combined to form a \textit{model sequence}. All of the individual components of a model sequence are understood to be applied, rather than considered to be mutually exclusive alternatives. For example, we might wish to define two different behaviours for a \hyperref[TEI.note]{<note>} element: the inline behaviour should be used to display the value of the note number (given by its {\itshape n} attribute), while a different behaviour (here called \textsf{footnote}) should be used to display the content of the element at a specified place, given by the parameter \textsf{place}. Because both of these actions are required, the two models are grouped by a \hyperref[TEI.modelSequence]{<modelSequence>} element: \par\bgroup\index{elementSpec=<elementSpec>|exampleindex}\index{ident=@ident!<elementSpec>|exampleindex}\index{mode=@mode!<elementSpec>|exampleindex}\index{modelSequence=<modelSequence>|exampleindex}\index{output=@output!<modelSequence>|exampleindex}\index{model=<model>|exampleindex}\index{behaviour=@behaviour!<model>|exampleindex}\index{param=<param>|exampleindex}\index{name=@name!<param>|exampleindex}\index{value=@value!<param>|exampleindex}\index{model=<model>|exampleindex}\index{behaviour=@behaviour!<model>|exampleindex}\index{param=<param>|exampleindex}\index{name=@name!<param>|exampleindex}\index{value=@value!<param>|exampleindex}\exampleFont \begin{shaded}\noindent\mbox{}{<\textbf{elementSpec}\hspace*{1em}{ident}="{note}"\hspace*{1em}{mode}="{change}">}\mbox{}\newline 
\hspace*{1em}{<\textbf{modelSequence}\hspace*{1em}{output}="{print}">}\mbox{}\newline 
\hspace*{1em}\hspace*{1em}{<\textbf{model}\hspace*{1em}{behaviour}="{inline}">}\mbox{}\newline 
\hspace*{1em}\hspace*{1em}\hspace*{1em}{<\textbf{param}\hspace*{1em}{name}="{content}"\hspace*{1em}{value}="{@n}"/>}\mbox{}\newline 
\hspace*{1em}\hspace*{1em}{</\textbf{model}>}\mbox{}\newline 
\hspace*{1em}\hspace*{1em}{<\textbf{model}\hspace*{1em}{behaviour}="{footnote}">}\mbox{}\newline 
\hspace*{1em}\hspace*{1em}\hspace*{1em}{<\textbf{param}\hspace*{1em}{name}="{place}"\hspace*{1em}{value}="{'foot'}"/>}\mbox{}\newline 
\hspace*{1em}\hspace*{1em}{</\textbf{model}>}\mbox{}\newline 
\hspace*{1em}{</\textbf{modelSequence}>}\mbox{}\newline 
{</\textbf{elementSpec}>}\end{shaded}\egroup\par \par
The value of the parameter called \textsf{place} above is an XPath expression supplying an arbitrary string (‘foot’), which is therefore quoted. It is left to implementors to validate or constrain the possible values for such expressions.
\paragraph[{Defining a processing model}]{Defining a processing model}\label{TDPMDEF}\par
The processing model for an element is defined using some combination of \hyperref[TEI.model]{<model>}, \hyperref[TEI.modelSequence]{<modelSequence>}, or \hyperref[TEI.modelGrp]{<modelGrp>} elements within the \hyperref[TEI.elementSpec]{<elementSpec>} element containing its specification. The processing to be carried out is defined by means of the behaviour specified for each \hyperref[TEI.model]{<model>} element, possibly supplying specified values for a number of named parameters. The parameters available for a given behaviour are specified using a number of \hyperref[TEI.param]{<param>} elements grouped together in a \hyperref[TEI.paramList]{<paramList>} element. This \hyperref[TEI.paramList]{<paramList>} is supplied within the \hyperref[TEI.valItem]{<valItem>} used to document and name the behaviour. Here for example is the \hyperref[TEI.valItem]{<valItem>} which defines the \textsf{link} behaviour presented above: \par\bgroup\index{valItem=<valItem>|exampleindex}\index{ident=@ident!<valItem>|exampleindex}\index{desc=<desc>|exampleindex}\index{paramList=<paramList>|exampleindex}\index{paramSpec=<paramSpec>|exampleindex}\index{ident=@ident!<paramSpec>|exampleindex}\index{desc=<desc>|exampleindex}\index{paramSpec=<paramSpec>|exampleindex}\index{ident=@ident!<paramSpec>|exampleindex}\index{desc=<desc>|exampleindex}\exampleFont \begin{shaded}\noindent\mbox{}{<\textbf{valItem}\hspace*{1em}{ident}="{link}">}\mbox{}\newline 
\hspace*{1em}{<\textbf{desc}>}create a hyperlink{</\textbf{desc}>}\mbox{}\newline 
\hspace*{1em}{<\textbf{paramList}>}\mbox{}\newline 
\hspace*{1em}\hspace*{1em}{<\textbf{paramSpec}\hspace*{1em}{ident}="{content}">}\mbox{}\newline 
\hspace*{1em}\hspace*{1em}\hspace*{1em}{<\textbf{desc}>}supplies the location of some content describing the link{</\textbf{desc}>}\mbox{}\newline 
\hspace*{1em}\hspace*{1em}{</\textbf{paramSpec}>}\mbox{}\newline 
\hspace*{1em}\hspace*{1em}{<\textbf{paramSpec}\hspace*{1em}{ident}="{uri}">}\mbox{}\newline 
\hspace*{1em}\hspace*{1em}\hspace*{1em}{<\textbf{desc}>}supplies the location of the intended hyperlink{</\textbf{desc}>}\mbox{}\newline 
\hspace*{1em}\hspace*{1em}{</\textbf{paramSpec}>}\mbox{}\newline 
\hspace*{1em}{</\textbf{paramList}>}\mbox{}\newline 
{</\textbf{valItem}>}\end{shaded}\egroup\par \par
Similarly the \hyperref[TEI.valItem]{<valItem>} which defines the behaviour named \textsf{alternate} includes specifications for two parameters: one also called \textsf{alternate} and the other called \textsf{default} \par\bgroup\index{valItem=<valItem>|exampleindex}\index{ident=@ident!<valItem>|exampleindex}\index{desc=<desc>|exampleindex}\index{versionDate=@versionDate!<desc>|exampleindex}\index{paramList=<paramList>|exampleindex}\index{paramSpec=<paramSpec>|exampleindex}\index{ident=@ident!<paramSpec>|exampleindex}\index{desc=<desc>|exampleindex}\index{versionDate=@versionDate!<desc>|exampleindex}\index{paramSpec=<paramSpec>|exampleindex}\index{ident=@ident!<paramSpec>|exampleindex}\index{desc=<desc>|exampleindex}\index{versionDate=@versionDate!<desc>|exampleindex}\exampleFont \begin{shaded}\noindent\mbox{}{<\textbf{valItem}\hspace*{1em}{ident}="{alternate}">}\mbox{}\newline 
\hspace*{1em}{<\textbf{desc}\hspace*{1em}{versionDate}="{2015-08-21}"\mbox{}\newline 
\hspace*{1em}\hspace*{1em}{xml:lang}="{en}">}support display of alternative visualisations, for\mbox{}\newline 
\hspace*{1em}\hspace*{1em} example by displaying the preferred content, by displaying both in parallel, or by toggling\mbox{}\newline 
\hspace*{1em}\hspace*{1em} between the two.{</\textbf{desc}>}\mbox{}\newline 
\hspace*{1em}{<\textbf{paramList}>}\mbox{}\newline 
\hspace*{1em}\hspace*{1em}{<\textbf{paramSpec}\hspace*{1em}{ident}="{default}">}\mbox{}\newline 
\hspace*{1em}\hspace*{1em}\hspace*{1em}{<\textbf{desc}\hspace*{1em}{versionDate}="{2015-08-21}"\mbox{}\newline 
\hspace*{1em}\hspace*{1em}\hspace*{1em}\hspace*{1em}{xml:lang}="{en}">}supplies the location of the preferred\mbox{}\newline 
\hspace*{1em}\hspace*{1em}\hspace*{1em}\hspace*{1em}\hspace*{1em}\hspace*{1em} content{</\textbf{desc}>}\mbox{}\newline 
\hspace*{1em}\hspace*{1em}{</\textbf{paramSpec}>}\mbox{}\newline 
\hspace*{1em}\hspace*{1em}{<\textbf{paramSpec}\hspace*{1em}{ident}="{alternate}">}\mbox{}\newline 
\hspace*{1em}\hspace*{1em}\hspace*{1em}{<\textbf{desc}\hspace*{1em}{versionDate}="{2015-08-21}"\mbox{}\newline 
\hspace*{1em}\hspace*{1em}\hspace*{1em}\hspace*{1em}{xml:lang}="{en}">}supplies the location of the alternative\mbox{}\newline 
\hspace*{1em}\hspace*{1em}\hspace*{1em}\hspace*{1em}\hspace*{1em}\hspace*{1em} content{</\textbf{desc}>}\mbox{}\newline 
\hspace*{1em}\hspace*{1em}{</\textbf{paramSpec}>}\mbox{}\newline 
\hspace*{1em}{</\textbf{paramList}>}\mbox{}\newline 
{</\textbf{valItem}>}\end{shaded}\egroup\par \par
The suggested behaviours provided by the \hyperref[TEI.model]{<model>} element are informally defined using commonly understood terminology, but specific details of how they should be implemented are left to the implementor. Such decisions may vary greatly depending on the kind of processing environment, the kind of output envisaged, etc. The intention is to reduce as far as possible any requirement for the implementor to be aware of TEI-specific rules, and to maximize the ability of the ODD to express processing intentions without fully specifying an implementation.
\paragraph[{Implementation of Processing Models}]{Implementation of Processing Models}\label{TDPMIP}\par
As elsewhere in these Guidelines, the recommendations of this section are intended to be implementation-agnostic, not favouring any particular implementation method or technology. An implementor may choose, for example, whether to pre-process processing model specifications into a free standing ‘pipeline’, or to extract and process them dynamically during document processing. However, although implementors are generally free to interpret the processing model documentation according to their own requirements, some general assumptions underlie the recommendations made here: \begin{itemize}
\item If a \hyperref[TEI.model]{<model>} element has no child \hyperref[TEI.param]{<param>} elements, the action specified by its behaviour should be applied to the whole element node concerned, including its child nodes of whatever type. If that behaviour requires distinguishing particular parts of the input, an implementation may choose either to make those distinctions itself, or to raise an error.
\item If a \hyperref[TEI.model]{<model>} element has no {\itshape predicate} or {\itshape output} attribute then it is assumed to apply to all instances of the element defined in its parent \hyperref[TEI.elementSpec]{<elementSpec>} for all outputs. Otherwise its applicability depends on these attributes.
\item Only one of the \hyperref[TEI.model]{<model>}s is to be applied for a particular instance of the element, except when they appear within a \hyperref[TEI.modelSequence]{<modelSequence>}
\item A ‘matching’ model is one where the element to be processed satisfies the XPath in the {\itshape predicate} attribute of the \hyperref[TEI.model]{<model>} or \hyperref[TEI.modelSequence]{<modelSequence>} and the current output method matches the method specified in the {\itshape output} attribute of the \hyperref[TEI.model]{<model>}, \hyperref[TEI.modelSequence]{<modelSequence>}, or a containing \hyperref[TEI.modelGrp]{<modelGrp>}. A \hyperref[TEI.model]{<model>} or \hyperref[TEI.modelSequence]{<modelSequence>} without a {\itshape predicate} always matches the element to be processed. A \hyperref[TEI.model]{<model>}, \hyperref[TEI.modelGrp]{<modelGrp>}, or \hyperref[TEI.modelSequence]{<modelSequence>} without an {\itshape output} attribute matches any output method.
\item Processing Model implementations must execute only the first matching \hyperref[TEI.model]{<model>} or \hyperref[TEI.modelSequence]{<modelSequence>} in document order.
\item If there are two or more \hyperref[TEI.model]{<model>} elements provided for an \hyperref[TEI.elementSpec]{<elementSpec>} but they have different {\itshape output} attributes then the implementation should choose the \hyperref[TEI.model]{<model>} appropriate to the desired output.
\item If there are two or more \hyperref[TEI.model]{<model>} elements provided for an \hyperref[TEI.elementSpec]{<elementSpec>} but they have different {\itshape predicate} attributes then the implementation should choose the \hyperref[TEI.model]{<model>} whose {\itshape predicate} provides the most specific context (where \textit{specific} is understood in the same way as in XSLT) 
\end{itemize}  In the following example, which shows part of the element specification for the \hyperref[TEI.head]{<head>} element, there are two \hyperref[TEI.model]{<model>} elements, one with and one without a {\itshape predicate} attribute: \par\bgroup\index{model=<model>|exampleindex}\index{behaviour=@behaviour!<model>|exampleindex}\index{predicate=@predicate!<model>|exampleindex}\index{desc=<desc>|exampleindex}\index{versionDate=@versionDate!<desc>|exampleindex}\index{model=<model>|exampleindex}\index{behaviour=@behaviour!<model>|exampleindex}\index{desc=<desc>|exampleindex}\index{versionDate=@versionDate!<desc>|exampleindex}\exampleFont \begin{shaded}\noindent\mbox{}{<\textbf{model}\hspace*{1em}{behaviour}="{inline}"\mbox{}\newline 
\hspace*{1em}{predicate}="{parent::list}">}\mbox{}\newline 
\hspace*{1em}{<\textbf{desc}\hspace*{1em}{versionDate}="{2015-03-02}"\mbox{}\newline 
\hspace*{1em}\hspace*{1em}{xml:lang}="{en}">}Model for list headings{</\textbf{desc}>}\mbox{}\newline 
\textit{<!-- ... -->}\mbox{}\newline 
{</\textbf{model}>}\mbox{}\newline 
{<\textbf{model}\hspace*{1em}{behaviour}="{heading}">}\mbox{}\newline 
\hspace*{1em}{<\textbf{desc}\hspace*{1em}{versionDate}="{2016-03-02}"\mbox{}\newline 
\hspace*{1em}\hspace*{1em}{xml:lang}="{en}">}Default model for all headings.{</\textbf{desc}>}\mbox{}\newline 
\textit{<!-- ... -->}\mbox{}\newline 
{</\textbf{model}>}\end{shaded}\egroup\par \noindent  In this example, an implementation should use the first processing model only for \hyperref[TEI.head]{<head>} elements with a \hyperref[TEI.list]{<list>} element as parent; for all other \hyperref[TEI.head]{<head>} elements, the second processing model should be used.
\subsection[{Class Specifications}]{Class Specifications}\label{TDCLA}\par
The element \hyperref[TEI.classSpec]{<classSpec>} is used to document either an \textit{attribute class} or a ‘model class’, as defined in section \textit{\hyperref[STEC]{1.3.\ The TEI Class System}}. A corresponding \hyperref[TEI.classRef]{<classRef>} element may be used to select a specific named class from those available. 
\begin{sansreflist}
  
\item [\textbf{<classSpec>}] (class specification) contains reference information for a TEI element class; that is a group of elements which appear together in content models, or which share some common attribute, or both.\hfil\\[-10pt]\begin{sansreflist}
    \item[@{\itshape type}]
  indicates whether this is a model class or an attribute class
\end{sansreflist}  
\item [\textbf{<classRef>}] points to the specification for an attribute or model class which is to be included in a schema\hfil\\[-10pt]\begin{sansreflist}
    \item[@{\itshape expand}]
  indicates how references to this class within a content model should be interpreted.
\end{sansreflist}  
\item [\textbf{<attList>}] (attribute list) contains documentation for all the attributes associated with this element, as a series of \hyperref[TEI.attDef]{<attDef>} elements.\hfil\\[-10pt]\begin{sansreflist}
    \item[@{\itshape org}]
  (organization) specifies whether all the attributes in the list are available (org="group") or only one of them (org="choice")
\end{sansreflist}  
\end{sansreflist}
\par
A model class specification does not list all of its members. Instead, its members declare that they belong to it by means of a \hyperref[TEI.classes]{<classes>} element contained within the relevant \hyperref[TEI.elementSpec]{<elementSpec>}. This will contain a \hyperref[TEI.memberOf]{<memberOf>} element for each class of which the relevant element is a member, supplying the name of the relevant class. For example, the \hyperref[TEI.elementSpec]{<elementSpec>} for the element \hyperref[TEI.hi]{<hi>} contains the following: \par\bgroup\index{classes=<classes>|exampleindex}\index{memberOf=<memberOf>|exampleindex}\index{key=@key!<memberOf>|exampleindex}\exampleFont \begin{shaded}\noindent\mbox{}{<\textbf{classes}>}\mbox{}\newline 
\hspace*{1em}{<\textbf{memberOf}\hspace*{1em}{key}="{model.hiLike}"/>}\mbox{}\newline 
{</\textbf{classes}>}\end{shaded}\egroup\par \noindent  This indicates that the \hyperref[TEI.hi]{<hi>} element is a member of the class with identifier \textsf{model.hiLike}. The \hyperref[TEI.classSpec]{<classSpec>} element that documents this class contains the following declarations: \par\bgroup\index{classSpec=<classSpec>|exampleindex}\index{type=@type!<classSpec>|exampleindex}\index{ident=@ident!<classSpec>|exampleindex}\index{desc=<desc>|exampleindex}\index{classes=<classes>|exampleindex}\index{memberOf=<memberOf>|exampleindex}\index{key=@key!<memberOf>|exampleindex}\exampleFont \begin{shaded}\noindent\mbox{}{<\textbf{classSpec}\hspace*{1em}{type}="{model}"\mbox{}\newline 
\hspace*{1em}{ident}="{model.hiLike}">}\mbox{}\newline 
\hspace*{1em}{<\textbf{desc}>}groups phrase-level elements related to highlighting that have no specific semantics {</\textbf{desc}>}\mbox{}\newline 
\hspace*{1em}{<\textbf{classes}>}\mbox{}\newline 
\hspace*{1em}\hspace*{1em}{<\textbf{memberOf}\hspace*{1em}{key}="{model.highlighted}"/>}\mbox{}\newline 
\hspace*{1em}{</\textbf{classes}>}\mbox{}\newline 
{</\textbf{classSpec}>}\end{shaded}\egroup\par \noindent  which indicate that the class \textsf{model.hiLike} is actually a member (or subclass) of the class \textsf{model.highlighted}.\par
The function of a model class declaration is to provide another way of referring to a group of elements. It does not confer any other properties on the elements which constitute its membership. \par
The attribute {\itshape type} is used to distinguish between ‘model’ and ‘attribute’ classes. In the case of attribute classes, the attributes provided by membership in the class are documented by an \hyperref[TEI.attList]{<attList>} element contained within the \hyperref[TEI.classSpec]{<classSpec>}. In the case of model classes, no further information is needed to define the class beyond its description, its identifier, and optionally any classes of which it is a member.\par
When a model class is referenced in the content model of an element (i.e. by means of a \hyperref[TEI.classRef]{<classRef>} element within the \hyperref[TEI.content]{<content>} of an \hyperref[TEI.elementSpec]{<elementSpec>}), its meaning will depend on the value of its {\itshape expand} attribute.\par
If this attribute is not specified, the \hyperref[TEI.classRef]{<classRef>} is interpreted to mean an alternated list of all the current members of the class named. For example, suppose that the members of the class \textsf{model.hiLike} are elements \hyperref[TEI.hi]{<hi>}, \texttt{<it>}, and \texttt{<bo>}. Then a content model such as \par\bgroup\index{content=<content>|exampleindex}\index{classRef=<classRef>|exampleindex}\index{key=@key!<classRef>|exampleindex}\exampleFont \begin{shaded}\noindent\mbox{}{<\textbf{content}>}\mbox{}\newline 
\hspace*{1em}{<\textbf{classRef}\hspace*{1em}{key}="{model.hiLike}"/>}\mbox{}\newline 
{</\textbf{content}>}\end{shaded}\egroup\par \noindent  would be equivalent to the explicit content model: \par\bgroup\index{content=<content>|exampleindex}\index{alternate=<alternate>|exampleindex}\index{elementRef=<elementRef>|exampleindex}\index{key=@key!<elementRef>|exampleindex}\index{elementRef=<elementRef>|exampleindex}\index{key=@key!<elementRef>|exampleindex}\index{elementRef=<elementRef>|exampleindex}\index{key=@key!<elementRef>|exampleindex}\exampleFont \begin{shaded}\noindent\mbox{}{<\textbf{content}>}\mbox{}\newline 
\hspace*{1em}{<\textbf{alternate}>}\mbox{}\newline 
\hspace*{1em}\hspace*{1em}{<\textbf{elementRef}\hspace*{1em}{key}="{hi}"/>}\mbox{}\newline 
\hspace*{1em}\hspace*{1em}{<\textbf{elementRef}\hspace*{1em}{key}="{it}"/>}\mbox{}\newline 
\hspace*{1em}\hspace*{1em}{<\textbf{elementRef}\hspace*{1em}{key}="{bo}"/>}\mbox{}\newline 
\hspace*{1em}{</\textbf{alternate}>}\mbox{}\newline 
{</\textbf{content}>}\end{shaded}\egroup\par \noindent  (or, to use RELAX NG compact syntax, \texttt{(hi|it|bo)*}). However, a content model of \texttt{<classRef expand="sequence"/>} would be equivalent to the following explicit content model: \par\bgroup\index{content=<content>|exampleindex}\index{sequence=<sequence>|exampleindex}\index{elementRef=<elementRef>|exampleindex}\index{key=@key!<elementRef>|exampleindex}\index{elementRef=<elementRef>|exampleindex}\index{key=@key!<elementRef>|exampleindex}\index{elementRef=<elementRef>|exampleindex}\index{key=@key!<elementRef>|exampleindex}\exampleFont \begin{shaded}\noindent\mbox{}{<\textbf{content}>}\mbox{}\newline 
\hspace*{1em}{<\textbf{sequence}>}\mbox{}\newline 
\hspace*{1em}\hspace*{1em}{<\textbf{elementRef}\hspace*{1em}{key}="{hi}"/>}\mbox{}\newline 
\hspace*{1em}\hspace*{1em}{<\textbf{elementRef}\hspace*{1em}{key}="{it}"/>}\mbox{}\newline 
\hspace*{1em}\hspace*{1em}{<\textbf{elementRef}\hspace*{1em}{key}="{bo}"/>}\mbox{}\newline 
\hspace*{1em}{</\textbf{sequence}>}\mbox{}\newline 
{</\textbf{content}>}\end{shaded}\egroup\par \noindent  (or, in RELAX NG compact syntax, \texttt{(hi,it,bo)*}).\par
An attribute class (a \hyperref[TEI.classSpec]{<classSpec>} of {\itshape type} atts) contains an \hyperref[TEI.attList]{<attList>} element which lists the attributes that all the members of that class inherit from it. For example, the class \textsf{att.interpLike} defines a small set of attributes common to all elements which are members of that class: those attributes are listed by the \hyperref[TEI.attList]{<attList>} element contained by the \hyperref[TEI.classSpec]{<classSpec>} for \textsf{att.interpLike}. When processing the documentation elements for elements which are members of that class, an ODD processor is required to extend the \hyperref[TEI.attList]{<attList>} (or equivalent) for such elements to include any attributes defined by the \hyperref[TEI.classSpec]{<classSpec>} elements concerned. There is a single global attribute class, \textsf{att.global}, to which some modules contribute additional attributes when they are included in a schema.
\subsection[{Macro Specifications}]{Macro Specifications}\label{TDENT}\par
The \hyperref[TEI.macroSpec]{<macroSpec>} element is used to declare and document predefined strings or patterns not otherwise documented by the elements described in this section. A corresponding \hyperref[TEI.macroRef]{<macroRef>} element may be used to select a specific named pattern from those available. Patterns are used as a shorthand chiefly to describe common content models and datatypes, but may be used for any purpose. The following elements are used to represent patterns: 
\begin{sansreflist}
  
\item [\textbf{<macroSpec>}] (macro specification) documents the function and implementation of a pattern.
\item [\textbf{<macroRef>}] points to the specification for some pattern which is to be included in a schema\hfil\\[-10pt]\begin{sansreflist}
    \item[@{\itshape key}]
  the identifier used for the required pattern within the source indicated.
\end{sansreflist}  
\end{sansreflist}

\subsection[{Building a TEI Schema}]{Building a TEI Schema}\label{TDB1}\par
The specification elements, and some of their children, are all members of the \textsf{att.identified} class, from which they inherit the following attributes: 
\begin{sansreflist}
  
\item [\textbf{att.identified}] provides the identifying attribute for elements which can be subsequently referenced by means of a {\itshape key} attribute.\hfil\\[-10pt]\begin{sansreflist}
    \item[@{\itshape ident}]
  supplies the identifier by which this element may be referenced.
    \item[@{\itshape predeclare}]
  says whether this object should be predeclared in the \textsf{tei} infrastructure module.
    \item[@{\itshape module}]
  supplies a name for the module in which this object is to be declared.
\end{sansreflist}  
\end{sansreflist}
 This attribute class is a subclass of the \textsf{att.combinable} class from which it (and some other elements) inherits the following attribute: 
\begin{sansreflist}
  
\item [\textbf{att.combinable}] provides attributes indicating how multiple references to the same object in a schema should be combined\hfil\\[-10pt]\begin{sansreflist}
    \item[@{\itshape mode}]
  specifies the effect of this declaration on its parent object.
\end{sansreflist}  
\end{sansreflist}
 This attribute class, in turn, is a subclass of the \textsf{att.deprecated} class, from which it inherits the following attribute: 
\begin{sansreflist}
  
\item [\textbf{att.deprecated}] provides attributes indicating how a deprecated feature will be treated in future releases.\hfil\\[-10pt]\begin{sansreflist}
    \item[@{\itshape validUntil}]
  provides a date before which the construct being defined will not be removed.
\end{sansreflist}  
\end{sansreflist}
 The {\itshape validUntil} attribute may be used to signal an intent to remove a construct from future versions of the schema being specified.\par
The \hyperref[TEI.elementSpec]{<elementSpec>}, \hyperref[TEI.attDef]{<attDef>} and \hyperref[TEI.schemaSpec]{<schemaSpec>} specification elements also have an attribute which determines which namespace to which the object being created will belong. In the case of \hyperref[TEI.schemaSpec]{<schemaSpec>}, this namespace is inherited by all the elements created in the schema, unless they have their own {\itshape ns}. 
\begin{sansreflist}
  
\item [\textbf{att.namespaceable}] provides an attribute indicating the target namespace for an object being created
\end{sansreflist}
\par
These attributes are used by an ODD processor to determine how declarations are to be combined to form a schema or DTD, as further discussed in this section.
\subsubsection[{TEI customizations}]{TEI customizations}\label{TDbuild}\par
As noted above, a TEI schema is defined by a \hyperref[TEI.schemaSpec]{<schemaSpec>} element containing an arbitrary mixture of explicit declarations for objects (i.e. elements, classes, patterns, or macro specifications) and references to other objects containing such declarations (i.e. references to specification groups, or to modules). A major purpose of this mechanism is to simplify the process of defining user customizations, by providing a formal method for the user to combine new declarations with existing ones, or to modify particular parts of existing declarations.\par
In the simplest case, a user-defined schema might just combine all the declarations from two nominated modules: \par\bgroup\index{schemaSpec=<schemaSpec>|exampleindex}\index{ident=@ident!<schemaSpec>|exampleindex}\index{moduleRef=<moduleRef>|exampleindex}\index{key=@key!<moduleRef>|exampleindex}\index{moduleRef=<moduleRef>|exampleindex}\index{key=@key!<moduleRef>|exampleindex}\exampleFont \begin{shaded}\noindent\mbox{}{<\textbf{schemaSpec}\hspace*{1em}{ident}="{example}">}\mbox{}\newline 
\hspace*{1em}{<\textbf{moduleRef}\hspace*{1em}{key}="{core}"/>}\mbox{}\newline 
\hspace*{1em}{<\textbf{moduleRef}\hspace*{1em}{key}="{linking}"/>}\mbox{}\newline 
{</\textbf{schemaSpec}>}\end{shaded}\egroup\par \noindent  An ODD processor, given such a document, should combine the declarations which belong to the named modules, and deliver the result as a schema of the requested type. It may also generate documentation for the elements declared by those modules. No source is specified for the modules, and the schema will therefore combine the declarations found in the most recent release version of the TEI Guidelines known to the ODD processor in use.\par
The value specified for the {\itshape source} attribute, when it is supplied as a URL, specifies any convenient location from which the relevant ODD files may be obtained. For the current release of the TEI Guidelines, a URL in the form \texttt{http://www.tei-c.org/Vault/P5/x.y.z/xml/tei/odd/p5subset.xml} may be used, where \texttt{x.y.z} represents the P5 version number, e.g. \texttt{1.3.0}. Alternatively, if the ODD files are locally installed, it may be more convenient to supply a value such as ../ODDs/p5subset.xml". \par
The value for the {\itshape source} attribute may be any form of URI. A set of TEI-conformant specifications in a form directly usable by an ODD processor must be available at the location indicated. When no {\itshape source} value is supplied, an ODD processor may either raise an error or assume that the location of the current release of the TEI Guidelines is intended.\par
If the source is specified in the form of a private URI, the form recommended is \texttt{aaa:x.y.z}, where \texttt{aaa} is a prefix indicating the markup language in use, and \texttt{x.y.z} indicates the version number. For example, \texttt{tei:1.2.1} should be used to reference release 1.2.1 of the current TEI Guidelines. When such a URI is used, it will usually be necessary to translate it before such a file can be used in blind interchange.\par
The effect of a \hyperref[TEI.moduleRef]{<moduleRef>} element is to include in the schema all declarations provided by that module. This may be modified by means of the attributes {\itshape include} and {\itshape except} which allow the encoder to supply an explicit lists of elements from the stated module which are to be included or excluded respectively. For example: \par\bgroup\index{schemaSpec=<schemaSpec>|exampleindex}\index{ident=@ident!<schemaSpec>|exampleindex}\index{moduleRef=<moduleRef>|exampleindex}\index{key=@key!<moduleRef>|exampleindex}\index{except=@except!<moduleRef>|exampleindex}\index{moduleRef=<moduleRef>|exampleindex}\index{key=@key!<moduleRef>|exampleindex}\index{include=@include!<moduleRef>|exampleindex}\exampleFont \begin{shaded}\noindent\mbox{}{<\textbf{schemaSpec}\hspace*{1em}{ident}="{example}">}\mbox{}\newline 
\hspace*{1em}{<\textbf{moduleRef}\hspace*{1em}{key}="{core}"\mbox{}\newline 
\hspace*{1em}\hspace*{1em}{except}="{add del orig reg}"/>}\mbox{}\newline 
\hspace*{1em}{<\textbf{moduleRef}\hspace*{1em}{key}="{linking}"\mbox{}\newline 
\hspace*{1em}\hspace*{1em}{include}="{linkGroup link}"/>}\mbox{}\newline 
{</\textbf{schemaSpec}>}\end{shaded}\egroup\par \noindent  The schema specified here will include all the elements supplied by the core module except for \hyperref[TEI.add]{<add>}, \hyperref[TEI.del]{<del>}, \hyperref[TEI.orig]{<orig>}, and \hyperref[TEI.reg]{<reg>}. It will also include only the \texttt{<linkGroup>} and \hyperref[TEI.link]{<link>} elements from the linking module.\par
Alternatively, the element \hyperref[TEI.elementRef]{<elementRef>} may be used to indicate explicitly which elements are to be included in a schema. The same effect as the preceding example might thus be achieved by the following: \par\bgroup\index{schemaSpec=<schemaSpec>|exampleindex}\index{ident=@ident!<schemaSpec>|exampleindex}\index{moduleRef=<moduleRef>|exampleindex}\index{key=@key!<moduleRef>|exampleindex}\index{except=@except!<moduleRef>|exampleindex}\index{elementRef=<elementRef>|exampleindex}\index{key=@key!<elementRef>|exampleindex}\index{elementRef=<elementRef>|exampleindex}\index{key=@key!<elementRef>|exampleindex}\exampleFont \begin{shaded}\noindent\mbox{}{<\textbf{schemaSpec}\hspace*{1em}{ident}="{example}">}\mbox{}\newline 
\hspace*{1em}{<\textbf{moduleRef}\hspace*{1em}{key}="{core}"\mbox{}\newline 
\hspace*{1em}\hspace*{1em}{except}="{add del orig reg}"/>}\mbox{}\newline 
\hspace*{1em}{<\textbf{elementRef}\hspace*{1em}{key}="{linkGroup}"/>}\mbox{}\newline 
\hspace*{1em}{<\textbf{elementRef}\hspace*{1em}{key}="{link}"/>}\mbox{}\newline 
{</\textbf{schemaSpec}>}\end{shaded}\egroup\par \noindent  Note that in this last case, there is no need to specify the name of the module from which the two element declarations are to be found; in the TEI scheme, element names are unique across all modules. The module is simply a convenient way of grouping together a number of related declarations.\par
In the same way, a schema may select a subset of the attributes available in a specific class, using the \hyperref[TEI.classRef]{<classRef>} element to point to an attribute class: \par\bgroup\index{schemaSpec=<schemaSpec>|exampleindex}\index{ident=@ident!<schemaSpec>|exampleindex}\index{moduleRef=<moduleRef>|exampleindex}\index{key=@key!<moduleRef>|exampleindex}\index{classRef=<classRef>|exampleindex}\index{key=@key!<classRef>|exampleindex}\index{include=@include!<classRef>|exampleindex}\exampleFont \begin{shaded}\noindent\mbox{}{<\textbf{schemaSpec}\hspace*{1em}{ident}="{example}">}\mbox{}\newline 
\hspace*{1em}{<\textbf{moduleRef}\hspace*{1em}{key}="{tei}"/>}\mbox{}\newline 
\textit{<!-- ... -->}\mbox{}\newline 
\hspace*{1em}{<\textbf{classRef}\hspace*{1em}{key}="{att.global.linking}"\mbox{}\newline 
\hspace*{1em}\hspace*{1em}{include}="{corresp}"/>}\mbox{}\newline 
\textit{<!-- ... -->}\mbox{}\newline 
{</\textbf{schemaSpec}>}\end{shaded}\egroup\par \noindent  Here, only the {\itshape corresp} attribute is included; the other attributes in the class are not available. The same effect can be achieved using {\itshape except}: \par\bgroup\index{schemaSpec=<schemaSpec>|exampleindex}\index{ident=@ident!<schemaSpec>|exampleindex}\index{moduleRef=<moduleRef>|exampleindex}\index{key=@key!<moduleRef>|exampleindex}\index{classRef=<classRef>|exampleindex}\index{key=@key!<classRef>|exampleindex}\index{except=@except!<classRef>|exampleindex}\exampleFont \begin{shaded}\noindent\mbox{}{<\textbf{schemaSpec}\hspace*{1em}{ident}="{example}">}\mbox{}\newline 
\hspace*{1em}{<\textbf{moduleRef}\hspace*{1em}{key}="{tei}"/>}\mbox{}\newline 
\textit{<!-- ... -->}\mbox{}\newline 
\hspace*{1em}{<\textbf{classRef}\hspace*{1em}{key}="{att.global.linking}"\mbox{}\newline 
\hspace*{1em}\hspace*{1em}{except}="{copyOf exclude next prev sameAs select synch}"/>}\mbox{}\newline 
\textit{<!-- ... -->}\mbox{}\newline 
{</\textbf{schemaSpec}>}\end{shaded}\egroup\par \par
A schema may also include declarations for new elements, as in the following example: \par\bgroup\index{schemaSpec=<schemaSpec>|exampleindex}\index{ident=@ident!<schemaSpec>|exampleindex}\index{moduleRef=<moduleRef>|exampleindex}\index{key=@key!<moduleRef>|exampleindex}\index{moduleRef=<moduleRef>|exampleindex}\index{key=@key!<moduleRef>|exampleindex}\index{elementSpec=<elementSpec>|exampleindex}\index{ident=@ident!<elementSpec>|exampleindex}\index{classes=<classes>|exampleindex}\index{memberOf=<memberOf>|exampleindex}\index{key=@key!<memberOf>|exampleindex}\exampleFont \begin{shaded}\noindent\mbox{}{<\textbf{schemaSpec}\hspace*{1em}{ident}="{example}">}\mbox{}\newline 
\hspace*{1em}{<\textbf{moduleRef}\hspace*{1em}{key}="{header}"/>}\mbox{}\newline 
\hspace*{1em}{<\textbf{moduleRef}\hspace*{1em}{key}="{verse}"/>}\mbox{}\newline 
\hspace*{1em}{<\textbf{elementSpec}\hspace*{1em}{ident}="{soundClip}">}\mbox{}\newline 
\hspace*{1em}\hspace*{1em}{<\textbf{classes}>}\mbox{}\newline 
\hspace*{1em}\hspace*{1em}\hspace*{1em}{<\textbf{memberOf}\hspace*{1em}{key}="{model.pPart.data}"/>}\mbox{}\newline 
\hspace*{1em}\hspace*{1em}{</\textbf{classes}>}\mbox{}\newline 
\hspace*{1em}{</\textbf{elementSpec}>}\mbox{}\newline 
{</\textbf{schemaSpec}>}\end{shaded}\egroup\par \noindent  A declaration for the element \texttt{<soundClip>}, which is not defined in the TEI scheme, will be added to the output schema. This element will also be added to the existing TEI class \textsf{model.pPart.data}, and will thus be available in TEI conformant documents. Attributes from existing TEI classes could be added to the new element using \hyperref[TEI.attRef]{<attRef>}: \par\bgroup\index{schemaSpec=<schemaSpec>|exampleindex}\index{ident=@ident!<schemaSpec>|exampleindex}\index{moduleRef=<moduleRef>|exampleindex}\index{key=@key!<moduleRef>|exampleindex}\index{moduleRef=<moduleRef>|exampleindex}\index{key=@key!<moduleRef>|exampleindex}\index{elementSpec=<elementSpec>|exampleindex}\index{ident=@ident!<elementSpec>|exampleindex}\index{classes=<classes>|exampleindex}\index{memberOf=<memberOf>|exampleindex}\index{key=@key!<memberOf>|exampleindex}\index{attList=<attList>|exampleindex}\index{attRef=<attRef>|exampleindex}\index{class=@class!<attRef>|exampleindex}\index{name=@name!<attRef>|exampleindex}\exampleFont \begin{shaded}\noindent\mbox{}{<\textbf{schemaSpec}\hspace*{1em}{ident}="{example}">}\mbox{}\newline 
\hspace*{1em}{<\textbf{moduleRef}\hspace*{1em}{key}="{header}"/>}\mbox{}\newline 
\hspace*{1em}{<\textbf{moduleRef}\hspace*{1em}{key}="{verse}"/>}\mbox{}\newline 
\hspace*{1em}{<\textbf{elementSpec}\hspace*{1em}{ident}="{soundClip}">}\mbox{}\newline 
\hspace*{1em}\hspace*{1em}{<\textbf{classes}>}\mbox{}\newline 
\hspace*{1em}\hspace*{1em}\hspace*{1em}{<\textbf{memberOf}\hspace*{1em}{key}="{model.pPart.data}"/>}\mbox{}\newline 
\hspace*{1em}\hspace*{1em}{</\textbf{classes}>}\mbox{}\newline 
\hspace*{1em}\hspace*{1em}{<\textbf{attList}>}\mbox{}\newline 
\hspace*{1em}\hspace*{1em}\hspace*{1em}{<\textbf{attRef}\hspace*{1em}{class}="{att.global.source}"\mbox{}\newline 
\hspace*{1em}\hspace*{1em}\hspace*{1em}\hspace*{1em}{name}="{source}"/>}\mbox{}\newline 
\hspace*{1em}\hspace*{1em}{</\textbf{attList}>}\mbox{}\newline 
\hspace*{1em}{</\textbf{elementSpec}>}\mbox{}\newline 
{</\textbf{schemaSpec}>}\end{shaded}\egroup\par \noindent  This will provide the {\itshape source} attribute from the \textsf{att.global.source} class on the new \texttt{<soundClip>} element.\par
A schema might also include re-declarations of existing elements, as in the following example: \par\bgroup\index{schemaSpec=<schemaSpec>|exampleindex}\index{ident=@ident!<schemaSpec>|exampleindex}\index{moduleRef=<moduleRef>|exampleindex}\index{key=@key!<moduleRef>|exampleindex}\index{moduleRef=<moduleRef>|exampleindex}\index{key=@key!<moduleRef>|exampleindex}\index{elementSpec=<elementSpec>|exampleindex}\index{ident=@ident!<elementSpec>|exampleindex}\index{mode=@mode!<elementSpec>|exampleindex}\index{content=<content>|exampleindex}\index{macroRef=<macroRef>|exampleindex}\index{key=@key!<macroRef>|exampleindex}\exampleFont \begin{shaded}\noindent\mbox{}{<\textbf{schemaSpec}\hspace*{1em}{ident}="{example}">}\mbox{}\newline 
\hspace*{1em}{<\textbf{moduleRef}\hspace*{1em}{key}="{header}"/>}\mbox{}\newline 
\hspace*{1em}{<\textbf{moduleRef}\hspace*{1em}{key}="{teistructure}"/>}\mbox{}\newline 
\hspace*{1em}{<\textbf{elementSpec}\hspace*{1em}{ident}="{head}"\hspace*{1em}{mode}="{change}">}\mbox{}\newline 
\hspace*{1em}\hspace*{1em}{<\textbf{content}>}\mbox{}\newline 
\hspace*{1em}\hspace*{1em}\hspace*{1em}{<\textbf{macroRef}\hspace*{1em}{key}="{macro.xtext}"/>}\mbox{}\newline 
\hspace*{1em}\hspace*{1em}{</\textbf{content}>}\mbox{}\newline 
\hspace*{1em}{</\textbf{elementSpec}>}\mbox{}\newline 
{</\textbf{schemaSpec}>}\end{shaded}\egroup\par \noindent  The effect of this is to redefine the content model for the element \hyperref[TEI.head]{<head>} as plain text, by over-riding the \hyperref[TEI.content]{<content>} child of the selected \hyperref[TEI.elementSpec]{<elementSpec>}. The attribute specification \texttt{mode="change"} has the effect of over-riding only those children elements of the \hyperref[TEI.elementSpec]{<elementSpec>} which appear both in the original specification and in the new specification supplied above: \hyperref[TEI.content]{<content>} in this example. Note that if the value for {\itshape mode} were replace, the effect would be to replace all children elements of the original specification with the the children elements of the new specification, and thus (in this example) to delete all of them except \hyperref[TEI.content]{<content>}.\par
A schema may not contain more than two declarations for any given component. The value of the {\itshape mode} attribute is used to determine exactly how the second declaration (and its constituents) should be combined with the first. The following table summarizes how a processor should resolve duplicate declarations; the term \textit{identifiable} refers to those elements which can have a {\itshape mode} attribute:  \par 
\begin{longtable}{P{0.08912621359223301\textwidth}P{0.06601941747572815\textwidth}P{0.6948543689320388\textwidth}}
\rowcolor{label}mode value\tabcellsep existing declaration\tabcellsep effect\\\hline 
add\tabcellsep no\tabcellsep add new declaration to schema; process its children in add mode\\
add\tabcellsep yes\tabcellsep raise error\\
replace\tabcellsep no\tabcellsep raise error\\
replace\tabcellsep yes\tabcellsep retain existing declaration; process new children in replace mode; ignore existing children\\
change\tabcellsep no\tabcellsep raise error\\
change\tabcellsep yes\tabcellsep process identifiable children according to their modes; process unidentifiable children in replace mode; retain existing children where no replacement or change is provided \\
delete\tabcellsep no\tabcellsep raise error\\
delete\tabcellsep yes\tabcellsep ignore existing declaration and its children\end{longtable} \par
 
\subsubsection[{Combining TEI and Non-TEI Modules}]{Combining TEI and Non-TEI Modules}\label{ST-aliens}\par
In the simplest case, all that is needed to include a non-TEI module in a schema is to reference its RELAX NG source using the {\itshape url} attribute on \hyperref[TEI.moduleRef]{<moduleRef>}. The following specification, for example, creates a schema in which declarations from the non-TEI module \textsf{svg11.rng} (defining Standard Vector Graphics) are included. To avoid any risk of name clashes, the schema specifies that all TEI patterns generated should be prefixed by the string "TEI\textunderscore ". \par\bgroup\index{schemaSpec=<schemaSpec>|exampleindex}\index{prefix=@prefix!<schemaSpec>|exampleindex}\index{ident=@ident!<schemaSpec>|exampleindex}\index{start=@start!<schemaSpec>|exampleindex}\index{moduleRef=<moduleRef>|exampleindex}\index{key=@key!<moduleRef>|exampleindex}\index{moduleRef=<moduleRef>|exampleindex}\index{key=@key!<moduleRef>|exampleindex}\index{moduleRef=<moduleRef>|exampleindex}\index{key=@key!<moduleRef>|exampleindex}\index{moduleRef=<moduleRef>|exampleindex}\index{key=@key!<moduleRef>|exampleindex}\index{moduleRef=<moduleRef>|exampleindex}\index{url=@url!<moduleRef>|exampleindex}\exampleFont \begin{shaded}\noindent\mbox{}{<\textbf{schemaSpec}\hspace*{1em}{prefix}="{TEI\textunderscore }"\hspace*{1em}{ident}="{testsvg}"\mbox{}\newline 
\hspace*{1em}{start}="{TEI svg}">}\mbox{}\newline 
\hspace*{1em}{<\textbf{moduleRef}\hspace*{1em}{key}="{header}"/>}\mbox{}\newline 
\hspace*{1em}{<\textbf{moduleRef}\hspace*{1em}{key}="{core}"/>}\mbox{}\newline 
\hspace*{1em}{<\textbf{moduleRef}\hspace*{1em}{key}="{tei}"/>}\mbox{}\newline 
\hspace*{1em}{<\textbf{moduleRef}\hspace*{1em}{key}="{textstructure}"/>}\mbox{}\newline 
\hspace*{1em}{<\textbf{moduleRef}\hspace*{1em}{url}="{svg11.rng}"/>}\mbox{}\newline 
{</\textbf{schemaSpec}>}\end{shaded}\egroup\par \par
This specification generates a single schema which might be used to validate either a TEI document (with the root element \hyperref[TEI.TEI]{<TEI>}), or an SVG document (with a root element \texttt{<svg:svg>}), but would \textit{not} validate a TEI document containing \texttt{<svg:svg>} or other elements from the SVG language. For that to be possible, the \texttt{<svg:svg>} element must become a member of a TEI model class (\textit{\hyperref[STEC]{1.3.\ The TEI Class System}}), so that it may be referenced by other TEI elements. To achieve this, we modify the last \hyperref[TEI.moduleRef]{<moduleRef>} in the above example as follows: \par\bgroup\index{moduleRef=<moduleRef>|exampleindex}\index{url=@url!<moduleRef>|exampleindex}\index{content=<content>|exampleindex}\exampleFont \begin{shaded}\noindent\mbox{}{<\textbf{moduleRef}\hspace*{1em}{url}="{svg11.rng}">}\mbox{}\newline 
\hspace*{1em}{<\textbf{content}>}\mbox{}\newline 
\hspace*{1em}\hspace*{1em}{<\textbf{rng:define}\hspace*{1em}{name}="{TEI\textunderscore model.graphicLike}"\mbox{}\newline 
\hspace*{1em}\hspace*{1em}\hspace*{1em}{combine}="{choice}">}\mbox{}\newline 
\hspace*{1em}\hspace*{1em}\hspace*{1em}{<\textbf{rng:ref}\hspace*{1em}{name}="{svg}"/>}\mbox{}\newline 
\hspace*{1em}\hspace*{1em}{</\textbf{rng:define}>}\mbox{}\newline 
\hspace*{1em}{</\textbf{content}>}\mbox{}\newline 
{</\textbf{moduleRef}>}\end{shaded}\egroup\par \par
This states that when the declarations from the \textsf{svg11.rng} module are combined with those from the other modules, the declaration for the model class \textsf{model.graphicLike} in the TEI module should be extended to include the element \texttt{<svg:svg>} as an alternative. This has the effect that elements in the TEI scheme which define their content model in terms of that element class (notably \hyperref[TEI.figure]{<figure>}) can now include it. A RELAX NG schema generated from such a specification can be used to validate documents in which the TEI \hyperref[TEI.figure]{<figure>} element contains any valid SVG representation of a graphic, embedded within an \texttt{<svg:svg>} element.
\subsubsection[{Linking Schemas to XML Documents}]{Linking Schemas to XML Documents}\label{TD-LinkingSchemas}\par
Schemas can be linked to XML documents by means of the <?xml-model?> processing instruction described in the W3C Working Group Note \textit{Associating Schemas with XML documents} (\url{http://www.w3.org/TR/xml-model/}). <?xml-model?> can be used for any type of schema, and may be used for multiple schemas: \par\hfill\bgroup\exampleFont\vskip 10pt\begin{shaded}
\obeyspaces <?xml-model href="tei\textunderscore tite.rng" type="application/xml" ?>\newline
<?xml-model href="checkLinks.sch" type="application/xml" schematypens="http://purl.oclc.org/dsdl/schematron" ?>\newline
<?xml-model href="tei\textunderscore tite.odd" type="application/tei+xml" schematypens="http://www.tei-c.org/ns/1.0" ?>\end{shaded}
\par\egroup 
 This example includes a standard RELAX NG schema, a Schematron schema which might be used for checking that all pointing attributes point at existing targets, and also a link to the TEI ODD file from which the RELAX NG schema was generated. See also \textit{\hyperref[HDSCHSPEC]{2.3.10.\ The Schema Specification}} for details of another method of linking an ODD specification into your file by including a \hyperref[TEI.schemaSpec]{<schemaSpec>} element in \hyperref[TEI.encodingDesc]{<encodingDesc>}.
\subsection[{Module for Documentation Elements}]{Module for Documentation Elements}\label{TDformal}\par
The module described in this chapter makes available the following components: \begin{description}

\item[{Module tagdocs: Documentation of TEI and other XML markup languages}]\hspace{1em}\hfill\linebreak
\mbox{}\\[-10pt] \begin{itemize}
\item {\itshape Elements defined}: \hyperref[TEI.altIdent]{altIdent} \hyperref[TEI.alternate]{alternate} \hyperref[TEI.anyElement]{anyElement} \hyperref[TEI.att]{att} \hyperref[TEI.attDef]{attDef} \hyperref[TEI.attList]{attList} \hyperref[TEI.attRef]{attRef} \hyperref[TEI.classRef]{classRef} \hyperref[TEI.classSpec]{classSpec} \hyperref[TEI.classes]{classes} \hyperref[TEI.code]{code} \hyperref[TEI.constraint]{constraint} \hyperref[TEI.constraintSpec]{constraintSpec} \hyperref[TEI.content]{content} \hyperref[TEI.dataFacet]{dataFacet} \hyperref[TEI.dataRef]{dataRef} \hyperref[TEI.dataSpec]{dataSpec} \hyperref[TEI.datatype]{datatype} \hyperref[TEI.defaultVal]{defaultVal} \hyperref[TEI.eg]{eg} \hyperref[TEI.egXML]{egXML} \hyperref[TEI.elementRef]{elementRef} \hyperref[TEI.elementSpec]{elementSpec} \hyperref[TEI.empty]{empty} \hyperref[TEI.equiv]{equiv} \hyperref[TEI.exemplum]{exemplum} \hyperref[TEI.gi]{gi} \hyperref[TEI.ident]{ident} \hyperref[TEI.listRef]{listRef} \hyperref[TEI.macroRef]{macroRef} \hyperref[TEI.macroSpec]{macroSpec} \hyperref[TEI.memberOf]{memberOf} \hyperref[TEI.model]{model} \hyperref[TEI.modelGrp]{modelGrp} \hyperref[TEI.modelSequence]{modelSequence} \hyperref[TEI.moduleRef]{moduleRef} \hyperref[TEI.moduleSpec]{moduleSpec} \hyperref[TEI.outputRendition]{outputRendition} \hyperref[TEI.param]{param} \hyperref[TEI.paramList]{paramList} \hyperref[TEI.paramSpec]{paramSpec} \hyperref[TEI.remarks]{remarks} \hyperref[TEI.schemaSpec]{schemaSpec} \hyperref[TEI.sequence]{sequence} \hyperref[TEI.specDesc]{specDesc} \hyperref[TEI.specGrp]{specGrp} \hyperref[TEI.specGrpRef]{specGrpRef} \hyperref[TEI.specList]{specList} \hyperref[TEI.tag]{tag} \hyperref[TEI.textNode]{textNode} \hyperref[TEI.val]{val} \hyperref[TEI.valDesc]{valDesc} \hyperref[TEI.valItem]{valItem} \hyperref[TEI.valList]{valList}
\item {\itshape Classes defined}: \hyperref[TEI.att.combinable]{att.combinable} \hyperref[TEI.att.deprecated]{att.deprecated} \hyperref[TEI.att.identified]{att.identified} \hyperref[TEI.att.namespaceable]{att.namespaceable} \hyperref[TEI.att.predicate]{att.predicate} \hyperref[TEI.att.repeatable]{att.repeatable} \hyperref[TEI.att.translatable]{att.translatable} \hyperref[TEI.model.contentPart]{model.contentPart}
\end{itemize} 
\end{description}  The selection and combination of modules to form a TEI schema is described in \textit{\hyperref[STIN]{1.2.\ Defining a TEI Schema}}.\par
The elements described in this chapter are all members of one of three classes: \textsf{model.oddDecl}, \textsf{model.oddRef}, or \textsf{model.phrase.xml}, with the exceptions of \hyperref[TEI.schemaSpec]{<schemaSpec>} (a member of \textsf{model.divPart}) and both \hyperref[TEI.eg]{<eg>} and \hyperref[TEI.egXML]{<egXML>} (members of \textsf{model.common} and \textsf{model.egLike}). All of these classes are declared along with the other general TEI classes, in the basic structure module documented in \textit{\hyperref[ST]{1.\ The TEI Infrastructure}}.\par
In addition, some elements are members of the \textsf{att.identified} class, which is documented in \textit{\hyperref[TDbuild]{22.8.1.\ TEI customizations}} above.