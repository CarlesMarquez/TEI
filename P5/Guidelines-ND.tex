
\section[{Names, Dates, People, and Places}]{Names, Dates, People, and Places}\label{ND}\par
This chapter describes a module which may be used for the encoding of names and other phrases descriptive of persons, places, or organizations, in a manner more detailed than that possible using the elements already provided for these purposes in the Core module. In section \textit{\hyperref[CONA]{3.6.\ Names, Numbers, Dates, Abbreviations, and Addresses}} it was noted that the elements provided in the core module allow an encoder to specify that a given text segment is a proper noun, or a \textit{referring string}, and to specify the kind of object named or referred to only by supplying a value for the {\itshape type} attribute. The elements provided by the present module allow the encoder to supply a detailed sub-structure for such referring strings, and to distinguish explicitly between names of persons, places, and organizations.\par
This module also provides elements for the representation of information about the person, place, or organization to which a given name is understood to refer and to represent the name itself, independently of its application. In simple terms, where the core module allows one simply to represent that a given piece of text is a \textit{name}, this module allows one further to represent a \textit{personal name}, to represent the \textit{person} being named, and to represent the \textit{canonical name} being used. A similar range is provided for names of places and organizations. The main intended applications for this module are in biographical, historical, or geographical data systems such as gazetteers and biographical databases, where these are to be integrated with encoded texts.\par
The chapter begins by discussing attributes common to many of the elements discussed in the remaining parts of the chapter (\textit{\hyperref[NDATTS]{13.1.\ Attribute Classes Defined by This Module}}) before discussing specifically the elements provided for the encoding of component parts of personal names (section \textit{\hyperref[NDPER]{13.2.1.\ Personal Names}}), place names (section \textit{\hyperref[NDPLAC]{13.2.3.\ Place Names}}) and organizational names (section \textit{\hyperref[NDORG]{13.2.2.\ Organizational Names}}). Elements for encoding personal and organizational data are discussed in section \textit{\hyperref[NDPERS]{13.3.\ Biographical and Prosopographical Data}}. Elements for the encoding of geographical data are discussed in section \textit{\hyperref[NDGEOG]{13.3.4.\ Places}}. Finally, elements for encoding onomastic data are discussed in \textit{\hyperref[NDNYM]{13.3.6.\ Names and Nyms}}, and the detailed encoding of dates and times is described in section \textit{\hyperref[NDDATE]{13.4.\ Dates}}.
\subsection[{Attribute Classes Defined by This Module}]{Attribute Classes Defined by This Module}\label{NDATTS}\par
Most of the elements made available by this chapter share some important characteristics which are expressed by their membership in specific attribute classes. Members of the class \textsf{att.naming} have specialized attributes which support linkage of a naming element with the entity (person, place, organization) being named; members of the class \textsf{att.datable} have specialized attributes which support a number of ways of normalizing the date or time of the data encoded by the element concerned.
\subsubsection[{Linking Names and Their Referents}]{Linking Names and Their Referents}\label{NDATTSnr}\par
The class \textsf{att.naming} is a subclass of the class \textsf{att.canonical}, from which it inherits the following attributes: 
\begin{sansreflist}
  
\item [\textbf{att.canonical}] provides attributes which can be used to associate a representation such as a name or title with canonical information about the object being named or referenced.\hfil\\[-10pt]\begin{sansreflist}
    \item[@{\itshape key}]
  provides an externally-defined means of identifying the entity (or entities) being named, using a coded value of some kind.
    \item[@{\itshape ref}]
  (reference) provides an explicit means of locating a full definition or identity for the entity being named by means of one or more URIs.
\end{sansreflist}  
\end{sansreflist}
 As discussed in \textit{\hyperref[CONARS]{3.6.1.\ Referring Strings}}, these attributes provide two different ways of associating any sort of name with its referent. For cases where all that is required is to provide some minimal information about the person name, for example their occupation or status, the \textsf{att.naming} class also provides a simple {\itshape role} attribute. It also provides an additional attribute, which allows the name itself to be associated with a base or canonical form: 
\begin{sansreflist}
  
\item [\textbf{att.naming}] provides attributes common to elements which refer to named persons, places, organizations etc.\hfil\\[-10pt]\begin{sansreflist}
    \item[@{\itshape role}]
  may be used to specify further information about the entity referenced by this name in the form of a set of whitespace-separated values, for example the occupation of a person, or the status of a place.
    \item[@{\itshape nymRef}]
  (reference to the canonical name) provides a means of locating the canonical form (\textit{nym}) of the names associated with the object named by the element bearing it.
\end{sansreflist}  
\end{sansreflist}
 The encoder may use these attributes in combination as appropriate. For example: \par\bgroup\index{name=<name>|exampleindex}\index{role=@role!<name>|exampleindex}\index{type=@type!<name>|exampleindex}\exampleFont \begin{shaded}\noindent\mbox{}That silly man {<\textbf{name}\hspace*{1em}{role}="{politician}"\hspace*{1em}{type}="{person}">}David Paul Brown{</\textbf{name}>}\mbox{}\newline 
 has suffered ...\end{shaded}\egroup\par \noindent  The {\itshape ref} attribute should be used wherever it is possible to supply a direct link such as a URI to indicate the location of canonical information about the referent. \par\bgroup\index{name=<name>|exampleindex}\index{ref=@ref!<name>|exampleindex}\index{type=@type!<name>|exampleindex}\exampleFont \begin{shaded}\noindent\mbox{}That silly man {<\textbf{name}\hspace*{1em}{ref}="{\#DPB1}"\hspace*{1em}{type}="{person}">}David Paul Brown{</\textbf{name}>} has\mbox{}\newline 
 suffered ...\end{shaded}\egroup\par \noindent  This encoding requires that there exist somewhere a \hyperref[TEI.person]{<person>} element with the identifier \texttt{DPB1}, which will contain canonical information about this particular person, marked up using the elements discussed in \textit{\hyperref[NDPERS]{13.3.\ Biographical and Prosopographical Data}} below. The same element might alternatively be provided by some other document, of course, which the same attribute could refer to by means of a URI, as explained in \textit{\hyperref[SAXP]{16.2.\ Pointing Mechanisms}}: \par\bgroup\index{name=<name>|exampleindex}\index{ref=@ref!<name>|exampleindex}\index{type=@type!<name>|exampleindex}\exampleFont \begin{shaded}\noindent\mbox{}That silly man {<\textbf{name}\hspace*{1em}{ref}="{http://www.example.com/personography.xml\#DPB1}"\mbox{}\newline 
\hspace*{1em}{type}="{person}">}David Paul Brown{</\textbf{name}>} has suffered ...\end{shaded}\egroup\par \noindent More than one URI may be supplied if the name refers to more than one person. For example, assuming the existence of another \hyperref[TEI.person]{<person>} element for Mrs Brown, with identifier \texttt{EBB1}, a reference to ‘the Browns’ might be encoded \par\bgroup\index{name=<name>|exampleindex}\index{ref=@ref!<name>|exampleindex}\index{type=@type!<name>|exampleindex}\exampleFont \begin{shaded}\noindent\mbox{}That wretched pair {<\textbf{name}\hspace*{1em}{ref}="{\#DPB1 \#EBB1}"\hspace*{1em}{type}="{person}">}the Browns{</\textbf{name}>} came\mbox{}\newline 
 to dine ...\end{shaded}\egroup\par \par
The {\itshape key} attribute is provided for cases where no such direct link is required: for example because resolution of the reference is carried out by some local convention, or because the encoder judges that no such resolution is necessary. As an example of the first case, a project might maintain its own local database system containing canonical information about persons and places, each entry in which is accessed by means of some system-specific identifier constructed in a project-specific way from the value supplied for the {\itshape key} attribute.\footnote{In the module described by chapter \textit{\hyperref[TD]{22.\ Documentation Elements}} a similar method is used to link element descriptions to the modules or classes to which they belong, for example.} As an example of the second case, consider the use of well-established codifications such as country or airport codes, which it is probably unnecessary for an encoder to expand further: \par\bgroup\index{name=<name>|exampleindex}\index{key=@key!<name>|exampleindex}\index{type=@type!<name>|exampleindex}\index{name=<name>|exampleindex}\index{key=@key!<name>|exampleindex}\index{type=@type!<name>|exampleindex}\exampleFont \begin{shaded}\noindent\mbox{} I never fly from {<\textbf{name}\hspace*{1em}{key}="{LHR}"\hspace*{1em}{type}="{place}">}Heathrow Airport{</\textbf{name}>} to \mbox{}\newline 
{<\textbf{name}\hspace*{1em}{key}="{FR}"\hspace*{1em}{type}="{place}">}France{</\textbf{name}>}\end{shaded}\egroup\par \par
However, as explained in \textit{\hyperref[CONARS]{3.6.1.\ Referring Strings}}, interchange is improved by use of tag URIs in {\itshape ref} instead of {\itshape key}.\par
The {\itshape nymRef} attribute has a more specialized use, where it is the name itself which is of interest rather than the person, place, or organization being named. See section \textit{\hyperref[NDNYM]{13.3.6.\ Names and Nyms}} for further discussion.\par
All members of the \textsf{att.naming} class inherit the following attributes from the \textsf{att.global.responsibility} class: 
\begin{sansreflist}
  
\item [\textbf{att.global.responsibility}] provides attributes indicating the agent responsible for some aspect of the text, the markup or something asserted by the markup, and the degree of certainty associated with it.\hfil\\[-10pt]\begin{sansreflist}
    \item[@{\itshape resp}]
  (responsible party) indicates the agency responsible for the intervention or interpretation, for example an editor or transcriber.
    \item[@{\itshape cert}]
  (certainty) signifies the degree of certainty associated with the intervention or interpretation.
\end{sansreflist}  
\end{sansreflist}
 This enables an encoder to record the agency responsible for a given assertion (for example, the name) and the confidence placed in that assertion by the encoder. Examples are given below.
\subsubsection[{Dating Attributes}]{Dating Attributes}\label{NDATTSda}\par
Members of the \textsf{att.datable} class share the following attributes: 
\begin{sansreflist}
  
\item [\textbf{att.datable}] provides attributes for normalization of elements that contain dates, times, or datable events.\hfil\\[-10pt]\begin{sansreflist}
    \item[@{\itshape period}]
  supplies a pointer to some location defining a named period of time within which the datable item is understood to have occurred.
\end{sansreflist}  
\item [\textbf{att.datable.w3c}] provides attributes for normalization of elements that contain datable events conforming to the W3C \textit{XML Schema Part 2: Datatypes Second Edition}.\hfil\\[-10pt]\begin{sansreflist}
    \item[@{\itshape when}]
  supplies the value of the date or time in a standard form, e.g. yyyy-mm-dd.
    \item[@{\itshape notBefore}]
  specifies the earliest possible date for the event in standard form, e.g. yyyy-mm-dd.
    \item[@{\itshape notAfter}]
  specifies the latest possible date for the event in standard form, e.g. yyyy-mm-dd.
    \item[@{\itshape from}]
  indicates the starting point of the period in standard form, e.g. yyyy-mm-dd.
    \item[@{\itshape to}]
  indicates the ending point of the period in standard form, e.g. yyyy-mm-dd.
\end{sansreflist}  
\end{sansreflist}
\par
The {\itshape when} attribute is used to specify a normalized form for any temporal expression, independently of how it is represented in the text, as in the following example: \par\bgroup\index{date=<date>|exampleindex}\index{when=@when!<date>|exampleindex}\index{date=<date>|exampleindex}\index{when=@when!<date>|exampleindex}\exampleFont \begin{shaded}\noindent\mbox{}{<\textbf{date}\hspace*{1em}{when}="{1807-06-09}">}June 9th{</\textbf{date}>} The period is approaching which will terminate my present\mbox{}\newline 
 copartnership. On the {<\textbf{date}\hspace*{1em}{when}="{1808-01-01}">}1st Jany.{</\textbf{date}>} next, it expires by its own limitation.\end{shaded}\egroup\par \par
The {\itshape period} attribute provides a convenient way of associating an event or date with a named period. Its value is a pointer which should indicate some other element where the period concerned is more precisely defined. A convenient location for such definitions is the \hyperref[TEI.taxonomy]{<taxonomy>} element in the \hyperref[TEI.classDecl]{<classDecl>} (classification declaration) in the \hyperref[TEI.encodingDesc]{<encodingDesc>} of a TEI header. A \hyperref[TEI.taxonomy]{<taxonomy>} may contain simply a bibliographic reference to an external definition for it. More usefully, it may also contain a series of \hyperref[TEI.category]{<category>} elements, each with an identifier and a description. The identifier can then be used as the target for a {\itshape period} attribute. For example, a taxonomy of named periods might be defined as follows: \par\bgroup\index{taxonomy=<taxonomy>|exampleindex}\index{category=<category>|exampleindex}\index{catDesc=<catDesc>|exampleindex}\index{category=<category>|exampleindex}\index{catDesc=<catDesc>|exampleindex}\index{category=<category>|exampleindex}\index{catDesc=<catDesc>|exampleindex}\index{ref=<ref>|exampleindex}\index{target=@target!<ref>|exampleindex}\index{date=<date>|exampleindex}\index{notBefore=@notBefore!<date>|exampleindex}\index{notAfter=@notAfter!<date>|exampleindex}\index{category=<category>|exampleindex}\index{catDesc=<catDesc>|exampleindex}\index{ref=<ref>|exampleindex}\index{target=@target!<ref>|exampleindex}\index{category=<category>|exampleindex}\index{catDesc=<catDesc>|exampleindex}\index{date=<date>|exampleindex}\index{when=@when!<date>|exampleindex}\exampleFont \begin{shaded}\noindent\mbox{}{<\textbf{taxonomy}\hspace*{1em}{xml:id}="{greekperiods}">}\mbox{}\newline 
\hspace*{1em}{<\textbf{category}\hspace*{1em}{xml:id}="{tyranny}">}\mbox{}\newline 
\hspace*{1em}\hspace*{1em}{<\textbf{catDesc}>}Before 510 BC{</\textbf{catDesc}>}\mbox{}\newline 
\hspace*{1em}{</\textbf{category}>}\mbox{}\newline 
\hspace*{1em}{<\textbf{category}\hspace*{1em}{xml:id}="{classical}">}\mbox{}\newline 
\hspace*{1em}\hspace*{1em}{<\textbf{catDesc}>}Between 510 and 323 BC{</\textbf{catDesc}>}\mbox{}\newline 
\hspace*{1em}{</\textbf{category}>}\mbox{}\newline 
\hspace*{1em}{<\textbf{category}\hspace*{1em}{xml:id}="{hellenistic}">}\mbox{}\newline 
\hspace*{1em}\hspace*{1em}{<\textbf{catDesc}>}\mbox{}\newline 
\hspace*{1em}\hspace*{1em}\hspace*{1em}{<\textbf{ref}\hspace*{1em}{target}="{http://www.wikipedia.com/wiki/Hellenistic}">}Hellenistic{</\textbf{ref}>}. Commonly treated as {<\textbf{date}\hspace*{1em}{notBefore}="{-0323}"\hspace*{1em}{notAfter}="{-0031}">}from the death of Alexander to the Roman conquest.{</\textbf{date}>}\mbox{}\newline 
\hspace*{1em}\hspace*{1em}{</\textbf{catDesc}>}\mbox{}\newline 
\hspace*{1em}{</\textbf{category}>}\mbox{}\newline 
\hspace*{1em}{<\textbf{category}\hspace*{1em}{xml:id}="{roman}">}\mbox{}\newline 
\hspace*{1em}\hspace*{1em}{<\textbf{catDesc}>}\mbox{}\newline 
\hspace*{1em}\hspace*{1em}\hspace*{1em}{<\textbf{ref}\hspace*{1em}{target}="{http://www.wikipedia.com/wiki/Roman\textunderscore Empire}">}Roman{</\textbf{ref}>}\mbox{}\newline 
\hspace*{1em}\hspace*{1em}{</\textbf{catDesc}>}\mbox{}\newline 
\hspace*{1em}{</\textbf{category}>}\mbox{}\newline 
\hspace*{1em}{<\textbf{category}\hspace*{1em}{xml:id}="{christian}">}\mbox{}\newline 
\hspace*{1em}\hspace*{1em}{<\textbf{catDesc}>} The Christian period technically starts at the birth of Jesus, but in practice is considered to date from\mbox{}\newline 
\hspace*{1em}\hspace*{1em}\hspace*{1em}\hspace*{1em} the conversion of Constantine in {<\textbf{date}\hspace*{1em}{when}="{0312}">}312 AD{</\textbf{date}>}. {</\textbf{catDesc}>}\mbox{}\newline 
\hspace*{1em}{</\textbf{category}>}\mbox{}\newline 
{</\textbf{taxonomy}>}\end{shaded}\egroup\par \par
With these definitions in place, any datable element may be associated with a specific period: \par\bgroup\index{placeName=<placeName>|exampleindex}\index{period=@period!<placeName>|exampleindex}\exampleFont \begin{shaded}\noindent\mbox{}{<\textbf{placeName}\hspace*{1em}{period}="{\#christian}">}Stauropolis{</\textbf{placeName}>}\end{shaded}\egroup\par \par
The other dating attributes provided by this class support a wide range of methods of specifying temporal information in a normalized form. The {\itshape from} and {\itshape to} attributes may be used to express the begining and ending of a period of time, for example: \par\bgroup\index{event=<event>|exampleindex}\index{from=@from!<event>|exampleindex}\index{to=@to!<event>|exampleindex}\index{label=<label>|exampleindex}\index{desc=<desc>|exampleindex}\index{placeName=<placeName>|exampleindex}\index{ref=@ref!<placeName>|exampleindex}\exampleFont \begin{shaded}\noindent\mbox{}{<\textbf{event}\hspace*{1em}{xml:id}="{eMBB}"\hspace*{1em}{from}="{1955-12-01}"\mbox{}\newline 
\hspace*{1em}{to}="{1956-12-20}">}\mbox{}\newline 
\hspace*{1em}{<\textbf{label}>}Montgomery Bus Boycott{</\textbf{label}>}\mbox{}\newline 
\hspace*{1em}{<\textbf{desc}>}A political and social protest campaign against the policy of racial segregation on the public transit system of\mbox{}\newline 
\hspace*{1em}\hspace*{1em} the city of {<\textbf{placeName}\hspace*{1em}{ref}="{\#MONT}">}Montgomery{</\textbf{placeName}>}.{</\textbf{desc}>}\mbox{}\newline 
{</\textbf{event}>}\end{shaded}\egroup\par \par
The {\itshape notBefore} and {\itshape notAfter} attributes may be used to express a range of possibilities for a particular date (or time). For example the following element, extracted from an imaginary prosopographic entry for Anne Calthorpe, indicates that although the exact date of her death is not known, it can be narrowed down to a particular range: from 22 August 1579 to 28 March 1582, inclusive. Ostensibly the encoder has evidence that Anne Calthorpe was alive on the 22nd of August 1579 and evidence that she was already dead on the 28th of March 1582. \par\bgroup\index{death=<death>|exampleindex}\index{notBefore=@notBefore!<death>|exampleindex}\index{notAfter=@notAfter!<death>|exampleindex}\exampleFont \begin{shaded}\noindent\mbox{}{<\textbf{death}\hspace*{1em}{notBefore}="{1579-08-22}"\mbox{}\newline 
\hspace*{1em}{notAfter}="{1582-03-28}"/>}\end{shaded}\egroup\par \par
Since {\itshape when} is used for a particular date or time, {\itshape from} and {\itshape to} for a duration, and {\itshape notBefore} and {\itshape notAfter} for a date or time within a range, it makes no sense to use {\itshape when} in combination with one or more of the others. Thus these Guidelines at present recommend against the use of {\itshape when} in combination with any of {\itshape from}, {\itshape to}, {\itshape notBefore}, or {\itshape notAfter}.\par
The {\itshape from} or {\itshape to} attributes imply that the temporal expression to which they are attached signifies a duration, so the use of either with {\itshape notBefore} or {\itshape notAfter} means a duration is indicated.  \par 
\begin{longtable}{P{0.12839879154078548\textwidth}P{0.3338368580060423\textwidth}P{0.3877643504531722\textwidth}}
\rowcolor{label}\tabcellsep {\itshape notBefore}\tabcellsep {\itshape from}\\\hline 
\Panel{{\itshape\bfseries notAfter}}{label}{1}{l}\tabcellsep range of possibilities, inclusive\tabcellsep duration from {\itshape from} to sometime before {\itshape notAfter}, inclusive\\
\Panel{{\itshape\bfseries to}}{label}{1}{l}\tabcellsep duration from sometime after {\itshape notBefore} to {\itshape to}, inclusive\tabcellsep duration from {\itshape from} to {\itshape to}, inclusive\end{longtable} \par
 \par
Some further self-explanatory examples follow: \par\bgroup\index{birth=<birth>|exampleindex}\index{when=@when!<birth>|exampleindex}\exampleFont \begin{shaded}\noindent\mbox{}{<\textbf{birth}\hspace*{1em}{when}="{1857-03-15}">}15 March 1857.{</\textbf{birth}>}\end{shaded}\egroup\par \noindent  \par\bgroup\index{birth=<birth>|exampleindex}\index{notBefore=@notBefore!<birth>|exampleindex}\index{notAfter=@notAfter!<birth>|exampleindex}\exampleFont \begin{shaded}\noindent\mbox{}{<\textbf{birth}\hspace*{1em}{notBefore}="{1857-03-01}"\mbox{}\newline 
\hspace*{1em}{notAfter}="{1857-04-30}">}Some time in March or April of 1857.{</\textbf{birth}>}\end{shaded}\egroup\par \noindent  \par\bgroup\index{residence=<residence>|exampleindex}\index{from=@from!<residence>|exampleindex}\index{to=@to!<residence>|exampleindex}\exampleFont \begin{shaded}\noindent\mbox{}{<\textbf{residence}\hspace*{1em}{from}="{1857-03-01}"\mbox{}\newline 
\hspace*{1em}{to}="{1857-04-30}">}Lived in Amsterdam during March and April of 1857.{</\textbf{residence}>}\end{shaded}\egroup\par \noindent  \par\bgroup\index{date=<date>|exampleindex}\index{from=@from!<date>|exampleindex}\index{notAfter=@notAfter!<date>|exampleindex}\exampleFont \begin{shaded}\noindent\mbox{}{<\textbf{date}\hspace*{1em}{from}="{1857-03-01}"\mbox{}\newline 
\hspace*{1em}{notAfter}="{1857-04-30}">}From the 1st of March to some time later in March or April of 1857.{</\textbf{date}>}\end{shaded}\egroup\par \noindent  \par\bgroup\index{residence=<residence>|exampleindex}\index{notBefore=@notBefore!<residence>|exampleindex}\index{to=@to!<residence>|exampleindex}\exampleFont \begin{shaded}\noindent\mbox{}{<\textbf{residence}\hspace*{1em}{notBefore}="{1857-03-01}"\mbox{}\newline 
\hspace*{1em}{to}="{1857-04-30}">}From the 1st of March or sometime later to the end of April,\mbox{}\newline 
 1857.{</\textbf{residence}>}\end{shaded}\egroup\par \noindent  \par\bgroup\index{residence=<residence>|exampleindex}\index{from=@from!<residence>|exampleindex}\index{to=@to!<residence>|exampleindex}\exampleFont \begin{shaded}\noindent\mbox{}{<\textbf{residence}\hspace*{1em}{from}="{1856-03}"\hspace*{1em}{to}="{1858-04}">}From sometime in March of 1856 to sometime in April of 1858.{</\textbf{residence}>}\end{shaded}\egroup\par \par
Normalization of date and time values permits the efficient processing of data (for example, to determine whether one event precedes or follows another). These examples all use the W3C standard format for representation of dates and times. Further examples, and discussion of some alternative approaches to normalization are given in section \textit{\hyperref[NDDATEISO]{13.4.3.\ More Expressive Normalizations}} below.
\subsection[{Names}]{Names}\label{NDNA}
\subsubsection[{Personal Names}]{Personal Names}\label{NDPER}\par
The core \hyperref[TEI.rs]{<rs>} and \hyperref[TEI.name]{<name>} elements can distinguish names in a text but are insufficiently powerful to mark their internal components or structure. To conduct nominal record linkage or even to create an alphabetically sorted list of personal names, it is important to distinguish between a family name, a forename and an honorary title. Similarly, when confronted with a string such as ‘John, by the grace of God, king of England, lord of Ireland, duke of Normandy and Aquitaine, and count of Anjou’, the analyst will often wish to distinguish amongst the various constituent elements present, since they provide additional information about the status, occupation, or residence of the person to whom the name belongs. The following elements are provided for these and related purposes: 
\begin{sansreflist}
  
\item [\textbf{<persName>}] (personal name) contains a proper noun or proper-noun phrase referring to a person, possibly including one or more of the person's forenames, surnames, honorifics, added names, etc.
\item [\textbf{<surname>}] (surname) contains a family (inherited) name, as opposed to a given, baptismal, or nick name.
\item [\textbf{<forename>}] (forename) contains a forename, given or baptismal name.
\item [\textbf{<roleName>}] (role name) contains a name component which indicates that the referent has a particular role or position in society, such as an official title or rank.
\item [\textbf{<addName>}] (additional name) contains an additional name component, such as a nickname, epithet, or alias, or any other descriptive phrase used within a personal name.
\item [\textbf{<nameLink>}] (name link) contains a connecting phrase or link used within a name but not regarded as part of it, such as \textit{van der} or \textit{of}.
\item [\textbf{<genName>}] (generational name component) contains a name component used to distinguish otherwise similar names on the basis of the relative ages or generations of the persons named.
\end{sansreflist}
\par
In addition to the \textsf{att.naming} attributes mentioned above, all of the above elements are members of the class \textsf{att.personal}, and thus share the following attributes: 
\begin{sansreflist}
  
\item [\textbf{att.personal}] (attributes for components of names usually, but not necessarily, personal names) common attributes for those elements which form part of a name usually, but not necessarily, a personal name.\hfil\\[-10pt]\begin{sansreflist}
    \item[@{\itshape full}]
  indicates whether the name component is given in full, as an abbreviation or simply as an initial.
    \item[@{\itshape sort}]
  (sort) specifies the sort order of the name component in relation to others within the name.
\end{sansreflist}  
\end{sansreflist}
\par
The \hyperref[TEI.persName]{<persName>} element may be used in preference to the general \hyperref[TEI.name]{<name>} element irrespective of whether or not the components of the personal name are also to be marked. The element \hyperref[TEI.persName]{<persName>} is synonymous with the element <name type="person">, except that its {\itshape type} attribute allows for further subcategorization of the personal name itself, for example as a married, birth, pen, pseudo, or religious name. Consequently the following examples are equivalent: \par\bgroup\index{rs=<rs>|exampleindex}\index{ref=@ref!<rs>|exampleindex}\index{type=@type!<rs>|exampleindex}\exampleFont \begin{shaded}\noindent\mbox{}That silly man {<\textbf{rs}\hspace*{1em}{ref}="{tag:projectname.org,2012:DPB1}"\mbox{}\newline 
\hspace*{1em}{type}="{person}">}David Paul Brown{</\textbf{rs}>} has suffered the furniture of his office to be seized the third time for\mbox{}\newline 
 rent.\end{shaded}\egroup\par \noindent  \par\bgroup\index{rs=<rs>|exampleindex}\index{ref=@ref!<rs>|exampleindex}\index{type=@type!<rs>|exampleindex}\index{name=<name>|exampleindex}\exampleFont \begin{shaded}\noindent\mbox{}That silly man {<\textbf{rs}\hspace*{1em}{ref}="{tag:projectname.org,2012:DPB1}"\mbox{}\newline 
\hspace*{1em}{type}="{person}">}\mbox{}\newline 
\hspace*{1em}{<\textbf{name}>}David Paul Brown{</\textbf{name}>}\mbox{}\newline 
{</\textbf{rs}>} has suffered ...\end{shaded}\egroup\par \noindent  \par\bgroup\index{name=<name>|exampleindex}\index{ref=@ref!<name>|exampleindex}\index{type=@type!<name>|exampleindex}\exampleFont \begin{shaded}\noindent\mbox{}That silly man {<\textbf{name}\hspace*{1em}{ref}="{tag:projectname.org,2012:DPB1}"\mbox{}\newline 
\hspace*{1em}{type}="{person}">}David Paul Brown{</\textbf{name}>} has suffered ...\end{shaded}\egroup\par \noindent  \par\bgroup\index{persName=<persName>|exampleindex}\index{ref=@ref!<persName>|exampleindex}\exampleFont \begin{shaded}\noindent\mbox{}That silly man {<\textbf{persName}\hspace*{1em}{ref}="{tag:projectname.org,2012:DPB1}">}David Paul\mbox{}\newline 
 Brown{</\textbf{persName}>} has suffered ...\end{shaded}\egroup\par \par
The \hyperref[TEI.persName]{<persName>} element is more powerful than the \hyperref[TEI.rs]{<rs>} and \hyperref[TEI.name]{<name>} elements because distinctive name components occurring within it can be marked as such.\par
Many cultures distinguish between a family or inherited \textit{surname} and additional personal names, often known as \textit{given names}. These should be tagged using the \hyperref[TEI.surname]{<surname>} and \hyperref[TEI.forename]{<forename>} elements respectively and may occur in any order: \par\bgroup\index{persName=<persName>|exampleindex}\index{surname=<surname>|exampleindex}\index{forename=<forename>|exampleindex}\index{forename=<forename>|exampleindex}\index{persName=<persName>|exampleindex}\index{forename=<forename>|exampleindex}\index{forename=<forename>|exampleindex}\index{surname=<surname>|exampleindex}\exampleFont \begin{shaded}\noindent\mbox{}{<\textbf{persName}>}\mbox{}\newline 
\hspace*{1em}{<\textbf{surname}>}Roosevelt{</\textbf{surname}>}, {<\textbf{forename}>}Franklin{</\textbf{forename}>}\mbox{}\newline 
\hspace*{1em}{<\textbf{forename}>}Delano{</\textbf{forename}>}\mbox{}\newline 
{</\textbf{persName}>}\mbox{}\newline 
{<\textbf{persName}>}\mbox{}\newline 
\hspace*{1em}{<\textbf{forename}>}Franklin{</\textbf{forename}>}\mbox{}\newline 
\hspace*{1em}{<\textbf{forename}>}Delano{</\textbf{forename}>}\mbox{}\newline 
\hspace*{1em}{<\textbf{surname}>}Roosevelt{</\textbf{surname}>}\mbox{}\newline 
{</\textbf{persName}>}\end{shaded}\egroup\par \par
The {\itshape type} attribute may be used with both \hyperref[TEI.forename]{<forename>} and \hyperref[TEI.surname]{<surname>} elements to provide further culture- or project-specific detail about the name component, for example: \par\bgroup\index{persName=<persName>|exampleindex}\index{forename=<forename>|exampleindex}\index{type=@type!<forename>|exampleindex}\index{forename=<forename>|exampleindex}\index{type=@type!<forename>|exampleindex}\index{surname=<surname>|exampleindex}\index{persName=<persName>|exampleindex}\index{forename=<forename>|exampleindex}\index{type=@type!<forename>|exampleindex}\index{forename=<forename>|exampleindex}\index{type=@type!<forename>|exampleindex}\index{surname=<surname>|exampleindex}\index{type=@type!<surname>|exampleindex}\index{surname=<surname>|exampleindex}\index{type=@type!<surname>|exampleindex}\index{persName=<persName>|exampleindex}\index{type=@type!<persName>|exampleindex}\index{persName=<persName>|exampleindex}\index{forename=<forename>|exampleindex}\index{surname=<surname>|exampleindex}\index{type=@type!<surname>|exampleindex}\exampleFont \begin{shaded}\noindent\mbox{}{<\textbf{persName}>}\mbox{}\newline 
\hspace*{1em}{<\textbf{forename}\hspace*{1em}{type}="{first}">}Franklin{</\textbf{forename}>}\mbox{}\newline 
\hspace*{1em}{<\textbf{forename}\hspace*{1em}{type}="{middle}">}Delano{</\textbf{forename}>}\mbox{}\newline 
\hspace*{1em}{<\textbf{surname}>}Roosevelt{</\textbf{surname}>}\mbox{}\newline 
{</\textbf{persName}>}\mbox{}\newline 
{<\textbf{persName}>}\mbox{}\newline 
\hspace*{1em}{<\textbf{forename}\hspace*{1em}{type}="{given}">}Margaret{</\textbf{forename}>}\mbox{}\newline 
\hspace*{1em}{<\textbf{forename}\hspace*{1em}{type}="{unused}">}Hilda{</\textbf{forename}>}\mbox{}\newline 
\hspace*{1em}{<\textbf{surname}\hspace*{1em}{type}="{birth}">}Roberts{</\textbf{surname}>}\mbox{}\newline 
\hspace*{1em}{<\textbf{surname}\hspace*{1em}{type}="{married}">}Thatcher{</\textbf{surname}>}\mbox{}\newline 
{</\textbf{persName}>}\mbox{}\newline 
{<\textbf{persName}\hspace*{1em}{type}="{religious}">}Muhammad Ali{</\textbf{persName}>}\mbox{}\newline 
{<\textbf{persName}>}\mbox{}\newline 
\hspace*{1em}{<\textbf{forename}>}Norman{</\textbf{forename}>}\mbox{}\newline 
\hspace*{1em}{<\textbf{surname}\hspace*{1em}{type}="{complex}">}St John Stevas{</\textbf{surname}>}\mbox{}\newline 
{</\textbf{persName}>}\end{shaded}\egroup\par \noindent  Values for the {\itshape type} attribute are not constrained, and may be chosen as appropriate to the encoding needs of the project. They may be used to distinguish different kinds of forename or surname, as well as to indicate the function a name component fills within the whole. In this example, we indicate that a surname is toponymic, and also point to the specific place name from which it is derived: \par\bgroup\index{persName=<persName>|exampleindex}\index{forename=<forename>|exampleindex}\index{surname=<surname>|exampleindex}\index{type=@type!<surname>|exampleindex}\index{ref=@ref!<surname>|exampleindex}\index{placeName=<placeName>|exampleindex}\exampleFont \begin{shaded}\noindent\mbox{}{<\textbf{persName}>}\mbox{}\newline 
\hspace*{1em}{<\textbf{forename}>}Johan{</\textbf{forename}>}\mbox{}\newline 
\hspace*{1em}{<\textbf{surname}\hspace*{1em}{type}="{toponymic}"\hspace*{1em}{ref}="{\#dystvold}">}Dystvold{</\textbf{surname}>}\mbox{}\newline 
{</\textbf{persName}>}\mbox{}\newline 
\textit{<!-- ... -->}\mbox{}\newline 
{<\textbf{placeName}\hspace*{1em}{xml:id}="{dystvold}">}Dystvold{</\textbf{placeName}>}\end{shaded}\egroup\par \par
The value complex was suggested above for the not uncommon case where the whole of a surname is composed of several other surname elements. These nested surnames may be individually tagged as well, together with appropriate type values: \par\bgroup\index{persName=<persName>|exampleindex}\index{forename=<forename>|exampleindex}\index{surname=<surname>|exampleindex}\index{type=@type!<surname>|exampleindex}\index{surname=<surname>|exampleindex}\index{type=@type!<surname>|exampleindex}\index{surname=<surname>|exampleindex}\index{type=@type!<surname>|exampleindex}\exampleFont \begin{shaded}\noindent\mbox{}{<\textbf{persName}>}\mbox{}\newline 
\hspace*{1em}{<\textbf{forename}>}Kara{</\textbf{forename}>}\mbox{}\newline 
\hspace*{1em}{<\textbf{surname}\hspace*{1em}{type}="{complex}">}\mbox{}\newline 
\hspace*{1em}\hspace*{1em}{<\textbf{surname}\hspace*{1em}{type}="{paternal}">}Hattersley{</\textbf{surname}>}- {<\textbf{surname}\hspace*{1em}{type}="{maternal}">}Smith{</\textbf{surname}>}\mbox{}\newline 
\hspace*{1em}{</\textbf{surname}>}\mbox{}\newline 
{</\textbf{persName}>}\end{shaded}\egroup\par \par
The {\itshape full} attribute may be used to indicate whether a name is an abbreviation, initials, or given in full: \par\bgroup\index{persName=<persName>|exampleindex}\index{forename=<forename>|exampleindex}\index{full=@full!<forename>|exampleindex}\index{surname=<surname>|exampleindex}\exampleFont \begin{shaded}\noindent\mbox{}{<\textbf{persName}>}\mbox{}\newline 
\hspace*{1em}{<\textbf{forename}\hspace*{1em}{full}="{abb}">}Maggie{</\textbf{forename}>}\mbox{}\newline 
\hspace*{1em}{<\textbf{surname}>}Thatcher{</\textbf{surname}>}\mbox{}\newline 
{</\textbf{persName}>}\end{shaded}\egroup\par \par
These elements may be applied as the encoder considers appropriate, including cases where phrases or expressions are used to stand for surnames or forenames, as in the following: \par\bgroup\index{s=<s>|exampleindex}\index{persName=<persName>|exampleindex}\index{forename=<forename>|exampleindex}\index{surname=<surname>|exampleindex}\index{persName=<persName>|exampleindex}\index{forename=<forename>|exampleindex}\index{surname=<surname>|exampleindex}\exampleFont \begin{shaded}\noindent\mbox{}{<\textbf{s}>}\mbox{}\newline 
\hspace*{1em}{<\textbf{persName}>}\mbox{}\newline 
\hspace*{1em}\hspace*{1em}{<\textbf{forename}>}Peter{</\textbf{forename}>}\mbox{}\newline 
\hspace*{1em}\hspace*{1em}{<\textbf{surname}>}son of Herbert{</\textbf{surname}>}\mbox{}\newline 
\hspace*{1em}{</\textbf{persName}>} gives the king 40 m. for having custody of the land and heir of\mbox{}\newline 
{<\textbf{persName}>}\mbox{}\newline 
\hspace*{1em}\hspace*{1em}{<\textbf{forename}>}John{</\textbf{forename}>}\mbox{}\newline 
\hspace*{1em}\hspace*{1em}{<\textbf{surname}>}son of Hugh{</\textbf{surname}>}\mbox{}\newline 
\hspace*{1em}{</\textbf{persName}>}...\mbox{}\newline 
{</\textbf{s}>}\end{shaded}\egroup\par \par
Similarly, patronymics may be treated as forenames, thus: \par\bgroup\index{persName=<persName>|exampleindex}\index{forename=<forename>|exampleindex}\index{forename=<forename>|exampleindex}\exampleFont \begin{shaded}\noindent\mbox{}... but it remained for {<\textbf{persName}>}\mbox{}\newline 
\hspace*{1em}{<\textbf{forename}>}Snorri{</\textbf{forename}>}\mbox{}\newline 
\hspace*{1em}{<\textbf{forename}>}Sturluson{</\textbf{forename}>}\mbox{}\newline 
{</\textbf{persName}>} to combine the two traditions in cyclic form.\end{shaded}\egroup\par \noindent  When a patronymic is used as a surname, however (e.g. by an individual who otherwise would have no surname, but lives in a culture which requires surnames), it may be tagged as such: \par\bgroup\index{persName=<persName>|exampleindex}\index{forename=<forename>|exampleindex}\index{surname=<surname>|exampleindex}\exampleFont \begin{shaded}\noindent\mbox{}Even {<\textbf{persName}>}\mbox{}\newline 
\hspace*{1em}{<\textbf{forename}>}Finnur{</\textbf{forename}>}\mbox{}\newline 
\hspace*{1em}{<\textbf{surname}>}Jonsson{</\textbf{surname}>}\mbox{}\newline 
{</\textbf{persName}>} acknowledged the artificiality of the procedure...\end{shaded}\egroup\par \noindent  Alternatively, it may be felt more appropriate to mark a patronymic as a distinct kind of name, neither a forename nor a surname, using the \hyperref[TEI.addName]{<addName>} element: \par\bgroup\index{persName=<persName>|exampleindex}\index{forename=<forename>|exampleindex}\index{addName=<addName>|exampleindex}\index{type=@type!<addName>|exampleindex}\exampleFont \begin{shaded}\noindent\mbox{}{<\textbf{persName}>}\mbox{}\newline 
\hspace*{1em}{<\textbf{forename}>}Egill{</\textbf{forename}>}\mbox{}\newline 
\hspace*{1em}{<\textbf{addName}\hspace*{1em}{type}="{patronym}">}Skallagrmsson{</\textbf{addName}>}\mbox{}\newline 
{</\textbf{persName}>}\end{shaded}\egroup\par \noindent  In the following example, the {\itshape type} attribute is used to distinguish a patronymic from other forenames: \par\bgroup\index{persName=<persName>|exampleindex}\index{ref=@ref!<persName>|exampleindex}\index{forename=<forename>|exampleindex}\index{sort=@sort!<forename>|exampleindex}\index{forename=<forename>|exampleindex}\index{sort=@sort!<forename>|exampleindex}\index{type=@type!<forename>|exampleindex}\index{surname=<surname>|exampleindex}\index{sort=@sort!<surname>|exampleindex}\exampleFont \begin{shaded}\noindent\mbox{}{<\textbf{persName}\hspace*{1em}{ref}="{tag:projectname.org,2012:pn9}">}\mbox{}\newline 
\hspace*{1em}{<\textbf{forename}\hspace*{1em}{sort}="{2}">}Sergei{</\textbf{forename}>}\mbox{}\newline 
\hspace*{1em}{<\textbf{forename}\hspace*{1em}{sort}="{3}"\hspace*{1em}{type}="{patronym}">}Mikhailovic{</\textbf{forename}>}\mbox{}\newline 
\hspace*{1em}{<\textbf{surname}\hspace*{1em}{sort}="{1}">}Uspensky{</\textbf{surname}>}\mbox{}\newline 
{</\textbf{persName}>}\end{shaded}\egroup\par \par
This example also demonstrates the use of the {\itshape sort} attribute common to all members of the \textsf{model.persNamePart} class; its effect is to state the sequence in which \hyperref[TEI.forename]{<forename>} and \hyperref[TEI.surname]{<surname>} elements should be combined when constructing a sort key for the name.\par
Some names include generational or dynastic information, such as a number, or phrases such as ‘Junior’, or ‘the Elder’; these qualifications may also be used to distinguish similarly named but unrelated people. In either case, the \hyperref[TEI.genName]{<genName>} element may be used to distinguish such labels from other parts of the name, as in the following examples: \par\bgroup\index{persName=<persName>|exampleindex}\index{ref=@ref!<persName>|exampleindex}\index{surname=<surname>|exampleindex}\index{genName=<genName>|exampleindex}\index{forename=<forename>|exampleindex}\exampleFont \begin{shaded}\noindent\mbox{}{<\textbf{persName}\hspace*{1em}{ref}="{tag:projectname.org,2012:HEMA1}">}\mbox{}\newline 
\hspace*{1em}{<\textbf{surname}>}Marques{</\textbf{surname}>}\mbox{}\newline 
\hspace*{1em}{<\textbf{genName}>}Junior{</\textbf{genName}>}, {<\textbf{forename}>}Henrique{</\textbf{forename}>}\mbox{}\newline 
{</\textbf{persName}>}\end{shaded}\egroup\par \noindent  \par\bgroup\index{persName=<persName>|exampleindex}\index{forename=<forename>|exampleindex}\index{genName=<genName>|exampleindex}\exampleFont \begin{shaded}\noindent\mbox{}{<\textbf{persName}>}\mbox{}\newline 
\hspace*{1em}{<\textbf{forename}>}Charles{</\textbf{forename}>}\mbox{}\newline 
\hspace*{1em}{<\textbf{genName}>}II{</\textbf{genName}>}\mbox{}\newline 
{</\textbf{persName}>}\end{shaded}\egroup\par \noindent  \par\bgroup\index{persName=<persName>|exampleindex}\index{forename=<forename>|exampleindex}\index{genName=<genName>|exampleindex}\index{surname=<surname>|exampleindex}\exampleFont \begin{shaded}\noindent\mbox{}{<\textbf{persName}\hspace*{1em}{xml:lang}="{de}">}\mbox{}\newline 
\hspace*{1em}{<\textbf{forename}>}Rudolf{</\textbf{forename}>}\mbox{}\newline 
\hspace*{1em}{<\textbf{genName}>}II{</\textbf{genName}>}\mbox{}\newline 
\hspace*{1em}{<\textbf{surname}>}von Habsburg{</\textbf{surname}>}\mbox{}\newline 
{</\textbf{persName}>}\end{shaded}\egroup\par \noindent  \par\bgroup\index{persName=<persName>|exampleindex}\index{surname=<surname>|exampleindex}\index{genName=<genName>|exampleindex}\exampleFont \begin{shaded}\noindent\mbox{}{<\textbf{persName}>}\mbox{}\newline 
\hspace*{1em}{<\textbf{surname}>}Smith{</\textbf{surname}>}\mbox{}\newline 
\hspace*{1em}{<\textbf{genName}>}Minor{</\textbf{genName}>}\mbox{}\newline 
{</\textbf{persName}>}\end{shaded}\egroup\par \par
It is also often convenient to distinguish phrases (historically similar to the generational labels mentioned above) used to link parts of a name together, such as ‘von’, ‘of’, ‘de’ etc. It is often a matter of arbitrary choice whether such components are regarded as part of the surname or not; the \hyperref[TEI.nameLink]{<nameLink>} element is provided as a means of making clear what the correct usage should be in a given case, as in the following examples: \par\bgroup\index{persName=<persName>|exampleindex}\index{ref=@ref!<persName>|exampleindex}\index{roleName=<roleName>|exampleindex}\index{type=@type!<roleName>|exampleindex}\index{full=@full!<roleName>|exampleindex}\index{nameLink=<nameLink>|exampleindex}\index{surname=<surname>|exampleindex}\exampleFont \begin{shaded}\noindent\mbox{}{<\textbf{persName}\hspace*{1em}{ref}="{tag:projectname.org,2012:DUDO1}">}\mbox{}\newline 
\hspace*{1em}{<\textbf{roleName}\hspace*{1em}{type}="{honorific}"\hspace*{1em}{full}="{abb}">}Mme{</\textbf{roleName}>}\mbox{}\newline 
\hspace*{1em}{<\textbf{nameLink}>}de la{</\textbf{nameLink}>}\mbox{}\newline 
\hspace*{1em}{<\textbf{surname}>}Rochefoucault{</\textbf{surname}>}\mbox{}\newline 
{</\textbf{persName}>}\end{shaded}\egroup\par \noindent  \par\bgroup\index{persName=<persName>|exampleindex}\index{forename=<forename>|exampleindex}\index{surname=<surname>|exampleindex}\exampleFont \begin{shaded}\noindent\mbox{}{<\textbf{persName}>}\mbox{}\newline 
\hspace*{1em}{<\textbf{forename}>}Walter{</\textbf{forename}>}\mbox{}\newline 
\hspace*{1em}{<\textbf{surname}>}de la Mare{</\textbf{surname}>}\mbox{}\newline 
{</\textbf{persName}>}\end{shaded}\egroup\par \par
Finally, the \hyperref[TEI.addName]{<addName>} and \hyperref[TEI.roleName]{<roleName>} elements are used to mark all name components other than those already listed. The distinction between them is that a \hyperref[TEI.roleName]{<roleName>} encloses an associated name component such as an aristocratic or official title which exists in some sense independently of its bearer. The distinction is not always a clear one. As elsewhere, the {\itshape type} attribute may be used with either element to supply culture- or application- specific distinctions. Some typical values for this attribute for names in the Western European tradition follow: \begin{description}

\item[{nobility}]An inherited or life-time title of nobility such as \textit{Lord}, \textit{Viscount}, \textit{Baron}, etc.
\item[{honorific}]An academic or other honorific prefixed to a name e.g. \textit{Doctor}, \textit{Professor}, \textit{Mrs.}, etc.
\item[{office}]Membership of some elected or appointed organization such as \textit{President}, \textit{Governor}, etc.
\item[{military}]Military rank such as \textit{Colonel}.
\item[{epithet}]A traditional descriptive phrase or nick-name such as \textit{The Hammer}, \textit{The Great}, etc.
\end{description}  Note, however, that the \textit{role} a person has in a given context (such as \textit{witness}, \textit{defendant}, etc. in a legal document) should not be encoded using the \hyperref[TEI.roleName]{<roleName>} element, since this is intended to mark roles which function as part of a person's name, not the role of the person bearing the name in general. Information about roles, occupations, etc. of a person are encoded within the \hyperref[TEI.person]{<person>} element discussed below in \textit{\hyperref[NDPERS]{13.3.\ Biographical and Prosopographical Data}}.\par
Here are some further examples of the usage of these elements: \par\bgroup\index{persName=<persName>|exampleindex}\index{ref=@ref!<persName>|exampleindex}\index{roleName=<roleName>|exampleindex}\index{type=@type!<roleName>|exampleindex}\index{forename=<forename>|exampleindex}\exampleFont \begin{shaded}\noindent\mbox{}{<\textbf{persName}\hspace*{1em}{ref}="{tag:projectname.org,2012:PGK1}">}\mbox{}\newline 
\hspace*{1em}{<\textbf{roleName}\hspace*{1em}{type}="{nobility}">}Princess{</\textbf{roleName}>}\mbox{}\newline 
\hspace*{1em}{<\textbf{forename}>}Grace{</\textbf{forename}>}\mbox{}\newline 
{</\textbf{persName}>}\end{shaded}\egroup\par \noindent  \par\bgroup\index{persName=<persName>|exampleindex}\index{ref=@ref!<persName>|exampleindex}\index{type=@type!<persName>|exampleindex}\index{addName=<addName>|exampleindex}\index{type=@type!<addName>|exampleindex}\index{surname=<surname>|exampleindex}\exampleFont \begin{shaded}\noindent\mbox{}{<\textbf{persName}\hspace*{1em}{ref}="{tag:projectname.org,2012:GRMO1}"\mbox{}\newline 
\hspace*{1em}{type}="{pseudo}">}\mbox{}\newline 
\hspace*{1em}{<\textbf{addName}\hspace*{1em}{type}="{honorific}">}Grandma{</\textbf{addName}>}\mbox{}\newline 
\hspace*{1em}{<\textbf{surname}>}Moses{</\textbf{surname}>}\mbox{}\newline 
{</\textbf{persName}>}\end{shaded}\egroup\par \noindent  \par\bgroup\index{persName=<persName>|exampleindex}\index{ref=@ref!<persName>|exampleindex}\index{roleName=<roleName>|exampleindex}\index{type=@type!<roleName>|exampleindex}\index{forename=<forename>|exampleindex}\index{surname=<surname>|exampleindex}\exampleFont \begin{shaded}\noindent\mbox{}{<\textbf{persName}\hspace*{1em}{ref}="{tag:projectname.org,2012:SLWICL1}">}\mbox{}\newline 
\hspace*{1em}{<\textbf{roleName}\hspace*{1em}{type}="{office}">}President{</\textbf{roleName}>}\mbox{}\newline 
\hspace*{1em}{<\textbf{forename}>}Bill{</\textbf{forename}>}\mbox{}\newline 
\hspace*{1em}{<\textbf{surname}>}Clinton{</\textbf{surname}>}\mbox{}\newline 
{</\textbf{persName}>}\end{shaded}\egroup\par \noindent  \par\bgroup\index{persName=<persName>|exampleindex}\index{ref=@ref!<persName>|exampleindex}\index{roleName=<roleName>|exampleindex}\index{type=@type!<roleName>|exampleindex}\index{surname=<surname>|exampleindex}\exampleFont \begin{shaded}\noindent\mbox{}{<\textbf{persName}\hspace*{1em}{ref}="{tag:projectname.org,2012:MOGA1}">}\mbox{}\newline 
\hspace*{1em}{<\textbf{roleName}\hspace*{1em}{type}="{military}">}Colonel{</\textbf{roleName}>}\mbox{}\newline 
\hspace*{1em}{<\textbf{surname}>}Gaddafi{</\textbf{surname}>}\mbox{}\newline 
{</\textbf{persName}>}\end{shaded}\egroup\par \noindent  \par\bgroup\index{persName=<persName>|exampleindex}\index{ref=@ref!<persName>|exampleindex}\index{forename=<forename>|exampleindex}\index{addName=<addName>|exampleindex}\index{type=@type!<addName>|exampleindex}\exampleFont \begin{shaded}\noindent\mbox{}{<\textbf{persName}\hspace*{1em}{ref}="{tag:projectname.org,2012:FRTG1}">}\mbox{}\newline 
\hspace*{1em}{<\textbf{forename}>}Frederick{</\textbf{forename}>}\mbox{}\newline 
\hspace*{1em}{<\textbf{addName}\hspace*{1em}{type}="{epithet}">}the Great{</\textbf{addName}>}\mbox{}\newline 
{</\textbf{persName}>}\end{shaded}\egroup\par \par
A name may have any combination of the above elements: \par\bgroup\index{persName=<persName>|exampleindex}\index{ref=@ref!<persName>|exampleindex}\index{roleName=<roleName>|exampleindex}\index{type=@type!<roleName>|exampleindex}\index{forename=<forename>|exampleindex}\index{sort=@sort!<forename>|exampleindex}\index{forename=<forename>|exampleindex}\index{full=@full!<forename>|exampleindex}\index{sort=@sort!<forename>|exampleindex}\index{addName=<addName>|exampleindex}\index{type=@type!<addName>|exampleindex}\index{addName=<addName>|exampleindex}\index{type=@type!<addName>|exampleindex}\index{surname=<surname>|exampleindex}\index{sort=@sort!<surname>|exampleindex}\index{genName=<genName>|exampleindex}\index{full=@full!<genName>|exampleindex}\exampleFont \begin{shaded}\noindent\mbox{}{<\textbf{persName}\hspace*{1em}{ref}="{tag:projectname.org,2012:EGBR1}">}\mbox{}\newline 
\hspace*{1em}{<\textbf{roleName}\hspace*{1em}{type}="{office}">}Governor{</\textbf{roleName}>}\mbox{}\newline 
\hspace*{1em}{<\textbf{forename}\hspace*{1em}{sort}="{2}">}Edmund{</\textbf{forename}>}\mbox{}\newline 
\hspace*{1em}{<\textbf{forename}\hspace*{1em}{full}="{init}"\hspace*{1em}{sort}="{3}">}G.{</\textbf{forename}>}\mbox{}\newline 
\hspace*{1em}{<\textbf{addName}\hspace*{1em}{type}="{nick}">}Jerry{</\textbf{addName}>}\mbox{}\newline 
\hspace*{1em}{<\textbf{addName}\hspace*{1em}{type}="{epithet}">}Moonbeam{</\textbf{addName}>}\mbox{}\newline 
\hspace*{1em}{<\textbf{surname}\hspace*{1em}{sort}="{1}">}Brown{</\textbf{surname}>}\mbox{}\newline 
\hspace*{1em}{<\textbf{genName}\hspace*{1em}{full}="{abb}">}Jr{</\textbf{genName}>}. \mbox{}\newline 
{</\textbf{persName}>}\end{shaded}\egroup\par \par
Although highly flexible, these mechanisms for marking personal name components will not cater for every personal name, nor for every processing need. Where the internal structure of personal names is highly complex or where name components are particularly ambiguous, feature structures are recommended as the most appropriate mechanism to mark and analyze them, as further discussed in chapter \textit{\hyperref[FS]{18.\ Feature Structures}}.\par
White space is allowed and therefore significant between elements within \hyperref[TEI.name]{<name>}, \hyperref[TEI.persName]{<persName>}, \hyperref[TEI.orgName]{<orgName>}, and \hyperref[TEI.placeName]{<placeName>}. Therefore \par\hfill\bgroup\exampleFont\vskip 10pt\begin{shaded}
\obeyspaces <persName> <forename>Mary</forename> <forename>Ann</forename> <nameLink>De</nameLink><surname>Mint</surname> </persName>\end{shaded}
\par\egroup 
 encodes ‘Mary Ann DeMint’ and \par\hfill\bgroup\exampleFont\vskip 10pt\begin{shaded}
\obeyspaces <persName> <forename>Mary</forename><forename>Ann</forename> <nameLink>De</nameLink> <surname>Mint</surname> </persName>\end{shaded}
\par\egroup 
 encodes ‘MaryAnn De Mint’. See \textit{\hyperref[STGAxs]{1.3.1.1.6.\ XML Whitespace}} for more information on whitespace in XML.
\subsubsection[{Organizational Names}]{Organizational Names}\label{NDORG}\par
In these Guidelines, we use the term ‘organization’ for any named collection of people regarded as a single unit. Typical examples include institutions such as ‘Harvard College’ or ‘the BBC’ and businesses such as ‘Apple’ or ‘Google’ but also racial or ethnic groupings or political factions where these are regarded as forming a single agency such as ‘the Scythians’ or ‘the Militant Tendency’. Giving a loosely-defined group of individuals a name often serves a particular political or social agenda and an analysis of the way such phrases are constructed and used may therefore be of considerable importance to the social historian, even where the objective existence of an ‘organization’ in this sense is harder to demonstrate than that of (say) a named person. In the case of businesses or other formally constituted institutions, the component parts of an organizational name may help to characterize the organization in terms of its perceived geographical location, ownership, likely number of employees, management structure, etc.\par
Like names of persons or places, organizational names can be marked up as referring strings or as proper names with the \hyperref[TEI.rs]{<rs>} or \hyperref[TEI.name]{<name>} elements respectively. The element \hyperref[TEI.orgName]{<orgName>} is provided for use where it is desired to distinguish organizational names more explicitly. 
\begin{sansreflist}
  
\item [\textbf{<orgName>}] (organization name) contains an organizational name.
\end{sansreflist}
 This element is a member of the same attribute classes as \hyperref[TEI.persName]{<persName>}, as discussed above in \textit{\hyperref[NDATTSnr]{13.1.1.\ Linking Names and Their Referents}}.\par
The \hyperref[TEI.orgName]{<orgName>} element may be used to mark up any form of organizational name: \par\bgroup\index{orgName=<orgName>|exampleindex}\index{type=@type!<orgName>|exampleindex}\index{ref=@ref!<orgName>|exampleindex}\exampleFont \begin{shaded}\noindent\mbox{}About a year back, a question of considerable interest\mbox{}\newline 
 was agitated in the {<\textbf{orgName}\hspace*{1em}{type}="{voluntary}"\mbox{}\newline 
\hspace*{1em}{ref}="{tag:projectname.org,2012:PAS1}">}Pennsyla. Abolition\mbox{}\newline 
 Society{</\textbf{orgName}>}\end{shaded}\egroup\par \noindent  This encoding is equivalent to, but more specific than, either of the following representations: \par\bgroup\index{rs=<rs>|exampleindex}\index{ref=@ref!<rs>|exampleindex}\index{type=@type!<rs>|exampleindex}\index{name=<name>|exampleindex}\exampleFont \begin{shaded}\noindent\mbox{}About a year back, a question of considerable interest\mbox{}\newline 
 was agitated in the {<\textbf{rs}\hspace*{1em}{ref}="{tag:projectname.org,2012:PAS1}"\mbox{}\newline 
\hspace*{1em}{type}="{org}">}\mbox{}\newline 
\hspace*{1em}{<\textbf{name}>}Pennsyla. Abolition Society{</\textbf{name}>}\mbox{}\newline 
{</\textbf{rs}>}.\end{shaded}\egroup\par \noindent  \par\bgroup\index{name=<name>|exampleindex}\index{ref=@ref!<name>|exampleindex}\index{type=@type!<name>|exampleindex}\exampleFont \begin{shaded}\noindent\mbox{}About a year back, a question of considerable interest was agitated in the\mbox{}\newline 
{<\textbf{name}\hspace*{1em}{ref}="{tag:projectname.org,2012:PAS1}"\mbox{}\newline 
\hspace*{1em}{type}="{org}">}Pennsyla. Abolition Society{</\textbf{name}>}.\end{shaded}\egroup\par \noindent  As shown above, like the \hyperref[TEI.rs]{<rs>} and \hyperref[TEI.name]{<name>} elements, the \hyperref[TEI.orgName]{<orgName>} element has a {\itshape key} attribute with which an external identifier such as a database key can be assigned to the organization name, and also a {\itshape ref} attribute which can be used to point directly to an \hyperref[TEI.org]{<org>} element containing information about the organization itself (see further \textit{\hyperref[ND-org]{13.3.3.\ Organizational Data}}). Its {\itshape type} attribute should be used to characterize the name (rather than the organization), for example as an acronym: \par\bgroup\index{orgName=<orgName>|exampleindex}\index{type=@type!<orgName>|exampleindex}\index{orgName=<orgName>|exampleindex}\index{type=@type!<orgName>|exampleindex}\exampleFont \begin{shaded}\noindent\mbox{}Mr Frost will be able to earn an\mbox{}\newline 
 extra fee from {<\textbf{orgName}\hspace*{1em}{type}="{acronym}">}BSkyB{</\textbf{orgName}>} rather than the \mbox{}\newline 
{<\textbf{orgName}\hspace*{1em}{type}="{acronym}">}BBC{</\textbf{orgName}>}\end{shaded}\egroup\par \noindent  as a phrase: \par\bgroup\index{country=<country>|exampleindex}\index{orgName=<orgName>|exampleindex}\index{type=@type!<orgName>|exampleindex}\index{orgName=<orgName>|exampleindex}\index{type=@type!<orgName>|exampleindex}\exampleFont \begin{shaded}\noindent\mbox{} The feeling in\mbox{}\newline 
{<\textbf{country}>}Canada{</\textbf{country}>} is one of strong aversion to the {<\textbf{orgName}\hspace*{1em}{type}="{phrase}">}United States Government{</\textbf{orgName}>},\mbox{}\newline 
 and of predilection for self-government under the \mbox{}\newline 
{<\textbf{orgName}\hspace*{1em}{type}="{phrase}">}English Crown{</\textbf{orgName}>}\end{shaded}\egroup\par \noindent  \par\bgroup\index{orgName=<orgName>|exampleindex}\exampleFont \begin{shaded}\noindent\mbox{}{<\textbf{orgName}>}The Justified Ancients of Mu Mu{</\textbf{orgName}>}\end{shaded}\egroup\par \noindent  or as a composite of other kinds of name: \par\bgroup\index{orgName=<orgName>|exampleindex}\index{type=@type!<orgName>|exampleindex}\index{surname=<surname>|exampleindex}\index{surname=<surname>|exampleindex}\exampleFont \begin{shaded}\noindent\mbox{}{<\textbf{orgName}\hspace*{1em}{type}="{partnerNames}">}\mbox{}\newline 
\hspace*{1em}{<\textbf{surname}>}Ernst{</\textbf{surname}>} \& {<\textbf{surname}>}Young{</\textbf{surname}>}\mbox{}\newline 
{</\textbf{orgName}>}\end{shaded}\egroup\par \par
The components of an organization's name may include place names as well as personal names: \par\bgroup\index{orgName=<orgName>|exampleindex}\index{type=@type!<orgName>|exampleindex}\index{orgName=<orgName>|exampleindex}\index{country=<country>|exampleindex}\exampleFont \begin{shaded}\noindent\mbox{}A spokesman from {<\textbf{orgName}\hspace*{1em}{type}="{regional}">}\mbox{}\newline 
\hspace*{1em}{<\textbf{orgName}>}IBM{</\textbf{orgName}>}\mbox{}\newline 
\hspace*{1em}{<\textbf{country}>}UK{</\textbf{country}>}\mbox{}\newline 
{</\textbf{orgName}>} said ... \end{shaded}\egroup\par \noindent  or role names: \par\bgroup\index{orgName=<orgName>|exampleindex}\index{name=<name>|exampleindex}\index{type=@type!<name>|exampleindex}\index{hi=<hi>|exampleindex}\exampleFont \begin{shaded}\noindent\mbox{}THE TICKET which you will\mbox{}\newline 
 receive herewith has been formed by the {<\textbf{orgName}>}Democratic Whig {<\textbf{name}\hspace*{1em}{type}="{role}">}party{</\textbf{name}>}\mbox{}\newline 
{</\textbf{orgName}>} after the most careful deliberation, with a reference to all the great objects of NATIONAL, STATE, COUNTY and\mbox{}\newline 
 CITY concern, and with a single eye to the {<\textbf{hi}>}Welfare and Best Interests of the Community{</\textbf{hi}>}.\end{shaded}\egroup\par \par
As indicated above, organizational names may also be specified hierarchically particularly where the named organization is itself a department or a branch of a larger organizational entity. ‘The Department of Modern History, Glasgow University’ is an example: \par\bgroup\index{orgName=<orgName>|exampleindex}\index{orgName=<orgName>|exampleindex}\index{orgName=<orgName>|exampleindex}\index{name=<name>|exampleindex}\index{type=@type!<name>|exampleindex}\index{name=<name>|exampleindex}\index{type=@type!<name>|exampleindex}\exampleFont \begin{shaded}\noindent\mbox{}{<\textbf{orgName}>}\mbox{}\newline 
\hspace*{1em}{<\textbf{orgName}>}Department of Modern History{</\textbf{orgName}>}\mbox{}\newline 
\hspace*{1em}{<\textbf{orgName}>}\mbox{}\newline 
\hspace*{1em}\hspace*{1em}{<\textbf{name}\hspace*{1em}{type}="{city}">}Glasgow{</\textbf{name}>}\mbox{}\newline 
\hspace*{1em}\hspace*{1em}{<\textbf{name}\hspace*{1em}{type}="{role}">}University{</\textbf{name}>}\mbox{}\newline 
\hspace*{1em}{</\textbf{orgName}>}\mbox{}\newline 
{</\textbf{orgName}>}\end{shaded}\egroup\par \par

\subsubsection[{Place Names}]{Place Names}\label{NDPLAC}\par
Like other proper nouns or noun phrases used as names, place names can simply be marked up with the \hyperref[TEI.rs]{<rs>} element, or with the \hyperref[TEI.name]{<name>} element. For cartographers and historical geographers, however, the component parts of a place name provide important information about the relation between the name and some spot in space and time. They also provide important evidence in historical linguistics.\par
These Guidelines distinguish three ways of referring to places. A place name (represented using the \hyperref[TEI.placeName]{<placeName>} element) may consist of one or more names for hierarchically-organized geo-political or administrative units (see section \textit{\hyperref[NDPLGU]{13.2.3.1.\ Geo-political Place Names}}). A place named simply in terms of geographical features such as mountains or rivers is represented using the \hyperref[TEI.geogName]{<geogName>} element (see section \textit{\hyperref[NDPLGF]{13.2.3.2.\ Geographic Names}}). Finally, an expression consisting of phrases expressing spatial or other kinds of relationship between other kinds of named place may itself be regarded as a way of referring to a place, and hence as a kind of named place (see section \textit{\hyperref[NDPLR]{13.2.3.3.\ Relative Place Names}}). 
\begin{sansreflist}
  
\item [\textbf{<placeName>}] (place name) contains an absolute or relative place name.
\item [\textbf{<geogName>}] (geographical name) identifies a name associated with some geographical feature such as Windrush Valley or Mount Sinai.
\end{sansreflist}
\par
As members of the \textsf{att.naming} class, all of these elements bear the attributes {\itshape key}, {\itshape ref}, and {\itshape nymRef} mentioned above. These attributes are primarily useful as a means of linking a place name with information about a specific place. Recommendations for the encoding of information about a place, as distinct from its name, are provided in \textit{\hyperref[NDGEOG]{13.3.4.\ Places}} below.\par
Like the \hyperref[TEI.persName]{<persName>} element discussed in section \textit{\hyperref[NDPER]{13.2.1.\ Personal Names}}, the \hyperref[TEI.placeName]{<placeName>} element may be regarded simply as an abbreviation for the elements <name type="place"> or <rs type="place">. The following encodings are thus equivalent:\footnote{Strictly, a suitable value such as figurative should be added to the two place names which are presented periphrastically in the second version of this example. This would preserve the distinction indicated by the choice of \hyperref[TEI.rs]{<rs>} rather than \hyperref[TEI.name]{<name>} to encode them in the first version of this example.} \par\bgroup\index{rs=<rs>|exampleindex}\index{ref=@ref!<rs>|exampleindex}\index{type=@type!<rs>|exampleindex}\index{name=<name>|exampleindex}\index{ref=@ref!<name>|exampleindex}\index{type=@type!<name>|exampleindex}\index{name=<name>|exampleindex}\index{ref=@ref!<name>|exampleindex}\index{type=@type!<name>|exampleindex}\index{rs=<rs>|exampleindex}\index{ref=@ref!<rs>|exampleindex}\index{type=@type!<rs>|exampleindex}\exampleFont \begin{shaded}\noindent\mbox{}After spending some time in our {<\textbf{rs}\hspace*{1em}{ref}="{tag:projectname.org,2012:NY1}"\mbox{}\newline 
\hspace*{1em}{type}="{place}">}modern {<\textbf{name}\hspace*{1em}{ref}="{tag:projectname.org,2012:BA1}"\mbox{}\newline 
\hspace*{1em}\hspace*{1em}{type}="{place}">}Babylon{</\textbf{name}>}\mbox{}\newline 
{</\textbf{rs}>}, {<\textbf{name}\hspace*{1em}{ref}="{tag:projectname.org,2012:NY1}"\mbox{}\newline 
\hspace*{1em}{type}="{place}">}New York{</\textbf{name}>}, I have proceeded to the {<\textbf{rs}\hspace*{1em}{ref}="{tag:projectname.org,2012:PH1}"\mbox{}\newline 
\hspace*{1em}{type}="{place}">}City of Brotherly Love{</\textbf{rs}>}.\end{shaded}\egroup\par \noindent  \par\bgroup\index{placeName=<placeName>|exampleindex}\index{ref=@ref!<placeName>|exampleindex}\index{placeName=<placeName>|exampleindex}\index{ref=@ref!<placeName>|exampleindex}\index{placeName=<placeName>|exampleindex}\index{ref=@ref!<placeName>|exampleindex}\index{placeName=<placeName>|exampleindex}\index{ref=@ref!<placeName>|exampleindex}\exampleFont \begin{shaded}\noindent\mbox{}After spending some time in our {<\textbf{placeName}\hspace*{1em}{ref}="{tag:projectname.org,2012:NY1}">}modern {<\textbf{placeName}\hspace*{1em}{ref}="{tag:projectname.org,2012:BA1}">}Babylon{</\textbf{placeName}>}\mbox{}\newline 
{</\textbf{placeName}>}, {<\textbf{placeName}\hspace*{1em}{ref}="{tag:projectname.org,2012:NY1}">}New York{</\textbf{placeName}>}, I have proceeded\mbox{}\newline 
 to the {<\textbf{placeName}\hspace*{1em}{ref}="{tag:projectname.org,2012:PH1}">}City of Brotherly Love{</\textbf{placeName}>}.\end{shaded}\egroup\par 
\paragraph[{Geo-political Place Names}]{Geo-political Place Names}\label{NDPLGU}\par
A place name may contain text with no indication of its internal structure: \par\bgroup\index{placeName=<placeName>|exampleindex}\exampleFont \begin{shaded}\noindent\mbox{}{<\textbf{placeName}>}Rochester, NY{</\textbf{placeName}>}\end{shaded}\egroup\par \noindent  More usually however, a place name of this kind will be further analysed in terms of its constitutive geo-political or administrative units. These may be arranged in ascending sequence according to their size or administrative importance, for example: ‘Rochester, New York’, or as a single such unit, for example ‘Belgium’. These Guidelines provide a hierarchy of generic element names, each of which may be more exactly specified by means of a {\itshape type} attribute: 
\begin{sansreflist}
  
\item [\textbf{<district>}] (district) contains the name of any kind of subdivision of a settlement, such as a parish, ward, or other administrative or geographic unit.
\item [\textbf{<settlement>}] (settlement) contains the name of a settlement such as a city, town, or village identified as a single geo-political or administrative unit.
\item [\textbf{<region>}] (region) contains the name of an administrative unit such as a state, province, or county, larger than a settlement, but smaller than a country.
\item [\textbf{<country>}] (country) contains the name of a geo-political unit, such as a nation, country, colony, or commonwealth, larger than or administratively superior to a region and smaller than a bloc.
\item [\textbf{<bloc>}] (bloc) contains the name of a geo-political unit consisting of two or more nation states or countries.
\end{sansreflist}
\par
These elements are all members of the \textsf{model.placeNamePart} class, members of which may be used anywhere that text is permitted, including within each other as in the following examples: \par\bgroup\index{placeName=<placeName>|exampleindex}\index{settlement=<settlement>|exampleindex}\index{type=@type!<settlement>|exampleindex}\index{region=<region>|exampleindex}\index{type=@type!<region>|exampleindex}\exampleFont \begin{shaded}\noindent\mbox{}{<\textbf{placeName}>}\mbox{}\newline 
\hspace*{1em}{<\textbf{settlement}\hspace*{1em}{type}="{city}">}Rochester{</\textbf{settlement}>}, {<\textbf{region}\hspace*{1em}{type}="{state}">}New York{</\textbf{region}>}\mbox{}\newline 
{</\textbf{placeName}>}\end{shaded}\egroup\par \noindent  \par\bgroup\index{placeName=<placeName>|exampleindex}\index{ref=@ref!<placeName>|exampleindex}\index{country=<country>|exampleindex}\index{type=@type!<country>|exampleindex}\index{bloc=<bloc>|exampleindex}\index{type=@type!<bloc>|exampleindex}\exampleFont \begin{shaded}\noindent\mbox{}{<\textbf{placeName}\hspace*{1em}{ref}="{tag:projectname.org,2012:LSEA1}">}\mbox{}\newline 
\hspace*{1em}{<\textbf{country}\hspace*{1em}{type}="{nation}">}Laos{</\textbf{country}>}, {<\textbf{bloc}\hspace*{1em}{type}="{sub-continent}">}Southeast Asia{</\textbf{bloc}>}\mbox{}\newline 
{</\textbf{placeName}>}\end{shaded}\egroup\par \noindent  \par\bgroup\index{placeName=<placeName>|exampleindex}\index{district=<district>|exampleindex}\index{type=@type!<district>|exampleindex}\index{settlement=<settlement>|exampleindex}\index{type=@type!<settlement>|exampleindex}\index{country=<country>|exampleindex}\exampleFont \begin{shaded}\noindent\mbox{}{<\textbf{placeName}>}\mbox{}\newline 
\hspace*{1em}{<\textbf{district}\hspace*{1em}{type}="{arondissement}">}6ème{</\textbf{district}>}\mbox{}\newline 
\hspace*{1em}{<\textbf{settlement}\hspace*{1em}{type}="{city}">}Paris, {</\textbf{settlement}>}\mbox{}\newline 
\hspace*{1em}{<\textbf{country}>}France{</\textbf{country}>}\mbox{}\newline 
{</\textbf{placeName}>}\end{shaded}\egroup\par 
\paragraph[{Geographic Names}]{Geographic Names}\label{NDPLGF}\par
Places may also be named in terms of geographic features such as mountains, lakes, or rivers, independently of geo-political units. The \hyperref[TEI.geogName]{<geogName>} is provided to mark up such names, as an alternative to the \hyperref[TEI.placeName]{<placeName>} element discussed above. For example: \par\bgroup\index{geogName=<geogName>|exampleindex}\index{ref=@ref!<geogName>|exampleindex}\index{type=@type!<geogName>|exampleindex}\exampleFont \begin{shaded}\noindent\mbox{}{<\textbf{geogName}\hspace*{1em}{ref}="{tag:projectname.org,2012:MIRI1}"\mbox{}\newline 
\hspace*{1em}{type}="{river}">}Mississippi River{</\textbf{geogName}>}\end{shaded}\egroup\par \par
In addition to the usual phrase level elements, the \hyperref[TEI.geogName]{<geogName>} element may contain the following specialized element: 
\begin{sansreflist}
  
\item [\textbf{<geogFeat>}] (geographical feature name) contains a common noun identifying some geographical feature contained within a geographic name, such as valley, mount, etc.
\end{sansreflist}
\par
Where the \hyperref[TEI.geogFeat]{<geogFeat>} element is used to characterize the kind of geographic feature being named, the \hyperref[TEI.name]{<name>} element will generally also be used to mark the associated proper noun or noun phrase: \par\bgroup\index{geogName=<geogName>|exampleindex}\index{ref=@ref!<geogName>|exampleindex}\index{type=@type!<geogName>|exampleindex}\index{name=<name>|exampleindex}\index{geogFeat=<geogFeat>|exampleindex}\exampleFont \begin{shaded}\noindent\mbox{}{<\textbf{geogName}\hspace*{1em}{ref}="{tag:projectname.org,2012:MIRI1}"\mbox{}\newline 
\hspace*{1em}{type}="{river}">}\mbox{}\newline 
\hspace*{1em}{<\textbf{name}>}Mississippi{</\textbf{name}>}\mbox{}\newline 
\hspace*{1em}{<\textbf{geogFeat}>}River{</\textbf{geogFeat}>}\mbox{}\newline 
{</\textbf{geogName}>}\end{shaded}\egroup\par \noindent  A more complex example, showing a variety of practices, follows: \par\bgroup\index{name=<name>|exampleindex}\index{ref=@ref!<name>|exampleindex}\index{type=@type!<name>|exampleindex}\index{geogName=<geogName>|exampleindex}\index{ref=@ref!<geogName>|exampleindex}\index{type=@type!<geogName>|exampleindex}\index{geogFeat=<geogFeat>|exampleindex}\index{name=<name>|exampleindex}\index{geogName=<geogName>|exampleindex}\index{ref=@ref!<geogName>|exampleindex}\index{type=@type!<geogName>|exampleindex}\index{geogFeat=<geogFeat>|exampleindex}\index{name=<name>|exampleindex}\index{geogName=<geogName>|exampleindex}\index{ref=@ref!<geogName>|exampleindex}\index{type=@type!<geogName>|exampleindex}\index{geogFeat=<geogFeat>|exampleindex}\index{name=<name>|exampleindex}\exampleFont \begin{shaded}\noindent\mbox{}The isolated ridge separates two great corridors which run from {<\textbf{name}\hspace*{1em}{ref}="{tag:projectname.org,2012:GLCO1}"\mbox{}\newline 
\hspace*{1em}{type}="{place}">}Glencoe{</\textbf{name}>} into {<\textbf{geogName}\hspace*{1em}{ref}="{tag:projectname.org,2012:GLET1}"\mbox{}\newline 
\hspace*{1em}{type}="{glen}">}\mbox{}\newline 
\hspace*{1em}{<\textbf{geogFeat}>}Glen{</\textbf{geogFeat}>}\mbox{}\newline 
\hspace*{1em}{<\textbf{name}>}Etive{</\textbf{name}>}\mbox{}\newline 
{</\textbf{geogName}>}, the {<\textbf{geogName}\hspace*{1em}{ref}="{tag:projectname.org,2012:LAGA1}"\mbox{}\newline 
\hspace*{1em}{type}="{hill}">}\mbox{}\newline 
\hspace*{1em}{<\textbf{geogFeat}\hspace*{1em}{xml:lang}="{gd}">}Lairig{</\textbf{geogFeat}>}\mbox{}\newline 
\hspace*{1em}{<\textbf{name}>}Gartain{</\textbf{name}>}\mbox{}\newline 
{</\textbf{geogName}>} and the \mbox{}\newline 
{<\textbf{geogName}\hspace*{1em}{ref}="{tag:projectname.org,2012:LAEI1}"\mbox{}\newline 
\hspace*{1em}{type}="{hill}">}\mbox{}\newline 
\hspace*{1em}{<\textbf{geogFeat}\hspace*{1em}{xml:lang}="{gd}">}Lairig{</\textbf{geogFeat}>}\mbox{}\newline 
\hspace*{1em}{<\textbf{name}>}Eilde{</\textbf{name}>}\mbox{}\newline 
{</\textbf{geogName}>}\end{shaded}\egroup\par \par
The Gaelic word \textit{lairig} may be glossed as  \textit{sloping hill face}. The most efficient way of including this information in the above encoding would be to create a separate \hyperref[TEI.nym]{<nym>} element for this component of the name and then point to it using the {\itshape nymRef} attribute, as further discussed in \textit{\hyperref[NDNYM]{13.3.6.\ Names and Nyms}}.
\paragraph[{Relative Place Names}]{Relative Place Names}\label{NDPLR}\par
All the place name specifications so far discussed are \textit{absolute}, in the sense that they define only one place. A place may however be specified in terms of its relationship to another place, for example ‘10 miles northeast of Paris’ or ‘near the top of Mount Sinai’. These \textit{relative place names} will contain a place name which acts as a referent (e.g. ‘Paris’ and ‘Mount Sinai’). They will also contain a word or phrase indicating the position of the place being named in relation to the referent (e.g. ‘the top of’, ‘north of’). A distance, possibly only vaguely specified, between the referent place and the place being indicated may also be present (e.g. ‘10 miles’, ‘near’).\par
Relative place names may be encoded using the following elements in combination with either a \hyperref[TEI.placeName]{<placeName>} or a \hyperref[TEI.geogName]{<geogName>} element. 
\begin{sansreflist}
  
\item [\textbf{<offset>}] (offset) marks that part of a relative temporal or spatial expression which indicates the direction of the offset between the two place names, dates, or times involved in the expression.
\item [\textbf{<measure>}] (measure) contains a word or phrase referring to some quantity of an object or commodity, usually comprising a number, a unit, and a commodity name.
\end{sansreflist}
 Some examples of relative place names are: \par\bgroup\index{placeName=<placeName>|exampleindex}\index{ref=@ref!<placeName>|exampleindex}\index{offset=<offset>|exampleindex}\index{geogName=<geogName>|exampleindex}\index{geogFeat=<geogFeat>|exampleindex}\index{name=<name>|exampleindex}\exampleFont \begin{shaded}\noindent\mbox{}{<\textbf{placeName}\hspace*{1em}{ref}="{tag:projectname.org,2012:NRPA1}">}\mbox{}\newline 
\hspace*{1em}{<\textbf{offset}>}near the top of{</\textbf{offset}>}\mbox{}\newline 
\hspace*{1em}{<\textbf{geogName}>}\mbox{}\newline 
\hspace*{1em}\hspace*{1em}{<\textbf{geogFeat}>}Mount{</\textbf{geogFeat}>}\mbox{}\newline 
\hspace*{1em}\hspace*{1em}{<\textbf{name}>}Sinai{</\textbf{name}>}\mbox{}\newline 
\hspace*{1em}{</\textbf{geogName}>}\mbox{}\newline 
{</\textbf{placeName}>}\end{shaded}\egroup\par \noindent  \par\bgroup\index{placeName=<placeName>|exampleindex}\index{measure=<measure>|exampleindex}\index{offset=<offset>|exampleindex}\index{settlement=<settlement>|exampleindex}\index{type=@type!<settlement>|exampleindex}\exampleFont \begin{shaded}\noindent\mbox{}{<\textbf{placeName}>}\mbox{}\newline 
\hspace*{1em}{<\textbf{measure}>}20 km{</\textbf{measure}>}\mbox{}\newline 
\hspace*{1em}{<\textbf{offset}>}north of{</\textbf{offset}>}\mbox{}\newline 
\hspace*{1em}{<\textbf{settlement}\hspace*{1em}{type}="{city}">}Paris{</\textbf{settlement}>}\mbox{}\newline 
{</\textbf{placeName}>}\end{shaded}\egroup\par \noindent  If desired, the distance specified may be normalized using the {\itshape unit} and {\itshape quantity} attributes of \hyperref[TEI.measure]{<measure>}: \par\bgroup\index{placeName=<placeName>|exampleindex}\index{ref=@ref!<placeName>|exampleindex}\index{measure=<measure>|exampleindex}\index{unit=@unit!<measure>|exampleindex}\index{quantity=@quantity!<measure>|exampleindex}\index{offset=<offset>|exampleindex}\index{settlement=<settlement>|exampleindex}\index{type=@type!<settlement>|exampleindex}\index{region=<region>|exampleindex}\index{type=@type!<region>|exampleindex}\exampleFont \begin{shaded}\noindent\mbox{}{<\textbf{placeName}\hspace*{1em}{ref}="{tag:projectname.org,2012:Duncan}">}\mbox{}\newline 
\hspace*{1em}{<\textbf{measure}\hspace*{1em}{unit}="{km}"\hspace*{1em}{quantity}="{17.7}">}11 miles{</\textbf{measure}>}\mbox{}\newline 
\hspace*{1em}{<\textbf{offset}>}Northwest of{</\textbf{offset}>}\mbox{}\newline 
\hspace*{1em}{<\textbf{settlement}\hspace*{1em}{type}="{city}">}Providence{</\textbf{settlement}>}, {<\textbf{region}\hspace*{1em}{type}="{state}">}RI{</\textbf{region}>}\mbox{}\newline 
{</\textbf{placeName}>}\end{shaded}\egroup\par \par
The internal structure of place names is like that of personal names—complex and subject to an enormous amount of variation across time and different cultures. The recommendations in this section should however be adequate for a majority of users and applications; they may be extended using the mechanisms described in chapter \textit{\hyperref[MD]{23.3.\ Customization}} to add new elements to the existing classes. When the focus of interest is on the name components themselves, as in place name studies for example, the elements discussed in \textit{\hyperref[NDNYM]{13.3.6.\ Names and Nyms}} may also be of use. Alternatively, the meaning structure itself may be represented using feature structures (\textit{\hyperref[FS]{18.\ Feature Structures}}).\par

\subsubsection[{Object Names}]{Object Names}\label{NDOBJN}\par

\begin{sansreflist}
  
\item [\textbf{<objectName>}] (name of an object) contains a proper noun or noun phrase used to refer to an object.
\end{sansreflist}
\par
As with other proper nouns or noun phrases used as names, the names of objects may be marked up simply with the \hyperref[TEI.name]{<name>} element. For those working with a variety of named objects the \hyperref[TEI.objectName]{<objectName>} element provides more flexibility. \par\bgroup\index{p=<p>|exampleindex}\index{objectName=<objectName>|exampleindex}\index{ref=@ref!<objectName>|exampleindex}\index{objectName=<objectName>|exampleindex}\index{ref=@ref!<objectName>|exampleindex}\index{placeName=<placeName>|exampleindex}\index{ref=@ref!<placeName>|exampleindex}\index{placeName=<placeName>|exampleindex}\index{ref=@ref!<placeName>|exampleindex}\index{orgName=<orgName>|exampleindex}\index{ref=@ref!<orgName>|exampleindex}\exampleFont \begin{shaded}\noindent\mbox{}{<\textbf{p}>}The {<\textbf{objectName}\hspace*{1em}{ref}="{\#MinsterLovellJewel}">}Minster Lovell Jewel{</\textbf{objectName}>} is probably the most similar to the\mbox{}\newline 
{<\textbf{objectName}\hspace*{1em}{ref}="{\#AlfredJewel}">}Alfred Jewel{</\textbf{objectName}>} and was found in {<\textbf{placeName}\hspace*{1em}{ref}="{\#MinsterLovell}">}Minster\mbox{}\newline 
\hspace*{1em}\hspace*{1em} Lovell{</\textbf{placeName}>} in {<\textbf{placeName}\hspace*{1em}{ref}="{\#Oxfordshire}">}Oxfordshire{</\textbf{placeName}>} and is kept at the {<\textbf{orgName}\hspace*{1em}{ref}="{\#AshmoleanMuseum}">}Ashmolean Museum{</\textbf{orgName}>}.{</\textbf{p}>}\end{shaded}\egroup\par \par
The \hyperref[TEI.objectName]{<objectName>} element may be used to encode any named object whether or not this is a text-bearing object. The use of \hyperref[TEI.objectName]{<objectName>} by itself does not categorize the object referenced, but this may be done further with the {\itshape type} and {\itshape subtype} attributes or through reference to a \hyperref[TEI.taxonomy]{<taxonomy>}. Additionally, the use of the \hyperref[TEI.objectName]{<objectName>} element says nothing about the physical reality of the object – that is whether it is real, fictional, purported, or missing – and this may be one aspect that some may wish to record through the {\itshape type} attribute. Where more detailed information is available for a named object the {\itshape ref} attribute should be used to point to an \hyperref[TEI.object]{<object>} element or other source of information about this object. The \hyperref[TEI.objectName]{<objectName>} element is intended for named objects; where an object is mentioned through a descriptive phrase but not named explicitly the \hyperref[TEI.rs]{<rs>} element should be used.
\subsection[{Biographical and Prosopographical Data}]{Biographical and Prosopographical Data}\label{NDPERS}\par
This module defines a number of special purpose elements which can be used to markup biographical, historical, and prosopographical data. We envisage a number of users and uses for these elements. For example, an encoder may be interested in creating or converting a set of biographical records, for example of the type found in a Dictionary of National Biography. Another use is the creation or conversion of a database-like collection of information about a group of people, such as the people referenced in a marked-up collection of documents, or persons who have served as informants in the creation of spoken corpora. It is also appropriate to use these elements to register information relating to those who have taken part in the creation of a TEI document.\par
To cater for this diversity, these Guidelines propose a flexible strategy, in which encoders may choose for themselves the approach appropriate to their needs. If one were interested, for example, in converting existing DNB-type records, and wanted to preserve the text as is, the \hyperref[TEI.person]{<person>} element (see \textit{\hyperref[NDPERSE]{13.3.2.\ The Person Element}}) could simply contain the text of an article, placed within \hyperref[TEI.p]{<p>} elements, possibly using elements such as \hyperref[TEI.name]{<name>} or \hyperref[TEI.date]{<date>} to mark up features of that text. For a more structured entry, however, one would extract the data and place information contained in the text, and encode it directly using the more specific elements described in this section.
\subsubsection[{Basic Principles}]{Basic Principles}\label{NDPERSbp}\par
Information about people, places, and organizations, of whatever type, essentially comprises a series of statements or assertions relating to: \begin{itemize}
\item characteristics or \textit{traits} which do not, by and large, change over time
\item characteristics or \textit{states} which hold true only at a specific time
\item \textit{events} or incidents which may lead to a change of state or, less frequently, trait,
\item external resources where other information on the subject can be found.
\end{itemize} \par
‘Characteristics’ or ‘traits’ are typically independent of an individual's volition or action and can be either physical, such as sex or hair and eye colour, or cultural, such as ethnicity, caste, or faith. The distinction is not entirely straightforward, however: while sex is fairly obviously a physical trait, gender should rather be regarded as culturally determined, and the division of mankind into different ‘races’, proposed by early (white European) anthropologists on the basis of physical characteristics such as skin colour, hair type and skull measurements, is now considered to be more a social or mental construct. Furthermore, while some characteristics will obviously change over time, hair colour for example, none, in principle—not even sex—is immutable.\par
‘States’ include, for example, marital status, place of residence and position or occupation. Such states have a definite duration, that is, they have a beginning and an end and are typically a consequence of the individual's own action or that of others.\par
By ‘changes in state’ are meant the events in a person's life such as birth, marriage, or appointment to office; such events will normally be associated with a specific date or a fairly narrow date-range. Changes in states can also cause or be caused by changes in characteristics. Any statement or assertion on any of these aspects of a person's life will be based on some source, possibly multiple sources, possibly contradictory. Taking all this into account it follows that each such statement or assertion needs to be able to be documented, put into a time frame and be relatable to other statements or assertions of the same or any of the other types.\par
The elements defined by the module described in this chapter may, for the most part, all be regarded as specializations of one or other of the above three classes. Generic elements for state, trait, and event are also defined: 
\begin{sansreflist}
  
\item [\textbf{<state>}] (state) contains a description of some status or quality attributed to a person, place, or organization often at some specific time or for a specific date range.
\item [\textbf{<trait>}] (trait) contains a description of some status or quality attributed to a person, place, or organization typically, but not necessarily, independent of the volition or action of the holder and usually not at some specific time or for a specific date range.
\item [\textbf{<event>}] (event) contains data relating to any kind of significant event associated with a person, place, or organization.\hfil\\[-10pt]\begin{sansreflist}
    \item[@{\itshape where}]
  indicates the location of an event by pointing to a \hyperref[TEI.place]{<place>} element
\end{sansreflist}  
\item [\textbf{<listEvent>}] (list of events) contains a list of descriptions, each of which provides information about an identifiable event.
\end{sansreflist}
\par
When developing a prosopography record of a named entity it is a common practice to refer explicitly to other resources, for example the Library of Congress Name Authority File, Virtual Internationl Authority File (VIAF), a gazetteer of places like Pleiades, or a printed book. 
\begin{sansreflist}
  
\item [\textbf{<idno>}] (identifier) supplies any form of identifier used to identify some object, such as a bibliographic item, a person, a title, an organization, etc. in a standardized way.\hfil\\[-10pt]\begin{sansreflist}
    \item[@{\itshape type}]
  categorizes the identifier, for example as an ISBN, Social Security number, etc.
\end{sansreflist}  
\end{sansreflist}
\par
Here is a simple example: \par\bgroup\index{place=<place>|exampleindex}\index{placeName=<placeName>|exampleindex}\index{location=<location>|exampleindex}\index{geo=<geo>|exampleindex}\index{idno=<idno>|exampleindex}\index{type=@type!<idno>|exampleindex}\index{note=<note>|exampleindex}\exampleFont \begin{shaded}\noindent\mbox{}{<\textbf{place}\hspace*{1em}{xml:id}="{Rome}">}\mbox{}\newline 
\hspace*{1em}{<\textbf{placeName}>}Rome{</\textbf{placeName}>}\mbox{}\newline 
\hspace*{1em}{<\textbf{location}>}\mbox{}\newline 
\hspace*{1em}\hspace*{1em}{<\textbf{geo}>}41.891775, 12.486137{</\textbf{geo}>}\mbox{}\newline 
\hspace*{1em}{</\textbf{location}>}\mbox{}\newline 
\hspace*{1em}{<\textbf{idno}\hspace*{1em}{type}="{Pleiades}">}423025{</\textbf{idno}>}\mbox{}\newline 
\hspace*{1em}{<\textbf{note}>}capital of the Roman Empire{</\textbf{note}>}\mbox{}\newline 
{</\textbf{place}>}\end{shaded}\egroup\par 
\subsubsection[{The Person Element}]{The Person Element}\label{NDPERSE}\par
Information about a person, as distinct from references to a person, for example by name, is grouped together within a \hyperref[TEI.person]{<person>} element. Information about a group of people regarded as a single entity (for example ‘the audience’ of a performance) may be encoded using the \hyperref[TEI.personGrp]{<personGrp>} element. Note however that information about a group of people with a distinct identity (for example a named theatrical troupe) should be recorded using the \hyperref[TEI.org]{<org>} element described in section \textit{\hyperref[ND-org]{13.3.3.\ Organizational Data}} below.\par
These elements may appear only within a \hyperref[TEI.listPerson]{<listPerson>} element, which groups such descriptions together, and optionally also describes relationships amongst the people listed. 
\begin{sansreflist}
  
\item [\textbf{<listPerson>}] (list of persons) contains a list of descriptions, each of which provides information about an identifiable person or a group of people, for example the participants in a language interaction, or the people referred to in a historical source.
\item [\textbf{<listRelation>}] provides information about relationships identified amongst people, places, and organizations, either informally as prose or as formally expressed relation links.
\end{sansreflist}
\par
One or more \hyperref[TEI.listPerson]{<listPerson>} elements may be supplied within the \hyperref[TEI.standOff]{<standOff>} element (see \textit{\hyperref[SASOstdf]{16.10.\ The standOff Container}}) or, when used to list the participants in a linguistic interaction, within the \hyperref[TEI.particDesc]{<particDesc>} (participant description) element in the \hyperref[TEI.profileDesc]{<profileDesc>} element of a TEI header. Like other forms of list, however, \hyperref[TEI.listPerson]{<listPerson>} can also appear within the body of a text when the module defined by this chapter is included in a schema.\par
The {\itshape type} attribute may be used to distinguish lists of people of different kinds where this is considered convenient: \par\bgroup\index{standOff=<standOff>|exampleindex}\index{listPerson=<listPerson>|exampleindex}\index{type=@type!<listPerson>|exampleindex}\index{person=<person>|exampleindex}\index{persName=<persName>|exampleindex}\index{note=<note>|exampleindex}\index{placeName=<placeName>|exampleindex}\index{title=<title>|exampleindex}\index{person=<person>|exampleindex}\index{persName=<persName>|exampleindex}\index{note=<note>|exampleindex}\index{choice=<choice>|exampleindex}\index{abbr=<abbr>|exampleindex}\index{expan=<expan>|exampleindex}\index{title=<title>|exampleindex}\index{title=<title>|exampleindex}\index{person=<person>|exampleindex}\index{persName=<persName>|exampleindex}\index{note=<note>|exampleindex}\index{placeName=<placeName>|exampleindex}\index{title=<title>|exampleindex}\index{person=<person>|exampleindex}\index{persName=<persName>|exampleindex}\index{note=<note>|exampleindex}\index{placeName=<placeName>|exampleindex}\index{title=<title>|exampleindex}\index{title=<title>|exampleindex}\index{title=<title>|exampleindex}\index{person=<person>|exampleindex}\index{persName=<persName>|exampleindex}\index{note=<note>|exampleindex}\index{placeName=<placeName>|exampleindex}\index{title=<title>|exampleindex}\index{title=<title>|exampleindex}\index{person=<person>|exampleindex}\index{persName=<persName>|exampleindex}\index{note=<note>|exampleindex}\index{choice=<choice>|exampleindex}\index{abbr=<abbr>|exampleindex}\index{expan=<expan>|exampleindex}\index{title=<title>|exampleindex}\index{listPerson=<listPerson>|exampleindex}\index{type=@type!<listPerson>|exampleindex}\index{person=<person>|exampleindex}\index{persName=<persName>|exampleindex}\index{note=<note>|exampleindex}\index{person=<person>|exampleindex}\index{persName=<persName>|exampleindex}\index{note=<note>|exampleindex}\index{person=<person>|exampleindex}\index{persName=<persName>|exampleindex}\index{note=<note>|exampleindex}\index{person=<person>|exampleindex}\index{persName=<persName>|exampleindex}\index{note=<note>|exampleindex}\index{person=<person>|exampleindex}\index{persName=<persName>|exampleindex}\index{note=<note>|exampleindex}\index{person=<person>|exampleindex}\index{persName=<persName>|exampleindex}\index{note=<note>|exampleindex}\exampleFont \begin{shaded}\noindent\mbox{}{<\textbf{standOff}>}\mbox{}\newline 
\hspace*{1em}{<\textbf{listPerson}\hspace*{1em}{type}="{fictional}">}\mbox{}\newline 
\hspace*{1em}\hspace*{1em}{<\textbf{person}\hspace*{1em}{xml:id}="{person\textunderscore FAS}">}\mbox{}\newline 
\hspace*{1em}\hspace*{1em}\hspace*{1em}{<\textbf{persName}>}Adam Schiff{</\textbf{persName}>}\mbox{}\newline 
\hspace*{1em}\hspace*{1em}\hspace*{1em}{<\textbf{note}>}District Attorney for {<\textbf{placeName}>}Manhattan{</\textbf{placeName}>} in\mbox{}\newline 
\hspace*{1em}\hspace*{1em}\hspace*{1em}\hspace*{1em}\hspace*{1em}\hspace*{1em} seasons 1 to 10 of {<\textbf{title}>}Law and Order{</\textbf{title}>}.{</\textbf{note}>}\mbox{}\newline 
\hspace*{1em}\hspace*{1em}{</\textbf{person}>}\mbox{}\newline 
\hspace*{1em}\hspace*{1em}{<\textbf{person}\hspace*{1em}{xml:id}="{person\textunderscore FML}">}\mbox{}\newline 
\hspace*{1em}\hspace*{1em}\hspace*{1em}{<\textbf{persName}>}Mike Logan{</\textbf{persName}>}\mbox{}\newline 
\hspace*{1em}\hspace*{1em}\hspace*{1em}{<\textbf{note}>}\mbox{}\newline 
\hspace*{1em}\hspace*{1em}\hspace*{1em}\hspace*{1em}{<\textbf{choice}>}\mbox{}\newline 
\hspace*{1em}\hspace*{1em}\hspace*{1em}\hspace*{1em}\hspace*{1em}{<\textbf{abbr}>}NYPD{</\textbf{abbr}>}\mbox{}\newline 
\hspace*{1em}\hspace*{1em}\hspace*{1em}\hspace*{1em}\hspace*{1em}{<\textbf{expan}>}New York Police\mbox{}\newline 
\hspace*{1em}\hspace*{1em}\hspace*{1em}\hspace*{1em}\hspace*{1em}\hspace*{1em}\hspace*{1em}\hspace*{1em}\hspace*{1em}\hspace*{1em} Department{</\textbf{expan}>}\mbox{}\newline 
\hspace*{1em}\hspace*{1em}\hspace*{1em}\hspace*{1em}{</\textbf{choice}>} Detective regularly appearing in\mbox{}\newline 
\hspace*{1em}\hspace*{1em}\hspace*{1em}\hspace*{1em}\hspace*{1em}\hspace*{1em} seasons 1 to 5 of {<\textbf{title}>}Law and Order{</\textbf{title}>} and seasons 5 to 7\mbox{}\newline 
\hspace*{1em}\hspace*{1em}\hspace*{1em}\hspace*{1em}\hspace*{1em}\hspace*{1em} of {<\textbf{title}>}Law and Order: Criminal Intent{</\textbf{title}>}.{</\textbf{note}>}\mbox{}\newline 
\hspace*{1em}\hspace*{1em}{</\textbf{person}>}\mbox{}\newline 
\hspace*{1em}\hspace*{1em}{<\textbf{person}\hspace*{1em}{xml:id}="{person\textunderscore FBS}">}\mbox{}\newline 
\hspace*{1em}\hspace*{1em}\hspace*{1em}{<\textbf{persName}>}Benjamin Stone{</\textbf{persName}>}\mbox{}\newline 
\hspace*{1em}\hspace*{1em}\hspace*{1em}{<\textbf{note}>}Executive Assistant District Attorney for\mbox{}\newline 
\hspace*{1em}\hspace*{1em}\hspace*{1em}{<\textbf{placeName}>}Manhattan{</\textbf{placeName}>} in seasons 1 to 4 of {<\textbf{title}>}Law\mbox{}\newline 
\hspace*{1em}\hspace*{1em}\hspace*{1em}\hspace*{1em}\hspace*{1em}\hspace*{1em}\hspace*{1em}\hspace*{1em} and Order{</\textbf{title}>}\mbox{}\newline 
\hspace*{1em}\hspace*{1em}\hspace*{1em}{</\textbf{note}>}\mbox{}\newline 
\hspace*{1em}\hspace*{1em}{</\textbf{person}>}\mbox{}\newline 
\hspace*{1em}\hspace*{1em}{<\textbf{person}\hspace*{1em}{xml:id}="{person\textunderscore FJM}">}\mbox{}\newline 
\hspace*{1em}\hspace*{1em}\hspace*{1em}{<\textbf{persName}>}Jack McCoy{</\textbf{persName}>}\mbox{}\newline 
\hspace*{1em}\hspace*{1em}\hspace*{1em}{<\textbf{note}>}An Executive Assistant District Attorney then District\mbox{}\newline 
\hspace*{1em}\hspace*{1em}\hspace*{1em}\hspace*{1em}\hspace*{1em}\hspace*{1em} Attorney for {<\textbf{placeName}>}Manhattan{</\textbf{placeName}>} in seasons 5 to 10\mbox{}\newline 
\hspace*{1em}\hspace*{1em}\hspace*{1em}\hspace*{1em}\hspace*{1em}\hspace*{1em} of {<\textbf{title}>}Law and Order{</\textbf{title}>}, in seasons 1, 9, 11, and 19 of\mbox{}\newline 
\hspace*{1em}\hspace*{1em}\hspace*{1em}{<\textbf{title}>}Law and Order: Special Victims Unit{</\textbf{title}>}, and in\mbox{}\newline 
\hspace*{1em}\hspace*{1em}\hspace*{1em}\hspace*{1em}\hspace*{1em}\hspace*{1em} season 1 of {<\textbf{title}>}Law and Order: Trial by Jury{</\textbf{title}>}.{</\textbf{note}>}\mbox{}\newline 
\hspace*{1em}\hspace*{1em}{</\textbf{person}>}\mbox{}\newline 
\hspace*{1em}\hspace*{1em}{<\textbf{person}\hspace*{1em}{xml:id}="{person\textunderscore FJR}">}\mbox{}\newline 
\hspace*{1em}\hspace*{1em}\hspace*{1em}{<\textbf{persName}>}Jamie Ross{</\textbf{persName}>}\mbox{}\newline 
\hspace*{1em}\hspace*{1em}\hspace*{1em}{<\textbf{note}>}An Assistant District Attorney for\mbox{}\newline 
\hspace*{1em}\hspace*{1em}\hspace*{1em}{<\textbf{placeName}>}Manhattan{</\textbf{placeName}>} in seasons 7 \& 8 of\mbox{}\newline 
\hspace*{1em}\hspace*{1em}\hspace*{1em}{<\textbf{title}>}Law and Order{</\textbf{title}>}, and a defense attorney in seasons\mbox{}\newline 
\hspace*{1em}\hspace*{1em}\hspace*{1em}\hspace*{1em}\hspace*{1em}\hspace*{1em} 10 \& 11, and then a judge in {<\textbf{title}>}Law and Order: Trial by\mbox{}\newline 
\hspace*{1em}\hspace*{1em}\hspace*{1em}\hspace*{1em}\hspace*{1em}\hspace*{1em}\hspace*{1em}\hspace*{1em} Jury{</\textbf{title}>}.{</\textbf{note}>}\mbox{}\newline 
\hspace*{1em}\hspace*{1em}{</\textbf{person}>}\mbox{}\newline 
\hspace*{1em}\hspace*{1em}{<\textbf{person}\hspace*{1em}{xml:id}="{person\textunderscore FJF}">}\mbox{}\newline 
\hspace*{1em}\hspace*{1em}\hspace*{1em}{<\textbf{persName}>}Joe Fontana{</\textbf{persName}>}\mbox{}\newline 
\hspace*{1em}\hspace*{1em}\hspace*{1em}{<\textbf{note}>}\mbox{}\newline 
\hspace*{1em}\hspace*{1em}\hspace*{1em}\hspace*{1em}{<\textbf{choice}>}\mbox{}\newline 
\hspace*{1em}\hspace*{1em}\hspace*{1em}\hspace*{1em}\hspace*{1em}{<\textbf{abbr}>}NYPD{</\textbf{abbr}>}\mbox{}\newline 
\hspace*{1em}\hspace*{1em}\hspace*{1em}\hspace*{1em}\hspace*{1em}{<\textbf{expan}>}New York Police\mbox{}\newline 
\hspace*{1em}\hspace*{1em}\hspace*{1em}\hspace*{1em}\hspace*{1em}\hspace*{1em}\hspace*{1em}\hspace*{1em}\hspace*{1em}\hspace*{1em} Department{</\textbf{expan}>}\mbox{}\newline 
\hspace*{1em}\hspace*{1em}\hspace*{1em}\hspace*{1em}{</\textbf{choice}>} Detective regularly appearing\mbox{}\newline 
\hspace*{1em}\hspace*{1em}\hspace*{1em}\hspace*{1em}\hspace*{1em}\hspace*{1em} in seasons 15 \& 16 of {<\textbf{title}>}Law and Order{</\textbf{title}>}.{</\textbf{note}>}\mbox{}\newline 
\hspace*{1em}\hspace*{1em}{</\textbf{person}>}\mbox{}\newline 
\hspace*{1em}{</\textbf{listPerson}>}\mbox{}\newline 
\textit{<!-- == == -->}\mbox{}\newline 
\hspace*{1em}{<\textbf{listPerson}\hspace*{1em}{type}="{real}">}\mbox{}\newline 
\hspace*{1em}\hspace*{1em}{<\textbf{person}\hspace*{1em}{xml:id}="{person\textunderscore RAS}">}\mbox{}\newline 
\hspace*{1em}\hspace*{1em}\hspace*{1em}{<\textbf{persName}>}Adam Schiff{</\textbf{persName}>}\mbox{}\newline 
\hspace*{1em}\hspace*{1em}\hspace*{1em}{<\textbf{note}>}U.S. Representative from California since 2013.{</\textbf{note}>}\mbox{}\newline 
\hspace*{1em}\hspace*{1em}{</\textbf{person}>}\mbox{}\newline 
\hspace*{1em}\hspace*{1em}{<\textbf{person}\hspace*{1em}{xml:id}="{person\textunderscore RML}">}\mbox{}\newline 
\hspace*{1em}\hspace*{1em}\hspace*{1em}{<\textbf{persName}>}Mike Logan{</\textbf{persName}>}\mbox{}\newline 
\hspace*{1em}\hspace*{1em}\hspace*{1em}{<\textbf{note}>}Gridiron football player for the Pittsburgh Steelers from\mbox{}\newline 
\hspace*{1em}\hspace*{1em}\hspace*{1em}\hspace*{1em}\hspace*{1em}\hspace*{1em} 2001 to 2006.{</\textbf{note}>}\mbox{}\newline 
\hspace*{1em}\hspace*{1em}{</\textbf{person}>}\mbox{}\newline 
\hspace*{1em}\hspace*{1em}{<\textbf{person}\hspace*{1em}{xml:id}="{person\textunderscore RBS}">}\mbox{}\newline 
\hspace*{1em}\hspace*{1em}\hspace*{1em}{<\textbf{persName}>}Benjamin Stone{</\textbf{persName}>}\mbox{}\newline 
\hspace*{1em}\hspace*{1em}\hspace*{1em}{<\textbf{note}>}Michigan State Senator from 1968 to 1979.{</\textbf{note}>}\mbox{}\newline 
\hspace*{1em}\hspace*{1em}{</\textbf{person}>}\mbox{}\newline 
\hspace*{1em}\hspace*{1em}{<\textbf{person}\hspace*{1em}{xml:id}="{person\textunderscore RJM}">}\mbox{}\newline 
\hspace*{1em}\hspace*{1em}\hspace*{1em}{<\textbf{persName}>}Jack McCoy{</\textbf{persName}>}\mbox{}\newline 
\hspace*{1em}\hspace*{1em}\hspace*{1em}{<\textbf{note}>}Iowa State Representative from 1955 to 1959.{</\textbf{note}>}\mbox{}\newline 
\hspace*{1em}\hspace*{1em}{</\textbf{person}>}\mbox{}\newline 
\hspace*{1em}\hspace*{1em}{<\textbf{person}\hspace*{1em}{xml:id}="{person\textunderscore RJR}">}\mbox{}\newline 
\hspace*{1em}\hspace*{1em}\hspace*{1em}{<\textbf{persName}>}Jamie Ross{</\textbf{persName}>}\mbox{}\newline 
\hspace*{1em}\hspace*{1em}\hspace*{1em}{<\textbf{note}>}Broadway actor, with occasional forays into television,\mbox{}\newline 
\hspace*{1em}\hspace*{1em}\hspace*{1em}\hspace*{1em}\hspace*{1em}\hspace*{1em} from 1971 to roughly 2007.{</\textbf{note}>}\mbox{}\newline 
\hspace*{1em}\hspace*{1em}{</\textbf{person}>}\mbox{}\newline 
\hspace*{1em}\hspace*{1em}{<\textbf{person}\hspace*{1em}{xml:id}="{person\textunderscore RJF}">}\mbox{}\newline 
\hspace*{1em}\hspace*{1em}\hspace*{1em}{<\textbf{persName}>}Joe Fontana{</\textbf{persName}>}\mbox{}\newline 
\hspace*{1em}\hspace*{1em}\hspace*{1em}{<\textbf{note}>}A member of Canada’s House of Commons from 1987 to 2006,\mbox{}\newline 
\hspace*{1em}\hspace*{1em}\hspace*{1em}\hspace*{1em}\hspace*{1em}\hspace*{1em} and mayor of London, Ontario from 2010 to 2014.{</\textbf{note}>}\mbox{}\newline 
\hspace*{1em}\hspace*{1em}{</\textbf{person}>}\mbox{}\newline 
\hspace*{1em}{</\textbf{listPerson}>}\mbox{}\newline 
{</\textbf{standOff}>}\end{shaded}\egroup\par \par
The \hyperref[TEI.person]{<person>} element carries several attributes. As a member of the classes \textsf{att.global.responsibility}, \textsf{att.editLike}, and \textsf{att.global.source} class, it carries the usual attributes for providing details about the information recorded for that person, such as its reliability or source: 
\begin{sansreflist}
  
\item [\textbf{att.global.responsibility}] provides attributes indicating the agent responsible for some aspect of the text, the markup or something asserted by the markup, and the degree of certainty associated with it.\hfil\\[-10pt]\begin{sansreflist}
    \item[@{\itshape cert}]
  (certainty) signifies the degree of certainty associated with the intervention or interpretation.
    \item[@{\itshape resp}]
  (responsible party) indicates the agency responsible for the intervention or interpretation, for example an editor or transcriber.
\end{sansreflist}  
\item [\textbf{att.editLike}] provides attributes describing the nature of an encoded scholarly intervention or interpretation of any kind.\hfil\\[-10pt]\begin{sansreflist}
    \item[@{\itshape evidence}]
  indicates the nature of the evidence supporting the reliability or accuracy of the intervention or interpretation.
\end{sansreflist}  
\item [\textbf{att.global.source}] provides an attribute used by elements to point to an external source.\hfil\\[-10pt]\begin{sansreflist}
    \item[@{\itshape source}]
  specifies the source from which some aspect of this element is drawn.
\end{sansreflist}  
\end{sansreflist}
 In addition, a small number of very commonly used personal properties may be recorded using attributes specific to \hyperref[TEI.person]{<person>} and \hyperref[TEI.personGrp]{<personGrp>}: 
\begin{sansreflist}
  
\item [\textbf{<person>}] (person) provides information about an identifiable individual, for example a participant in a language interaction, or a person referred to in a historical source.\hfil\\[-10pt]\begin{sansreflist}
    \item[@{\itshape role}]
  specifies a primary role or classification for the person.
    \item[@{\itshape sex}]
  specifies the sex of the person.
    \item[@{\itshape age}]
  specifies an age group for the person.
\end{sansreflist}  
\item [\textbf{<personGrp>}] (personal group) describes a group of individuals treated as a single person for analytic purposes.
\end{sansreflist}
\par
These attributes are intended for use where only a small amount of data is to be encoded in a more or less normalized form, possibly for many person elements, for example when encoding basic facts about respondents to a questionnaire. When however a more detailed encoding is required for all kinds of information about a person, for example in a historical gazetteer, then it will be more appropriate to use the elements \hyperref[TEI.age]{<age>}, \hyperref[TEI.sex]{<sex>} and others described elsewhere in this chapter.\par
Note that the {\itshape age} attribute is not intended to record the person's age expressed in years, months, or other temporal unit. Rather it is intended to record into which age bracket, for the purposes of some analysis, the person falls. A simple (perhaps too simple to be useful) binary classification of age brackets would be child and adult. The actual age brackets useful to various projects are likely to be varied and idiosyncratic, and thus these Guidelines make no particular recommendation as to possible values. Instead, individual projects are recommended to define the values they use in their own customization file, using a declaration like the following: \par\bgroup\index{elementSpec=<elementSpec>|exampleindex}\index{ident=@ident!<elementSpec>|exampleindex}\index{module=@module!<elementSpec>|exampleindex}\index{mode=@mode!<elementSpec>|exampleindex}\index{attList=<attList>|exampleindex}\index{attDef=<attDef>|exampleindex}\index{mode=@mode!<attDef>|exampleindex}\index{ident=@ident!<attDef>|exampleindex}\index{datatype=<datatype>|exampleindex}\index{dataRef=<dataRef>|exampleindex}\index{key=@key!<dataRef>|exampleindex}\index{valList=<valList>|exampleindex}\index{type=@type!<valList>|exampleindex}\index{valItem=<valItem>|exampleindex}\index{ident=@ident!<valItem>|exampleindex}\index{desc=<desc>|exampleindex}\index{valItem=<valItem>|exampleindex}\index{ident=@ident!<valItem>|exampleindex}\index{desc=<desc>|exampleindex}\index{valItem=<valItem>|exampleindex}\index{ident=@ident!<valItem>|exampleindex}\index{desc=<desc>|exampleindex}\exampleFont \begin{shaded}\noindent\mbox{}{<\textbf{elementSpec}\hspace*{1em}{ident}="{person}"\mbox{}\newline 
\hspace*{1em}{module}="{namesdates}"\hspace*{1em}{mode}="{change}">}\mbox{}\newline 
\hspace*{1em}{<\textbf{attList}>}\mbox{}\newline 
\hspace*{1em}\hspace*{1em}{<\textbf{attDef}\hspace*{1em}{mode}="{replace}"\hspace*{1em}{ident}="{age}">}\mbox{}\newline 
\hspace*{1em}\hspace*{1em}\hspace*{1em}{<\textbf{datatype}>}\mbox{}\newline 
\hspace*{1em}\hspace*{1em}\hspace*{1em}\hspace*{1em}{<\textbf{dataRef}\hspace*{1em}{key}="{teidata.enumerated}"/>}\mbox{}\newline 
\hspace*{1em}\hspace*{1em}\hspace*{1em}{</\textbf{datatype}>}\mbox{}\newline 
\hspace*{1em}\hspace*{1em}\hspace*{1em}{<\textbf{valList}\hspace*{1em}{type}="{closed}">}\mbox{}\newline 
\hspace*{1em}\hspace*{1em}\hspace*{1em}\hspace*{1em}{<\textbf{valItem}\hspace*{1em}{ident}="{child}">}\mbox{}\newline 
\hspace*{1em}\hspace*{1em}\hspace*{1em}\hspace*{1em}\hspace*{1em}{<\textbf{desc}>}less than 18 years of age{</\textbf{desc}>}\mbox{}\newline 
\hspace*{1em}\hspace*{1em}\hspace*{1em}\hspace*{1em}{</\textbf{valItem}>}\mbox{}\newline 
\hspace*{1em}\hspace*{1em}\hspace*{1em}\hspace*{1em}{<\textbf{valItem}\hspace*{1em}{ident}="{adult}">}\mbox{}\newline 
\hspace*{1em}\hspace*{1em}\hspace*{1em}\hspace*{1em}\hspace*{1em}{<\textbf{desc}>}18 to 65 years of age{</\textbf{desc}>}\mbox{}\newline 
\hspace*{1em}\hspace*{1em}\hspace*{1em}\hspace*{1em}{</\textbf{valItem}>}\mbox{}\newline 
\hspace*{1em}\hspace*{1em}\hspace*{1em}\hspace*{1em}{<\textbf{valItem}\hspace*{1em}{ident}="{retired}">}\mbox{}\newline 
\hspace*{1em}\hspace*{1em}\hspace*{1em}\hspace*{1em}\hspace*{1em}{<\textbf{desc}>}over 65 years of age{</\textbf{desc}>}\mbox{}\newline 
\hspace*{1em}\hspace*{1em}\hspace*{1em}\hspace*{1em}{</\textbf{valItem}>}\mbox{}\newline 
\hspace*{1em}\hspace*{1em}\hspace*{1em}{</\textbf{valList}>}\mbox{}\newline 
\hspace*{1em}\hspace*{1em}{</\textbf{attDef}>}\mbox{}\newline 
\hspace*{1em}{</\textbf{attList}>}\mbox{}\newline 
{</\textbf{elementSpec}>}\end{shaded}\egroup\par \noindent  The above declaration, were it properly placed in a customization file, establishes that the {\itshape age} attribute of \hyperref[TEI.person]{<person>} has only three possible values, child, adult, and retired. For more information on customization see \textit{\hyperref[MD]{23.3.\ Customization}}.\par
The \hyperref[TEI.person]{<person>} element may contain many sub-elements, each specifying a different property of the person being described. The remainder of this section describes these more specific elements. For convenience, these elements are grouped into three classes, corresponding with the tripartite division outlined above: one for traits, one for states and one for events. Each class may contain specific elements for common types of biographical information, and contains a generic element for other, user-defined, types of information.\par
All the elements in these three classes belong to the attribute class \textsf{att.datable}, which provides the following attributes: 
\begin{sansreflist}
  
\item [\textbf{att.datable.w3c}] provides attributes for normalization of elements that contain datable events conforming to the W3C \textit{XML Schema Part 2: Datatypes Second Edition}.\hfil\\[-10pt]\begin{sansreflist}
    \item[@{\itshape when}]
  supplies the value of the date or time in a standard form, e.g. yyyy-mm-dd.
    \item[@{\itshape notBefore}]
  specifies the earliest possible date for the event in standard form, e.g. yyyy-mm-dd.
    \item[@{\itshape notAfter}]
  specifies the latest possible date for the event in standard form, e.g. yyyy-mm-dd.
    \item[@{\itshape from}]
  indicates the starting point of the period in standard form, e.g. yyyy-mm-dd.
    \item[@{\itshape to}]
  indicates the ending point of the period in standard form, e.g. yyyy-mm-dd.
\end{sansreflist}  
\end{sansreflist}
 as discussed in \textit{\hyperref[NDATTS]{13.1.\ Attribute Classes Defined by This Module}} above.
\paragraph[{Personal Characteristics}]{Personal Characteristics}\label{NDPERSEpc}\par
The \textsf{model.persStateLike} class contains elements describing physical or socially-constructed characteristics, traits, or states of a person. Members of the class comprise the following specific elements:  
\begin{sansreflist}
  
\item [\textbf{<faith>}] (faith) specifies the faith, religion, or belief set of a person.
\item [\textbf{<langKnowledge>}] (language knowledge) summarizes the state of a person's linguistic knowledge, either as prose or by a list of \hyperref[TEI.langKnown]{<langKnown>} elements.
\item [\textbf{<nationality>}] (nationality) contains an informal description of a person's present or past nationality or citizenship.
\item [\textbf{<persPronouns>}] (personal pronouns) indicates the personal pronouns used, or assumed to be used, by the individual being described.
\item [\textbf{<sex>}] (sex) specifies the sex of a person.
\item [\textbf{<age>}] (age) specifies the age of a person.
\item [\textbf{<socecStatus>}] (socio-economic status) contains an informal description of a person's perceived social or economic status.
\item [\textbf{<persName>}] (personal name) contains a proper noun or proper-noun phrase referring to a person, possibly including one or more of the person's forenames, surnames, honorifics, added names, etc.
\item [\textbf{<occupation>}] (occupation) contains an informal description of a person's trade, profession or occupation.
\item [\textbf{<residence>}] (residence) describes a person's present or past places of residence.
\item [\textbf{<affiliation>}] (affiliation) contains an informal description of a person's present or past affiliation with some organization, for example an employer or sponsor.
\item [\textbf{<education>}] (education) contains a description of the educational experience of a person.
\item [\textbf{<floruit>}] (floruit) contains information about a person's period of activity.
\item [\textbf{<persona>}] provides information about one of the personalities identified for a given individual, where an individual has multiple personalities.
\item [\textbf{<state>}] (state) contains a description of some status or quality attributed to a person, place, or organization often at some specific time or for a specific date range.
\item [\textbf{<trait>}] (trait) contains a description of some status or quality attributed to a person, place, or organization typically, but not necessarily, independent of the volition or action of the holder and usually not at some specific time or for a specific date range.
\end{sansreflist}
 All, apart from \hyperref[TEI.langKnowledge]{<langKnowledge>} and \hyperref[TEI.persona]{<persona>}, allow content of ordinary prose containing phrase-level elements. \par\bgroup\index{socecStatus=<socecStatus>|exampleindex}\index{ref=@ref!<socecStatus>|exampleindex}\exampleFont \begin{shaded}\noindent\mbox{}{<\textbf{socecStatus}\hspace*{1em}{ref}="{tag:projectname.org,2012:AB1}">}Status AB1 in the RG Classification scheme{</\textbf{socecStatus}>}\end{shaded}\egroup\par \noindent  \par
The meanings of concepts such as sex, nationality, or age are highly culturally-dependent, and the encoder should take particular care to be explicit about any assumptions underlying their usage of them. For example, when recording personal age in different cultures, there may be different assumptions about the point from which age is reckoned. A statement of the practice adopted in a given encoding may usefully be provided in the \hyperref[TEI.editorialDecl]{<editorialDecl>} element discussed in \textit{\hyperref[HD53]{2.3.3.\ The Editorial Practices Declaration}}.\par
The \hyperref[TEI.langKnowledge]{<langKnowledge>} element contains either paragraphs or a number of \hyperref[TEI.langKnown]{<langKnown>} elements; it may take a {\itshape tags} attribute, which provides one or more standard codes or ‘tag’s for the languages. The \hyperref[TEI.langKnown]{<langKnown>} element must have a {\itshape tag} attribute, which indicates the language with the same kind of ‘language tag’. These ‘language tags’ are discussed in detail in \textit{\hyperref[CHSH]{vi.1\ Language Identification}}.\par
Furthermore, the \hyperref[TEI.langKnown]{<langKnown>} element also has a {\itshape level} attribute to indicate the level of the person's competence in the language. It is thus possible either to say: \par\bgroup\index{langKnowledge=<langKnowledge>|exampleindex}\index{tags=@tags!<langKnowledge>|exampleindex}\index{p=<p>|exampleindex}\exampleFont \begin{shaded}\noindent\mbox{}{<\textbf{langKnowledge}\hspace*{1em}{tags}="{ff fr wo en}">}\mbox{}\newline 
\hspace*{1em}{<\textbf{p}>}Speaks fluent Fulani, Wolof, and French. Some knowledge of\mbox{}\newline 
\hspace*{1em}\hspace*{1em} English.{</\textbf{p}>}\mbox{}\newline 
{</\textbf{langKnowledge}>}\end{shaded}\egroup\par \noindent  or \par\bgroup\index{langKnowledge=<langKnowledge>|exampleindex}\index{langKnown=<langKnown>|exampleindex}\index{level=@level!<langKnown>|exampleindex}\index{tag=@tag!<langKnown>|exampleindex}\index{langKnown=<langKnown>|exampleindex}\index{level=@level!<langKnown>|exampleindex}\index{tag=@tag!<langKnown>|exampleindex}\index{langKnown=<langKnown>|exampleindex}\index{level=@level!<langKnown>|exampleindex}\index{tag=@tag!<langKnown>|exampleindex}\index{langKnown=<langKnown>|exampleindex}\index{level=@level!<langKnown>|exampleindex}\index{tag=@tag!<langKnown>|exampleindex}\exampleFont \begin{shaded}\noindent\mbox{}{<\textbf{langKnowledge}>}\mbox{}\newline 
\hspace*{1em}{<\textbf{langKnown}\hspace*{1em}{level}="{fluent}"\hspace*{1em}{tag}="{ff}">}Fulani{</\textbf{langKnown}>}\mbox{}\newline 
\hspace*{1em}{<\textbf{langKnown}\hspace*{1em}{level}="{fluent}"\hspace*{1em}{tag}="{wo}">}Wolof{</\textbf{langKnown}>}\mbox{}\newline 
\hspace*{1em}{<\textbf{langKnown}\hspace*{1em}{level}="{fluent}"\hspace*{1em}{tag}="{fr}">}French{</\textbf{langKnown}>}\mbox{}\newline 
\hspace*{1em}{<\textbf{langKnown}\hspace*{1em}{level}="{basic}"\hspace*{1em}{tag}="{en}">}English{</\textbf{langKnown}>}\mbox{}\newline 
{</\textbf{langKnowledge}>}\end{shaded}\egroup\par \par
The \hyperref[TEI.persona]{<persona>} element may contain the same component elements as a \hyperref[TEI.person]{<person>} element. Its function is to document a distinct persona assumed by the \hyperref[TEI.person]{<person>} element containing it. A person, not necessarily fictional, may take on different personas at different times or in different situations, each persona having different personal characteristics, such as name, age, sex etc. We distinguish a persona, which is a set of characteristics associated with one specific individual, from a role, which is a set of characteristics that many different people can assume. An actor does not change their persona when adopting a different role, but none of the personas associated with one person can properly be associated with another.\par
The \hyperref[TEI.persPronouns]{<persPronouns>} element may be used to indicate the personal pronouns used, or assumed to be used, by the individual being described. It is common practice in email signatures and biographies, for people to include their preferred personal pronouns along with their name or handle. This allows transgender and gender variant people to express how they wish to be identified, without having to share their gender identity (though some do). Cisgender people have also adopted the practice, which normalizes the idea that a person's personal pronouns should not be inferred by their name, sex, gender, or gender presentation. The \hyperref[TEI.persPronouns]{<persPronouns>} element may be used either in transcribed content to encode a phrase used to indicate preferred personal pronouns, or may be used inside a \hyperref[TEI.person]{<person>} or \hyperref[TEI.persona]{<persona>} element to indicate the associated pronouns.\par
For example, the following entry from a hypothetical prosopography lists only the nominative case of the preferred pronouns as identified by Miss Major Griffin-Gracy, a historical figure.   \par\bgroup\index{person=<person>|exampleindex}\index{persName=<persName>|exampleindex}\index{forename=<forename>|exampleindex}\index{surname=<surname>|exampleindex}\index{birth=<birth>|exampleindex}\index{when=@when!<birth>|exampleindex}\index{sex=<sex>|exampleindex}\index{value=@value!<sex>|exampleindex}\index{evidence=@evidence!<sex>|exampleindex}\index{persPronouns=<persPronouns>|exampleindex}\index{value=@value!<persPronouns>|exampleindex}\index{evidence=@evidence!<persPronouns>|exampleindex}\index{note=<note>|exampleindex}\index{p=<p>|exampleindex}\exampleFont \begin{shaded}\noindent\mbox{}{<\textbf{person}>}\mbox{}\newline 
\hspace*{1em}{<\textbf{persName}>}\mbox{}\newline 
\hspace*{1em}\hspace*{1em}{<\textbf{forename}>}Miss Major{</\textbf{forename}>}\mbox{}\newline 
\hspace*{1em}\hspace*{1em}{<\textbf{surname}>}Griffin-Gracy{</\textbf{surname}>}\mbox{}\newline 
\hspace*{1em}{</\textbf{persName}>}\mbox{}\newline 
\hspace*{1em}{<\textbf{birth}\hspace*{1em}{when}="{1940-10-25}"/>}\mbox{}\newline 
\hspace*{1em}{<\textbf{sex}\hspace*{1em}{value}="{transFemale}"\mbox{}\newline 
\hspace*{1em}\hspace*{1em}{evidence}="{selfIdentification}">}trans woman{</\textbf{sex}>}\mbox{}\newline 
\hspace*{1em}{<\textbf{persPronouns}\hspace*{1em}{value}="{she}"\mbox{}\newline 
\hspace*{1em}\hspace*{1em}{evidence}="{selfIdentification}"/>}\mbox{}\newline 
\hspace*{1em}{<\textbf{note}>}\mbox{}\newline 
\hspace*{1em}\hspace*{1em}{<\textbf{p}>}Veteran of the Stonewall Riots. Founder of the\mbox{}\newline 
\hspace*{1em}\hspace*{1em}\hspace*{1em}\hspace*{1em} Griffin-Gracy Educational Retreat and Historical\mbox{}\newline 
\hspace*{1em}\hspace*{1em}\hspace*{1em}\hspace*{1em} Center (the House of GG). Activist and advocate for\mbox{}\newline 
\hspace*{1em}\hspace*{1em}\hspace*{1em}\hspace*{1em} transgender and gender-nonconforming people of\mbox{}\newline 
\hspace*{1em}\hspace*{1em}\hspace*{1em}\hspace*{1em} color.{</\textbf{p}>}\mbox{}\newline 
\hspace*{1em}{</\textbf{note}>}\mbox{}\newline 
{</\textbf{person}>}\end{shaded}\egroup\par \par
Personal pronouns often occur as part of the closer of an email, post, or other electronic communication.  \par\bgroup\index{div=<div>|exampleindex}\index{type=@type!<div>|exampleindex}\index{opener=<opener>|exampleindex}\index{salute=<salute>|exampleindex}\index{p=<p>|exampleindex}\index{closer=<closer>|exampleindex}\index{lb=<lb>|exampleindex}\index{persName=<persName>|exampleindex}\index{lb=<lb>|exampleindex}\index{roleName=<roleName>|exampleindex}\index{lb=<lb>|exampleindex}\index{roleName=<roleName>|exampleindex}\index{lb=<lb>|exampleindex}\index{orgName=<orgName>|exampleindex}\index{lb=<lb>|exampleindex}\index{email=<email>|exampleindex}\index{lb=<lb>|exampleindex}\index{persPronouns=<persPronouns>|exampleindex}\index{lb=<lb>|exampleindex}\index{lb=<lb>|exampleindex}\index{roleName=<roleName>|exampleindex}\index{lb=<lb>|exampleindex}\index{orgName=<orgName>|exampleindex}\index{orgName=<orgName>|exampleindex}\index{lb=<lb>|exampleindex}\index{roleName=<roleName>|exampleindex}\index{orgName=<orgName>|exampleindex}\exampleFont \begin{shaded}\noindent\mbox{}{<\textbf{div}\hspace*{1em}{type}="{email}">}\mbox{}\newline 
\hspace*{1em}{<\textbf{opener}>}\mbox{}\newline 
\hspace*{1em}\hspace*{1em}{<\textbf{salute}>}Dear all,{</\textbf{salute}>}\mbox{}\newline 
\hspace*{1em}{</\textbf{opener}>}\mbox{}\newline 
\hspace*{1em}{<\textbf{p}>}With apologies for length. I'm expanding a schema …{</\textbf{p}>}\mbox{}\newline 
\textit{<!-- ... -->}\mbox{}\newline 
\hspace*{1em}{<\textbf{closer}>}\mbox{}\newline 
\hspace*{1em}\hspace*{1em}{<\textbf{lb}/>}\mbox{}\newline 
\hspace*{1em}\hspace*{1em}{<\textbf{persName}>}Diane Jakacki, Ph.D.{</\textbf{persName}>}\mbox{}\newline 
\hspace*{1em}\hspace*{1em}{<\textbf{lb}/>}\mbox{}\newline 
\hspace*{1em}\hspace*{1em}{<\textbf{roleName}>}Digital Scholarship Coordinator{</\textbf{roleName}>}\mbox{}\newline 
\hspace*{1em}\hspace*{1em}{<\textbf{lb}/>}\mbox{}\newline 
\hspace*{1em}\hspace*{1em}{<\textbf{roleName}>}Affiliate Faculty in Comparative \& Digital Humanities{</\textbf{roleName}>}\mbox{}\newline 
\hspace*{1em}\hspace*{1em}{<\textbf{lb}/>}\mbox{}\newline 
\hspace*{1em}\hspace*{1em}{<\textbf{orgName}>}Bucknell University{</\textbf{orgName}>}\mbox{}\newline 
\hspace*{1em}\hspace*{1em}{<\textbf{lb}/>}\mbox{}\newline 
\hspace*{1em}\hspace*{1em}{<\textbf{email}>}d…@….edu{</\textbf{email}>}\mbox{}\newline 
\hspace*{1em}\hspace*{1em}{<\textbf{lb}/>}({<\textbf{persPronouns}>}she/her/hers{</\textbf{persPronouns}>})\mbox{}\newline 
\hspace*{1em}{<\textbf{lb}/>}\mbox{}\newline 
\hspace*{1em}\hspace*{1em}{<\textbf{lb}/>}\mbox{}\newline 
\hspace*{1em}\hspace*{1em}{<\textbf{roleName}>}Principal Investigator{</\textbf{roleName}>},\mbox{}\newline 
\hspace*{1em}{<\textbf{lb}/>}\mbox{}\newline 
\hspace*{1em}\hspace*{1em}{<\textbf{orgName}>}LAB Cooperative{</\textbf{orgName}>} and {<\textbf{orgName}>}REED London Online{</\textbf{orgName}>}\mbox{}\newline 
\hspace*{1em}\hspace*{1em}{<\textbf{lb}/>}\mbox{}\newline 
\hspace*{1em}\hspace*{1em}{<\textbf{roleName}>}Chair{</\textbf{roleName}>}, {<\textbf{orgName}>}ADHO Conference Coordinating Committee{</\textbf{orgName}>}\mbox{}\newline 
\hspace*{1em}{</\textbf{closer}>}\mbox{}\newline 
{</\textbf{div}>}\end{shaded}\egroup\par \par
The \hyperref[TEI.sex]{<sex>} element carries a {\itshape value} attribute to give values from a project-internal taxonomy, or an external standard, such as vCard's sex property \url{http://microformats.org/wiki/gender-formats} (in which M indicates male, F indicates female, O indicates other, N indicates none or not applicable, U indicates unknown) or the often used ISO 5218:2004 \textit{Representation of Human Sexes} \url{http://standards.iso.org/ittf/PubliclyAvailableStandards/c036266\textunderscore ISO\textunderscore IEC\textunderscore 5218\textunderscore 2004(E\textunderscore F).zip} (in which 0 indicates unknown; 1 indicates male; 2 indicates female; and 9 indicates not applicable, although the ISO standard is widely considered inadequate). \par\bgroup\index{sex=<sex>|exampleindex}\index{value=@value!<sex>|exampleindex}\exampleFont \begin{shaded}\noindent\mbox{}{<\textbf{sex}\hspace*{1em}{value}="{F}">}female{</\textbf{sex}>}\end{shaded}\egroup\par \noindent  As elsewhere, these coded values may be used as an alternative to or normalization of the actual descriptive text contained in the element. The previous example might equally well be given as \par\bgroup\index{sex=<sex>|exampleindex}\index{value=@value!<sex>|exampleindex}\exampleFont \begin{shaded}\noindent\mbox{}{<\textbf{sex}\hspace*{1em}{value}="{F}"/>}\end{shaded}\egroup\par \par
The generic \hyperref[TEI.trait]{<trait>} and \hyperref[TEI.state]{<state>} elements are also members of this class, 
\begin{sansreflist}
  
\item [\textbf{<trait>}] (trait) contains a description of some status or quality attributed to a person, place, or organization typically, but not necessarily, independent of the volition or action of the holder and usually not at some specific time or for a specific date range.
\item [\textbf{<state>}] (state) contains a description of some status or quality attributed to a person, place, or organization often at some specific time or for a specific date range.
\end{sansreflist}
 These element can be used to extend the range of information supplied about an individual's personal characteristics. Either may contain an optional \hyperref[TEI.label]{<label>} element, used to provide a human-readable specification for the characteristic concerned and a description of the feature itself supplied within a \hyperref[TEI.desc]{<desc>} element. These may be followed by or one or more \hyperref[TEI.p]{<p>} elements supplying more detailed information about the trait. In either case, these may be followed by one or more notes or bibliographical references. The {\itshape type}, {\itshape ref}, and {\itshape key} attributes may be used to indicate a fuller definition of the combination of feature and value. \par\bgroup\index{trait=<trait>|exampleindex}\index{type=@type!<trait>|exampleindex}\index{key=@key!<trait>|exampleindex}\index{label=<label>|exampleindex}\index{desc=<desc>|exampleindex}\exampleFont \begin{shaded}\noindent\mbox{}{<\textbf{trait}\hspace*{1em}{type}="{ethnicity}"\hspace*{1em}{key}="{alb}">}\mbox{}\newline 
\hspace*{1em}{<\textbf{label}>}Ethnicity{</\textbf{label}>}\mbox{}\newline 
\hspace*{1em}{<\textbf{desc}>}Ethnic Albanian.{</\textbf{desc}>}\mbox{}\newline 
{</\textbf{trait}>}\end{shaded}\egroup\par \par
These elements are provided as a simple means of extending the set of descriptive features available in a standardized way. For example, there are no predefined elements for such features as eye or hair colour. If these are to be recorded, they may simply be added as new types of trait: \par\bgroup\index{trait=<trait>|exampleindex}\index{type=@type!<trait>|exampleindex}\index{label=<label>|exampleindex}\index{desc=<desc>|exampleindex}\index{trait=<trait>|exampleindex}\index{type=@type!<trait>|exampleindex}\index{label=<label>|exampleindex}\index{desc=<desc>|exampleindex}\exampleFont \begin{shaded}\noindent\mbox{}{<\textbf{trait}\hspace*{1em}{type}="{physical}">}\mbox{}\newline 
\hspace*{1em}{<\textbf{label}>}eye colour{</\textbf{label}>}\mbox{}\newline 
\hspace*{1em}{<\textbf{desc}>}blue{</\textbf{desc}>}\mbox{}\newline 
{</\textbf{trait}>}\mbox{}\newline 
{<\textbf{trait}\hspace*{1em}{type}="{physical}">}\mbox{}\newline 
\hspace*{1em}{<\textbf{label}>}hair colour{</\textbf{label}>}\mbox{}\newline 
\hspace*{1em}{<\textbf{desc}>}brown{</\textbf{desc}>}\mbox{}\newline 
{</\textbf{trait}>}\end{shaded}\egroup\par \par
If none of the more specialized elements listed above is appropriate, then a choice must be made between the two generic elements \hyperref[TEI.trait]{<trait>} and \hyperref[TEI.state]{<state>}. If you wish to distinguish between characteristics that are generally perceived to be transient and those which are generally considered unchanging, use \hyperref[TEI.state]{<state>} for the former, and \hyperref[TEI.trait]{<trait>} for the latter. It may also be helpful to note that traits are typically, but not necessarily, independent of the volition or action of the holder. If the distinction between state and trait is not considered relevant or useful, use \hyperref[TEI.state]{<state>}.\par
The \hyperref[TEI.persName]{<persName>} element is repeatable and can, like all TEI elements, take the attribute {\itshape xml:lang} to indicate the language of the content of the element, as well as a {\itshape type} attribute to indicate the type of name, whether a nickname, maiden or birth name, alternative form, etc. This is useful in cases where, for example, a person is known by a Latin name and also by any number of vernacular names, many or all of which may have claims to ‘authenticity’. In order to ensure uniformity, the method generally employed in the library world has been to accept the form found in some authority file, for example that of the American Library of Congress, as the ‘base’ or ‘neutral’ form. Feelings can run high on this matter, however, and people are often reluctant to accept as ‘neutral’ an overtly foreign form of the name of their local saint or hero. Within the \hyperref[TEI.person]{<person>} element any number of variant forms of a name can be given, with no prioritization, and hence less likelihood of offence. The Icelandic scholar and manuscript collector Árni Magnússon, to give his name in standard modern Icelandic spelling, is known in Danish as Arne Magnusson, the form which he himself, as a long term resident of Denmark, generally used; there is also a Latinized form, Arnas Magnæus, which he used in his scholarly writings. All three forms can be given, and in any order: \par\bgroup\index{person=<person>|exampleindex}\index{persName=<persName>|exampleindex}\index{persName=<persName>|exampleindex}\index{persName=<persName>|exampleindex}\exampleFont \begin{shaded}\noindent\mbox{}{<\textbf{person}\hspace*{1em}{xml:id}="{ArnMag}">}\mbox{}\newline 
\hspace*{1em}{<\textbf{persName}\hspace*{1em}{xml:lang}="{is}">}Árni Magnússon{</\textbf{persName}>}\mbox{}\newline 
\hspace*{1em}{<\textbf{persName}\hspace*{1em}{xml:lang}="{da}">}Arne Magnusson{</\textbf{persName}>}\mbox{}\newline 
\hspace*{1em}{<\textbf{persName}\hspace*{1em}{xml:lang}="{la}">}Arnas Magnæus{</\textbf{persName}>}\mbox{}\newline 
{</\textbf{person}>}\end{shaded}\egroup\par \par
At the other extreme, a person may be named periphrastically as in the following example: \par\bgroup\index{person=<person>|exampleindex}\index{persName=<persName>|exampleindex}\index{residence=<residence>|exampleindex}\index{placeName=<placeName>|exampleindex}\index{region=<region>|exampleindex}\index{floruit=<floruit>|exampleindex}\index{notBefore=@notBefore!<floruit>|exampleindex}\index{notAfter=@notAfter!<floruit>|exampleindex}\exampleFont \begin{shaded}\noindent\mbox{}{<\textbf{person}\hspace*{1em}{xml:id}="{simon\textunderscore son\textunderscore of\textunderscore richard2}">}\mbox{}\newline 
\hspace*{1em}{<\textbf{persName}>}Simon, son of Richard{</\textbf{persName}>}\mbox{}\newline 
\hspace*{1em}{<\textbf{residence}>}\mbox{}\newline 
\hspace*{1em}\hspace*{1em}{<\textbf{placeName}>}\mbox{}\newline 
\hspace*{1em}\hspace*{1em}\hspace*{1em}{<\textbf{region}>}Essex{</\textbf{region}>}\mbox{}\newline 
\hspace*{1em}\hspace*{1em}{</\textbf{placeName}>}\mbox{}\newline 
\hspace*{1em}{</\textbf{residence}>}\mbox{}\newline 
\hspace*{1em}{<\textbf{floruit}\hspace*{1em}{notBefore}="{1219}"\hspace*{1em}{notAfter}="{1223}">}1219-1223{</\textbf{floruit}>}\mbox{}\newline 
{</\textbf{person}>}\end{shaded}\egroup\par \par
Alternatively, the generic \hyperref[TEI.name]{<name>} element may be used for all of the naming components in a description. For example, a description of the first living held by the Icelandic clergyman and poet Jón Oddsson Hjaltalín might be tagged as follows: \par\bgroup\index{state=<state>|exampleindex}\index{type=@type!<state>|exampleindex}\index{from=@from!<state>|exampleindex}\index{to=@to!<state>|exampleindex}\index{p=<p>|exampleindex}\index{name=<name>|exampleindex}\index{type=@type!<name>|exampleindex}\index{name=<name>|exampleindex}\index{type=@type!<name>|exampleindex}\index{q=<q>|exampleindex}\index{name=<name>|exampleindex}\index{type=@type!<name>|exampleindex}\index{ref=@ref!<name>|exampleindex}\index{name=<name>|exampleindex}\index{type=@type!<name>|exampleindex}\index{name=<name>|exampleindex}\index{type=@type!<name>|exampleindex}\index{bibl=<bibl>|exampleindex}\index{bibl=<bibl>|exampleindex}\exampleFont \begin{shaded}\noindent\mbox{}{<\textbf{state}\hspace*{1em}{type}="{office}"\hspace*{1em}{from}="{1777-04-07}"\mbox{}\newline 
\hspace*{1em}{to}="{1780-07-12}">}\mbox{}\newline 
\hspace*{1em}{<\textbf{p}>}Jón's first living — which he apparently accepted rather reluctantly — was at {<\textbf{name}\hspace*{1em}{type}="{place}">}Háls í\mbox{}\newline 
\hspace*{1em}\hspace*{1em}\hspace*{1em}\hspace*{1em} Hamarsfirði{</\textbf{name}>}, {<\textbf{name}\hspace*{1em}{type}="{place}">}Múlasýsla{</\textbf{name}>}, to which he was presented on 7 April 1777. He was\mbox{}\newline 
\hspace*{1em}\hspace*{1em} ordained the following month and spent three years at Háls, but was never happy there, due largely to the general\mbox{}\newline 
\hspace*{1em}\hspace*{1em} penury in which he was forced to live — a recurrent theme throughout the early part of his life. In June of 1780\mbox{}\newline 
\hspace*{1em}\hspace*{1em} the bishop recommended that Jón should {<\textbf{q}\hspace*{1em}{xml:lang}="{da}">}promoveres til andet bedre kald, end det hand hidindtil\mbox{}\newline 
\hspace*{1em}\hspace*{1em}\hspace*{1em}\hspace*{1em} har havt{</\textbf{q}>}, and on 12 July it was agreed that he should exchange livings with {<\textbf{name}\hspace*{1em}{type}="{person}"\mbox{}\newline 
\hspace*{1em}\hspace*{1em}\hspace*{1em}{ref}="{tag:projectname.org,2012:ThorJon}">}sr. Þórður Jónsson{</\textbf{name}>} at {<\textbf{name}\hspace*{1em}{type}="{place}">}Kálfafell á Síðu{</\textbf{name}>},\mbox{}\newline 
\hspace*{1em}{<\textbf{name}\hspace*{1em}{type}="{place}">}Skaftafellssýsla{</\textbf{name}>}.{</\textbf{p}>}\mbox{}\newline 
\hspace*{1em}{<\textbf{bibl}>}ÞÍ, Stms I.15, p. 733.{</\textbf{bibl}>}\mbox{}\newline 
\hspace*{1em}{<\textbf{bibl}>}ÞÍ, Stms I.17, p.\mbox{}\newline 
\hspace*{1em}\hspace*{1em} 102.{</\textbf{bibl}>}\mbox{}\newline 
{</\textbf{state}>}\end{shaded}\egroup\par \par
Similarly, the generic \hyperref[TEI.state]{<state>} or \hyperref[TEI.trait]{<trait>} element may be used in preference to the more specific elements listed above: \par\bgroup\index{state=<state>|exampleindex}\index{type=@type!<state>|exampleindex}\index{notBefore=@notBefore!<state>|exampleindex}\index{label=<label>|exampleindex}\index{desc=<desc>|exampleindex}\exampleFont \begin{shaded}\noindent\mbox{}{<\textbf{state}\hspace*{1em}{type}="{nationality}"\mbox{}\newline 
\hspace*{1em}{notBefore}="{2002-01-15}">}\mbox{}\newline 
\hspace*{1em}{<\textbf{label}>}Nationality{</\textbf{label}>}\mbox{}\newline 
\hspace*{1em}{<\textbf{desc}>}American citizen from 15 January 2002.{</\textbf{desc}>}\mbox{}\newline 
{</\textbf{state}>}\end{shaded}\egroup\par \noindent  is the same as: \par\bgroup\index{nationality=<nationality>|exampleindex}\index{notBefore=@notBefore!<nationality>|exampleindex}\exampleFont \begin{shaded}\noindent\mbox{}{<\textbf{nationality}\hspace*{1em}{notBefore}="{2002-01-15}">}American citizen from 15 January 2002.{</\textbf{nationality}>}\end{shaded}\egroup\par \noindent  or even: \par\bgroup\index{nationality=<nationality>|exampleindex}\index{notBefore=@notBefore!<nationality>|exampleindex}\index{key=@key!<nationality>|exampleindex}\exampleFont \begin{shaded}\noindent\mbox{}{<\textbf{nationality}\hspace*{1em}{notBefore}="{2002-01-15}"\mbox{}\newline 
\hspace*{1em}{key}="{US}"/>}\end{shaded}\egroup\par 
\paragraph[{Personal Events}]{Personal Events}\label{NDPERSEpe}\par
Events in a person's history are not characteristics of an individual, but often cause an individual to gain such characteristics, or to enter a new state. Most such events, for example marriage, appointment, promotion, or a journey may be recorded using the generic element \hyperref[TEI.event]{<event>}, which may be grouped with \hyperref[TEI.listEvent]{<listEvent>}, and has a content model similar to that of \hyperref[TEI.state]{<state>} and \hyperref[TEI.trait]{<trait>}. The chief difference is that \hyperref[TEI.event]{<event>} can include a \hyperref[TEI.placeName]{<placeName>} element to identify the name of the place where the event occurred.\par
Two particular events in a persons life, namely birth and death, are both ubiquitous and usually considered particularly important, and thus may be represented by specialized elements for the purpose: 
\begin{sansreflist}
  
\item [\textbf{<birth>}] (birth) contains information about a person's birth, such as its date and place.
\item [\textbf{<death>}] (death) contains information about a person's death, such as its date and place.
\end{sansreflist}
\par
In the following example, we give a brief summary of the wedding of Jane Burden to the English writer, designer, and socialist William Morris, encoded as an \hyperref[TEI.event]{<event>} element embedded within the \hyperref[TEI.person]{<person>} element used to record data about Morris, though we could equally well have embedded the \hyperref[TEI.event]{<event>} element within the \hyperref[TEI.person]{<person>} element for Burden, or have encoded it independently of either \hyperref[TEI.person]{<person>} element: \par\bgroup\index{person=<person>|exampleindex}\index{event=<event>|exampleindex}\index{type=@type!<event>|exampleindex}\index{when=@when!<event>|exampleindex}\index{label=<label>|exampleindex}\index{desc=<desc>|exampleindex}\index{name=<name>|exampleindex}\index{type=@type!<name>|exampleindex}\index{ref=@ref!<name>|exampleindex}\index{name=<name>|exampleindex}\index{type=@type!<name>|exampleindex}\index{ref=@ref!<name>|exampleindex}\index{name=<name>|exampleindex}\index{type=@type!<name>|exampleindex}\index{date=<date>|exampleindex}\index{when=@when!<date>|exampleindex}\index{name=<name>|exampleindex}\index{type=@type!<name>|exampleindex}\index{ref=@ref!<name>|exampleindex}\index{name=<name>|exampleindex}\index{type=@type!<name>|exampleindex}\index{ref=@ref!<name>|exampleindex}\index{name=<name>|exampleindex}\index{type=@type!<name>|exampleindex}\index{ref=@ref!<name>|exampleindex}\index{name=<name>|exampleindex}\index{type=@type!<name>|exampleindex}\index{ref=@ref!<name>|exampleindex}\index{name=<name>|exampleindex}\index{type=@type!<name>|exampleindex}\index{ref=@ref!<name>|exampleindex}\index{quote=<quote>|exampleindex}\index{said=<said>|exampleindex}\index{quote=<quote>|exampleindex}\index{bibl=<bibl>|exampleindex}\index{title=<title>|exampleindex}\index{person=<person>|exampleindex}\index{persName=<persName>|exampleindex}\index{person=<person>|exampleindex}\index{persName=<persName>|exampleindex}\index{person=<person>|exampleindex}\index{persName=<persName>|exampleindex}\index{person=<person>|exampleindex}\index{persName=<persName>|exampleindex}\index{forename=<forename>|exampleindex}\index{surname=<surname>|exampleindex}\index{person=<person>|exampleindex}\index{persName=<persName>|exampleindex}\exampleFont \begin{shaded}\noindent\mbox{}{<\textbf{person}\hspace*{1em}{xml:id}="{WM}">}\mbox{}\newline 
\textit{<!-- ... -->}\mbox{}\newline 
\hspace*{1em}{<\textbf{event}\hspace*{1em}{type}="{marriage}"\hspace*{1em}{when}="{1859-04-26}">}\mbox{}\newline 
\hspace*{1em}\hspace*{1em}{<\textbf{label}>}Marriage{</\textbf{label}>}\mbox{}\newline 
\hspace*{1em}\hspace*{1em}{<\textbf{desc}>}\mbox{}\newline 
\hspace*{1em}\hspace*{1em}\hspace*{1em}{<\textbf{name}\hspace*{1em}{type}="{person}"\hspace*{1em}{ref}="{\#WM}">}William Morris{</\textbf{name}>} and {<\textbf{name}\hspace*{1em}{type}="{person}"\mbox{}\newline 
\hspace*{1em}\hspace*{1em}\hspace*{1em}\hspace*{1em}{ref}="{http://en.wikipedia.org/wiki/Jane\textunderscore Burden}">}Jane Burden{</\textbf{name}>} were married at {<\textbf{name}\hspace*{1em}{type}="{place}">}St\mbox{}\newline 
\hspace*{1em}\hspace*{1em}\hspace*{1em}\hspace*{1em}\hspace*{1em}\hspace*{1em} Michael's Church, Ship Street, Oxford{</\textbf{name}>} on {<\textbf{date}\hspace*{1em}{when}="{1859-04-26}">}26 April 1859{</\textbf{date}>}. The wedding was\mbox{}\newline 
\hspace*{1em}\hspace*{1em}\hspace*{1em}\hspace*{1em} conducted by Morris's friend {<\textbf{name}\hspace*{1em}{type}="{person}"\hspace*{1em}{ref}="{\#RWD}">}R. W. Dixon{</\textbf{name}>} with {<\textbf{name}\hspace*{1em}{type}="{person}"\hspace*{1em}{ref}="{\#CBF}">}Charles Faulkner{</\textbf{name}>} as the best man. The bride was given away by her father, {<\textbf{name}\hspace*{1em}{type}="{person}"\hspace*{1em}{ref}="{\#RB}">}Robert Burden{</\textbf{name}>}. According to the account that {<\textbf{name}\hspace*{1em}{type}="{person}"\mbox{}\newline 
\hspace*{1em}\hspace*{1em}\hspace*{1em}\hspace*{1em}{ref}="{http://en.wikipedia.org/wiki/Edward\textunderscore Burne-Jones}">}Burne-Jones{</\textbf{name}>} gave {<\textbf{name}\hspace*{1em}{type}="{person}"\hspace*{1em}{ref}="{\#JWM}">}Mackail{</\textbf{name}>}\mbox{}\newline 
\hspace*{1em}\hspace*{1em}\hspace*{1em}{<\textbf{quote}>}M. said to Dixon beforehand {<\textbf{said}>}Mind you don't call her Mary{</\textbf{said}>} but he did{</\textbf{quote}>}. The entry in\mbox{}\newline 
\hspace*{1em}\hspace*{1em}\hspace*{1em}\hspace*{1em} the Register reads: {<\textbf{quote}>}William Morris, 25, Bachelor Gentleman, 13 George Street, son of William Morris decd.\mbox{}\newline 
\hspace*{1em}\hspace*{1em}\hspace*{1em}\hspace*{1em}\hspace*{1em}\hspace*{1em} Gentleman. Jane Burden, minor, spinster, 65 Holywell Street, d. of Robert Burden, Groom.{</\textbf{quote}>} The witnesses\mbox{}\newline 
\hspace*{1em}\hspace*{1em}\hspace*{1em}\hspace*{1em} were Jane's parents and Faulkner. None of Morris's family attended the ceremony. Morris presented Jane with a\mbox{}\newline 
\hspace*{1em}\hspace*{1em}\hspace*{1em}\hspace*{1em} plain gold ring bearing the London hallmark for 1858. She gave her husband a double-handled antique silver\mbox{}\newline 
\hspace*{1em}\hspace*{1em}\hspace*{1em}\hspace*{1em} cup.{</\textbf{desc}>}\mbox{}\newline 
\hspace*{1em}\hspace*{1em}{<\textbf{bibl}>}J. W. Mackail, {<\textbf{title}>}The Life of William Morris{</\textbf{title}>}, 1899.{</\textbf{bibl}>}\mbox{}\newline 
\hspace*{1em}{</\textbf{event}>}\mbox{}\newline 
{</\textbf{person}>}\mbox{}\newline 
{<\textbf{person}\hspace*{1em}{xml:id}="{RB}">}\mbox{}\newline 
\hspace*{1em}{<\textbf{persName}>}Robert Burden{</\textbf{persName}>}\mbox{}\newline 
{</\textbf{person}>}\mbox{}\newline 
{<\textbf{person}\hspace*{1em}{xml:id}="{RWD}">}\mbox{}\newline 
\hspace*{1em}{<\textbf{persName}>}R.W. Dixon{</\textbf{persName}>}\mbox{}\newline 
{</\textbf{person}>}\mbox{}\newline 
{<\textbf{person}\hspace*{1em}{xml:id}="{CBF}">}\mbox{}\newline 
\hspace*{1em}{<\textbf{persName}>}Charles Faulkner{</\textbf{persName}>}\mbox{}\newline 
{</\textbf{person}>}\mbox{}\newline 
{<\textbf{person}\hspace*{1em}{xml:id}="{EBJ}">}\mbox{}\newline 
\hspace*{1em}{<\textbf{persName}>}\mbox{}\newline 
\hspace*{1em}\hspace*{1em}{<\textbf{forename}>}Edward{</\textbf{forename}>}\mbox{}\newline 
\hspace*{1em}\hspace*{1em}{<\textbf{surname}>}Burne-Jones{</\textbf{surname}>}\mbox{}\newline 
\hspace*{1em}{</\textbf{persName}>}\mbox{}\newline 
{</\textbf{person}>}\mbox{}\newline 
{<\textbf{person}\hspace*{1em}{xml:id}="{JWM}">}\mbox{}\newline 
\hspace*{1em}{<\textbf{persName}>}J.W. Mackail{</\textbf{persName}>}\mbox{}\newline 
{</\textbf{person}>}\end{shaded}\egroup\par \noindent  In this example the {\itshape ref} attributes on the various \hyperref[TEI.name]{<name>} elements point either to an external source or to a \hyperref[TEI.person]{<person>} element within which other information about the person named may be found. As further discussed below (\textit{\hyperref[NDPERSREL]{13.3.2.3.\ Personal Relationships}}), a \hyperref[TEI.relation]{<relation>} element may then be used to link them in a more meaningful way: \par\bgroup\index{relation=<relation>|exampleindex}\index{name=@name!<relation>|exampleindex}\index{mutual=@mutual!<relation>|exampleindex}\index{relation=<relation>|exampleindex}\index{name=@name!<relation>|exampleindex}\index{mutual=@mutual!<relation>|exampleindex}\index{relation=<relation>|exampleindex}\index{name=@name!<relation>|exampleindex}\index{active=@active!<relation>|exampleindex}\index{passive=@passive!<relation>|exampleindex}\exampleFont \begin{shaded}\noindent\mbox{}{<\textbf{relation}\hspace*{1em}{name}="{spouse}"\hspace*{1em}{mutual}="{\#WM \#JBM}"/>}\mbox{}\newline 
{<\textbf{relation}\hspace*{1em}{name}="{friend}"\hspace*{1em}{mutual}="{\#WM \#RWD}"/>}\mbox{}\newline 
{<\textbf{relation}\hspace*{1em}{name}="{parent}"\hspace*{1em}{active}="{\#RB}"\mbox{}\newline 
\hspace*{1em}{passive}="{\#JBM}"/>}\end{shaded}\egroup\par \par
As mentioned above, all these elements, both the specific and the generic, are members of the \textsf{att.datable} attribute class, which means they can be limited in terms of time. The following encoding, for example, demonstrates that the person named David Jones changed his name in 1966 to David Bowie: \par\bgroup\index{person=<person>|exampleindex}\index{persName=<persName>|exampleindex}\index{notAfter=@notAfter!<persName>|exampleindex}\index{persName=<persName>|exampleindex}\index{notBefore=@notBefore!<persName>|exampleindex}\exampleFont \begin{shaded}\noindent\mbox{}{<\textbf{person}\hspace*{1em}{xml:id}="{DB}">}\mbox{}\newline 
\hspace*{1em}{<\textbf{persName}\hspace*{1em}{notAfter}="{1966}">}David Jones{</\textbf{persName}>}\mbox{}\newline 
\hspace*{1em}{<\textbf{persName}\hspace*{1em}{notBefore}="{1966}">}David Bowie{</\textbf{persName}>}\mbox{}\newline 
{</\textbf{person}>}\end{shaded}\egroup\par \par
All the generic elements are also members of the \textsf{att.global.responsibility} and \textsf{att.editLike} classes. These classes make available the attributes {\itshape cert}, to indicate the degree of certainty, {\itshape resp}, the agency responsible, {\itshape evidence}, the nature of the evidence used, and {\itshape source}, a pointer to a resource from which the information derives. In this way it is possible, in the case of multiple and conflicting sources, to provide more than one view of what happened, as in the following example: \par\bgroup\index{event=<event>|exampleindex}\index{type=@type!<event>|exampleindex}\index{resp=@resp!<event>|exampleindex}\index{cert=@cert!<event>|exampleindex}\index{p=<p>|exampleindex}\index{name=<name>|exampleindex}\index{type=@type!<name>|exampleindex}\index{event=<event>|exampleindex}\index{type=@type!<event>|exampleindex}\index{resp=@resp!<event>|exampleindex}\index{cert=@cert!<event>|exampleindex}\index{p=<p>|exampleindex}\index{name=<name>|exampleindex}\index{type=@type!<name>|exampleindex}\exampleFont \begin{shaded}\noindent\mbox{}{<\textbf{event}\hspace*{1em}{type}="{birth}"\hspace*{1em}{resp}="{\#XYZ}"\hspace*{1em}{cert}="{high}">}\mbox{}\newline 
\hspace*{1em}{<\textbf{p}>}Born in {<\textbf{name}\hspace*{1em}{type}="{place}">}Brixton{</\textbf{name}>} on 8 January\mbox{}\newline 
\hspace*{1em}\hspace*{1em} 1947.{</\textbf{p}>}\mbox{}\newline 
{</\textbf{event}>}\mbox{}\newline 
{<\textbf{event}\hspace*{1em}{type}="{birth}"\hspace*{1em}{resp}="{\#ABC}"\hspace*{1em}{cert}="{low}">}\mbox{}\newline 
\hspace*{1em}{<\textbf{p}>}Born in {<\textbf{name}\hspace*{1em}{type}="{place}">}Berkhamsted{</\textbf{name}>} on 9 January\mbox{}\newline 
\hspace*{1em}\hspace*{1em} 1947.{</\textbf{p}>}\mbox{}\newline 
{</\textbf{event}>}\end{shaded}\egroup\par 
\paragraph[{Personal Relationships}]{Personal Relationships}\label{NDPERSREL}\par
When the module defined by this chapter is included in a schema, the following two elements may be used to document relationships amongst the persons, places, or organizations identified: 
\begin{sansreflist}
  
\item [\textbf{<listRelation>}] provides information about relationships identified amongst people, places, and organizations, either informally as prose or as formally expressed relation links.
\item [\textbf{<relation>}] (relationship) describes any kind of relationship or linkage amongst a specified group of places, events, persons, objects or other items.\hfil\\[-10pt]\begin{sansreflist}
    \item[@{\itshape name}]
  supplies a name for the kind of relationship of which this is an instance.
    \item[@{\itshape active}]
  identifies the ‘active’ participants in a non-mutual relationship, or all the participants in a mutual one.
    \item[@{\itshape mutual}]
  supplies a list of participants amongst all of whom the relationship holds equally.
    \item[@{\itshape passive}]
  identifies the ‘passive’ participants in a non-mutual relationship.
\end{sansreflist}  
\end{sansreflist}
 These elements are both members of the \textsf{att.typed} class, from which they inherit the {\itshape type} and {\itshape subtype} attributes in the usual way. The value specified for either attribute on a \hyperref[TEI.listRelation]{<listRelation>} element is implicitly applicable to all of its child \hyperref[TEI.relation]{<relation>} elements, unless overridden.\par
A \textit{relationship}, as defined here, may be any kind of describable link between specified participants. A participant (in this sense) might be a person, a place, or an organization. In the case of persons, therefore, a relationship might be a social relationship (such as employer/employee), a personal relationship (such as sibling, spouse, etc.) or something less precise such as ‘possessing shared knowledge’. A relationship may be \textit{mutual}, in that all the participants engage in it on an equal footing (for example the ‘sibling’ relationship); or it may not be if participants are not identical with respect to their role in the relationship (for example, the ‘employer’ relationship). For non-mutual relationships, only two kinds of role are currently supported; they are named \textit{active} and \textit{passive}. These names are chosen to reflect the fact that non-mutual relations are \textit{directed}, in the sense that they are most readily described by a transitive verb, or a verb phrase of the form \textit{is X of} or \textit{is X to}. The subject of the verb is classed as \textit{active}; the direct object of the verb, or the object of the concluding preposition, as \textit{passive}. Thus parents are ‘active’ and children ‘passive’ in the relationship ‘parent’ (interpreted as \textit{is parent of}); the employer is ‘active’, the employee ‘passive’, in the relationship \textit{employs}. These relationships can be inverted: parents are ‘passive’ and children ‘active’ in the relationship \textit{is child of}; similarly ‘works for’ inverts the active and passive roles of ‘employs’.\par
For example: \par\bgroup\index{listRelation=<listRelation>|exampleindex}\index{relation=<relation>|exampleindex}\index{name=@name!<relation>|exampleindex}\index{active=@active!<relation>|exampleindex}\index{passive=@passive!<relation>|exampleindex}\index{relation=<relation>|exampleindex}\index{name=@name!<relation>|exampleindex}\index{mutual=@mutual!<relation>|exampleindex}\index{relation=<relation>|exampleindex}\index{type=@type!<relation>|exampleindex}\index{name=@name!<relation>|exampleindex}\index{active=@active!<relation>|exampleindex}\index{passive=@passive!<relation>|exampleindex}\exampleFont \begin{shaded}\noindent\mbox{}{<\textbf{listRelation}>}\mbox{}\newline 
\hspace*{1em}{<\textbf{relation}\hspace*{1em}{name}="{parent}"\hspace*{1em}{active}="{\#P1 \#P2}"\mbox{}\newline 
\hspace*{1em}\hspace*{1em}{passive}="{\#P3 \#P4}"/>}\mbox{}\newline 
\hspace*{1em}{<\textbf{relation}\hspace*{1em}{name}="{spouse}"\hspace*{1em}{mutual}="{\#P1 \#P2}"/>}\mbox{}\newline 
\hspace*{1em}{<\textbf{relation}\hspace*{1em}{type}="{social}"\hspace*{1em}{name}="{employer}"\mbox{}\newline 
\hspace*{1em}\hspace*{1em}{active}="{\#P1}"\hspace*{1em}{passive}="{\#P3 \#P4}"/>}\mbox{}\newline 
{</\textbf{listRelation}>}\end{shaded}\egroup\par \noindent  This example defines the relationships amongst a number of people not further described here; we assume however that each person has been allocated an identifier such as P1, P2, etc. which can be linked to using references such as \#P1, \#P2, etc. Then the above set of \hyperref[TEI.relation]{<relation>} elements describe the following three relationships amongst the people referenced: \begin{itemize}
\item P1 and P2 are parents of P3 and P4.
\item P1 and P2 are linked in a mutual relationship called ‘spouse’—that is, P2 is the spouse of P1, and P1 is the spouse of P2.
\item P1 has the social relationship ‘employer’ with respect to P3 and P4.
\end{itemize} \par
Relationships within places and organizations are further discussed in the relevant sections below. Relationships between for example organizations and places, or places and persons, may be handled in exactly the same way. 
\subsubsection[{Organizational Data}]{Organizational Data}\label{ND-org}\par
The \hyperref[TEI.org]{<org>} and \hyperref[TEI.listOrg]{<listOrg>} elements are used to store data about an organization such as its preferred name, its locations, or key persons within it. 
\begin{sansreflist}
  
\item [\textbf{<org>}] (organization) provides information about an identifiable organization such as a business, a tribe, or any other grouping of people.
\item [\textbf{<listOrg>}] (list of organizations) contains a list of elements, each of which provides information about an identifiable organization.
\end{sansreflist}
 These elements are intended to be used in a way analogous to the \hyperref[TEI.place]{<place>} and \hyperref[TEI.person]{<person>} elements discussed elsewhere in this chapter, that is to provide a unique wrapper element for information about an entity, distinct from references to that entity which are typically encoded using a naming element such as <name type="org"> or \hyperref[TEI.orgName]{<orgName>}. The content of a naming element will represent the way an organization is named in a given context; the content of an \hyperref[TEI.org]{<org>} represents the information known to the encoder about that organization, gathered together in a single place, and independent of its textual realization.\par
An organization is not the same thing as a list or group of people because it has an identity of its own. That identity may be expressed solely in the existence of a name (for example ‘The Scythians’), but is likely to consist in the combination of that name with a number of events, traits, or states which are considered to apply to the organization itself, rather than any of its members. For example, a sports team might be described in terms of its membership (a \hyperref[TEI.listPerson]{<listPerson>}), its fixtures (a \hyperref[TEI.listPlace]{<listPlace>}), its geographical affiliation (a \hyperref[TEI.placeName]{<placeName>}), or any combination of these. It will also have properties which may be used to categorize it in some way such as the kind of sport played, whether the team is amateur or professional, and so on: these are probably best dealt with by means of the {\itshape type} attribute. However, it is the name of the sports team alone which identifies it.\par
The content model for \hyperref[TEI.org]{<org>} permits any mixture of generic \hyperref[TEI.state]{<state>}, \hyperref[TEI.trait]{<trait>}, or \hyperref[TEI.event]{<event>} elements: the presence of the \hyperref[TEI.orgName]{<orgName>} element described in \textit{\hyperref[NDORG]{13.2.2.\ Organizational Names}} is however strongly recommended.\par
In other respects, the \hyperref[TEI.org]{<org>} element is used in much the same way as \hyperref[TEI.place]{<place>} or \hyperref[TEI.person]{<person>}. An organization may have different names at different times: \par\bgroup\index{org=<org>|exampleindex}\index{orgName=<orgName>|exampleindex}\index{notAfter=@notAfter!<orgName>|exampleindex}\index{orgName=<orgName>|exampleindex}\index{from=@from!<orgName>|exampleindex}\exampleFont \begin{shaded}\noindent\mbox{}{<\textbf{org}\hspace*{1em}{xml:id}="{fab4}">}\mbox{}\newline 
\hspace*{1em}{<\textbf{orgName}\hspace*{1em}{notAfter}="{1960}">}The Silver Beetles{</\textbf{orgName}>}\mbox{}\newline 
\hspace*{1em}{<\textbf{orgName}\hspace*{1em}{from}="{1960-08}">}The Beatles{</\textbf{orgName}>}\mbox{}\newline 
{</\textbf{org}>}\end{shaded}\egroup\par \par
The names of the people making up an organization can also change over time, (if they are known at all). For example: \par\bgroup\index{org=<org>|exampleindex}\index{orgName=<orgName>|exampleindex}\index{notAfter=@notAfter!<orgName>|exampleindex}\index{orgName=<orgName>|exampleindex}\index{notBefore=@notBefore!<orgName>|exampleindex}\index{state=<state>|exampleindex}\index{type=@type!<state>|exampleindex}\index{from=@from!<state>|exampleindex}\index{to=@to!<state>|exampleindex}\index{desc=<desc>|exampleindex}\index{persName=<persName>|exampleindex}\index{persName=<persName>|exampleindex}\index{persName=<persName>|exampleindex}\index{persName=<persName>|exampleindex}\index{persName=<persName>|exampleindex}\index{state=<state>|exampleindex}\index{type=@type!<state>|exampleindex}\index{notBefore=@notBefore!<state>|exampleindex}\index{desc=<desc>|exampleindex}\index{persName=<persName>|exampleindex}\index{persName=<persName>|exampleindex}\index{persName=<persName>|exampleindex}\index{persName=<persName>|exampleindex}\exampleFont \begin{shaded}\noindent\mbox{}{<\textbf{org}\hspace*{1em}{xml:id}="{FAB4}">}\mbox{}\newline 
\hspace*{1em}{<\textbf{orgName}\hspace*{1em}{notAfter}="{1960}">}The Silver Beetles{</\textbf{orgName}>}\mbox{}\newline 
\hspace*{1em}{<\textbf{orgName}\hspace*{1em}{notBefore}="{1960}">}The Beatles{</\textbf{orgName}>}\mbox{}\newline 
\hspace*{1em}{<\textbf{state}\hspace*{1em}{type}="{membership}"\hspace*{1em}{from}="{1960-08}"\mbox{}\newline 
\hspace*{1em}\hspace*{1em}{to}="{1962-05}">}\mbox{}\newline 
\hspace*{1em}\hspace*{1em}{<\textbf{desc}>}\mbox{}\newline 
\hspace*{1em}\hspace*{1em}\hspace*{1em}{<\textbf{persName}>}John Lennon{</\textbf{persName}>}\mbox{}\newline 
\hspace*{1em}\hspace*{1em}\hspace*{1em}{<\textbf{persName}>}Paul McCartney{</\textbf{persName}>}\mbox{}\newline 
\hspace*{1em}\hspace*{1em}\hspace*{1em}{<\textbf{persName}>}George Harrison{</\textbf{persName}>}\mbox{}\newline 
\hspace*{1em}\hspace*{1em}\hspace*{1em}{<\textbf{persName}>}Stuart Sutcliffe{</\textbf{persName}>}\mbox{}\newline 
\hspace*{1em}\hspace*{1em}\hspace*{1em}{<\textbf{persName}>}Pete Best{</\textbf{persName}>}\mbox{}\newline 
\hspace*{1em}\hspace*{1em}{</\textbf{desc}>}\mbox{}\newline 
\hspace*{1em}{</\textbf{state}>}\mbox{}\newline 
\hspace*{1em}{<\textbf{state}\hspace*{1em}{type}="{membership}"\hspace*{1em}{notBefore}="{1963}">}\mbox{}\newline 
\hspace*{1em}\hspace*{1em}{<\textbf{desc}>}\mbox{}\newline 
\hspace*{1em}\hspace*{1em}\hspace*{1em}{<\textbf{persName}>}John Lennon{</\textbf{persName}>}\mbox{}\newline 
\hspace*{1em}\hspace*{1em}\hspace*{1em}{<\textbf{persName}>}Paul McCartney{</\textbf{persName}>}\mbox{}\newline 
\hspace*{1em}\hspace*{1em}\hspace*{1em}{<\textbf{persName}>}George Harrison{</\textbf{persName}>}\mbox{}\newline 
\hspace*{1em}\hspace*{1em}\hspace*{1em}{<\textbf{persName}>}Ringo Starr{</\textbf{persName}>}\mbox{}\newline 
\hspace*{1em}\hspace*{1em}{</\textbf{desc}>}\mbox{}\newline 
\hspace*{1em}{</\textbf{state}>}\mbox{}\newline 
{</\textbf{org}>}\end{shaded}\egroup\par \par
An \hyperref[TEI.org]{<org>} may contain subordinate \hyperref[TEI.org]{<org>}s: \par\bgroup\index{org=<org>|exampleindex}\index{orgName=<orgName>|exampleindex}\index{org=<org>|exampleindex}\index{orgName=<orgName>|exampleindex}\index{org=<org>|exampleindex}\index{orgName=<orgName>|exampleindex}\index{org=<org>|exampleindex}\index{orgName=<orgName>|exampleindex}\index{org=<org>|exampleindex}\index{orgName=<orgName>|exampleindex}\index{org=<org>|exampleindex}\index{orgName=<orgName>|exampleindex}\exampleFont \begin{shaded}\noindent\mbox{}{<\textbf{org}\hspace*{1em}{xml:id}="{OUCS}">}\mbox{}\newline 
\hspace*{1em}{<\textbf{orgName}>}Oxford University Computing Services{</\textbf{orgName}>}\mbox{}\newline 
\hspace*{1em}{<\textbf{org}\hspace*{1em}{xml:id}="{OUCSisg}">}\mbox{}\newline 
\hspace*{1em}\hspace*{1em}{<\textbf{orgName}>}Information and Support Group{</\textbf{orgName}>}\mbox{}\newline 
\hspace*{1em}{</\textbf{org}>}\mbox{}\newline 
\hspace*{1em}{<\textbf{org}\hspace*{1em}{xml:id}="{OUCSig}">}\mbox{}\newline 
\hspace*{1em}\hspace*{1em}{<\textbf{orgName}>}Infrastructure Group{</\textbf{orgName}>}\mbox{}\newline 
\hspace*{1em}\hspace*{1em}{<\textbf{org}\hspace*{1em}{xml:id}="{OUCSig.nt}">}\mbox{}\newline 
\hspace*{1em}\hspace*{1em}\hspace*{1em}{<\textbf{orgName}>}Networking Team{</\textbf{orgName}>}\mbox{}\newline 
\hspace*{1em}\hspace*{1em}{</\textbf{org}>}\mbox{}\newline 
\hspace*{1em}\hspace*{1em}{<\textbf{org}\hspace*{1em}{xml:id}="{OUCSig.sdt}">}\mbox{}\newline 
\hspace*{1em}\hspace*{1em}\hspace*{1em}{<\textbf{orgName}>}System Development Team{</\textbf{orgName}>}\mbox{}\newline 
\hspace*{1em}\hspace*{1em}{</\textbf{org}>}\mbox{}\newline 
\hspace*{1em}{</\textbf{org}>}\mbox{}\newline 
\hspace*{1em}{<\textbf{org}\hspace*{1em}{xml:id}="{OUCSltg}">}\mbox{}\newline 
\hspace*{1em}\hspace*{1em}{<\textbf{orgName}>}Learning Technologies Group{</\textbf{orgName}>}\mbox{}\newline 
\hspace*{1em}{</\textbf{org}>}\mbox{}\newline 
{</\textbf{org}>}\end{shaded}\egroup\par \par
The following example demonstrates the use of the \hyperref[TEI.listOrg]{<listOrg>} element to group together a number of \hyperref[TEI.org]{<org>} elements, each of which is defined solely by means of an informal description, itself containing other names. \par\bgroup\index{p=<p>|exampleindex}\index{listOrg=<listOrg>|exampleindex}\index{org=<org>|exampleindex}\index{orgName=<orgName>|exampleindex}\index{desc=<desc>|exampleindex}\index{name=<name>|exampleindex}\index{type=@type!<name>|exampleindex}\index{orgName=<orgName>|exampleindex}\index{org=<org>|exampleindex}\index{orgName=<orgName>|exampleindex}\index{desc=<desc>|exampleindex}\index{orgName=<orgName>|exampleindex}\index{orgName=<orgName>|exampleindex}\index{orgName=<orgName>|exampleindex}\index{org=<org>|exampleindex}\index{orgName=<orgName>|exampleindex}\index{desc=<desc>|exampleindex}\index{orgName=<orgName>|exampleindex}\index{orgName=<orgName>|exampleindex}\index{org=<org>|exampleindex}\index{orgName=<orgName>|exampleindex}\index{desc=<desc>|exampleindex}\index{orgName=<orgName>|exampleindex}\index{orgName=<orgName>|exampleindex}\exampleFont \begin{shaded}\noindent\mbox{}{<\textbf{p}>}The TEI institutional hosts are: {<\textbf{listOrg}>}\mbox{}\newline 
\hspace*{1em}\hspace*{1em}{<\textbf{org}\hspace*{1em}{xml:id}="{bu}">}\mbox{}\newline 
\hspace*{1em}\hspace*{1em}\hspace*{1em}{<\textbf{orgName}>}Brown University{</\textbf{orgName}>}\mbox{}\newline 
\hspace*{1em}\hspace*{1em}\hspace*{1em}{<\textbf{desc}>}The host contribution is made jointly by the {<\textbf{name}\hspace*{1em}{type}="{project}">}Brown University Women Writers\mbox{}\newline 
\hspace*{1em}\hspace*{1em}\hspace*{1em}\hspace*{1em}\hspace*{1em}\hspace*{1em}\hspace*{1em}\hspace*{1em} Project{</\textbf{name}>} and the {<\textbf{orgName}>}Brown University Library's Center for Digital Initiatives{</\textbf{orgName}>}.{</\textbf{desc}>}\mbox{}\newline 
\hspace*{1em}\hspace*{1em}{</\textbf{org}>}\mbox{}\newline 
\hspace*{1em}\hspace*{1em}{<\textbf{org}\hspace*{1em}{xml:id}="{na}">}\mbox{}\newline 
\hspace*{1em}\hspace*{1em}\hspace*{1em}{<\textbf{orgName}>}Nancy{</\textbf{orgName}>}\mbox{}\newline 
\hspace*{1em}\hspace*{1em}\hspace*{1em}{<\textbf{desc}>}Hosting is provided by a group of institutions located in Nancy, France, coordinated by\mbox{}\newline 
\hspace*{1em}\hspace*{1em}\hspace*{1em}{<\textbf{orgName}>}Loria{</\textbf{orgName}>} and also including {<\textbf{orgName}>}ATILF{</\textbf{orgName}>} and {<\textbf{orgName}>}INIST{</\textbf{orgName}>}.{</\textbf{desc}>}\mbox{}\newline 
\hspace*{1em}\hspace*{1em}{</\textbf{org}>}\mbox{}\newline 
\hspace*{1em}\hspace*{1em}{<\textbf{org}\hspace*{1em}{xml:id}="{ou}">}\mbox{}\newline 
\hspace*{1em}\hspace*{1em}\hspace*{1em}{<\textbf{orgName}>}Oxford University{</\textbf{orgName}>}\mbox{}\newline 
\hspace*{1em}\hspace*{1em}\hspace*{1em}{<\textbf{desc}>}Hosting is provided by the {<\textbf{orgName}>}Research Technologies Service{</\textbf{orgName}>} at {<\textbf{orgName}>}Oxford University\mbox{}\newline 
\hspace*{1em}\hspace*{1em}\hspace*{1em}\hspace*{1em}\hspace*{1em}\hspace*{1em}\hspace*{1em}\hspace*{1em} Computing Services{</\textbf{orgName}>}.{</\textbf{desc}>}\mbox{}\newline 
\hspace*{1em}\hspace*{1em}{</\textbf{org}>}\mbox{}\newline 
\hspace*{1em}\hspace*{1em}{<\textbf{org}\hspace*{1em}{xml:id}="{uv}">}\mbox{}\newline 
\hspace*{1em}\hspace*{1em}\hspace*{1em}{<\textbf{orgName}>}University of Virginia{</\textbf{orgName}>}\mbox{}\newline 
\hspace*{1em}\hspace*{1em}\hspace*{1em}{<\textbf{desc}>}Virginia's host support comes jointly from the {<\textbf{orgName}>}Institute for Advanced Technology in the\mbox{}\newline 
\hspace*{1em}\hspace*{1em}\hspace*{1em}\hspace*{1em}\hspace*{1em}\hspace*{1em}\hspace*{1em}\hspace*{1em} Humanities{</\textbf{orgName}>} and the {<\textbf{orgName}>}University of Virginia Library{</\textbf{orgName}>}.{</\textbf{desc}>}\mbox{}\newline 
\hspace*{1em}\hspace*{1em}{</\textbf{org}>}\mbox{}\newline 
\hspace*{1em}{</\textbf{listOrg}>}\mbox{}\newline 
{</\textbf{p}>}\end{shaded}\egroup\par \noindent  In a more elaborated version of this example, the organizational names tagged using \hyperref[TEI.orgName]{<orgName>} might be linked using the {\itshape key} or {\itshape ref} attribute to a unique \hyperref[TEI.org]{<org>} element elsewhere.
\subsubsection[{Places}]{Places}\label{NDGEOG}\par
In \textit{\hyperref[NDPLAC]{13.2.3.\ Place Names}} we discuss various ways of naming places such as towns, countries, etc. In much the same way as these Guidelines distinguish between the encoding of names for people and the encoding of other data about people, so they also distinguish between the encoding of names for places and the encoding of other data about places. In this section we present elements which may be used to record in a structured way data about places of any kind which might be named or referenced within a text. Such data may be useful as a way of normalizing or standardizing references to particular places, as the raw material for a gazetteer or similar reference document associated with a particular text or set of texts, or in conjunction with any form of geographical information system.\par
The following elements are provided for this purpose: 
\begin{sansreflist}
  
\item [\textbf{<listPlace>}] (list of places) contains a list of places, optionally followed by a list of relationships (other than containment) defined amongst them.
\item [\textbf{<place>}] (place) contains data about a geographic location
\end{sansreflist}
\par
The \textsf{model.placeStateLike} class contains elements describing characteristics of a place which have a definite duration, such as its name. Any member of the \textsf{model.placeNamePart} may be used for this purpose, since a \hyperref[TEI.place]{<place>} element will usually contain at least one, and possibly several, \hyperref[TEI.placeName]{<placeName>}-like elements indicating the names associated with it, by different people, in different languages, or at different times.\par
For example, the modern city of Lyon in France was in Roman times known as Lugdunum. Although the modern and the Roman city are not physically co-extensive, they have significant areas which overlap, and we may therefore wish to regard them as the same place, while supplying both names with an indication of the time period during which each was current.\par
Places usually have physical locations in addition to names. As with the example of Lyon, the precise geographic location and extent of a place may change over time, and so locations like names may need to be qualified with indications of the time period to which they apply. Locations may be specified in a number of ways: as a set of coordinates defining a point or an area on the surface of the earth, or by providing a description of how the place may be found, usually in terms of other place names. For example, we can identify the location of the Canadian city of London, either by specifying its latitude and longitude, or by specifying that we mean the city called London located in the province called Ontario within the country called Canada.\par
In addition we may wish to supply a brief characterization of the place identified, for example to state that it is a city, an administrative area such as a country, or a landmark of some kind such as a monument or a battlefield. If our typology of places is simple, the open ended {\itshape type} attribute is the easiest way to represent it: so we might say <place type="city">, <place type="battlefield"> etc.\par
Within the \hyperref[TEI.place]{<place>} element, the following elements may be used to provide more information about specific aspects of the place in a structured form: 
\begin{sansreflist}
  
\item [\textbf{<placeName>}] (place name) contains an absolute or relative place name.
\item [\textbf{<location>}] (location) defines the location of a place as a set of geographical coordinates, in terms of other named geo-political entities, or as an address.
\end{sansreflist}

\paragraph[{Varieties of Location}]{Varieties of Location}\label{NDGEOGva}\par
A location may be specified in one or more of the following ways: \begin{enumerate}
\item by supplying a string representing its coordinates in some standardized way within a \hyperref[TEI.geo]{<geo>} element, as shown below
\item by supplying one or more place name component elements (e.g. \hyperref[TEI.country]{<country>}, \hyperref[TEI.settlement]{<settlement>} etc.) to place it within a geo-political context
\item by supplying a postal address, e.g. using the \hyperref[TEI.address]{<address>} element
\item by supplying a brief textual description, e.g. using the \hyperref[TEI.desc]{<desc>} element
\item by using a non-TEI XML vocabulary such as the Geography Markup Language
\end{enumerate} We give examples of all of these methods in the remainder of this section.\par
The simplest method of specifying a location is by means of its geographic coordinates, supplied within the \hyperref[TEI.geo]{<geo>} element. This may be used to supply any kind of positional information, using one of the many different geodetic systems available. Such systems vary in their format, in their scope or coverage, and more fundamentally in the reference frame (the ‘datum’) used for the coordinate system itself. The default recommended by these Guidelines is to supply a string containing two real numbers separated by whitespace, of which the first indicates latitude and the second longitude according to the 1984 World Geodetic System (WGS84); this is the system currently used by most GPS applications which TEI users are likely to encounter.\footnote{See \url{http://earth-info.nga.mil/GandG/wgs84/index.html}. The most recent revision of this standard is known as the Earth Gravity Model 1996.}We might therefore record the information about the place known as ‘Lyon’ as follows: \par\bgroup\index{place=<place>|exampleindex}\index{type=@type!<place>|exampleindex}\index{placeName=<placeName>|exampleindex}\index{notBefore=@notBefore!<placeName>|exampleindex}\index{placeName=<placeName>|exampleindex}\index{notAfter=@notAfter!<placeName>|exampleindex}\index{location=<location>|exampleindex}\index{geo=<geo>|exampleindex}\exampleFont \begin{shaded}\noindent\mbox{}{<\textbf{place}\hspace*{1em}{xml:id}="{LYON1}"\hspace*{1em}{type}="{city}">}\mbox{}\newline 
\hspace*{1em}{<\textbf{placeName}\hspace*{1em}{notBefore}="{1400}">}Lyon{</\textbf{placeName}>}\mbox{}\newline 
\hspace*{1em}{<\textbf{placeName}\hspace*{1em}{notAfter}="{0640}">}Lugdunum{</\textbf{placeName}>}\mbox{}\newline 
\hspace*{1em}{<\textbf{location}>}\mbox{}\newline 
\hspace*{1em}\hspace*{1em}{<\textbf{geo}>}45.769559 4.834843{</\textbf{geo}>}\mbox{}\newline 
\hspace*{1em}{</\textbf{location}>}\mbox{}\newline 
{</\textbf{place}>}\end{shaded}\egroup\par \par
Identifying Lyon by its geo-political status as a settlement within a country forming part of a larger political entity, we might represent the same ‘place’ as follows: \par\bgroup\index{place=<place>|exampleindex}\index{placeName=<placeName>|exampleindex}\index{notBefore=@notBefore!<placeName>|exampleindex}\index{placeName=<placeName>|exampleindex}\index{notAfter=@notAfter!<placeName>|exampleindex}\index{location=<location>|exampleindex}\index{bloc=<bloc>|exampleindex}\index{country=<country>|exampleindex}\exampleFont \begin{shaded}\noindent\mbox{}{<\textbf{place}\hspace*{1em}{xml:id}="{LYON2}">}\mbox{}\newline 
\hspace*{1em}{<\textbf{placeName}\hspace*{1em}{notBefore}="{1400}">}Lyon{</\textbf{placeName}>}\mbox{}\newline 
\hspace*{1em}{<\textbf{placeName}\hspace*{1em}{notAfter}="{0640}">}Lugdunum{</\textbf{placeName}>}\mbox{}\newline 
\hspace*{1em}{<\textbf{location}>}\mbox{}\newline 
\hspace*{1em}\hspace*{1em}{<\textbf{bloc}>}EU{</\textbf{bloc}>}\mbox{}\newline 
\hspace*{1em}\hspace*{1em}{<\textbf{country}>}France{</\textbf{country}>}\mbox{}\newline 
\hspace*{1em}{</\textbf{location}>}\mbox{}\newline 
{</\textbf{place}>}\end{shaded}\egroup\par \noindent  Elements such as \hyperref[TEI.bloc]{<bloc>} are specialized forms of \hyperref[TEI.placeName]{<placeName>}, as discussed in \textit{\hyperref[NDPLGU]{13.2.3.1.\ Geo-political Place Names}}.\par
We may use the same procedure to represent the location of smaller places, such as a street or even an individual building: \par\bgroup\index{place=<place>|exampleindex}\index{type=@type!<place>|exampleindex}\index{placeName=<placeName>|exampleindex}\index{location=<location>|exampleindex}\index{country=<country>|exampleindex}\index{key=@key!<country>|exampleindex}\index{settlement=<settlement>|exampleindex}\index{type=@type!<settlement>|exampleindex}\index{district=<district>|exampleindex}\index{type=@type!<district>|exampleindex}\index{district=<district>|exampleindex}\index{type=@type!<district>|exampleindex}\index{placeName=<placeName>|exampleindex}\index{type=@type!<placeName>|exampleindex}\index{num=<num>|exampleindex}\exampleFont \begin{shaded}\noindent\mbox{}{<\textbf{place}\hspace*{1em}{xml:id}="{BGbldg}"\hspace*{1em}{type}="{building}">}\mbox{}\newline 
\hspace*{1em}{<\textbf{placeName}>}Brasserie Georges{</\textbf{placeName}>}\mbox{}\newline 
\hspace*{1em}{<\textbf{location}>}\mbox{}\newline 
\hspace*{1em}\hspace*{1em}{<\textbf{country}\hspace*{1em}{key}="{FR}"/>}\mbox{}\newline 
\hspace*{1em}\hspace*{1em}{<\textbf{settlement}\hspace*{1em}{type}="{city}">}Lyon{</\textbf{settlement}>}\mbox{}\newline 
\hspace*{1em}\hspace*{1em}{<\textbf{district}\hspace*{1em}{type}="{arrondissement}">}IIème{</\textbf{district}>}\mbox{}\newline 
\hspace*{1em}\hspace*{1em}{<\textbf{district}\hspace*{1em}{type}="{quartier}">}Perrache{</\textbf{district}>}\mbox{}\newline 
\hspace*{1em}\hspace*{1em}{<\textbf{placeName}\hspace*{1em}{type}="{street}">}\mbox{}\newline 
\hspace*{1em}\hspace*{1em}\hspace*{1em}{<\textbf{num}>}30{</\textbf{num}>}, Cours de Verdun{</\textbf{placeName}>}\mbox{}\newline 
\hspace*{1em}{</\textbf{location}>}\mbox{}\newline 
{</\textbf{place}>}\end{shaded}\egroup\par \noindent  Note the use of the {\itshape type} attribute to categorize more precisely both the kind of place concerned (a building) and the kind of name used to locate it, for example by characterizing the generic \hyperref[TEI.district]{<district>} as an ‘arrondissement’, or a ‘quartier’.\par
We may also treat imaginary places in the same way: \par\bgroup\index{place=<place>|exampleindex}\index{type=@type!<place>|exampleindex}\index{placeName=<placeName>|exampleindex}\index{location=<location>|exampleindex}\index{offset=<offset>|exampleindex}\index{placeName=<placeName>|exampleindex}\index{persName=<persName>|exampleindex}\exampleFont \begin{shaded}\noindent\mbox{}{<\textbf{place}\hspace*{1em}{xml:id}="{Atl}"\hspace*{1em}{type}="{imaginary}">}\mbox{}\newline 
\hspace*{1em}{<\textbf{placeName}>}Atlantis{</\textbf{placeName}>}\mbox{}\newline 
\hspace*{1em}{<\textbf{location}>}\mbox{}\newline 
\hspace*{1em}\hspace*{1em}{<\textbf{offset}>}beyond{</\textbf{offset}>}\mbox{}\newline 
\hspace*{1em}\hspace*{1em}{<\textbf{placeName}>}The Pillars of {<\textbf{persName}>}Hercules{</\textbf{persName}>}\mbox{}\newline 
\hspace*{1em}\hspace*{1em}{</\textbf{placeName}>}\mbox{}\newline 
\hspace*{1em}{</\textbf{location}>}\mbox{}\newline 
{</\textbf{place}>}\end{shaded}\egroup\par \par
A \hyperref[TEI.location]{<location>} sometimes resembles a set of instructions for finding a place: \par\bgroup\index{place=<place>|exampleindex}\index{placeName=<placeName>|exampleindex}\index{notAfter=@notAfter!<placeName>|exampleindex}\index{placeName=<placeName>|exampleindex}\index{notBefore=@notBefore!<placeName>|exampleindex}\index{location=<location>|exampleindex}\index{measure=<measure>|exampleindex}\index{offset=<offset>|exampleindex}\index{settlement=<settlement>|exampleindex}\index{region=<region>|exampleindex}\exampleFont \begin{shaded}\noindent\mbox{}{<\textbf{place}\hspace*{1em}{xml:id}="{MYF}">}\mbox{}\newline 
\hspace*{1em}{<\textbf{placeName}\hspace*{1em}{notAfter}="{1969}">}Yasgur's Farm{</\textbf{placeName}>}\mbox{}\newline 
\hspace*{1em}{<\textbf{placeName}\hspace*{1em}{notBefore}="{1969}">}Woodstock Festival Site{</\textbf{placeName}>}\mbox{}\newline 
\hspace*{1em}{<\textbf{location}>}\mbox{}\newline 
\hspace*{1em}\hspace*{1em}{<\textbf{measure}>}one mile{</\textbf{measure}>}\mbox{}\newline 
\hspace*{1em}\hspace*{1em}{<\textbf{offset}>}north west of{</\textbf{offset}>}\mbox{}\newline 
\hspace*{1em}\hspace*{1em}{<\textbf{settlement}>}Bethel{</\textbf{settlement}>}\mbox{}\newline 
\hspace*{1em}\hspace*{1em}{<\textbf{region}>}New York{</\textbf{region}>}\mbox{}\newline 
\hspace*{1em}{</\textbf{location}>}\mbox{}\newline 
{</\textbf{place}>}\end{shaded}\egroup\par \par
The element \hyperref[TEI.address]{<address>} may also be used to identify a location in terms of its postal or other address: \par\bgroup\index{place=<place>|exampleindex}\index{type=@type!<place>|exampleindex}\index{placeName=<placeName>|exampleindex}\index{placeName=<placeName>|exampleindex}\index{type=@type!<placeName>|exampleindex}\index{location=<location>|exampleindex}\index{type=@type!<location>|exampleindex}\index{country=<country>|exampleindex}\index{settlement=<settlement>|exampleindex}\index{district=<district>|exampleindex}\index{location=<location>|exampleindex}\index{type=@type!<location>|exampleindex}\index{address=<address>|exampleindex}\index{addrLine=<addrLine>|exampleindex}\index{addrLine=<addrLine>|exampleindex}\exampleFont \begin{shaded}\noindent\mbox{}{<\textbf{place}\hspace*{1em}{xml:id}="{locCA}"\hspace*{1em}{type}="{cemetery}">}\mbox{}\newline 
\hspace*{1em}{<\textbf{placeName}>}Protestant Cemetery{</\textbf{placeName}>}\mbox{}\newline 
\hspace*{1em}{<\textbf{placeName}\hspace*{1em}{type}="{official}"\hspace*{1em}{xml:lang}="{it}">}Cimitero Acattolico{</\textbf{placeName}>}\mbox{}\newline 
\hspace*{1em}{<\textbf{location}\hspace*{1em}{type}="{geopolitical}">}\mbox{}\newline 
\hspace*{1em}\hspace*{1em}{<\textbf{country}>}Italy{</\textbf{country}>}\mbox{}\newline 
\hspace*{1em}\hspace*{1em}{<\textbf{settlement}>}Rome{</\textbf{settlement}>}\mbox{}\newline 
\hspace*{1em}\hspace*{1em}{<\textbf{district}>}Testaccio{</\textbf{district}>}\mbox{}\newline 
\hspace*{1em}{</\textbf{location}>}\mbox{}\newline 
\hspace*{1em}{<\textbf{location}\hspace*{1em}{type}="{address}">}\mbox{}\newline 
\hspace*{1em}\hspace*{1em}{<\textbf{address}>}\mbox{}\newline 
\hspace*{1em}\hspace*{1em}\hspace*{1em}{<\textbf{addrLine}>}Via Caio Cestio, 6{</\textbf{addrLine}>}\mbox{}\newline 
\hspace*{1em}\hspace*{1em}\hspace*{1em}{<\textbf{addrLine}>}00153 Roma{</\textbf{addrLine}>}\mbox{}\newline 
\hspace*{1em}\hspace*{1em}{</\textbf{address}>}\mbox{}\newline 
\hspace*{1em}{</\textbf{location}>}\mbox{}\newline 
{</\textbf{place}>}\end{shaded}\egroup\par \noindent  When, as here, the same place is given multiple locations, the {\itshape type} attribute should be used to characterize the kind of location, as a means of indicating that these are alternative ways of identifying the same place, rather than that the place is spread across several locations.\par
The \hyperref[TEI.location]{<location>} element may thus identify a place to a greater or lesser degree of precision, using a variety of means: a name, a set of names, or a set of coordinates. The \hyperref[TEI.geo]{<geo>} element introduced earlier is by default understood to supply a value expressed in a specific (and widely used) notation. If a \hyperref[TEI.location]{<location>} contains more than one \hyperref[TEI.geo]{<geo>}, this is interpreted as being really the same place in the universe, but with different systems used to refer to it. If there is a lack of consensus about the location (of, for example, Camelot), more than one \hyperref[TEI.location]{<location>} should be used, each with its own \hyperref[TEI.geo]{<geo>}.\par
By default, the content of \hyperref[TEI.geo]{<geo>} is interpreted as following the standard known as the World Geodetic System (WGS). This may be modified, however, in two ways.\par
Firstly, the content of the \hyperref[TEI.geo]{<geo>} element can be expressed some other way, that is, according to some different geodetic system. The {\itshape decls} attribute is used point to a \hyperref[TEI.geoDecl]{<geoDecl>} element defined in the document header, which describes a different datum.\par
Secondly, the element \hyperref[TEI.geo]{<geo>} may be redefined to contain markup from a different XML vocabulary which is specifically designed to represent this kind of information. This technique is used throughout these Guidelines where specialized markup is required, for example to embed mathematical expressions or vector graphics, and is further described and exemplified in \textit{\hyperref[MDlite]{23.3.4.\ Examples of Modification }}. For geographic information, suitable non-TEI vocabularies include: \begin{itemize}
\item the OpenGIS Geography Markup Language (GML) being defined by the OGC\footnote{The OGC is an international voluntary consensus standards organization whose members maintain the Geography Markup Language standard. The OGC coordinates with the ISO TC 211 standards organization to maintain consistency between OGC and ISO standards work. GML is also an ISO standard (ISO 19136:2007).}
\item the Keyhole Markup Language (KML) used by Google Maps\footnote{See \url{https://developers.google.com/kml/documentation/}}
\end{itemize} \par
In the following example, we have defined the location of the place ‘Lyon’ using GML and indicated the two names associated with it at different times: \par\bgroup\index{place=<place>|exampleindex}\index{type=@type!<place>|exampleindex}\index{placeName=<placeName>|exampleindex}\index{notBefore=@notBefore!<placeName>|exampleindex}\index{placeName=<placeName>|exampleindex}\index{notAfter=@notAfter!<placeName>|exampleindex}\index{location=<location>|exampleindex}\index{geo=<geo>|exampleindex}\exampleFont \begin{shaded}\noindent\mbox{}{<\textbf{place}\hspace*{1em}{xml:id}="{locLyon}"\hspace*{1em}{type}="{city}"\mbox{}\newline 
   xmlns:gml="http://www.opengis.net/gml">}\mbox{}\newline 
\hspace*{1em}{<\textbf{placeName}\hspace*{1em}{notBefore}="{1400}">}Lyon{</\textbf{placeName}>}\mbox{}\newline 
\hspace*{1em}{<\textbf{placeName}\hspace*{1em}{notAfter}="{0640}">}Lugdunum{</\textbf{placeName}>}\mbox{}\newline 
\hspace*{1em}{<\textbf{location}>}\mbox{}\newline 
\hspace*{1em}\hspace*{1em}{<\textbf{geo}>}\mbox{}\newline 
\hspace*{1em}\hspace*{1em}\hspace*{1em}{<\textbf{gml:Polygon}>}\mbox{}\newline 
\hspace*{1em}\hspace*{1em}\hspace*{1em}\hspace*{1em}{<\textbf{gml:exterior}>}\mbox{}\newline 
\hspace*{1em}\hspace*{1em}\hspace*{1em}\hspace*{1em}\hspace*{1em}{<\textbf{gml:LinearRing}>} 45.256 -110.45 46.46 -109.48 43.84 -109.86 45.8 -109.2 45.256 -110.45 {</\textbf{gml:LinearRing}>}\mbox{}\newline 
\hspace*{1em}\hspace*{1em}\hspace*{1em}\hspace*{1em}{</\textbf{gml:exterior}>}\mbox{}\newline 
\hspace*{1em}\hspace*{1em}\hspace*{1em}{</\textbf{gml:Polygon}>}\mbox{}\newline 
\hspace*{1em}\hspace*{1em}{</\textbf{geo}>}\mbox{}\newline 
\hspace*{1em}{</\textbf{location}>}\mbox{}\newline 
{</\textbf{place}>}\end{shaded}\egroup\par \par
A \hyperref[TEI.bibl]{<bibl>} element may be used within \hyperref[TEI.location]{<location>} to indicate the source of the location information.\par\bgroup\index{location=<location>|exampleindex}\index{geo=<geo>|exampleindex}\index{bibl=<bibl>|exampleindex}\index{title=<title>|exampleindex}\index{idno=<idno>|exampleindex}\exampleFont \begin{shaded}\noindent\mbox{}{<\textbf{location}>}\mbox{}\newline 
\hspace*{1em}{<\textbf{geo}>}53.226658 -0.541254{</\textbf{geo}>}\mbox{}\newline 
\hspace*{1em}{<\textbf{bibl}>}\mbox{}\newline 
\hspace*{1em}\hspace*{1em}{<\textbf{title}>}Roman Inscriptions of Britain{</\textbf{title}>}, {<\textbf{idno}>}262{</\textbf{idno}>}\mbox{}\newline 
\hspace*{1em}{</\textbf{bibl}>}\mbox{}\newline 
{</\textbf{location}>}\end{shaded}\egroup\par 
\paragraph[{Multiple Places}]{Multiple Places}\label{NDGEOGmp}\par
A place may contain other places. This containment relation can be directly modelled in XML: thus we can say that the towns of Vilnius and Kaunas are both in a place called Lithuania (or Lietuva) as follows: \par\bgroup\index{place=<place>|exampleindex}\index{country=<country>|exampleindex}\index{country=<country>|exampleindex}\index{place=<place>|exampleindex}\index{settlement=<settlement>|exampleindex}\index{place=<place>|exampleindex}\index{settlement=<settlement>|exampleindex}\exampleFont \begin{shaded}\noindent\mbox{}{<\textbf{place}\hspace*{1em}{xml:id}="{locLith}">}\mbox{}\newline 
\hspace*{1em}{<\textbf{country}>}Lithuania{</\textbf{country}>}\mbox{}\newline 
\hspace*{1em}{<\textbf{country}\hspace*{1em}{xml:lang}="{lt}">}Lietuva{</\textbf{country}>}\mbox{}\newline 
\hspace*{1em}{<\textbf{place}>}\mbox{}\newline 
\hspace*{1em}\hspace*{1em}{<\textbf{settlement}>}Vilnius{</\textbf{settlement}>}\mbox{}\newline 
\hspace*{1em}{</\textbf{place}>}\mbox{}\newline 
\hspace*{1em}{<\textbf{place}>}\mbox{}\newline 
\hspace*{1em}\hspace*{1em}{<\textbf{settlement}>}Kaunas{</\textbf{settlement}>}\mbox{}\newline 
\hspace*{1em}{</\textbf{place}>}\mbox{}\newline 
{</\textbf{place}>}\end{shaded}\egroup\par \par
This does not, of course, imply that Vilnius and Kaunas are the only places constituting Lithuania; only that they are within it. A separate \hyperref[TEI.place]{<place>} element may indicate that it is a part of Lithuania by supplying a \hyperref[TEI.relation]{<relation>} element, as discussed below (\textit{\hyperref[place-rel]{13.3.4.4.\ Relations Between Places}}).\par
As a further example, the islands of Mauritius, Réunion, and Rodrigues are collectively known as the Mascarene Islands. Grouped together with Mauritius there are also several smaller offshore islands, with rather picturesque French names. These offshore islands do not however constitute an identifiable place as a whole. One way of representing this is as follows: \par\bgroup\index{place=<place>|exampleindex}\index{type=@type!<place>|exampleindex}\index{placeName=<placeName>|exampleindex}\index{placeName=<placeName>|exampleindex}\index{place=<place>|exampleindex}\index{type=@type!<place>|exampleindex}\index{placeName=<placeName>|exampleindex}\index{listPlace=<listPlace>|exampleindex}\index{type=@type!<listPlace>|exampleindex}\index{place=<place>|exampleindex}\index{placeName=<placeName>|exampleindex}\index{place=<place>|exampleindex}\index{placeName=<placeName>|exampleindex}\index{place=<place>|exampleindex}\index{type=@type!<place>|exampleindex}\index{placeName=<placeName>|exampleindex}\index{place=<place>|exampleindex}\index{type=@type!<place>|exampleindex}\index{placeName=<placeName>|exampleindex}\exampleFont \begin{shaded}\noindent\mbox{}{<\textbf{place}\hspace*{1em}{xml:id}="{locMascarenes}"\mbox{}\newline 
\hspace*{1em}{type}="{islandGroup}">}\mbox{}\newline 
\hspace*{1em}{<\textbf{placeName}>}Mascarene Islands{</\textbf{placeName}>}\mbox{}\newline 
\hspace*{1em}{<\textbf{placeName}>}Mascarenhas Archipelago{</\textbf{placeName}>}\mbox{}\newline 
\hspace*{1em}{<\textbf{place}\hspace*{1em}{type}="{island}">}\mbox{}\newline 
\hspace*{1em}\hspace*{1em}{<\textbf{placeName}>}Mauritius{</\textbf{placeName}>}\mbox{}\newline 
\hspace*{1em}\hspace*{1em}{<\textbf{listPlace}\hspace*{1em}{type}="{offshoreIslands}">}\mbox{}\newline 
\hspace*{1em}\hspace*{1em}\hspace*{1em}{<\textbf{place}>}\mbox{}\newline 
\hspace*{1em}\hspace*{1em}\hspace*{1em}\hspace*{1em}{<\textbf{placeName}>}La roche qui pleure{</\textbf{placeName}>}\mbox{}\newline 
\hspace*{1em}\hspace*{1em}\hspace*{1em}{</\textbf{place}>}\mbox{}\newline 
\hspace*{1em}\hspace*{1em}\hspace*{1em}{<\textbf{place}>}\mbox{}\newline 
\hspace*{1em}\hspace*{1em}\hspace*{1em}\hspace*{1em}{<\textbf{placeName}>}Île aux cerfs{</\textbf{placeName}>}\mbox{}\newline 
\hspace*{1em}\hspace*{1em}\hspace*{1em}{</\textbf{place}>}\mbox{}\newline 
\hspace*{1em}\hspace*{1em}{</\textbf{listPlace}>}\mbox{}\newline 
\hspace*{1em}{</\textbf{place}>}\mbox{}\newline 
\hspace*{1em}{<\textbf{place}\hspace*{1em}{type}="{island}">}\mbox{}\newline 
\hspace*{1em}\hspace*{1em}{<\textbf{placeName}>}Rodrigues{</\textbf{placeName}>}\mbox{}\newline 
\hspace*{1em}{</\textbf{place}>}\mbox{}\newline 
\hspace*{1em}{<\textbf{place}\hspace*{1em}{type}="{island}">}\mbox{}\newline 
\hspace*{1em}\hspace*{1em}{<\textbf{placeName}>}Réunion{</\textbf{placeName}>}\mbox{}\newline 
\hspace*{1em}{</\textbf{place}>}\mbox{}\newline 
{</\textbf{place}>}\end{shaded}\egroup\par \par
Here is a more complex example, showing the variety of names associated at different times and in different languages with a set of hierarchically grouped places—the settlement of Carmarthen Castle, within the town of Carmarthen, within the administrative county of Carmarthenshire, Wales. \par\bgroup\index{place=<place>|exampleindex}\index{type=@type!<place>|exampleindex}\index{placeName=<placeName>|exampleindex}\index{placeName=<placeName>|exampleindex}\index{placeName=<placeName>|exampleindex}\index{placeName=<placeName>|exampleindex}\index{placeName=<placeName>|exampleindex}\index{place=<place>|exampleindex}\index{type=@type!<place>|exampleindex}\index{region=<region>|exampleindex}\index{type=@type!<region>|exampleindex}\index{notBefore=@notBefore!<region>|exampleindex}\index{place=<place>|exampleindex}\index{type=@type!<place>|exampleindex}\index{placeName=<placeName>|exampleindex}\index{placeName=<placeName>|exampleindex}\index{notBefore=@notBefore!<placeName>|exampleindex}\index{notAfter=@notAfter!<placeName>|exampleindex}\index{placeName=<placeName>|exampleindex}\index{place=<place>|exampleindex}\index{type=@type!<place>|exampleindex}\index{settlement=<settlement>|exampleindex}\exampleFont \begin{shaded}\noindent\mbox{}{<\textbf{place}\hspace*{1em}{xml:id}="{wales}"\hspace*{1em}{type}="{country}">}\mbox{}\newline 
\hspace*{1em}{<\textbf{placeName}\hspace*{1em}{xml:lang}="{cy}">}Cymru{</\textbf{placeName}>}\mbox{}\newline 
\hspace*{1em}{<\textbf{placeName}\hspace*{1em}{xml:lang}="{en}">}Wales{</\textbf{placeName}>}\mbox{}\newline 
\hspace*{1em}{<\textbf{placeName}\hspace*{1em}{xml:lang}="{la}">}Wallie{</\textbf{placeName}>}\mbox{}\newline 
\hspace*{1em}{<\textbf{placeName}\hspace*{1em}{xml:lang}="{la}">}Wallia{</\textbf{placeName}>}\mbox{}\newline 
\hspace*{1em}{<\textbf{placeName}\hspace*{1em}{xml:lang}="{fro}">}Le Waleis{</\textbf{placeName}>}\mbox{}\newline 
\hspace*{1em}{<\textbf{place}\hspace*{1em}{xml:id}="{carmarthenshire}"\mbox{}\newline 
\hspace*{1em}\hspace*{1em}{type}="{region}">}\mbox{}\newline 
\hspace*{1em}\hspace*{1em}{<\textbf{region}\hspace*{1em}{type}="{county}"\hspace*{1em}{xml:lang}="{en}"\mbox{}\newline 
\hspace*{1em}\hspace*{1em}\hspace*{1em}{notBefore}="{1284}">}Carmarthenshire{</\textbf{region}>}\mbox{}\newline 
\hspace*{1em}\hspace*{1em}{<\textbf{place}\hspace*{1em}{xml:id}="{carmarthen}"\mbox{}\newline 
\hspace*{1em}\hspace*{1em}\hspace*{1em}{type}="{settlement}">}\mbox{}\newline 
\hspace*{1em}\hspace*{1em}\hspace*{1em}{<\textbf{placeName}\hspace*{1em}{xml:lang}="{en}">}Carmarthen{</\textbf{placeName}>}\mbox{}\newline 
\hspace*{1em}\hspace*{1em}\hspace*{1em}{<\textbf{placeName}\hspace*{1em}{xml:lang}="{la}"\mbox{}\newline 
\hspace*{1em}\hspace*{1em}\hspace*{1em}\hspace*{1em}{notBefore}="{1090}"\hspace*{1em}{notAfter}="{1300}">}Kaermerdin{</\textbf{placeName}>}\mbox{}\newline 
\hspace*{1em}\hspace*{1em}\hspace*{1em}{<\textbf{placeName}\hspace*{1em}{xml:lang}="{cy}">}Caerfyrddin{</\textbf{placeName}>}\mbox{}\newline 
\hspace*{1em}\hspace*{1em}\hspace*{1em}{<\textbf{place}\hspace*{1em}{xml:id}="{carmarthen\textunderscore castle}"\mbox{}\newline 
\hspace*{1em}\hspace*{1em}\hspace*{1em}\hspace*{1em}{type}="{castle}">}\mbox{}\newline 
\hspace*{1em}\hspace*{1em}\hspace*{1em}\hspace*{1em}{<\textbf{settlement}>}castle of Carmarthen{</\textbf{settlement}>}\mbox{}\newline 
\hspace*{1em}\hspace*{1em}\hspace*{1em}{</\textbf{place}>}\mbox{}\newline 
\hspace*{1em}\hspace*{1em}{</\textbf{place}>}\mbox{}\newline 
\hspace*{1em}{</\textbf{place}>}\mbox{}\newline 
{</\textbf{place}>}\end{shaded}\egroup\par \par
As noted previously, \hyperref[TEI.country]{<country>}, \hyperref[TEI.region]{<region>}, and \hyperref[TEI.settlement]{<settlement>} are all specializations of the generic \hyperref[TEI.placeName]{<placeName>} element; they are not specializations of the \hyperref[TEI.place]{<place>} element. If it is desired to distinguish amongst kinds of \textit{place} this can only be done by means of the {\itshape type} attribute as in the above example.\par
This use of multiple \hyperref[TEI.place]{<place>} elements should be distinguished from the (possibly simpler) case where a number of places with some property in common are being grouped together for convenience, for example, in a gazetteer. The \hyperref[TEI.listPlace]{<listPlace>} element is provided as a means of grouping places together where there is no implication that the grouped elements constitute a distinct place. For example: \par\bgroup\index{place=<place>|exampleindex}\index{type=@type!<place>|exampleindex}\index{placeName=<placeName>|exampleindex}\index{listPlace=<listPlace>|exampleindex}\index{type=@type!<listPlace>|exampleindex}\index{place=<place>|exampleindex}\index{placeName=<placeName>|exampleindex}\index{location=<location>|exampleindex}\index{geo=<geo>|exampleindex}\index{place=<place>|exampleindex}\index{placeName=<placeName>|exampleindex}\index{listPlace=<listPlace>|exampleindex}\index{type=@type!<listPlace>|exampleindex}\index{place=<place>|exampleindex}\index{placeName=<placeName>|exampleindex}\index{place=<place>|exampleindex}\index{placeName=<placeName>|exampleindex}\exampleFont \begin{shaded}\noindent\mbox{}{<\textbf{place}\hspace*{1em}{xml:id}="{pl-c-H}"\hspace*{1em}{type}="{county}">}\mbox{}\newline 
\hspace*{1em}{<\textbf{placeName}>}Herefordshire{</\textbf{placeName}>}\mbox{}\newline 
\hspace*{1em}{<\textbf{listPlace}\hspace*{1em}{type}="{villages}">}\mbox{}\newline 
\hspace*{1em}\hspace*{1em}{<\textbf{place}\hspace*{1em}{xml:id}="{pl-v-AD}">}\mbox{}\newline 
\hspace*{1em}\hspace*{1em}\hspace*{1em}{<\textbf{placeName}>}Abbey Dore{</\textbf{placeName}>}\mbox{}\newline 
\hspace*{1em}\hspace*{1em}\hspace*{1em}{<\textbf{location}>}\mbox{}\newline 
\hspace*{1em}\hspace*{1em}\hspace*{1em}\hspace*{1em}{<\textbf{geo}>}51.969604 -2.893146{</\textbf{geo}>}\mbox{}\newline 
\hspace*{1em}\hspace*{1em}\hspace*{1em}{</\textbf{location}>}\mbox{}\newline 
\hspace*{1em}\hspace*{1em}{</\textbf{place}>}\mbox{}\newline 
\hspace*{1em}\hspace*{1em}{<\textbf{place}\hspace*{1em}{xml:id}="{pl-v-AB}">}\mbox{}\newline 
\hspace*{1em}\hspace*{1em}\hspace*{1em}{<\textbf{placeName}>}Acton Beauchamp{</\textbf{placeName}>}\mbox{}\newline 
\hspace*{1em}\hspace*{1em}{</\textbf{place}>}\mbox{}\newline 
\textit{<!-- ... -->}\mbox{}\newline 
\hspace*{1em}{</\textbf{listPlace}>}\mbox{}\newline 
\hspace*{1em}{<\textbf{listPlace}\hspace*{1em}{type}="{towns}">}\mbox{}\newline 
\hspace*{1em}\hspace*{1em}{<\textbf{place}\hspace*{1em}{xml:id}="{pl-t-H}">}\mbox{}\newline 
\hspace*{1em}\hspace*{1em}\hspace*{1em}{<\textbf{placeName}>}Hereford{</\textbf{placeName}>}\mbox{}\newline 
\hspace*{1em}\hspace*{1em}{</\textbf{place}>}\mbox{}\newline 
\hspace*{1em}\hspace*{1em}{<\textbf{place}\hspace*{1em}{xml:id}="{pl-t-L}">}\mbox{}\newline 
\hspace*{1em}\hspace*{1em}\hspace*{1em}{<\textbf{placeName}>}Leominster{</\textbf{placeName}>}\mbox{}\newline 
\hspace*{1em}\hspace*{1em}{</\textbf{place}>}\mbox{}\newline 
\textit{<!-- ...  -->}\mbox{}\newline 
\hspace*{1em}{</\textbf{listPlace}>}\mbox{}\newline 
{</\textbf{place}>}\end{shaded}\egroup\par 
\paragraph[{States, Traits, and Events}]{States, Traits, and Events}\label{NDGEOGste}\par
There are many different kinds of information which it might be considered useful to record for a place in addition to its name and location, and the categories selected are likely to be very project-specific. As with persons therefore these Guidelines make no claim to comprehensiveness in this context. Instead, the generic \hyperref[TEI.state]{<state>}, \hyperref[TEI.trait]{<trait>}, and \hyperref[TEI.event]{<event>} elements defined by this module should be used. Each of these may be customized for particular needs by means of their {\itshape type} attribute. These are complemented by a small number of predefined elements of general utility: 
\begin{sansreflist}
  
\item [\textbf{<population>}] (population) contains information about the population of a place.
\item [\textbf{<climate>}] (climate) contains information about the physical climate of a place.
\item [\textbf{<terrain>}] (terrain) contains information about the physical terrain of a place.
\end{sansreflist}
\par
These are all specializations of the generic \hyperref[TEI.trait]{<trait>} element. This element may be used for almost any kind of event in the life of a place; no specialized version of this element is proposed, nor do we attempt to enumerate the possible values which might be appropriate for the {\itshape type} attribute on any of these generic elements.\par
Here is an example, showing how the specific and generic elements may be combined: \par\bgroup\index{place=<place>|exampleindex}\index{placeName=<placeName>|exampleindex}\index{placeName=<placeName>|exampleindex}\index{location=<location>|exampleindex}\index{geo=<geo>|exampleindex}\index{terrain=<terrain>|exampleindex}\index{desc=<desc>|exampleindex}\index{state=<state>|exampleindex}\index{type=@type!<state>|exampleindex}\index{notBefore=@notBefore!<state>|exampleindex}\index{p=<p>|exampleindex}\index{state=<state>|exampleindex}\index{type=@type!<state>|exampleindex}\index{notAfter=@notAfter!<state>|exampleindex}\index{p=<p>|exampleindex}\index{placeName=<placeName>|exampleindex}\index{key=@key!<placeName>|exampleindex}\index{event=<event>|exampleindex}\index{type=@type!<event>|exampleindex}\index{when=@when!<event>|exampleindex}\index{desc=<desc>|exampleindex}\index{state=<state>|exampleindex}\index{type=@type!<state>|exampleindex}\index{from=@from!<state>|exampleindex}\index{p=<p>|exampleindex}\exampleFont \begin{shaded}\noindent\mbox{}{<\textbf{place}\hspace*{1em}{xml:id}="{IS}">}\mbox{}\newline 
\hspace*{1em}{<\textbf{placeName}\hspace*{1em}{xml:lang}="{en}">}Iceland{</\textbf{placeName}>}\mbox{}\newline 
\hspace*{1em}{<\textbf{placeName}\hspace*{1em}{xml:lang}="{is}">}Ísland{</\textbf{placeName}>}\mbox{}\newline 
\hspace*{1em}{<\textbf{location}>}\mbox{}\newline 
\hspace*{1em}\hspace*{1em}{<\textbf{geo}>}65.00 -18.00{</\textbf{geo}>}\mbox{}\newline 
\hspace*{1em}{</\textbf{location}>}\mbox{}\newline 
\hspace*{1em}{<\textbf{terrain}>}\mbox{}\newline 
\hspace*{1em}\hspace*{1em}{<\textbf{desc}>}Area: 103,000 sq km{</\textbf{desc}>}\mbox{}\newline 
\hspace*{1em}{</\textbf{terrain}>}\mbox{}\newline 
\hspace*{1em}{<\textbf{state}\hspace*{1em}{type}="{governance}"\hspace*{1em}{notBefore}="{1944}">}\mbox{}\newline 
\hspace*{1em}\hspace*{1em}{<\textbf{p}>}Constitutional republic{</\textbf{p}>}\mbox{}\newline 
\hspace*{1em}{</\textbf{state}>}\mbox{}\newline 
\hspace*{1em}{<\textbf{state}\hspace*{1em}{type}="{governance}"\hspace*{1em}{notAfter}="{1944}">}\mbox{}\newline 
\hspace*{1em}\hspace*{1em}{<\textbf{p}>}Part of the kingdom of {<\textbf{placeName}\hspace*{1em}{key}="{DK}">}Denmark{</\textbf{placeName}>}\mbox{}\newline 
\hspace*{1em}\hspace*{1em}{</\textbf{p}>}\mbox{}\newline 
\hspace*{1em}{</\textbf{state}>}\mbox{}\newline 
\hspace*{1em}{<\textbf{event}\hspace*{1em}{type}="{governance}"\hspace*{1em}{when}="{1944-06-17}">}\mbox{}\newline 
\hspace*{1em}\hspace*{1em}{<\textbf{desc}>}Iceland became independent on 17 June 1944.{</\textbf{desc}>}\mbox{}\newline 
\hspace*{1em}{</\textbf{event}>}\mbox{}\newline 
\hspace*{1em}{<\textbf{state}\hspace*{1em}{type}="{governance}"\hspace*{1em}{from}="{1944-06-17}">}\mbox{}\newline 
\hspace*{1em}\hspace*{1em}{<\textbf{p}>}An independent republic since June 1944{</\textbf{p}>}\mbox{}\newline 
\hspace*{1em}{</\textbf{state}>}\mbox{}\newline 
{</\textbf{place}>}\end{shaded}\egroup\par \par
In the following example, the \hyperref[TEI.climate]{<climate>} example is used to provided a detailed discussion of this particular aspect of the information available about a particular place: \par\bgroup\index{place=<place>|exampleindex}\index{placeName=<placeName>|exampleindex}\index{climate=<climate>|exampleindex}\index{desc=<desc>|exampleindex}\index{climate=<climate>|exampleindex}\index{label=<label>|exampleindex}\index{desc=<desc>|exampleindex}\index{placeName=<placeName>|exampleindex}\index{placeName=<placeName>|exampleindex}\index{placeName=<placeName>|exampleindex}\index{climate=<climate>|exampleindex}\index{label=<label>|exampleindex}\index{desc=<desc>|exampleindex}\index{placeName=<placeName>|exampleindex}\index{offset=<offset>|exampleindex}\index{placeName=<placeName>|exampleindex}\index{placeName=<placeName>|exampleindex}\index{offset=<offset>|exampleindex}\index{placeName=<placeName>|exampleindex}\index{placeName=<placeName>|exampleindex}\index{offset=<offset>|exampleindex}\index{placeName=<placeName>|exampleindex}\index{placeName=<placeName>|exampleindex}\index{placeName=<placeName>|exampleindex}\index{placeName=<placeName>|exampleindex}\index{climate=<climate>|exampleindex}\index{label=<label>|exampleindex}\index{desc=<desc>|exampleindex}\index{placeName=<placeName>|exampleindex}\index{offset=<offset>|exampleindex}\index{offset=<offset>|exampleindex}\index{placeName=<placeName>|exampleindex}\index{placeName=<placeName>|exampleindex}\index{placeName=<placeName>|exampleindex}\index{placeName=<placeName>|exampleindex}\index{offset=<offset>|exampleindex}\exampleFont \begin{shaded}\noindent\mbox{}{<\textbf{place}\hspace*{1em}{xml:id}="{greece}">}\mbox{}\newline 
\hspace*{1em}{<\textbf{placeName}>}Greece{</\textbf{placeName}>}\mbox{}\newline 
\hspace*{1em}{<\textbf{climate}>}\mbox{}\newline 
\hspace*{1em}\hspace*{1em}{<\textbf{desc}>}Greece's climate is divided into three well defined classes:{</\textbf{desc}>}\mbox{}\newline 
\hspace*{1em}\hspace*{1em}{<\textbf{climate}>}\mbox{}\newline 
\hspace*{1em}\hspace*{1em}\hspace*{1em}{<\textbf{label}>}Mediterranean{</\textbf{label}>}\mbox{}\newline 
\hspace*{1em}\hspace*{1em}\hspace*{1em}{<\textbf{desc}>}It features mild, wet winters and hot, dry summers. Temperatures rarely reach extremes, although snowfalls\mbox{}\newline 
\hspace*{1em}\hspace*{1em}\hspace*{1em}\hspace*{1em}\hspace*{1em}\hspace*{1em} do occur occasionally even in {<\textbf{placeName}>}Athens{</\textbf{placeName}>}, {<\textbf{placeName}>}Cyclades{</\textbf{placeName}>} or\mbox{}\newline 
\hspace*{1em}\hspace*{1em}\hspace*{1em}{<\textbf{placeName}>}Crete{</\textbf{placeName}>} during the winter.{</\textbf{desc}>}\mbox{}\newline 
\hspace*{1em}\hspace*{1em}{</\textbf{climate}>}\mbox{}\newline 
\hspace*{1em}\hspace*{1em}{<\textbf{climate}>}\mbox{}\newline 
\hspace*{1em}\hspace*{1em}\hspace*{1em}{<\textbf{label}>}Alpine{</\textbf{label}>}\mbox{}\newline 
\hspace*{1em}\hspace*{1em}\hspace*{1em}{<\textbf{desc}>}It is found primarily in {<\textbf{placeName}>}\mbox{}\newline 
\hspace*{1em}\hspace*{1em}\hspace*{1em}\hspace*{1em}\hspace*{1em}{<\textbf{offset}>}Western{</\textbf{offset}>} Greece{</\textbf{placeName}>}\mbox{}\newline 
\hspace*{1em}\hspace*{1em}\hspace*{1em}\hspace*{1em}\hspace*{1em}\hspace*{1em} ({<\textbf{placeName}>}Epirus{</\textbf{placeName}>}, {<\textbf{placeName}>}\mbox{}\newline 
\hspace*{1em}\hspace*{1em}\hspace*{1em}\hspace*{1em}\hspace*{1em}{<\textbf{offset}>}Central{</\textbf{offset}>} Greece{</\textbf{placeName}>},\mbox{}\newline 
\hspace*{1em}\hspace*{1em}\hspace*{1em}{<\textbf{placeName}>}Thessaly{</\textbf{placeName}>}, {<\textbf{placeName}>}\mbox{}\newline 
\hspace*{1em}\hspace*{1em}\hspace*{1em}\hspace*{1em}\hspace*{1em}{<\textbf{offset}>}Western{</\textbf{offset}>} Macedonia{</\textbf{placeName}>} as well as\mbox{}\newline 
\hspace*{1em}\hspace*{1em}\hspace*{1em}\hspace*{1em}\hspace*{1em}\hspace*{1em} central parts of {<\textbf{placeName}>}Peloponnesus{</\textbf{placeName}>} like {<\textbf{placeName}>}Achaea{</\textbf{placeName}>},\mbox{}\newline 
\hspace*{1em}\hspace*{1em}\hspace*{1em}{<\textbf{placeName}>}Arcadia{</\textbf{placeName}>} and parts of {<\textbf{placeName}>}Laconia{</\textbf{placeName}>} where the Alpine range pass\mbox{}\newline 
\hspace*{1em}\hspace*{1em}\hspace*{1em}\hspace*{1em}\hspace*{1em}\hspace*{1em} by){</\textbf{desc}>}\mbox{}\newline 
\hspace*{1em}\hspace*{1em}{</\textbf{climate}>}\mbox{}\newline 
\hspace*{1em}\hspace*{1em}{<\textbf{climate}>}\mbox{}\newline 
\hspace*{1em}\hspace*{1em}\hspace*{1em}{<\textbf{label}>}Temperate{</\textbf{label}>}\mbox{}\newline 
\hspace*{1em}\hspace*{1em}\hspace*{1em}{<\textbf{desc}>}It is found in {<\textbf{placeName}>}\mbox{}\newline 
\hspace*{1em}\hspace*{1em}\hspace*{1em}\hspace*{1em}\hspace*{1em}{<\textbf{offset}>}Central{</\textbf{offset}>} and {<\textbf{offset}>}Eastern{</\textbf{offset}>} Macedonia{</\textbf{placeName}>} as\mbox{}\newline 
\hspace*{1em}\hspace*{1em}\hspace*{1em}\hspace*{1em}\hspace*{1em}\hspace*{1em} well as in {<\textbf{placeName}>}Thrace{</\textbf{placeName}>} at places like {<\textbf{placeName}>}Komotini{</\textbf{placeName}>},\mbox{}\newline 
\hspace*{1em}\hspace*{1em}\hspace*{1em}{<\textbf{placeName}>}Xanthi{</\textbf{placeName}>} and {<\textbf{placeName}>}\mbox{}\newline 
\hspace*{1em}\hspace*{1em}\hspace*{1em}\hspace*{1em}\hspace*{1em}{<\textbf{offset}>}northern{</\textbf{offset}>} Evros{</\textbf{placeName}>}. It features cold,\mbox{}\newline 
\hspace*{1em}\hspace*{1em}\hspace*{1em}\hspace*{1em}\hspace*{1em}\hspace*{1em} damp winters and hot, dry summers.{</\textbf{desc}>}\mbox{}\newline 
\hspace*{1em}\hspace*{1em}{</\textbf{climate}>}\mbox{}\newline 
\hspace*{1em}{</\textbf{climate}>}\mbox{}\newline 
{</\textbf{place}>}\end{shaded}\egroup\par \noindent  \par
As the above example shows, \hyperref[TEI.state]{<state>} and \hyperref[TEI.trait]{<trait>} elements, and others of the same class, can be nested hierarchically within each other. When this is done, values for the {\itshape type} attribute are to be understood as cumulatively inherited, as elsewhere in the TEI scheme (for example on \hyperref[TEI.category]{<category>} or \hyperref[TEI.linkGrp]{<linkGrp>}). In the following example, the outermost \hyperref[TEI.population]{<population>} element concerns the squirrel population between the dates given. This is then broken down into red and gray squirrel populations, and within that into male and female: \par\bgroup\index{population=<population>|exampleindex}\index{type=@type!<population>|exampleindex}\index{notBefore=@notBefore!<population>|exampleindex}\index{notAfter=@notAfter!<population>|exampleindex}\index{resp=@resp!<population>|exampleindex}\index{population=<population>|exampleindex}\index{type=@type!<population>|exampleindex}\index{when=@when!<population>|exampleindex}\index{population=<population>|exampleindex}\index{type=@type!<population>|exampleindex}\index{desc=<desc>|exampleindex}\index{population=<population>|exampleindex}\index{type=@type!<population>|exampleindex}\index{desc=<desc>|exampleindex}\index{population=<population>|exampleindex}\index{type=@type!<population>|exampleindex}\index{when=@when!<population>|exampleindex}\index{cert=@cert!<population>|exampleindex}\index{population=<population>|exampleindex}\index{type=@type!<population>|exampleindex}\index{desc=<desc>|exampleindex}\index{population=<population>|exampleindex}\index{type=@type!<population>|exampleindex}\index{cert=@cert!<population>|exampleindex}\index{resp=@resp!<population>|exampleindex}\index{desc=<desc>|exampleindex}\exampleFont \begin{shaded}\noindent\mbox{}{<\textbf{population}\hspace*{1em}{type}="{squirrel}"\mbox{}\newline 
\hspace*{1em}{notBefore}="{1901}"\hspace*{1em}{notAfter}="{1902-01-11}"\hspace*{1em}{resp}="{\#strabo}">}\mbox{}\newline 
\hspace*{1em}{<\textbf{population}\hspace*{1em}{type}="{red}"\hspace*{1em}{when}="{1901-01-10}">}\mbox{}\newline 
\hspace*{1em}\hspace*{1em}{<\textbf{population}\hspace*{1em}{type}="{female}">}\mbox{}\newline 
\hspace*{1em}\hspace*{1em}\hspace*{1em}{<\textbf{desc}>}12{</\textbf{desc}>}\mbox{}\newline 
\hspace*{1em}\hspace*{1em}{</\textbf{population}>}\mbox{}\newline 
\hspace*{1em}\hspace*{1em}{<\textbf{population}\hspace*{1em}{type}="{male}">}\mbox{}\newline 
\hspace*{1em}\hspace*{1em}\hspace*{1em}{<\textbf{desc}>}15{</\textbf{desc}>}\mbox{}\newline 
\hspace*{1em}\hspace*{1em}{</\textbf{population}>}\mbox{}\newline 
\hspace*{1em}{</\textbf{population}>}\mbox{}\newline 
\hspace*{1em}{<\textbf{population}\hspace*{1em}{type}="{gray}"\hspace*{1em}{when}="{1902-01-10}"\mbox{}\newline 
\hspace*{1em}\hspace*{1em}{cert}="{high}">}\mbox{}\newline 
\hspace*{1em}\hspace*{1em}{<\textbf{population}\hspace*{1em}{type}="{female}">}\mbox{}\newline 
\hspace*{1em}\hspace*{1em}\hspace*{1em}{<\textbf{desc}>}23{</\textbf{desc}>}\mbox{}\newline 
\hspace*{1em}\hspace*{1em}{</\textbf{population}>}\mbox{}\newline 
\hspace*{1em}\hspace*{1em}{<\textbf{population}\hspace*{1em}{type}="{male}"\hspace*{1em}{cert}="{low}"\mbox{}\newline 
\hspace*{1em}\hspace*{1em}\hspace*{1em}{resp}="{\#biber}">}\mbox{}\newline 
\hspace*{1em}\hspace*{1em}\hspace*{1em}{<\textbf{desc}>}45{</\textbf{desc}>}\mbox{}\newline 
\hspace*{1em}\hspace*{1em}{</\textbf{population}>}\mbox{}\newline 
\hspace*{1em}{</\textbf{population}>}\mbox{}\newline 
{</\textbf{population}>}\end{shaded}\egroup\par \noindent  The dating and responsibility attributes here behave slightly differently from the {\itshape type} attribute: responsibility is not an additive property, and therefore an element either states it explicitly, or inherits it from its nearest ancestor. Dating is slightly different again, in that a child element may specify a date more precisely than its parent, as in the example above\par
Events may also be subdivided into other events. For example, a two part meeting might be represented as follows: \par\bgroup\index{event=<event>|exampleindex}\index{type=@type!<event>|exampleindex}\index{when=@when!<event>|exampleindex}\index{desc=<desc>|exampleindex}\index{event=<event>|exampleindex}\index{type=@type!<event>|exampleindex}\index{notAfter=@notAfter!<event>|exampleindex}\index{desc=<desc>|exampleindex}\index{event=<event>|exampleindex}\index{type=@type!<event>|exampleindex}\index{notBefore=@notBefore!<event>|exampleindex}\index{desc=<desc>|exampleindex}\exampleFont \begin{shaded}\noindent\mbox{}{<\textbf{event}\hspace*{1em}{type}="{meeting}"\hspace*{1em}{when}="{2007-05-29}">}\mbox{}\newline 
\hspace*{1em}{<\textbf{desc}>}All day meeting to resolve content models{</\textbf{desc}>}\mbox{}\newline 
\hspace*{1em}{<\textbf{event}\hspace*{1em}{type}="{preamble}"\hspace*{1em}{notAfter}="{13:00:00}">}\mbox{}\newline 
\hspace*{1em}\hspace*{1em}{<\textbf{desc}>}first part{</\textbf{desc}>}\mbox{}\newline 
\hspace*{1em}{</\textbf{event}>}\mbox{}\newline 
\hspace*{1em}{<\textbf{event}\hspace*{1em}{type}="{conclusions}"\mbox{}\newline 
\hspace*{1em}\hspace*{1em}{notBefore}="{13:00:00}">}\mbox{}\newline 
\hspace*{1em}\hspace*{1em}{<\textbf{desc}>}second part{</\textbf{desc}>}\mbox{}\newline 
\hspace*{1em}{</\textbf{event}>}\mbox{}\newline 
{</\textbf{event}>}\end{shaded}\egroup\par \par
An \hyperref[TEI.event]{<event>} element is usually used to record information about a place, or a person; for this reason the element usually appears as content of a \hyperref[TEI.place]{<place>} or \hyperref[TEI.person]{<person>}. However, it is also possible to describe events independently of either a person or a place. This may be useful in such applications as chronologies, lists of significant events such as battles, legislation, etc.\par
The \hyperref[TEI.listEvent]{<listEvent>} element is a member of the \textsf{model.listLike} class, and may therefore appear inside \hyperref[TEI.standOff]{<standOff>}, or wherever else lists are permitted, in the same way as the \hyperref[TEI.listPerson]{<listPerson>}, \hyperref[TEI.listPlace]{<listPlace>} etc. elements described elsewhere in this chapter.\par\bgroup\index{standOff=<standOff>|exampleindex}\index{listEvent=<listEvent>|exampleindex}\index{event=<event>|exampleindex}\index{when=@when!<event>|exampleindex}\index{ref=@ref!<event>|exampleindex}\index{label=<label>|exampleindex}\index{desc=<desc>|exampleindex}\index{orgName=<orgName>|exampleindex}\index{placeName=<placeName>|exampleindex}\index{placeName=<placeName>|exampleindex}\index{placeName=<placeName>|exampleindex}\index{orgName=<orgName>|exampleindex}\index{type=@type!<orgName>|exampleindex}\index{placeName=<placeName>|exampleindex}\index{key=@key!<placeName>|exampleindex}\index{placeName=<placeName>|exampleindex}\index{key=@key!<placeName>|exampleindex}\index{event=<event>|exampleindex}\index{when=@when!<event>|exampleindex}\index{key=@key!<event>|exampleindex}\index{label=<label>|exampleindex}\index{desc=<desc>|exampleindex}\index{ref=<ref>|exampleindex}\index{event=<event>|exampleindex}\index{when=@when!<event>|exampleindex}\index{ref=@ref!<event>|exampleindex}\index{label=<label>|exampleindex}\index{desc=<desc>|exampleindex}\index{name=<name>|exampleindex}\index{type=@type!<name>|exampleindex}\index{orgName=<orgName>|exampleindex}\index{orgName=<orgName>|exampleindex}\index{type=@type!<orgName>|exampleindex}\exampleFont \begin{shaded}\noindent\mbox{}{<\textbf{standOff}>}\mbox{}\newline 
\hspace*{1em}{<\textbf{listEvent}>}\mbox{}\newline 
\hspace*{1em}\hspace*{1em}{<\textbf{event}\hspace*{1em}{when}="{1713}"\mbox{}\newline 
\hspace*{1em}\hspace*{1em}\hspace*{1em}{ref}="{http://eco.canadiana.ca/view/oocihm.9\textunderscore 01832}">}\mbox{}\newline 
\hspace*{1em}\hspace*{1em}\hspace*{1em}{<\textbf{label}>}Treaty of Utrecht{</\textbf{label}>}\mbox{}\newline 
\hspace*{1em}\hspace*{1em}\hspace*{1em}{<\textbf{desc}>}France ceded to Great Britain its claims to the {<\textbf{orgName}>}Hudson's Bay Company{</\textbf{orgName}>} territories in\mbox{}\newline 
\hspace*{1em}\hspace*{1em}\hspace*{1em}{<\textbf{placeName}>}Rupert's Land{</\textbf{placeName}>}, {<\textbf{placeName}>}Newfoundland{</\textbf{placeName}>}, and {<\textbf{placeName}>}Acadia{</\textbf{placeName}>} and\mbox{}\newline 
\hspace*{1em}\hspace*{1em}\hspace*{1em}\hspace*{1em}\hspace*{1em}\hspace*{1em} recognized British suzerainty over {<\textbf{orgName}\hspace*{1em}{type}="{tribe}">}the Iroquois{</\textbf{orgName}>} but retained its other pre-war\mbox{}\newline 
\hspace*{1em}\hspace*{1em}\hspace*{1em}\hspace*{1em}\hspace*{1em}\hspace*{1em} North American possessions, including {<\textbf{placeName}\hspace*{1em}{key}="{PEI}">}Île-Saint-Jean{</\textbf{placeName}>} (now {<\textbf{placeName}\hspace*{1em}{key}="{PEI}">}Prince Edward Island{</\textbf{placeName}>})...{</\textbf{desc}>}\mbox{}\newline 
\hspace*{1em}\hspace*{1em}{</\textbf{event}>}\mbox{}\newline 
\hspace*{1em}\hspace*{1em}{<\textbf{event}\hspace*{1em}{when}="{1774}"\hspace*{1em}{key}="{14-GeoIII-c83}">}\mbox{}\newline 
\hspace*{1em}\hspace*{1em}\hspace*{1em}{<\textbf{label}>}Quebec Act{</\textbf{label}>}\mbox{}\newline 
\hspace*{1em}\hspace*{1em}\hspace*{1em}{<\textbf{desc}>}This act of the British Parliament guaranteed free practice of the Catholic faith and restored use of the\mbox{}\newline 
\hspace*{1em}\hspace*{1em}\hspace*{1em}\hspace*{1em}\hspace*{1em}\hspace*{1em} French Civil Code for private matters throughout the Province of Quebec, which had been expanded in territory\mbox{}\newline 
\hspace*{1em}\hspace*{1em}\hspace*{1em}\hspace*{1em}\hspace*{1em}\hspace*{1em} following the {<\textbf{ref}>}Treaty of Paris{</\textbf{ref}>}.{</\textbf{desc}>}\mbox{}\newline 
\hspace*{1em}\hspace*{1em}{</\textbf{event}>}\mbox{}\newline 
\hspace*{1em}\hspace*{1em}{<\textbf{event}\hspace*{1em}{when}="{1778}"\mbox{}\newline 
\hspace*{1em}\hspace*{1em}\hspace*{1em}{ref}="{http://avalon.law.yale.edu/18th\textunderscore century/del1778.asp}">}\mbox{}\newline 
\hspace*{1em}\hspace*{1em}\hspace*{1em}{<\textbf{label}>}Treaty of Fort Pitt{</\textbf{label}>}\mbox{}\newline 
\hspace*{1em}\hspace*{1em}\hspace*{1em}{<\textbf{desc}>}Also known as the {<\textbf{name}\hspace*{1em}{type}="{event}">}Treaty with the Delawares{</\textbf{name}>}, this was the first written treaty\mbox{}\newline 
\hspace*{1em}\hspace*{1em}\hspace*{1em}\hspace*{1em}\hspace*{1em}\hspace*{1em} between the newly formed {<\textbf{orgName}>}United States{</\textbf{orgName}>} and any Native American people, in this case, the\mbox{}\newline 
\hspace*{1em}\hspace*{1em}\hspace*{1em}{<\textbf{orgName}\hspace*{1em}{type}="{tribe}">}Lenape{</\textbf{orgName}>} or Delawares.{</\textbf{desc}>}\mbox{}\newline 
\hspace*{1em}\hspace*{1em}{</\textbf{event}>}\mbox{}\newline 
\hspace*{1em}{</\textbf{listEvent}>}\mbox{}\newline 
{</\textbf{standOff}>}\end{shaded}\egroup\par 
\paragraph[{Relations Between Places}]{Relations Between Places}\label{place-rel}\par
The \hyperref[TEI.relation]{<relation>} element may also be used to express relationships of various kinds between places, or between places and persons, in much the same way as it is used to express relationships between persons alone. Returning to the Mascarene Islands example cited above, we might define the island group and its constituents separately, but indicate the relationship by means of a \hyperref[TEI.relation]{<relation>} element: \par\bgroup\index{listPlace=<listPlace>|exampleindex}\index{place=<place>|exampleindex}\index{placeName=<placeName>|exampleindex}\index{placeName=<placeName>|exampleindex}\index{place=<place>|exampleindex}\index{placeName=<placeName>|exampleindex}\index{place=<place>|exampleindex}\index{placeName=<placeName>|exampleindex}\index{place=<place>|exampleindex}\index{placeName=<placeName>|exampleindex}\index{relation=<relation>|exampleindex}\index{name=@name!<relation>|exampleindex}\index{active=@active!<relation>|exampleindex}\index{passive=@passive!<relation>|exampleindex}\exampleFont \begin{shaded}\noindent\mbox{}{<\textbf{listPlace}>}\mbox{}\newline 
\hspace*{1em}{<\textbf{place}\hspace*{1em}{xml:id}="{MASC}">}\mbox{}\newline 
\hspace*{1em}\hspace*{1em}{<\textbf{placeName}>}Mascarene islands{</\textbf{placeName}>}\mbox{}\newline 
\hspace*{1em}\hspace*{1em}{<\textbf{placeName}>}Mascarenhas Archipelago{</\textbf{placeName}>}\mbox{}\newline 
\hspace*{1em}{</\textbf{place}>}\mbox{}\newline 
\hspace*{1em}{<\textbf{place}\hspace*{1em}{xml:id}="{MRU}">}\mbox{}\newline 
\hspace*{1em}\hspace*{1em}{<\textbf{placeName}>}Mauritius{</\textbf{placeName}>}\mbox{}\newline 
\textit{<!-- ... -->}\mbox{}\newline 
\hspace*{1em}{</\textbf{place}>}\mbox{}\newline 
\hspace*{1em}{<\textbf{place}\hspace*{1em}{xml:id}="{ROD}">}\mbox{}\newline 
\hspace*{1em}\hspace*{1em}{<\textbf{placeName}>}Rodrigues{</\textbf{placeName}>}\mbox{}\newline 
\hspace*{1em}{</\textbf{place}>}\mbox{}\newline 
\hspace*{1em}{<\textbf{place}\hspace*{1em}{xml:id}="{REN}">}\mbox{}\newline 
\hspace*{1em}\hspace*{1em}{<\textbf{placeName}>}Réunion{</\textbf{placeName}>}\mbox{}\newline 
\hspace*{1em}{</\textbf{place}>}\mbox{}\newline 
\hspace*{1em}{<\textbf{relation}\hspace*{1em}{name}="{contains}"\hspace*{1em}{active}="{\#MASC}"\mbox{}\newline 
\hspace*{1em}\hspace*{1em}{passive}="{\#ROD \#MRU \#REN}"/>}\mbox{}\newline 
{</\textbf{listPlace}>}\end{shaded}\egroup\par \par
This ‘stand-off’ style of representation has the advantage that we can now also represent the fact that a place may be a ‘part of’ more than one other place; for example, Réunion is part of France, as well as part of the Mascarenes. If we add a declaration for France to the list above: \par\bgroup\index{place=<place>|exampleindex}\index{type=@type!<place>|exampleindex}\index{placeName=<placeName>|exampleindex}\exampleFont \begin{shaded}\noindent\mbox{}{<\textbf{place}\hspace*{1em}{type}="{country}"\hspace*{1em}{xml:id}="{FRA}">}\mbox{}\newline 
\hspace*{1em}{<\textbf{placeName}>}France{</\textbf{placeName}>}\mbox{}\newline 
{</\textbf{place}>}\end{shaded}\egroup\par \noindent  we can now model this dual allegiance by means of a \hyperref[TEI.relation]{<relation>} element: \par\bgroup\index{relation=<relation>|exampleindex}\index{name=@name!<relation>|exampleindex}\index{active=@active!<relation>|exampleindex}\index{passive=@passive!<relation>|exampleindex}\exampleFont \begin{shaded}\noindent\mbox{}{<\textbf{relation}\hspace*{1em}{name}="{partOf}"\hspace*{1em}{active}="{\#REN}"\mbox{}\newline 
\hspace*{1em}{passive}="{\#FRA \#MASC}"/>}\end{shaded}\egroup\par 
\subsubsection[{Objects}]{Objects}\label{NDOBJ}\par

\begin{sansreflist}
  
\item [\textbf{<object>}] contains a description of a single identifiable physical object.
\item [\textbf{<objectName>}] (name of an object) contains a proper noun or noun phrase used to refer to an object.
\item [\textbf{<listObject>}] (list of objects) contains a list of descriptions, each of which provides information about an identifiable physical object.
\end{sansreflist}
 An object is any material thing whether real, in existence, fictional, missing, or purported about which more information is known. Where objects have proper names the \hyperref[TEI.objectName]{<objectName>} element may be used to encode these. However, many objects are not named but the \hyperref[TEI.object]{<object>} element may still be used to provide a description of them. The \hyperref[TEI.object]{<object>} element is a more general descriptive form of the \hyperref[TEI.msDesc]{<msDesc>} element. The latter should be used for describing manuscripts and similar text-bearing objects but can be viewed as a more specific form of the \hyperref[TEI.object]{<object>} element.\par
 \textit{Please note:} The \hyperref[TEI.object]{<object>} element is a recent addition to TEI P5 Guidelines as of version 3.5.0 and as such might be more prone to further revision in the next few releases as its use develops. This may be particularly evident where its contents have been borrowed from \hyperref[TEI.msDesc]{<msDesc>} and have yet to be generalized from their use in the context of manuscript descriptions.\par
The \hyperref[TEI.object]{<object>} element usually appears inside the \hyperref[TEI.listObject]{<listObject>} element which is used to group descriptions of identifiable objects. The \hyperref[TEI.listObject]{<listObject>} element is a member of \textsf{model.listLike} and so may appear inside \hyperref[TEI.standOff]{<standOff>}, or anywhere else that \hyperref[TEI.list]{<list>} is allowed. This enables the flexibility of using \hyperref[TEI.listObject]{<listObject>} to contain a set of metadata descriptions stored in the TEI header, or as a list of objects transcribed from a source document. The equivalent list for manuscript descriptions is \hyperref[TEI.listBibl]{<listBibl>}.\par

\begin{sansreflist}
  
\item [\textbf{<objectIdentifier>}] (object identifier) groups one or more identifiers or pieces of locating information concerning a single object.
\item [\textbf{<msContents>}] (manuscript contents) describes the intellectual content of a manuscript, manuscript part, or other object either as a series of paragraphs or as a series of structured manuscript items.
\item [\textbf{<physDesc>}] (physical description) contains a full physical description of a manuscript, manuscript part, or other object optionally subdivided using more specialized elements from the \textsf{model.physDescPart} class.
\item [\textbf{<history>}] (history) groups elements describing the full history of a manuscript, manuscript part, or other object.
\item [\textbf{<additional>}] (additional) groups additional information, combining bibliographic information about a manuscript or other object, or surrogate copies of it, with curatorial or administrative information.
\end{sansreflist}
 Overall, the basic structure of an \hyperref[TEI.object]{<object>} element is akin to that of \hyperref[TEI.msDesc]{<msDesc>} in that it is providing a structured description of an object. After a group of identifying information, it has the option of paragraphs or, if the \textsf{msdescription} module is loaded, \hyperref[TEI.msContents]{<msContents>}, \hyperref[TEI.physDesc]{<physDesc>}, \hyperref[TEI.history]{<history>}, and \hyperref[TEI.additional]{<additional>} elements for descriptive metadata about this object. Although these elements originate from manuscript description the \hyperref[TEI.object]{<object>} element may be used for all forms of object (whether text-bearing or not).  Where descendents of \hyperref[TEI.object]{<object>} still have the hallmarks of their use in manuscript description, the descriptions as relating to manuscripts should be interpreted as applying to all forms of object (text-bearing) or not.\par
The \hyperref[TEI.objectIdentifier]{<objectIdentifier>} element is a general-purpose grouping element for location or identification information relating to a single object or resource. It is very similar to an \hyperref[TEI.msIdentifier]{<msIdentifier>} element with less contraints on the order of its contents. The \hyperref[TEI.objectIdentifier]{<objectIdentifier>} may be more or less detailed dependent on the needs of the encoder. In some cases an \hyperref[TEI.object]{<object>} may be used mostly as a common reference point for multiple \hyperref[TEI.objectName]{<objectName>} elements to refer back to. In situations, one might provide more detailed information in the \hyperref[TEI.objectIdentifier]{<objectIdentifier>} where it is available or desirable. Compare \par\bgroup\index{object=<object>|exampleindex}\index{objectIdentifier=<objectIdentifier>|exampleindex}\index{objectName=<objectName>|exampleindex}\index{p=<p>|exampleindex}\exampleFont \begin{shaded}\noindent\mbox{}{<\textbf{object}\hspace*{1em}{xml:id}="{Excalibur-shortIdentifier}">}\mbox{}\newline 
\hspace*{1em}{<\textbf{objectIdentifier}>}\mbox{}\newline 
\hspace*{1em}\hspace*{1em}{<\textbf{objectName}>}Excalibur{</\textbf{objectName}>}\mbox{}\newline 
\hspace*{1em}{</\textbf{objectIdentifier}>}\mbox{}\newline 
\hspace*{1em}{<\textbf{p}>}Excalibur is the name for the legendary sword of King Arthur.{</\textbf{p}>}\mbox{}\newline 
{</\textbf{object}>}\end{shaded}\egroup\par \noindent  where only a single \hyperref[TEI.objectName]{<objectName>} is provided and below where multiple versions are provided. \par\bgroup\index{object=<object>|exampleindex}\index{objectIdentifier=<objectIdentifier>|exampleindex}\index{objectName=<objectName>|exampleindex}\index{type=@type!<objectName>|exampleindex}\index{objectName=<objectName>|exampleindex}\index{type=@type!<objectName>|exampleindex}\index{objectName=<objectName>|exampleindex}\index{objectName=<objectName>|exampleindex}\index{objectName=<objectName>|exampleindex}\index{objectName=<objectName>|exampleindex}\index{country=<country>|exampleindex}\index{p=<p>|exampleindex}\index{mentioned=<mentioned>|exampleindex}\index{mentioned=<mentioned>|exampleindex}\index{mentioned=<mentioned>|exampleindex}\index{mentioned=<mentioned>|exampleindex}\index{q=<q>|exampleindex}\index{q=<q>|exampleindex}\exampleFont \begin{shaded}\noindent\mbox{}{<\textbf{object}\hspace*{1em}{xml:id}="{Excalibur-longerIdentifier}">}\mbox{}\newline 
\hspace*{1em}{<\textbf{objectIdentifier}>}\mbox{}\newline 
\hspace*{1em}\hspace*{1em}{<\textbf{objectName}\hspace*{1em}{type}="{main}">}Excalibur{</\textbf{objectName}>}\mbox{}\newline 
\hspace*{1em}\hspace*{1em}{<\textbf{objectName}\hspace*{1em}{type}="{alt}">}Caliburn{</\textbf{objectName}>}\mbox{}\newline 
\hspace*{1em}\hspace*{1em}{<\textbf{objectName}\hspace*{1em}{xml:lang}="{cy}">}Caledfwlch{</\textbf{objectName}>}\mbox{}\newline 
\hspace*{1em}\hspace*{1em}{<\textbf{objectName}\hspace*{1em}{xml:lang}="{cornu}">}Calesvol{</\textbf{objectName}>}\mbox{}\newline 
\hspace*{1em}\hspace*{1em}{<\textbf{objectName}\hspace*{1em}{xml:lang}="{br}">}Kaledvoulc'h{</\textbf{objectName}>}\mbox{}\newline 
\hspace*{1em}\hspace*{1em}{<\textbf{objectName}\hspace*{1em}{xml:lang}="{la}">}Caliburnus{</\textbf{objectName}>}\mbox{}\newline 
\hspace*{1em}\hspace*{1em}{<\textbf{country}>}Wales{</\textbf{country}>}\mbox{}\newline 
\hspace*{1em}{</\textbf{objectIdentifier}>}\mbox{}\newline 
\hspace*{1em}{<\textbf{p}>}Excalibur is the main English name for the legendary\mbox{}\newline 
\hspace*{1em}\hspace*{1em} sword of King Arthur. In Welsh it is called\mbox{}\newline 
\hspace*{1em}{<\textbf{mentioned}>}Caledfwlch{</\textbf{mentioned}>}, in Cornish it is called\mbox{}\newline 
\hspace*{1em}{<\textbf{mentioned}>}Calesvol{</\textbf{mentioned}>}, in Breton it is called\mbox{}\newline 
\hspace*{1em}{<\textbf{mentioned}>}Kaledvoulc'h{</\textbf{mentioned}>}, and in Latin it is\mbox{}\newline 
\hspace*{1em}\hspace*{1em} called {<\textbf{mentioned}>}Caliburnus{</\textbf{mentioned}>}. In some versions\mbox{}\newline 
\hspace*{1em}\hspace*{1em} of the legend, Excalibur’s blade was engraved with phrases on opposite\mbox{}\newline 
\hspace*{1em}\hspace*{1em} sides: {<\textbf{q}>}Take me up{</\textbf{q}>} and {<\textbf{q}>}Cast me away{</\textbf{q}>} (or similar).{</\textbf{p}>}\mbox{}\newline 
{</\textbf{object}>}\end{shaded}\egroup\par \par
Moreover, the \hyperref[TEI.objectIdentifier]{<objectIdentifier>} may include an \hyperref[TEI.address]{<address>} element to provide the address at which the object currently resides. The use of \hyperref[TEI.location]{<location>} within this enables the provision of geographical coordinates when describing objects not housed in traditional repositories or institutions. This may also be used to supplement more traditional repository location information if available and, for example, to enable providing outputs such as maps showing the location of encoded objects. \par\bgroup\index{objectIdentifier=<objectIdentifier>|exampleindex}\index{objectName=<objectName>|exampleindex}\index{idno=<idno>|exampleindex}\index{type=@type!<idno>|exampleindex}\index{idno=<idno>|exampleindex}\index{type=@type!<idno>|exampleindex}\index{idno=<idno>|exampleindex}\index{type=@type!<idno>|exampleindex}\index{institution=<institution>|exampleindex}\index{address=<address>|exampleindex}\index{street=<street>|exampleindex}\index{district=<district>|exampleindex}\index{settlement=<settlement>|exampleindex}\index{country=<country>|exampleindex}\index{location=<location>|exampleindex}\index{geo=<geo>|exampleindex}\exampleFont \begin{shaded}\noindent\mbox{}{<\textbf{objectIdentifier}>}\mbox{}\newline 
\hspace*{1em}{<\textbf{objectName}\hspace*{1em}{xml:lang}="{en}">}Mask of Tutankhamun{</\textbf{objectName}>}\mbox{}\newline 
\hspace*{1em}{<\textbf{idno}\hspace*{1em}{type}="{carter}">}256a{</\textbf{idno}>}\mbox{}\newline 
\hspace*{1em}{<\textbf{idno}\hspace*{1em}{type}="{JournalD'Entrée}">}60672{</\textbf{idno}>}\mbox{}\newline 
\hspace*{1em}{<\textbf{idno}\hspace*{1em}{type}="{exhibition}">}220{</\textbf{idno}>}\mbox{}\newline 
\hspace*{1em}{<\textbf{institution}>}Museum of Egyptian Antiquities{</\textbf{institution}>}\mbox{}\newline 
\hspace*{1em}{<\textbf{address}>}\mbox{}\newline 
\hspace*{1em}\hspace*{1em}{<\textbf{street}>}15 Meret Basha{</\textbf{street}>}\mbox{}\newline 
\hspace*{1em}\hspace*{1em}{<\textbf{district}>}Ismailia{</\textbf{district}>}\mbox{}\newline 
\hspace*{1em}\hspace*{1em}{<\textbf{settlement}>}Cairo{</\textbf{settlement}>}\mbox{}\newline 
\hspace*{1em}\hspace*{1em}{<\textbf{country}>}Egypt{</\textbf{country}>}\mbox{}\newline 
\hspace*{1em}\hspace*{1em}{<\textbf{location}>}\mbox{}\newline 
\hspace*{1em}\hspace*{1em}\hspace*{1em}{<\textbf{geo}>}30.047778, 31.233333{</\textbf{geo}>}\mbox{}\newline 
\hspace*{1em}\hspace*{1em}{</\textbf{location}>}\mbox{}\newline 
\hspace*{1em}{</\textbf{address}>}\mbox{}\newline 
{</\textbf{objectIdentifier}>}\end{shaded}\egroup\par \par
The \hyperref[TEI.msContents]{<msContents>} element is currently used to provide a description of the intellectual contents of any text on an object and, being optional, is not necessary if there are no intellectual contents to describe. (Such contents, especially in the case of artistic objects, may not always be textual.) The \hyperref[TEI.physDesc]{<physDesc>} element may be used to give a physical description of the object either in prose or using more structured elements as and where they apply to that kind of object. The \hyperref[TEI.history]{<history>} element provides the option to describe the history of the object as paragraphs or with more structure using the \hyperref[TEI.origin]{<origin>} element, as many \hyperref[TEI.provenance]{<provenance>} stages as are appropriate, and \hyperref[TEI.acquisition]{<acquisition>} to describe its current ownership. The \hyperref[TEI.additional]{<additional>} element may be used to provide information about surrogates for the object (such as digital facsimiles) as well as administrative and curatorial information. A full description of an object can provide more or less detail at any level to represent the state of knowledge about the object. \par\bgroup\index{listObject=<listObject>|exampleindex}\index{object=<object>|exampleindex}\index{objectIdentifier=<objectIdentifier>|exampleindex}\index{objectName=<objectName>|exampleindex}\index{idno=<idno>|exampleindex}\index{type=@type!<idno>|exampleindex}\index{idno=<idno>|exampleindex}\index{type=@type!<idno>|exampleindex}\index{idno=<idno>|exampleindex}\index{type=@type!<idno>|exampleindex}\index{institution=<institution>|exampleindex}\index{address=<address>|exampleindex}\index{street=<street>|exampleindex}\index{district=<district>|exampleindex}\index{settlement=<settlement>|exampleindex}\index{country=<country>|exampleindex}\index{location=<location>|exampleindex}\index{geo=<geo>|exampleindex}\index{msContents=<msContents>|exampleindex}\index{p=<p>|exampleindex}\index{title=<title>|exampleindex}\index{physDesc=<physDesc>|exampleindex}\index{p=<p>|exampleindex}\index{p=<p>|exampleindex}\index{history=<history>|exampleindex}\index{origin=<origin>|exampleindex}\index{p=<p>|exampleindex}\index{origPlace=<origPlace>|exampleindex}\index{origDate=<origDate>|exampleindex}\index{when=@when!<origDate>|exampleindex}\index{type=@type!<origDate>|exampleindex}\index{provenance=<provenance>|exampleindex}\index{p=<p>|exampleindex}\index{quote=<quote>|exampleindex}\index{acquisition=<acquisition>|exampleindex}\index{additional=<additional>|exampleindex}\index{adminInfo=<adminInfo>|exampleindex}\index{custodialHist=<custodialHist>|exampleindex}\index{custEvent=<custEvent>|exampleindex}\index{when=@when!<custEvent>|exampleindex}\index{custEvent=<custEvent>|exampleindex}\index{when=@when!<custEvent>|exampleindex}\index{custEvent=<custEvent>|exampleindex}\index{when=@when!<custEvent>|exampleindex}\exampleFont \begin{shaded}\noindent\mbox{}{<\textbf{listObject}>}\mbox{}\newline 
\hspace*{1em}{<\textbf{object}\hspace*{1em}{xml:id}="{TutankhamunMask}">}\mbox{}\newline 
\hspace*{1em}\hspace*{1em}{<\textbf{objectIdentifier}>}\mbox{}\newline 
\hspace*{1em}\hspace*{1em}\hspace*{1em}{<\textbf{objectName}\hspace*{1em}{xml:lang}="{en}">}Mask of Tutankhamun{</\textbf{objectName}>}\mbox{}\newline 
\hspace*{1em}\hspace*{1em}\hspace*{1em}{<\textbf{idno}\hspace*{1em}{type}="{carter}">}256a{</\textbf{idno}>}\mbox{}\newline 
\hspace*{1em}\hspace*{1em}\hspace*{1em}{<\textbf{idno}\hspace*{1em}{type}="{JournalD'Entrée}">}60672{</\textbf{idno}>}\mbox{}\newline 
\hspace*{1em}\hspace*{1em}\hspace*{1em}{<\textbf{idno}\hspace*{1em}{type}="{exhibition}">}220{</\textbf{idno}>}\mbox{}\newline 
\hspace*{1em}\hspace*{1em}\hspace*{1em}{<\textbf{institution}>}Museum of Egyptian Antiquities{</\textbf{institution}>}\mbox{}\newline 
\hspace*{1em}\hspace*{1em}\hspace*{1em}{<\textbf{address}>}\mbox{}\newline 
\hspace*{1em}\hspace*{1em}\hspace*{1em}\hspace*{1em}{<\textbf{street}>}15 Meret Basha{</\textbf{street}>}\mbox{}\newline 
\hspace*{1em}\hspace*{1em}\hspace*{1em}\hspace*{1em}{<\textbf{district}>}Ismailia{</\textbf{district}>}\mbox{}\newline 
\hspace*{1em}\hspace*{1em}\hspace*{1em}\hspace*{1em}{<\textbf{settlement}>}Cairo{</\textbf{settlement}>}\mbox{}\newline 
\hspace*{1em}\hspace*{1em}\hspace*{1em}\hspace*{1em}{<\textbf{country}>}Egypt{</\textbf{country}>}\mbox{}\newline 
\hspace*{1em}\hspace*{1em}\hspace*{1em}\hspace*{1em}{<\textbf{location}>}\mbox{}\newline 
\hspace*{1em}\hspace*{1em}\hspace*{1em}\hspace*{1em}\hspace*{1em}{<\textbf{geo}>}30.047778, 31.233333{</\textbf{geo}>}\mbox{}\newline 
\hspace*{1em}\hspace*{1em}\hspace*{1em}\hspace*{1em}{</\textbf{location}>}\mbox{}\newline 
\hspace*{1em}\hspace*{1em}\hspace*{1em}{</\textbf{address}>}\mbox{}\newline 
\hspace*{1em}\hspace*{1em}{</\textbf{objectIdentifier}>}\mbox{}\newline 
\hspace*{1em}\hspace*{1em}{<\textbf{msContents}>}\mbox{}\newline 
\hspace*{1em}\hspace*{1em}\hspace*{1em}{<\textbf{p}>}The back and shoulders of the mask is inscribed with a protective spell in Egyptian hieroglyphs formed of ten\mbox{}\newline 
\hspace*{1em}\hspace*{1em}\hspace*{1em}\hspace*{1em}\hspace*{1em}\hspace*{1em} vertical and horizontal lines. This spell first appeared on masks in the Middle Kingdom at least 500 years\mbox{}\newline 
\hspace*{1em}\hspace*{1em}\hspace*{1em}\hspace*{1em}\hspace*{1em}\hspace*{1em} before Tutankhamun, and comes from chapter 151 of the {<\textbf{title}>}Book of the Dead{</\textbf{title}>}.{</\textbf{p}>}\mbox{}\newline 
\hspace*{1em}\hspace*{1em}{</\textbf{msContents}>}\mbox{}\newline 
\hspace*{1em}\hspace*{1em}{<\textbf{physDesc}>}\mbox{}\newline 
\hspace*{1em}\hspace*{1em}\hspace*{1em}{<\textbf{p}>}The mask of Tutankhamun is 54cm x 39.3cm x 49cm. It is constructed from two layers of high-karat gold that\mbox{}\newline 
\hspace*{1em}\hspace*{1em}\hspace*{1em}\hspace*{1em}\hspace*{1em}\hspace*{1em} varies in thickness from 1.5-3mm. It weighs approximately 10.23kg and x-ray crystallography shows that it is\mbox{}\newline 
\hspace*{1em}\hspace*{1em}\hspace*{1em}\hspace*{1em}\hspace*{1em}\hspace*{1em} composed of two alloys of gold with a lighter 18.4 karat shade being used for the face and neck while a heavier\mbox{}\newline 
\hspace*{1em}\hspace*{1em}\hspace*{1em}\hspace*{1em}\hspace*{1em}\hspace*{1em} 22.5 karat gold was used for the rest of the mask.{</\textbf{p}>}\mbox{}\newline 
\hspace*{1em}\hspace*{1em}\hspace*{1em}{<\textbf{p}>}In the mask Tutankhamun wears a nemes headcloth which has the royal insignia of a cobra (Wadjet) and vulture\mbox{}\newline 
\hspace*{1em}\hspace*{1em}\hspace*{1em}\hspace*{1em}\hspace*{1em}\hspace*{1em} (Nekhbet) on it. These are thought respectively to symbolise Tutankhamun's rule of both Lower Egypt and Upper\mbox{}\newline 
\hspace*{1em}\hspace*{1em}\hspace*{1em}\hspace*{1em}\hspace*{1em}\hspace*{1em} Egypt. His ears are pierced for earrings. The mask has rich inlays of coloured glass and gemstones, including\mbox{}\newline 
\hspace*{1em}\hspace*{1em}\hspace*{1em}\hspace*{1em}\hspace*{1em}\hspace*{1em} lapis lazuli surrounding the eye and eyebrows, quartz for the eyes, obsidian for the pupils. The broad collar is\mbox{}\newline 
\hspace*{1em}\hspace*{1em}\hspace*{1em}\hspace*{1em}\hspace*{1em}\hspace*{1em} made up of carnelian, feldspar, turquoise, amazonite, faience and other stones.{</\textbf{p}>}\mbox{}\newline 
\hspace*{1em}\hspace*{1em}{</\textbf{physDesc}>}\mbox{}\newline 
\hspace*{1em}\hspace*{1em}{<\textbf{history}>}\mbox{}\newline 
\hspace*{1em}\hspace*{1em}\hspace*{1em}{<\textbf{origin}>}\mbox{}\newline 
\hspace*{1em}\hspace*{1em}\hspace*{1em}\hspace*{1em}{<\textbf{p}>}The mask of Tutankhamun was created in {<\textbf{origPlace}>}Egypt{</\textbf{origPlace}>} around {<\textbf{origDate}\hspace*{1em}{when}="{-1323}"\hspace*{1em}{type}="{circa}">}1323 BC{</\textbf{origDate}>}. It is a death mask of the 18th-dynasty ancient Egyptian Pharaoh Tutankhamun\mbox{}\newline 
\hspace*{1em}\hspace*{1em}\hspace*{1em}\hspace*{1em}\hspace*{1em}\hspace*{1em}\hspace*{1em}\hspace*{1em} who reigned 1332–1323 BC.{</\textbf{p}>}\mbox{}\newline 
\hspace*{1em}\hspace*{1em}\hspace*{1em}{</\textbf{origin}>}\mbox{}\newline 
\hspace*{1em}\hspace*{1em}\hspace*{1em}{<\textbf{provenance}>}\mbox{}\newline 
\hspace*{1em}\hspace*{1em}\hspace*{1em}\hspace*{1em}{<\textbf{p}>}The mask of Tutankhamun was found in his burial chamber at Theban Necropolis in the Valley of the Kings in\mbox{}\newline 
\hspace*{1em}\hspace*{1em}\hspace*{1em}\hspace*{1em}\hspace*{1em}\hspace*{1em}\hspace*{1em}\hspace*{1em} 1922. On 28 October 1925 the excavation team led by English archaeologist Howard Carter opened the heavy\mbox{}\newline 
\hspace*{1em}\hspace*{1em}\hspace*{1em}\hspace*{1em}\hspace*{1em}\hspace*{1em}\hspace*{1em}\hspace*{1em} sarcophagus and three coffins and were the first people in around 3,250 years to see the mask of Tutankhamun.\mbox{}\newline 
\hspace*{1em}\hspace*{1em}\hspace*{1em}\hspace*{1em}\hspace*{1em}\hspace*{1em}\hspace*{1em}\hspace*{1em} Carter wrote in his diary: {<\textbf{quote}>} The pins removed, the lid was raised. The penultimate scene was disclosed –\mbox{}\newline 
\hspace*{1em}\hspace*{1em}\hspace*{1em}\hspace*{1em}\hspace*{1em}\hspace*{1em}\hspace*{1em}\hspace*{1em}\hspace*{1em}\hspace*{1em} a very neatly wrapped mummy of the young king, with golden mask of sad but tranquil expression, symbolizing\mbox{}\newline 
\hspace*{1em}\hspace*{1em}\hspace*{1em}\hspace*{1em}\hspace*{1em}\hspace*{1em}\hspace*{1em}\hspace*{1em}\hspace*{1em}\hspace*{1em} Osiris … the mask bears that god's attributes, but the likeness is that of Tut.Ankh.Amen – placid and\mbox{}\newline 
\hspace*{1em}\hspace*{1em}\hspace*{1em}\hspace*{1em}\hspace*{1em}\hspace*{1em}\hspace*{1em}\hspace*{1em}\hspace*{1em}\hspace*{1em} beautiful, with the same features as we find upon his statues and coffins. The mask has fallen slightly\mbox{}\newline 
\hspace*{1em}\hspace*{1em}\hspace*{1em}\hspace*{1em}\hspace*{1em}\hspace*{1em}\hspace*{1em}\hspace*{1em}\hspace*{1em}\hspace*{1em} back, thus its gaze is straight up to the heavens. {</\textbf{quote}>}\mbox{}\newline 
\hspace*{1em}\hspace*{1em}\hspace*{1em}\hspace*{1em}{</\textbf{p}>}\mbox{}\newline 
\hspace*{1em}\hspace*{1em}\hspace*{1em}{</\textbf{provenance}>}\mbox{}\newline 
\hspace*{1em}\hspace*{1em}\hspace*{1em}{<\textbf{acquisition}>} In December 1925, the mask was removed from the tomb, placed in a crate and transported 635\mbox{}\newline 
\hspace*{1em}\hspace*{1em}\hspace*{1em}\hspace*{1em}\hspace*{1em}\hspace*{1em} kilometres (395 mi) to the Egyptian Museum in Cairo, where it remains on public display. {</\textbf{acquisition}>}\mbox{}\newline 
\hspace*{1em}\hspace*{1em}{</\textbf{history}>}\mbox{}\newline 
\hspace*{1em}\hspace*{1em}{<\textbf{additional}>}\mbox{}\newline 
\hspace*{1em}\hspace*{1em}\hspace*{1em}{<\textbf{adminInfo}>}\mbox{}\newline 
\hspace*{1em}\hspace*{1em}\hspace*{1em}\hspace*{1em}{<\textbf{custodialHist}>}\mbox{}\newline 
\hspace*{1em}\hspace*{1em}\hspace*{1em}\hspace*{1em}\hspace*{1em}{<\textbf{custEvent}\hspace*{1em}{when}="{1944}">}When it was discovered in 1925, the 2.5kg narrow gold beard was no longer attached to\mbox{}\newline 
\hspace*{1em}\hspace*{1em}\hspace*{1em}\hspace*{1em}\hspace*{1em}\hspace*{1em}\hspace*{1em}\hspace*{1em}\hspace*{1em}\hspace*{1em} the mask and was reattached to the chin by use of a wooden dowel in 1944.{</\textbf{custEvent}>}\mbox{}\newline 
\hspace*{1em}\hspace*{1em}\hspace*{1em}\hspace*{1em}\hspace*{1em}{<\textbf{custEvent}\hspace*{1em}{when}="{2014-08}">} In August 2014 when the mask was removed from its display case for cleaning, the\mbox{}\newline 
\hspace*{1em}\hspace*{1em}\hspace*{1em}\hspace*{1em}\hspace*{1em}\hspace*{1em}\hspace*{1em}\hspace*{1em}\hspace*{1em}\hspace*{1em} beard fell off again. Those working in the museum unadvisedly used a quick-drying epoxy to attempt to fix\mbox{}\newline 
\hspace*{1em}\hspace*{1em}\hspace*{1em}\hspace*{1em}\hspace*{1em}\hspace*{1em}\hspace*{1em}\hspace*{1em}\hspace*{1em}\hspace*{1em} it, but left the beard off-centre. {</\textbf{custEvent}>}\mbox{}\newline 
\hspace*{1em}\hspace*{1em}\hspace*{1em}\hspace*{1em}\hspace*{1em}{<\textbf{custEvent}\hspace*{1em}{when}="{2015-01}">}The damage was noticed and repaired in January 2015 by a German-Egyptian team who\mbox{}\newline 
\hspace*{1em}\hspace*{1em}\hspace*{1em}\hspace*{1em}\hspace*{1em}\hspace*{1em}\hspace*{1em}\hspace*{1em}\hspace*{1em}\hspace*{1em} used beeswax, a material known to be used as adhesives by the ancient Egyptians.{</\textbf{custEvent}>}\mbox{}\newline 
\hspace*{1em}\hspace*{1em}\hspace*{1em}\hspace*{1em}{</\textbf{custodialHist}>}\mbox{}\newline 
\hspace*{1em}\hspace*{1em}\hspace*{1em}{</\textbf{adminInfo}>}\mbox{}\newline 
\hspace*{1em}\hspace*{1em}{</\textbf{additional}>}\mbox{}\newline 
\hspace*{1em}{</\textbf{object}>}\mbox{}\newline 
{</\textbf{listObject}>}\end{shaded}\egroup\par \par
If the object is being referenced from elsewhere in the document, this is usually done with an \hyperref[TEI.objectName]{<objectName>}. For example here the Alfred-Jewel {\itshape xml:id} is referenced from a paragraph elsewhere in the document using the {\itshape ref} attribute on the \hyperref[TEI.objectName]{<objectName>} element. \par\bgroup\index{listObject=<listObject>|exampleindex}\index{object=<object>|exampleindex}\index{objectIdentifier=<objectIdentifier>|exampleindex}\index{country=<country>|exampleindex}\index{region=<region>|exampleindex}\index{settlement=<settlement>|exampleindex}\index{institution=<institution>|exampleindex}\index{repository=<repository>|exampleindex}\index{collection=<collection>|exampleindex}\index{idno=<idno>|exampleindex}\index{type=@type!<idno>|exampleindex}\index{idno=<idno>|exampleindex}\index{type=@type!<idno>|exampleindex}\index{objectName=<objectName>|exampleindex}\index{physDesc=<physDesc>|exampleindex}\index{p=<p>|exampleindex}\index{material=<material>|exampleindex}\index{material=<material>|exampleindex}\index{history=<history>|exampleindex}\index{origin=<origin>|exampleindex}\index{origDate=<origDate>|exampleindex}\index{origPlace=<origPlace>|exampleindex}\index{provenance=<provenance>|exampleindex}\index{when=@when!<provenance>|exampleindex}\index{provenance=<provenance>|exampleindex}\index{when=@when!<provenance>|exampleindex}\index{acquisition=<acquisition>|exampleindex}\index{p=<p>|exampleindex}\index{objectName=<objectName>|exampleindex}\index{ref=@ref!<objectName>|exampleindex}\index{objectName=<objectName>|exampleindex}\index{ref=@ref!<objectName>|exampleindex}\index{placeName=<placeName>|exampleindex}\index{ref=@ref!<placeName>|exampleindex}\index{placeName=<placeName>|exampleindex}\index{ref=@ref!<placeName>|exampleindex}\index{orgName=<orgName>|exampleindex}\index{ref=@ref!<orgName>|exampleindex}\exampleFont \begin{shaded}\noindent\mbox{}\mbox{}\newline 
\textit{<!-- Inside <standOff>: -->}{<\textbf{listObject}>}\mbox{}\newline 
\hspace*{1em}{<\textbf{object}\hspace*{1em}{xml:id}="{Alfred-Jewel}">}\mbox{}\newline 
\hspace*{1em}\hspace*{1em}{<\textbf{objectIdentifier}>}\mbox{}\newline 
\hspace*{1em}\hspace*{1em}\hspace*{1em}{<\textbf{country}>}United Kingdom{</\textbf{country}>}\mbox{}\newline 
\hspace*{1em}\hspace*{1em}\hspace*{1em}{<\textbf{region}>}Oxfordshire{</\textbf{region}>}\mbox{}\newline 
\hspace*{1em}\hspace*{1em}\hspace*{1em}{<\textbf{settlement}>}Oxford{</\textbf{settlement}>}\mbox{}\newline 
\hspace*{1em}\hspace*{1em}\hspace*{1em}{<\textbf{institution}>}University of Oxford{</\textbf{institution}>}\mbox{}\newline 
\hspace*{1em}\hspace*{1em}\hspace*{1em}{<\textbf{repository}>}Ashmolean Museum{</\textbf{repository}>}\mbox{}\newline 
\hspace*{1em}\hspace*{1em}\hspace*{1em}{<\textbf{collection}>}English Treasures{</\textbf{collection}>}\mbox{}\newline 
\hspace*{1em}\hspace*{1em}\hspace*{1em}{<\textbf{idno}\hspace*{1em}{type}="{ashmolean}">}AN1836p.135.371{</\textbf{idno}>}\mbox{}\newline 
\hspace*{1em}\hspace*{1em}\hspace*{1em}{<\textbf{idno}\hspace*{1em}{type}="{wikipedia}">}https://en.wikipedia.org/wiki/Alfred\textunderscore Jewel{</\textbf{idno}>}\mbox{}\newline 
\hspace*{1em}\hspace*{1em}\hspace*{1em}{<\textbf{objectName}>}Alfred Jewel{</\textbf{objectName}>}\mbox{}\newline 
\hspace*{1em}\hspace*{1em}{</\textbf{objectIdentifier}>}\mbox{}\newline 
\hspace*{1em}\hspace*{1em}{<\textbf{physDesc}>}\mbox{}\newline 
\hspace*{1em}\hspace*{1em}\hspace*{1em}{<\textbf{p}>}The Alfred Jewel is about 6.4 cm in length and is made of combination of filigreed {<\textbf{material}>}gold{</\textbf{material}>}\mbox{}\newline 
\hspace*{1em}\hspace*{1em}\hspace*{1em}\hspace*{1em}\hspace*{1em}\hspace*{1em} surrounding a polished teardrop shaped piece of transparent {<\textbf{material}>}quartz{</\textbf{material}>}. Underneath the rock\mbox{}\newline 
\hspace*{1em}\hspace*{1em}\hspace*{1em}\hspace*{1em}\hspace*{1em}\hspace*{1em} crystal is a cloisonné enamel image of a man with ecclesiastical symbols. The sides of the jewel holding the\mbox{}\newline 
\hspace*{1em}\hspace*{1em}\hspace*{1em}\hspace*{1em}\hspace*{1em}\hspace*{1em} crystal in place contain an openwork inscription saying "AELFRED MEC HEHT GEWYRCAN", meaning 'Alfred ordered me\mbox{}\newline 
\hspace*{1em}\hspace*{1em}\hspace*{1em}\hspace*{1em}\hspace*{1em}\hspace*{1em} made'.{</\textbf{p}>}\mbox{}\newline 
\hspace*{1em}\hspace*{1em}{</\textbf{physDesc}>}\mbox{}\newline 
\hspace*{1em}\hspace*{1em}{<\textbf{history}>}\mbox{}\newline 
\hspace*{1em}\hspace*{1em}\hspace*{1em}{<\textbf{origin}>}It is generally accepted that the Alfred Jewel dates from the {<\textbf{origDate}>}late 9th Century{</\textbf{origDate}>} and\mbox{}\newline 
\hspace*{1em}\hspace*{1em}\hspace*{1em}\hspace*{1em}\hspace*{1em}\hspace*{1em} was most likely made in {<\textbf{origPlace}>}England{</\textbf{origPlace}>}. {</\textbf{origin}>}\mbox{}\newline 
\hspace*{1em}\hspace*{1em}\hspace*{1em}{<\textbf{provenance}\hspace*{1em}{when}="{1693}">}The jewel was discovered in 1693 at Petherton Park, North Petherton in the English county\mbox{}\newline 
\hspace*{1em}\hspace*{1em}\hspace*{1em}\hspace*{1em}\hspace*{1em}\hspace*{1em} of Somerset, on land owned by Sir Thomas Wroth. North Petherton is about 8 miles away from Athelney, where King\mbox{}\newline 
\hspace*{1em}\hspace*{1em}\hspace*{1em}\hspace*{1em}\hspace*{1em}\hspace*{1em} Alfred founded a monastery. {</\textbf{provenance}>}\mbox{}\newline 
\hspace*{1em}\hspace*{1em}\hspace*{1em}{<\textbf{provenance}\hspace*{1em}{when}="{1698}">}A description of the Alfred Jewel was first published in 1698, in the Philosophical\mbox{}\newline 
\hspace*{1em}\hspace*{1em}\hspace*{1em}\hspace*{1em}\hspace*{1em}\hspace*{1em} Transactions of the Royal Society.{</\textbf{provenance}>}\mbox{}\newline 
\hspace*{1em}\hspace*{1em}\hspace*{1em}{<\textbf{acquisition}>} It was bequeathed to Oxford University by Colonel Nathaniel Palmer (c. 1661-1718) and today is in\mbox{}\newline 
\hspace*{1em}\hspace*{1em}\hspace*{1em}\hspace*{1em}\hspace*{1em}\hspace*{1em} the Ashmolean Museum in Oxford. {</\textbf{acquisition}>}\mbox{}\newline 
\hspace*{1em}\hspace*{1em}{</\textbf{history}>}\mbox{}\newline 
\hspace*{1em}{</\textbf{object}>}\mbox{}\newline 
{</\textbf{listObject}>}\mbox{}\newline 
\textit{<!-- Inside <text>: -->}\mbox{}\newline 
{<\textbf{p}>} The {<\textbf{objectName}\hspace*{1em}{ref}="{\#MinsterLovellJewel}">}Minster Lovell Jewel{</\textbf{objectName}>} is probably the most similar to the\mbox{}\newline 
{<\textbf{objectName}\hspace*{1em}{ref}="{\#Alfred-Jewel}">}Alfred Jewel{</\textbf{objectName}>} and was found in {<\textbf{placeName}\hspace*{1em}{ref}="{\#MinsterLovell}">}Minster\mbox{}\newline 
\hspace*{1em}\hspace*{1em} Lovell{</\textbf{placeName}>} in {<\textbf{placeName}\hspace*{1em}{ref}="{\#Oxfordshire}">}Oxfordshire{</\textbf{placeName}>} and is kept at the {<\textbf{orgName}\hspace*{1em}{ref}="{\#AshmoleanMuseum}">}Ashmolean Museum{</\textbf{orgName}>}.\mbox{}\newline 
{</\textbf{p}>}\end{shaded}\egroup\par \par
There is no restriction on the form, size, or type of object that may be described by an \hyperref[TEI.object]{<object>} element, however, some objects may be more adequately described by a \hyperref[TEI.place]{<place>} element depending on context. Where a description of an object is being provided in terms of identification, physical characteristics, or history, then an \hyperref[TEI.object]{<object>} element may be preferred. Where metadata is being recorded about the geo-political location, population, or similar traits, then the \hyperref[TEI.place]{<place>} element may be better suited. A corresponding relation between an object description and place may be recorded through the use of the {\itshape corresp} attribute. An example of a large object that might be described with the \hyperref[TEI.object]{<object>} element could be a building such as the Central Library of the National Autonomous University of Mexico.\par
\par\bgroup\index{object=<object>|exampleindex}\index{type=@type!<object>|exampleindex}\index{objectIdentifier=<objectIdentifier>|exampleindex}\index{objectName=<objectName>|exampleindex}\index{type=@type!<objectName>|exampleindex}\index{objectName=<objectName>|exampleindex}\index{type=@type!<objectName>|exampleindex}\index{objectName=<objectName>|exampleindex}\index{settlement=<settlement>|exampleindex}\index{region=<region>|exampleindex}\index{country=<country>|exampleindex}\index{physDesc=<physDesc>|exampleindex}\index{objectDesc=<objectDesc>|exampleindex}\index{p=<p>|exampleindex}\index{dim=<dim>|exampleindex}\index{unit=@unit!<dim>|exampleindex}\index{quantity=@quantity!<dim>|exampleindex}\index{type=@type!<dim>|exampleindex}\index{material=<material>|exampleindex}\index{objectType=<objectType>|exampleindex}\index{objectType=<objectType>|exampleindex}\index{p=<p>|exampleindex}\index{material=<material>|exampleindex}\index{p=<p>|exampleindex}\index{title=<title>|exampleindex}\index{persName=<persName>|exampleindex}\index{role=@role!<persName>|exampleindex}\index{decoDesc=<decoDesc>|exampleindex}\index{decoNote=<decoNote>|exampleindex}\index{label=<label>|exampleindex}\index{decoNote=<decoNote>|exampleindex}\index{label=<label>|exampleindex}\index{p=<p>|exampleindex}\index{title=<title>|exampleindex}\index{material=<material>|exampleindex}\index{persName=<persName>|exampleindex}\index{role=@role!<persName>|exampleindex}\index{ref=<ref>|exampleindex}\index{target=@target!<ref>|exampleindex}\index{history=<history>|exampleindex}\index{origin=<origin>|exampleindex}\index{origDate=<origDate>|exampleindex}\index{when=@when!<origDate>|exampleindex}\index{type=@type!<origDate>|exampleindex}\index{origDate=<origDate>|exampleindex}\index{when=@when!<origDate>|exampleindex}\index{type=@type!<origDate>|exampleindex}\index{additional=<additional>|exampleindex}\index{adminInfo=<adminInfo>|exampleindex}\index{custodialHist=<custodialHist>|exampleindex}\index{custEvent=<custEvent>|exampleindex}\index{from=@from!<custEvent>|exampleindex}\index{to=@to!<custEvent>|exampleindex}\exampleFont \begin{shaded}\noindent\mbox{}{<\textbf{object}\hspace*{1em}{type}="{building}"\hspace*{1em}{xml:lang}="{en}"\mbox{}\newline 
\hspace*{1em}{xml:id}="{UNAM-CL}">}\mbox{}\newline 
\hspace*{1em}{<\textbf{objectIdentifier}>}\mbox{}\newline 
\hspace*{1em}\hspace*{1em}{<\textbf{objectName}\hspace*{1em}{type}="{abbr}">}The Central Library of UNAM{</\textbf{objectName}>}\mbox{}\newline 
\hspace*{1em}\hspace*{1em}{<\textbf{objectName}\hspace*{1em}{type}="{full}">}The Central Library of the National Autonomous University of Mexico{</\textbf{objectName}>}\mbox{}\newline 
\hspace*{1em}\hspace*{1em}{<\textbf{objectName}\hspace*{1em}{xml:lang}="{es}">}La Biblioteca Central de la Universidad Nacional Autónoma de México{</\textbf{objectName}>}\mbox{}\newline 
\hspace*{1em}\hspace*{1em}{<\textbf{settlement}>}Mexico City{</\textbf{settlement}>}\mbox{}\newline 
\hspace*{1em}\hspace*{1em}{<\textbf{region}>}Coyoacán{</\textbf{region}>}\mbox{}\newline 
\hspace*{1em}\hspace*{1em}{<\textbf{country}>}Mexico{</\textbf{country}>}\mbox{}\newline 
\hspace*{1em}{</\textbf{objectIdentifier}>}\mbox{}\newline 
\hspace*{1em}{<\textbf{physDesc}>}\mbox{}\newline 
\hspace*{1em}\hspace*{1em}{<\textbf{objectDesc}>}\mbox{}\newline 
\hspace*{1em}\hspace*{1em}\hspace*{1em}{<\textbf{p}>}The Central Library encompasses an area of {<\textbf{dim}\hspace*{1em}{unit}="{m}"\hspace*{1em}{quantity}="{16000}"\mbox{}\newline 
\hspace*{1em}\hspace*{1em}\hspace*{1em}\hspace*{1em}\hspace*{1em}{type}="{area}">}16 thousand square\mbox{}\newline 
\hspace*{1em}\hspace*{1em}\hspace*{1em}\hspace*{1em}\hspace*{1em}\hspace*{1em}\hspace*{1em}\hspace*{1em} meters{</\textbf{dim}>} and is built on a three meter platform. The base contains two {<\textbf{material}>}basalt{</\textbf{material}>}\mbox{}\newline 
\hspace*{1em}\hspace*{1em}\hspace*{1em}\hspace*{1em}{<\textbf{objectType}>}fountains{</\textbf{objectType}>} and {<\textbf{objectType}>}decorative reliefs{</\textbf{objectType}>} inspired by\mbox{}\newline 
\hspace*{1em}\hspace*{1em}\hspace*{1em}\hspace*{1em}\hspace*{1em}\hspace*{1em} pre-Hispanic art.{</\textbf{p}>}\mbox{}\newline 
\hspace*{1em}\hspace*{1em}\hspace*{1em}{<\textbf{p}>}The library has ten windowless floors for book storage, each having enough space for 120 thousand volumes.\mbox{}\newline 
\hspace*{1em}\hspace*{1em}\hspace*{1em}\hspace*{1em}\hspace*{1em}\hspace*{1em} These storage areas have the necessary lighting, temperature and humidty conditions for book conservation.\mbox{}\newline 
\hspace*{1em}\hspace*{1em}\hspace*{1em}\hspace*{1em}\hspace*{1em}\hspace*{1em} In the reading room, flanked by a garden on each side, the diffuse and matte light is filtered through\mbox{}\newline 
\hspace*{1em}\hspace*{1em}\hspace*{1em}{<\textbf{material}>}thin tecali stone slabs{</\textbf{material}>}. The semi-basement of the building contains the service\mbox{}\newline 
\hspace*{1em}\hspace*{1em}\hspace*{1em}\hspace*{1em}\hspace*{1em}\hspace*{1em} and administrative offices of the library.{</\textbf{p}>}\mbox{}\newline 
\hspace*{1em}\hspace*{1em}\hspace*{1em}{<\textbf{p}>}The building facades are covered with one of the largest murals in the world and is made from naturally\mbox{}\newline 
\hspace*{1em}\hspace*{1em}\hspace*{1em}\hspace*{1em}\hspace*{1em}\hspace*{1em} colored stone tiles. It is entitled {<\textbf{title}>}Historical Representation of Culture{</\textbf{title}>} and is by\mbox{}\newline 
\hspace*{1em}\hspace*{1em}\hspace*{1em}{<\textbf{persName}\hspace*{1em}{role}="{artist}">}Juan O'Gorman{</\textbf{persName}>}.{</\textbf{p}>}\mbox{}\newline 
\hspace*{1em}\hspace*{1em}{</\textbf{objectDesc}>}\mbox{}\newline 
\hspace*{1em}\hspace*{1em}{<\textbf{decoDesc}>}\mbox{}\newline 
\hspace*{1em}\hspace*{1em}\hspace*{1em}{<\textbf{decoNote}>}\mbox{}\newline 
\hspace*{1em}\hspace*{1em}\hspace*{1em}\hspace*{1em}{<\textbf{label}>}Base{</\textbf{label}>} At the base of the building there are two basalt fountains and decorative reliefs\mbox{}\newline 
\hspace*{1em}\hspace*{1em}\hspace*{1em}\hspace*{1em}\hspace*{1em}\hspace*{1em} around the outside that are inspired by pre-Hispanic art. The color of the stone in these elements is left\mbox{}\newline 
\hspace*{1em}\hspace*{1em}\hspace*{1em}\hspace*{1em}\hspace*{1em}\hspace*{1em} exposed to take advantage of the stone's texture as an aesthetic and expressive element, and to give a\mbox{}\newline 
\hspace*{1em}\hspace*{1em}\hspace*{1em}\hspace*{1em}\hspace*{1em}\hspace*{1em} sense of continuity to the external pavement.{</\textbf{decoNote}>}\mbox{}\newline 
\hspace*{1em}\hspace*{1em}\hspace*{1em}{<\textbf{decoNote}>}\mbox{}\newline 
\hspace*{1em}\hspace*{1em}\hspace*{1em}\hspace*{1em}{<\textbf{label}>}Murals{</\textbf{label}>}\mbox{}\newline 
\hspace*{1em}\hspace*{1em}\hspace*{1em}\hspace*{1em}{<\textbf{p}>}The outside windowless portion of the building contains one of the largest murals in the world. This is\mbox{}\newline 
\hspace*{1em}\hspace*{1em}\hspace*{1em}\hspace*{1em}\hspace*{1em}\hspace*{1em}\hspace*{1em}\hspace*{1em} called {<\textbf{title}>}Historical Representation of the Culture{</\textbf{title}>} and is a {<\textbf{material}>}stone polychromatic\mbox{}\newline 
\hspace*{1em}\hspace*{1em}\hspace*{1em}\hspace*{1em}\hspace*{1em}\hspace*{1em}\hspace*{1em}\hspace*{1em}\hspace*{1em}\hspace*{1em} mosaic{</\textbf{material}>} based on the combination of 12 basic colors. The mural is created in an\mbox{}\newline 
\hspace*{1em}\hspace*{1em}\hspace*{1em}\hspace*{1em}\hspace*{1em}\hspace*{1em}\hspace*{1em}\hspace*{1em} impressionist style where the coloured tiles when seen from a distance form specific figures. The 12\mbox{}\newline 
\hspace*{1em}\hspace*{1em}\hspace*{1em}\hspace*{1em}\hspace*{1em}\hspace*{1em}\hspace*{1em}\hspace*{1em} colors where chosen from 150 samples of original stones with the criteria including the stone's\mbox{}\newline 
\hspace*{1em}\hspace*{1em}\hspace*{1em}\hspace*{1em}\hspace*{1em}\hspace*{1em}\hspace*{1em}\hspace*{1em} resistance to degredation by weather. According to the artist, {<\textbf{persName}\hspace*{1em}{role}="{artist}">}Juan\mbox{}\newline 
\hspace*{1em}\hspace*{1em}\hspace*{1em}\hspace*{1em}\hspace*{1em}\hspace*{1em}\hspace*{1em}\hspace*{1em}\hspace*{1em}\hspace*{1em} O'Gorman{</\textbf{persName}>}, in the mosaic he represented three fundamental historical facets of the Mexican\mbox{}\newline 
\hspace*{1em}\hspace*{1em}\hspace*{1em}\hspace*{1em}\hspace*{1em}\hspace*{1em}\hspace*{1em}\hspace*{1em} culture: the pre-Hispanic era, the Spanish colonial era, and the modern age. For example with the North\mbox{}\newline 
\hspace*{1em}\hspace*{1em}\hspace*{1em}\hspace*{1em}\hspace*{1em}\hspace*{1em}\hspace*{1em}\hspace*{1em} Wall, this represents the pre-Hispanic era and is dominated by mythical elements relating to the\mbox{}\newline 
\hspace*{1em}\hspace*{1em}\hspace*{1em}\hspace*{1em}\hspace*{1em}\hspace*{1em}\hspace*{1em}\hspace*{1em} life-death duality. The left side of the main axis there are\mbox{}\newline 
\hspace*{1em}\hspace*{1em}\hspace*{1em}\hspace*{1em}\hspace*{1em}\hspace*{1em}\hspace*{1em}\hspace*{1em} deities and scenes pertaining to the creation of life. The right hand side of the mural contains figures\mbox{}\newline 
\hspace*{1em}\hspace*{1em}\hspace*{1em}\hspace*{1em}\hspace*{1em}\hspace*{1em}\hspace*{1em}\hspace*{1em} relating to death. For a more detailed description see {<\textbf{ref}\hspace*{1em}{target}="{https://en.wikipedia.org/wiki/Central\textunderscore Library\textunderscore (UNAM)\#Murals}">}https://en.wikipedia.org/wiki/Central\textunderscore Library\textunderscore (UNAM)\#Murals{</\textbf{ref}>}.{</\textbf{p}>}\mbox{}\newline 
\hspace*{1em}\hspace*{1em}\hspace*{1em}{</\textbf{decoNote}>}\mbox{}\newline 
\hspace*{1em}\hspace*{1em}{</\textbf{decoDesc}>}\mbox{}\newline 
\hspace*{1em}{</\textbf{physDesc}>}\mbox{}\newline 
\hspace*{1em}{<\textbf{history}>}\mbox{}\newline 
\hspace*{1em}\hspace*{1em}{<\textbf{origin}>} In {<\textbf{origDate}\hspace*{1em}{when}="{1948}"\hspace*{1em}{type}="{conceptual}">}1948{</\textbf{origDate}>} the architect and artist Juan O'Gorman, in\mbox{}\newline 
\hspace*{1em}\hspace*{1em}\hspace*{1em}\hspace*{1em} collaboration with architects Gustavo Saavedra and Juan Martinez de Velasco designed the building with a\mbox{}\newline 
\hspace*{1em}\hspace*{1em}\hspace*{1em}\hspace*{1em} functionalist approach, as part of the greater project of the construction of the University City on the\mbox{}\newline 
\hspace*{1em}\hspace*{1em}\hspace*{1em}\hspace*{1em} grounds of the Pedregal de San Angel in Mexico City. Originally the building was planned to host the National\mbox{}\newline 
\hspace*{1em}\hspace*{1em}\hspace*{1em}\hspace*{1em} Library and National Newspaper Library of Mexico. The library finally opened its doors for the first time on\mbox{}\newline 
\hspace*{1em}\hspace*{1em}{<\textbf{origDate}\hspace*{1em}{when}="{1956-04-05}"\mbox{}\newline 
\hspace*{1em}\hspace*{1em}\hspace*{1em}\hspace*{1em}{type}="{opening}">}5 April 1956{</\textbf{origDate}>}. In July 2007 it was declared a UNESCO\mbox{}\newline 
\hspace*{1em}\hspace*{1em}\hspace*{1em}\hspace*{1em} world heritage site. {</\textbf{origin}>}\mbox{}\newline 
\hspace*{1em}{</\textbf{history}>}\mbox{}\newline 
\hspace*{1em}{<\textbf{additional}>}\mbox{}\newline 
\hspace*{1em}\hspace*{1em}{<\textbf{adminInfo}>}\mbox{}\newline 
\hspace*{1em}\hspace*{1em}\hspace*{1em}{<\textbf{custodialHist}>}\mbox{}\newline 
\hspace*{1em}\hspace*{1em}\hspace*{1em}\hspace*{1em}{<\textbf{custEvent}\hspace*{1em}{from}="{1981}"\hspace*{1em}{to}="{1983}">} The library was significantly remodelled from 1981 - 1983 with the\mbox{}\newline 
\hspace*{1em}\hspace*{1em}\hspace*{1em}\hspace*{1em}\hspace*{1em}\hspace*{1em}\hspace*{1em}\hspace*{1em} purpose of changing from closed shelving to open stacks, providing users more direct access to the\mbox{}\newline 
\hspace*{1em}\hspace*{1em}\hspace*{1em}\hspace*{1em}\hspace*{1em}\hspace*{1em}\hspace*{1em}\hspace*{1em} collections. {</\textbf{custEvent}>}\mbox{}\newline 
\hspace*{1em}\hspace*{1em}\hspace*{1em}{</\textbf{custodialHist}>}\mbox{}\newline 
\hspace*{1em}\hspace*{1em}{</\textbf{adminInfo}>}\mbox{}\newline 
\hspace*{1em}{</\textbf{additional}>}\mbox{}\newline 
{</\textbf{object}>}\end{shaded}\egroup\par 
\subsubsection[{Names and Nyms}]{Names and Nyms}\label{NDNYM}\par
So far we have discussed ways in which a name or referring string encountered in running text may be resolved by considering the object that the name refers to: in the case of a personal name, the name refers to a person; in the case of a place name, to a place, for example. The resolution of this reference is effected by means of the {\itshape key} or {\itshape ref} attributes available to all elements which are members of the \textsf{att.naming} class, such as \hyperref[TEI.persName]{<persName>} or \hyperref[TEI.placeName]{<placeName>} and their more specialized variants such as \hyperref[TEI.forename]{<forename>} or \hyperref[TEI.country]{<country>}. However, \textit{names} can also be regarded as objects in their own right, irrespective of the objects to which they are attached, notably in onomastic studies. From this point of view, the names \textit{John} in English, \textit{Jean} in French, and \textit{Ivan} in Russian might all be regarded as existing independently of any person to which they are attached, and also independently of any variant forms that might be attested in different sources (such as Jon or Johnny in English, or Jehan or Jojo in French). We use the term \textit{nym} to refer to the canonical or normalized form of a name regarded in such a way, and provide the following elements to encode it: 
\begin{sansreflist}
  
\item [\textbf{<listNym>}] (list of canonical names) contains a list of nyms, that is, standardized names for any thing.
\item [\textbf{<nym>}] (canonical name) contains the definition for a canonical name or name component of any kind.
\end{sansreflist}
\par
Any element which is a member of the \textsf{att.naming} class may use the attribute {\itshape nymRef} to indicate the nym with which it corresponds. Thus, given the following \hyperref[TEI.nym]{<nym>} for the name \textit{Antony}: \par\bgroup\index{listNym=<listNym>|exampleindex}\index{nym=<nym>|exampleindex}\index{form=<form>|exampleindex}\exampleFont \begin{shaded}\noindent\mbox{}{<\textbf{listNym}>}\mbox{}\newline 
\hspace*{1em}{<\textbf{nym}\hspace*{1em}{xml:id}="{N123}">}\mbox{}\newline 
\hspace*{1em}\hspace*{1em}{<\textbf{form}>}Antony{</\textbf{form}>}\mbox{}\newline 
\hspace*{1em}{</\textbf{nym}>}\mbox{}\newline 
\textit{<!-- other nym definitions here -->}\mbox{}\newline 
{</\textbf{listNym}>}\end{shaded}\egroup\par \noindent  an occurrence of this name in running text might be encoded as follows: \par\bgroup\index{forename=<forename>|exampleindex}\index{nymRef=@nymRef!<forename>|exampleindex}\exampleFont \begin{shaded}\noindent\mbox{}{<\textbf{forename}\hspace*{1em}{nymRef}="{\#N123}">}Tony{</\textbf{forename}>} Blair \end{shaded}\egroup\par \noindent  Note that this association (between "Tony" and "Antony") has nothing to do with any individual who might use the name.\par
The person identified by this particular Tony may however be indicated independently using the {\itshape ref} attribute, either on the forename or on the whole name component: \par\bgroup\index{forename=<forename>|exampleindex}\index{nymRef=@nymRef!<forename>|exampleindex}\index{ref=@ref!<forename>|exampleindex}\index{person=<person>|exampleindex}\index{persName=<persName>|exampleindex}\index{occupation=<occupation>|exampleindex}\exampleFont \begin{shaded}\noindent\mbox{}{<\textbf{forename}\hspace*{1em}{nymRef}="{\#N123}"\hspace*{1em}{ref}="{\#BLT}">}Tony{</\textbf{forename}>}\mbox{}\newline 
\textit{<!-- ... -->}\mbox{}\newline 
{<\textbf{person}\hspace*{1em}{xml:id}="{BLT}">}\mbox{}\newline 
\hspace*{1em}{<\textbf{persName}>}Tony Blair{</\textbf{persName}>}\mbox{}\newline 
\hspace*{1em}{<\textbf{occupation}>}politician{</\textbf{occupation}>}\mbox{}\newline 
{</\textbf{person}>}\end{shaded}\egroup\par \par
The \hyperref[TEI.nym]{<nym>} element may be thought of as providing a specialized kind of dictionary entry. Like a dictionary entry, it may contain any element from the \textsf{model.entryPart} class, such as \hyperref[TEI.form]{<form>}, \hyperref[TEI.etym]{<etym>}, etc. For example, we may show that the canonical form for a given nym has two orthographic variants in this way: \par\bgroup\index{nym=<nym>|exampleindex}\index{form=<form>|exampleindex}\index{orth=<orth>|exampleindex}\index{orth=<orth>|exampleindex}\exampleFont \begin{shaded}\noindent\mbox{}{<\textbf{nym}\hspace*{1em}{xml:id}="{J451}">}\mbox{}\newline 
\hspace*{1em}{<\textbf{form}>}\mbox{}\newline 
\hspace*{1em}\hspace*{1em}{<\textbf{orth}\hspace*{1em}{xml:lang}="{en-US}">}Ian{</\textbf{orth}>}\mbox{}\newline 
\hspace*{1em}\hspace*{1em}{<\textbf{orth}\hspace*{1em}{xml:lang}="{en-x-Scots}">}Iain{</\textbf{orth}>}\mbox{}\newline 
\hspace*{1em}{</\textbf{form}>}\mbox{}\newline 
{</\textbf{nym}>}\end{shaded}\egroup\par \par
Because a schema intending to make use of the \hyperref[TEI.nym]{<nym>} or \hyperref[TEI.listNym]{<listNym>} element must include the \textsf{dictionaries} module as well as the \textsf{namesdates} module, many other elements are available in addition to \hyperref[TEI.form]{<form>}. For example, to provide a more complex etymological decomposition of a name, we might use the existing \hyperref[TEI.etym]{<etym>} element, as follows: \par\bgroup\index{nym=<nym>|exampleindex}\index{form=<form>|exampleindex}\index{etym=<etym>|exampleindex}\index{gloss=<gloss>|exampleindex}\index{lang=<lang>|exampleindex}\index{mentioned=<mentioned>|exampleindex}\index{gloss=<gloss>|exampleindex}\index{mentioned=<mentioned>|exampleindex}\index{gloss=<gloss>|exampleindex}\exampleFont \begin{shaded}\noindent\mbox{}{<\textbf{nym}\hspace*{1em}{xml:id}="{XYZ}">}\mbox{}\newline 
\hspace*{1em}{<\textbf{form}>}Bogomil{</\textbf{form}>}\mbox{}\newline 
\hspace*{1em}{<\textbf{etym}>}Means {<\textbf{gloss}>}favoured by God{</\textbf{gloss}>} from the {<\textbf{lang}>}Slavic{</\textbf{lang}>} elements {<\textbf{mentioned}\hspace*{1em}{xml:lang}="{ru}">}bog{</\textbf{mentioned}>}\mbox{}\newline 
\hspace*{1em}\hspace*{1em}{<\textbf{gloss}>}God{</\textbf{gloss}>} and {<\textbf{mentioned}\hspace*{1em}{xml:lang}="{ru}">}mil{</\textbf{mentioned}>}\mbox{}\newline 
\hspace*{1em}\hspace*{1em}{<\textbf{gloss}>}favour{</\textbf{gloss}>}\mbox{}\newline 
\hspace*{1em}{</\textbf{etym}>}\mbox{}\newline 
{</\textbf{nym}>}\end{shaded}\egroup\par \par
Where it is necessary to mark the substructure of nyms, this may be done by \hyperref[TEI.seg]{<seg>} elements within the \hyperref[TEI.form]{<form>}:\par\bgroup\index{nym=<nym>|exampleindex}\index{form=<form>|exampleindex}\index{choice=<choice>|exampleindex}\index{seg=<seg>|exampleindex}\index{type=@type!<seg>|exampleindex}\index{seg=<seg>|exampleindex}\index{seg=<seg>|exampleindex}\index{seg=<seg>|exampleindex}\index{seg=<seg>|exampleindex}\index{type=@type!<seg>|exampleindex}\index{seg=<seg>|exampleindex}\index{seg=<seg>|exampleindex}\exampleFont \begin{shaded}\noindent\mbox{}{<\textbf{nym}\hspace*{1em}{xml:id}="{ABC}">}\mbox{}\newline 
\hspace*{1em}{<\textbf{form}>}\mbox{}\newline 
\hspace*{1em}\hspace*{1em}{<\textbf{choice}>}\mbox{}\newline 
\hspace*{1em}\hspace*{1em}\hspace*{1em}{<\textbf{seg}\hspace*{1em}{type}="{morph}">}\mbox{}\newline 
\hspace*{1em}\hspace*{1em}\hspace*{1em}\hspace*{1em}{<\textbf{seg}>}Bog{</\textbf{seg}>}\mbox{}\newline 
\hspace*{1em}\hspace*{1em}\hspace*{1em}\hspace*{1em}{<\textbf{seg}>}o{</\textbf{seg}>}\mbox{}\newline 
\hspace*{1em}\hspace*{1em}\hspace*{1em}\hspace*{1em}{<\textbf{seg}>}mil{</\textbf{seg}>}\mbox{}\newline 
\hspace*{1em}\hspace*{1em}\hspace*{1em}{</\textbf{seg}>}\mbox{}\newline 
\hspace*{1em}\hspace*{1em}\hspace*{1em}{<\textbf{seg}\hspace*{1em}{type}="{morph}">}\mbox{}\newline 
\hspace*{1em}\hspace*{1em}\hspace*{1em}\hspace*{1em}{<\textbf{seg}>}Bogo{</\textbf{seg}>}\mbox{}\newline 
\hspace*{1em}\hspace*{1em}\hspace*{1em}\hspace*{1em}{<\textbf{seg}>}mil{</\textbf{seg}>}\mbox{}\newline 
\hspace*{1em}\hspace*{1em}\hspace*{1em}{</\textbf{seg}>}\mbox{}\newline 
\hspace*{1em}\hspace*{1em}{</\textbf{choice}>}\mbox{}\newline 
\hspace*{1em}{</\textbf{form}>}\mbox{}\newline 
{</\textbf{nym}>}\end{shaded}\egroup\par \noindent   The \hyperref[TEI.seg]{<seg>} element used here is provided by the TEI \textsf{linking} module, which would therefore also need to be included in a schema built to validate such markup. Other possibilities for more detailed linguistic analysis are provided by elements included in that and the \textsf{analysis} (see \textit{\hyperref[AI]{17.\ Simple Analytic Mechanisms}}) or \textsf{iso-fs} modules (see \textit{\hyperref[FS]{18.\ Feature Structures}}).\par
Alternatively, each of the constituents of \textit{Bogomil} might be regarded as a nym in its own right: \par\bgroup\index{nym=<nym>|exampleindex}\index{type=@type!<nym>|exampleindex}\index{form=<form>|exampleindex}\index{nym=<nym>|exampleindex}\index{type=@type!<nym>|exampleindex}\index{form=<form>|exampleindex}\exampleFont \begin{shaded}\noindent\mbox{}{<\textbf{nym}\hspace*{1em}{xml:id}="{B1}"\hspace*{1em}{type}="{part}">}\mbox{}\newline 
\hspace*{1em}{<\textbf{form}>}bog{</\textbf{form}>}\mbox{}\newline 
{</\textbf{nym}>}\mbox{}\newline 
{<\textbf{nym}\hspace*{1em}{xml:id}="{M1}"\hspace*{1em}{type}="{part}">}\mbox{}\newline 
\hspace*{1em}{<\textbf{form}>}mil{</\textbf{form}>}\mbox{}\newline 
{</\textbf{nym}>}\end{shaded}\egroup\par \noindent  Within running text, a name can specify all the nyms associated with it: \par\bgroup\index{name=<name>|exampleindex}\index{nymRef=@nymRef!<name>|exampleindex}\exampleFont \begin{shaded}\noindent\mbox{} ...{<\textbf{name}\hspace*{1em}{nymRef}="{\#B1 \#M1}">}Bogomil{</\textbf{name}>}... \end{shaded}\egroup\par \noindent  Similarly, within a nym, the attribute {\itshape parts} is used to indicate its constituent parts, where these have been identified as distinct nyms: \par\bgroup\index{nym=<nym>|exampleindex}\index{parts=@parts!<nym>|exampleindex}\index{form=<form>|exampleindex}\exampleFont \begin{shaded}\noindent\mbox{}{<\textbf{nym}\hspace*{1em}{xml:id}="{BM1}"\hspace*{1em}{parts}="{\#B1 \#M1}">}\mbox{}\newline 
\hspace*{1em}{<\textbf{form}>}Bogomil{</\textbf{form}>}\mbox{}\newline 
{</\textbf{nym}>}\end{shaded}\egroup\par \par
The \hyperref[TEI.nym]{<nym>} element may also combine a number of other \hyperref[TEI.nym]{<nym>} elements together, where it is intended to show that they are all regarded as variations on the same root. Thus the different forms of the name John, all being derived from the same root, may be represented as a hierarchic structure like this: \par\bgroup\index{nym=<nym>|exampleindex}\index{form=<form>|exampleindex}\index{nym=<nym>|exampleindex}\index{form=<form>|exampleindex}\index{nym=<nym>|exampleindex}\index{form=<form>|exampleindex}\index{nym=<nym>|exampleindex}\index{form=<form>|exampleindex}\index{nym=<nym>|exampleindex}\index{form=<form>|exampleindex}\index{nym=<nym>|exampleindex}\index{form=<form>|exampleindex}\exampleFont \begin{shaded}\noindent\mbox{}{<\textbf{nym}\hspace*{1em}{xml:id}="{J45}">}\mbox{}\newline 
\hspace*{1em}{<\textbf{form}\hspace*{1em}{xml:lang}="{la}">}Iohannes{</\textbf{form}>}\mbox{}\newline 
\hspace*{1em}{<\textbf{nym}\hspace*{1em}{xml:id}="{J450}">}\mbox{}\newline 
\hspace*{1em}\hspace*{1em}{<\textbf{form}\hspace*{1em}{xml:lang}="{en}">}John{</\textbf{form}>}\mbox{}\newline 
\hspace*{1em}\hspace*{1em}{<\textbf{nym}\hspace*{1em}{xml:id}="{J4501}">}\mbox{}\newline 
\hspace*{1em}\hspace*{1em}\hspace*{1em}{<\textbf{form}>}Johnny{</\textbf{form}>}\mbox{}\newline 
\hspace*{1em}\hspace*{1em}{</\textbf{nym}>}\mbox{}\newline 
\hspace*{1em}\hspace*{1em}{<\textbf{nym}\hspace*{1em}{xml:id}="{J4502}">}\mbox{}\newline 
\hspace*{1em}\hspace*{1em}\hspace*{1em}{<\textbf{form}>}Jon{</\textbf{form}>}\mbox{}\newline 
\hspace*{1em}\hspace*{1em}{</\textbf{nym}>}\mbox{}\newline 
\hspace*{1em}{</\textbf{nym}>}\mbox{}\newline 
\hspace*{1em}{<\textbf{nym}\hspace*{1em}{xml:id}="{J455}">}\mbox{}\newline 
\hspace*{1em}\hspace*{1em}{<\textbf{form}\hspace*{1em}{xml:lang}="{ru}">}Ivan{</\textbf{form}>}\mbox{}\newline 
\hspace*{1em}{</\textbf{nym}>}\mbox{}\newline 
\hspace*{1em}{<\textbf{nym}\hspace*{1em}{xml:id}="{J453}">}\mbox{}\newline 
\hspace*{1em}\hspace*{1em}{<\textbf{form}\hspace*{1em}{xml:lang}="{fr}">}Jean{</\textbf{form}>}\mbox{}\newline 
\hspace*{1em}{</\textbf{nym}>}\mbox{}\newline 
{</\textbf{nym}>}\end{shaded}\egroup\par \par
The \hyperref[TEI.nym]{<nym>} element may be used for components of geographical or organizational names as well. For example: \par\bgroup\index{geogName=<geogName>|exampleindex}\index{ref=@ref!<geogName>|exampleindex}\index{type=@type!<geogName>|exampleindex}\index{geogFeat=<geogFeat>|exampleindex}\index{nymRef=@nymRef!<geogFeat>|exampleindex}\index{name=<name>|exampleindex}\index{nym=<nym>|exampleindex}\index{form=<form>|exampleindex}\index{def=<def>|exampleindex}\exampleFont \begin{shaded}\noindent\mbox{}{<\textbf{geogName}\hspace*{1em}{ref}="{tag:projectname.org,2012:LAEI1}"\mbox{}\newline 
\hspace*{1em}{type}="{hill}">}\mbox{}\newline 
\hspace*{1em}{<\textbf{geogFeat}\hspace*{1em}{xml:lang}="{gd}"\hspace*{1em}{nymRef}="{\#LAIRG}">}Lairig{</\textbf{geogFeat}>}\mbox{}\newline 
\hspace*{1em}{<\textbf{name}>}Eilde{</\textbf{name}>}\mbox{}\newline 
{</\textbf{geogName}>} ... {<\textbf{nym}\hspace*{1em}{xml:id}="{LAIRG}">}\mbox{}\newline 
\hspace*{1em}{<\textbf{form}\hspace*{1em}{xml:lang}="{gd}">}lairig{</\textbf{form}>}\mbox{}\newline 
\hspace*{1em}{<\textbf{def}>}sloping hill face{</\textbf{def}>}\mbox{}\newline 
{</\textbf{nym}>} ... \end{shaded}\egroup\par \par
As noted above, use of these elements implies that both the \textsf{dictionaries} and the \textsf{namesdates} modules are included in a schema.
\subsection[{Dates}]{Dates}\label{NDDATE}\par
The following elements for the encoding of dates and times were introduced in section \textit{\hyperref[CONADA]{3.6.4.\ Dates and Times}}: 
\begin{sansreflist}
  
\item [\textbf{<date>}] (date) contains a date in any format.
\item [\textbf{<time>}] (time) contains a phrase defining a time of day in any format.
\end{sansreflist}
\par
The current module \textsf{namesdates} provides a mechanism for more detailed encoding of relative dates and times. A \textit{relative temporal expression} describes a date or time with reference to some other (absolute) temporal expression, and thus may contain an \hyperref[TEI.offset]{<offset>} element in addition to one or more \hyperref[TEI.date]{<date>} or \hyperref[TEI.time]{<time>} elements: 
\begin{sansreflist}
  
\item [\textbf{<offset>}] (offset) marks that part of a relative temporal or spatial expression which indicates the direction of the offset between the two place names, dates, or times involved in the expression.
\end{sansreflist}
\par
As members of the \textsf{att.datable} and \textsf{att.duration} classes, which in turn are members of \textsf{att.datable.w3c} and \textsf{att.duration.w3c} respectively, the \hyperref[TEI.date]{<date>} and \hyperref[TEI.time]{<time>} elements share the following attributes: 
\begin{sansreflist}
  
\item [\textbf{att.datable.w3c}] provides attributes for normalization of elements that contain datable events conforming to the W3C \textit{XML Schema Part 2: Datatypes Second Edition}.\hfil\\[-10pt]\begin{sansreflist}
    \item[@{\itshape when}]
  supplies the value of the date or time in a standard form, e.g. yyyy-mm-dd.
\end{sansreflist}  
\item [\textbf{att.duration.w3c}] provides attributes for recording normalized temporal durations.\hfil\\[-10pt]\begin{sansreflist}
    \item[@{\itshape dur}]
  (duration) indicates the length of this element in time.
\end{sansreflist}  
\end{sansreflist}

\subsubsection[{Relative Dates and Times }]{Relative Dates and Times }\label{NDDATER}\par
As noted above, relative dates and times such as ‘in the Two Hundredth and First Year of the Republic’, ‘twenty minutes before noon’, and, more ambiguously, ‘after the lamented death of the Doctor’ or ‘an hour after the game’ have two distinct components. As well as the absolute temporal expression or event to which reference is made (e.g. ‘noon’, ‘the game’, ‘the death of the Doctor’, ‘[the foundation of] the Republic’), they also contain a description of the ‘distance’ between the time or date which is indicated and the referent expression (e.g. ‘the Two Hundredth and First Year’, ‘twenty minutes’, ‘an hour’); and (optionally) an ‘offset’ describing the direction of the distance between the time or date indicated and the referent expression (e.g. ‘of’ implying after, ‘before’, ‘after’).\par
The ‘distance’ component of a relative temporal expression may be encoded as a temporal element in its own right using either \hyperref[TEI.date]{<date>} or \hyperref[TEI.time]{<time>}, or with the more generic \hyperref[TEI.measure]{<measure>} element. A special element, \hyperref[TEI.offset]{<offset>}, is provided by this module for encoding the ‘offset’ component of a relative temporal expression. The absolute temporal expression contained within the relative expression may be encoded with a \hyperref[TEI.date]{<date>} or \hyperref[TEI.time]{<time>} element; in turn, those elements may of course be relative, and thus contain \hyperref[TEI.date]{<date>} or \hyperref[TEI.time]{<time>} elements within themselves. This allows for deeply nested structures such as ‘the third Sunday after the first Monday before Lammastide in the fifth year of the King's second marriage ...’ but so does natural language.\par
In the following examples, the {\itshape when} and {\itshape dur} attributes have been used to simplify processing of variant forms of expression: \par\bgroup\index{date=<date>|exampleindex}\index{when=@when!<date>|exampleindex}\index{date=<date>|exampleindex}\index{dur=@dur!<date>|exampleindex}\index{offset=<offset>|exampleindex}\index{date=<date>|exampleindex}\index{when=@when!<date>|exampleindex}\index{type=@type!<date>|exampleindex}\exampleFont \begin{shaded}\noindent\mbox{}{<\textbf{date}\hspace*{1em}{when}="{1786-12-11}">}\mbox{}\newline 
\hspace*{1em}{<\textbf{date}\hspace*{1em}{dur}="{P14D}">}A fortnight{</\textbf{date}>}\mbox{}\newline 
\hspace*{1em}{<\textbf{offset}>}before{</\textbf{offset}>}\mbox{}\newline 
\hspace*{1em}{<\textbf{date}\hspace*{1em}{when}="{1786-12-25}"\hspace*{1em}{type}="{holiday}">}Christmas 1786{</\textbf{date}>}\mbox{}\newline 
{</\textbf{date}>}\end{shaded}\egroup\par \noindent  \par\bgroup\index{time=<time>|exampleindex}\index{when=@when!<time>|exampleindex}\index{time=<time>|exampleindex}\index{dur=@dur!<time>|exampleindex}\index{offset=<offset>|exampleindex}\index{time=<time>|exampleindex}\index{when=@when!<time>|exampleindex}\index{type=@type!<time>|exampleindex}\exampleFont \begin{shaded}\noindent\mbox{}I reached the station {<\textbf{time}\hspace*{1em}{when}="{14:15:00}">}\mbox{}\newline 
\hspace*{1em}{<\textbf{time}\hspace*{1em}{dur}="{PT30M0S}">}precisely half an hour{</\textbf{time}>}\mbox{}\newline 
\hspace*{1em}{<\textbf{offset}>}after{</\textbf{offset}>}\mbox{}\newline 
\hspace*{1em}{<\textbf{time}\hspace*{1em}{when}="{13:45:00}"\hspace*{1em}{type}="{occasion}">}the departure of the afternoon train to Boston{</\textbf{time}>}\mbox{}\newline 
{</\textbf{time}>}\end{shaded}\egroup\par \par
In the following example, a nested \hyperref[TEI.date]{<date>} element is used to show that ‘my birthday’ and the cited date are parts of the same temporal expression, and hence to disambiguate the phrase ‘A week before my birthday on 9th December’: \par\bgroup\index{date=<date>|exampleindex}\index{when=@when!<date>|exampleindex}\index{date=<date>|exampleindex}\index{offset=<offset>|exampleindex}\index{date=<date>|exampleindex}\index{when=@when!<date>|exampleindex}\index{date=<date>|exampleindex}\index{type=@type!<date>|exampleindex}\index{date=<date>|exampleindex}\exampleFont \begin{shaded}\noindent\mbox{}{<\textbf{date}\hspace*{1em}{when}="{--12-02}">}\mbox{}\newline 
\hspace*{1em}{<\textbf{date}>}A week{</\textbf{date}>}\mbox{}\newline 
\hspace*{1em}{<\textbf{offset}>}before{</\textbf{offset}>}\mbox{}\newline 
\hspace*{1em}{<\textbf{date}\hspace*{1em}{when}="{--12-09}">}\mbox{}\newline 
\hspace*{1em}\hspace*{1em}{<\textbf{date}\hspace*{1em}{type}="{occasion}">}my birthday{</\textbf{date}>} on {<\textbf{date}>}9th December{</\textbf{date}>}\mbox{}\newline 
\hspace*{1em}{</\textbf{date}>}\mbox{}\newline 
{</\textbf{date}>}\end{shaded}\egroup\par \noindent  The alternative reading of this phrase could be encoded as follows: \par\bgroup\index{date=<date>|exampleindex}\index{when=@when!<date>|exampleindex}\index{date=<date>|exampleindex}\index{offset=<offset>|exampleindex}\index{date=<date>|exampleindex}\index{type=@type!<date>|exampleindex}\index{when=@when!<date>|exampleindex}\index{date=<date>|exampleindex}\exampleFont \begin{shaded}\noindent\mbox{}{<\textbf{date}\hspace*{1em}{when}="{--12-09}">}\mbox{}\newline 
\hspace*{1em}{<\textbf{date}>}A week{</\textbf{date}>}\mbox{}\newline 
\hspace*{1em}{<\textbf{offset}>}before{</\textbf{offset}>}\mbox{}\newline 
\hspace*{1em}{<\textbf{date}\hspace*{1em}{type}="{occasion}"\hspace*{1em}{when}="{--12-16}">}my birthday{</\textbf{date}>} on {<\textbf{date}>}9th December{</\textbf{date}>}\mbox{}\newline 
{</\textbf{date}>}\end{shaded}\egroup\par \par
Where more complex or ambiguous expressions are involved, and where it is desirable to make more explicit the interpretive processes required, the feature structure notation described in chapter \textit{\hyperref[FS]{18.\ Feature Structures}} may be used. Consider, for example, the following temporal expression which occurs in the \textit{Scottish Temperance Review} of August 1850, referring to the summer holiday known in Glasgow simply as ‘the Fair’: \par\bgroup\index{date=<date>|exampleindex}\index{ana=@ana!<date>|exampleindex}\exampleFont \begin{shaded}\noindent\mbox{}Not only is the city, {<\textbf{date}\hspace*{1em}{ana}="{\#gf50}">}during the Fair{</\textbf{date}>}, a horrible\mbox{}\newline 
 nucleus of immorality and wickedness; it sends our multitudes to pollute and demoralize the country.\end{shaded}\egroup\par \par
For the definition of the {\itshape ana} attribute, see chapter \textit{\hyperref[AI]{17.\ Simple Analytic Mechanisms}} (in particular \textit{\hyperref[AIATTS]{17.2.\ Global Attributes for Simple Analyses}}). It is used here to link the temporal phrase with an interpretation of it. Like most traditional fairs and market days, the Glasgow Fair was established by local custom and could vary from year to year. Consequently, in order to provide such an interpretation, it is necessary to draw upon additional information which may or may not be located in the particular text in question. In this case, it is necessary at least to know the spatial and temporal context (year and place) of the fair referred to. These and other features required for the analysis of this particular temporal expression may be combined together as one feature structure of type date-analysis: \par\bgroup\index{fs=<fs>|exampleindex}\index{type=@type!<fs>|exampleindex}\index{f=<f>|exampleindex}\index{name=@name!<f>|exampleindex}\index{string=<string>|exampleindex}\index{f=<f>|exampleindex}\index{name=@name!<f>|exampleindex}\index{string=<string>|exampleindex}\index{f=<f>|exampleindex}\index{name=@name!<f>|exampleindex}\index{numeric=<numeric>|exampleindex}\index{value=@value!<numeric>|exampleindex}\index{f=<f>|exampleindex}\index{name=@name!<f>|exampleindex}\index{string=<string>|exampleindex}\index{f=<f>|exampleindex}\index{name=@name!<f>|exampleindex}\index{string=<string>|exampleindex}\exampleFont \begin{shaded}\noindent\mbox{}{<\textbf{fs}\hspace*{1em}{xml:id}="{gf50}"\hspace*{1em}{type}="{date-analysis}">}\mbox{}\newline 
\hspace*{1em}{<\textbf{f}\hspace*{1em}{name}="{event}">}\mbox{}\newline 
\hspace*{1em}\hspace*{1em}{<\textbf{string}>}the Fair{</\textbf{string}>}\mbox{}\newline 
\hspace*{1em}{</\textbf{f}>}\mbox{}\newline 
\hspace*{1em}{<\textbf{f}\hspace*{1em}{name}="{place}">}\mbox{}\newline 
\hspace*{1em}\hspace*{1em}{<\textbf{string}>}Glasgow{</\textbf{string}>}\mbox{}\newline 
\hspace*{1em}{</\textbf{f}>}\mbox{}\newline 
\hspace*{1em}{<\textbf{f}\hspace*{1em}{name}="{year}">}\mbox{}\newline 
\hspace*{1em}\hspace*{1em}{<\textbf{numeric}\hspace*{1em}{value}="{1850}"/>}\mbox{}\newline 
\hspace*{1em}{</\textbf{f}>}\mbox{}\newline 
\hspace*{1em}{<\textbf{f}\hspace*{1em}{name}="{from-value}">}\mbox{}\newline 
\hspace*{1em}\hspace*{1em}{<\textbf{string}>}1850-08-08{</\textbf{string}>}\mbox{}\newline 
\hspace*{1em}{</\textbf{f}>}\mbox{}\newline 
\hspace*{1em}{<\textbf{f}\hspace*{1em}{name}="{to-value}">}\mbox{}\newline 
\hspace*{1em}\hspace*{1em}{<\textbf{string}>}1850-09-19{</\textbf{string}>}\mbox{}\newline 
\hspace*{1em}{</\textbf{f}>}\mbox{}\newline 
{</\textbf{fs}>}\end{shaded}\egroup\par \noindent  For further discussion of feature structure representation see chapter \textit{\hyperref[FS]{18.\ Feature Structures}}.
\subsubsection[{Absolute Dates and Times}]{Absolute Dates and Times}\label{NDDATEA}\par
The following are examples of absolute temporal expressions.\par
\par\bgroup\index{date=<date>|exampleindex}\index{when=@when!<date>|exampleindex}\exampleFont \begin{shaded}\noindent\mbox{}The university's view of American affairs\mbox{}\newline 
 produced a stinging attack by Edmund Burke in the Commons debate of {<\textbf{date}\hspace*{1em}{when}="{1775-10-26}">}26 October\mbox{}\newline 
 1775{</\textbf{date}>}\end{shaded}\egroup\par \noindent  \par\bgroup\index{date=<date>|exampleindex}\index{when=@when!<date>|exampleindex}\exampleFont \begin{shaded}\noindent\mbox{}{<\textbf{date}\hspace*{1em}{when}="{1993-05-14}">}Friday, 14 May 1993{</\textbf{date}>}\end{shaded}\egroup\par \par
It may be useful to categorize a temporal expression which is given in terms of a named event, such as a public holiday, or a named time such as ‘tea time’ or ‘matins’: \par\bgroup\index{date=<date>|exampleindex}\index{type=@type!<date>|exampleindex}\index{when=@when!<date>|exampleindex}\index{date=<date>|exampleindex}\index{when=@when!<date>|exampleindex}\index{type=@type!<date>|exampleindex}\exampleFont \begin{shaded}\noindent\mbox{}In\mbox{}\newline 
 New York, {<\textbf{date}\hspace*{1em}{type}="{occasion}"\hspace*{1em}{when}="{--01-01}">}New Year's Day{</\textbf{date}>} is the quietest of holidays, {<\textbf{date}\hspace*{1em}{when}="{--07-04}"\hspace*{1em}{type}="{occasion}">}Independence Day{</\textbf{date}>} the most turbulent.\end{shaded}\egroup\par \par
Absolute temporal expressions denoting times which are given in terms of seconds, minutes, hours, or of well-defined events (e.g. ‘noon’, ‘sunset’) may similarly be represented using the \hyperref[TEI.time]{<time>} element. \par\bgroup\index{time=<time>|exampleindex}\index{type=@type!<time>|exampleindex}\index{when=@when!<time>|exampleindex}\exampleFont \begin{shaded}\noindent\mbox{}The train leaves for Boston at {<\textbf{time}\hspace*{1em}{type}="{twentyfourHour}"\hspace*{1em}{when}="{13:45:00}">}13:45{</\textbf{time}>}\end{shaded}\egroup\par \noindent  \par\bgroup\index{time=<time>|exampleindex}\index{type=@type!<time>|exampleindex}\exampleFont \begin{shaded}\noindent\mbox{}At {<\textbf{time}\hspace*{1em}{type}="{occasion}">}sunset{</\textbf{time}>} we walked to the beach.\end{shaded}\egroup\par \noindent  \par\bgroup\index{time=<time>|exampleindex}\index{type=@type!<time>|exampleindex}\index{when=@when!<time>|exampleindex}\exampleFont \begin{shaded}\noindent\mbox{}The train leaves for Boston at {<\textbf{time}\hspace*{1em}{xml:lang}="{en-US}"\hspace*{1em}{type}="{descriptive}"\mbox{}\newline 
\hspace*{1em}{when}="{13:45:00-05:00}">} a quarter of two {</\textbf{time}>}\end{shaded}\egroup\par 
\subsubsection[{More Expressive Normalizations}]{More Expressive Normalizations}\label{NDDATEISO}\par
The attributes for normalization of dates and times so far described use a standard format defined by \textit{XML Schema Part 2: Datatypes Second Edition}. This format is widely accepted and has significant software support. It is essentially a profile of ISO 8601 \textit{Data elements and interchange formats — Information interchange — Representation of dates and times}. The full ISO standard provides formats not available in the W3C recommendation, for example, the capability to refer to a date by its ordinal date or week date, or to refer to a century. It also provides ways of indicating duration and range.\par
When this module is included in a schema, the following additional attributes are provided: 
\begin{sansreflist}
  
\item [\textbf{att.datable.iso}] provides attributes for normalization of elements that contain datable events using the ISO 8601 standard.\hfil\\[-10pt]\begin{sansreflist}
    \item[@{\itshape when-iso}]
  supplies the value of a date or time in a standard form.
    \item[@{\itshape notBefore-iso}]
  specifies the earliest possible date for the event in standard form, e.g. yyyy-mm-dd.
    \item[@{\itshape notAfter-iso}]
  specifies the latest possible date for the event in standard form, e.g. yyyy-mm-dd.
    \item[@{\itshape from-iso}]
  indicates the starting point of the period in standard form.
    \item[@{\itshape to-iso}]
  indicates the ending point of the period in standard form.
\end{sansreflist}  
\item [\textbf{att.duration.iso}] provides attributes for recording normalized temporal durations.\hfil\\[-10pt]\begin{sansreflist}
    \item[@{\itshape dur-iso}]
  (duration) indicates the length of this element in time.
\end{sansreflist}  
\end{sansreflist}
 These attributes may be used in preference to their W3C equivalent when it is necessary to provide a normalized value in some form not supported by the W3C attributes. For example, a century date in the W3C format must be expressed as a range, using the {\itshape from} attribute together with either the {\itshape to} attribute or the {\itshape dur} (duration) attribute: \par\bgroup\index{date=<date>|exampleindex}\index{from=@from!<date>|exampleindex}\index{to=@to!<date>|exampleindex}\index{date=<date>|exampleindex}\index{from=@from!<date>|exampleindex}\index{dur=@dur!<date>|exampleindex}\exampleFont \begin{shaded}\noindent\mbox{}{<\textbf{date}\hspace*{1em}{from}="{1301}"\hspace*{1em}{to}="{1400}">}fourteenth century{</\textbf{date}>}\mbox{}\newline 
{<\textbf{date}\hspace*{1em}{from}="{1301}"\hspace*{1em}{dur}="{P100Y}">}fourteenth century{</\textbf{date}>}\end{shaded}\egroup\par \noindent  With the attribute {\itshape when-iso}, however, it is possible to express the same normalized value in any of the following additional ways: \par\bgroup\index{date=<date>|exampleindex}\index{when-iso=@when-iso!<date>|exampleindex}\index{date=<date>|exampleindex}\index{when-iso=@when-iso!<date>|exampleindex}\index{date=<date>|exampleindex}\index{when-iso=@when-iso!<date>|exampleindex}\exampleFont \begin{shaded}\noindent\mbox{}{<\textbf{date}\hspace*{1em}{when-iso}="{13}">}fourteenth\mbox{}\newline 
 century{</\textbf{date}>}\mbox{}\newline 
{<\textbf{date}\hspace*{1em}{when-iso}="{1301/1400}">}fourteenth century{</\textbf{date}>}\mbox{}\newline 
{<\textbf{date}\hspace*{1em}{when-iso}="{1301/P100Y}">}fourteenth century{</\textbf{date}>}\end{shaded}\egroup\par 
\subsubsection[{Using Non-Gregorian Calendars}]{Using Non-Gregorian Calendars}\label{NDDATECUSTOM}\par
All date-related encoding described above makes use of the Gregorian calendar, on which both the ISO and W3C datetime formats are based. However, historical texts often pre-date the invention of the Gregorian calendar in the 16th century, or its adoption in Europe over the following centuries, and many other calendars are used in texts from other cultures and contexts. Non-Gregorian dates can be encoded using methods described below.\par
First, a Calendar Description element needs to be supplied in the \hyperref[TEI.teiHeader]{<teiHeader>} as described in \textit{\hyperref[HD44]{2.4.5.\ Calendar Description}}:\par\bgroup\index{calendarDesc=<calendarDesc>|exampleindex}\index{calendar=<calendar>|exampleindex}\index{p=<p>|exampleindex}\exampleFont \begin{shaded}\noindent\mbox{}{<\textbf{calendarDesc}>}\mbox{}\newline 
\hspace*{1em}{<\textbf{calendar}\hspace*{1em}{xml:id}="{Julian\textunderscore England}">}\mbox{}\newline 
\hspace*{1em}\hspace*{1em}{<\textbf{p}>}The Julian calendar, as used in late 16th-century England.{</\textbf{p}>}\mbox{}\newline 
\hspace*{1em}{</\textbf{calendar}>}\mbox{}\newline 
{</\textbf{calendarDesc}>}\end{shaded}\egroup\par \par
The following attributes can now be used to encode dates using this calendar: 
\begin{sansreflist}
  
\item [\textbf{att.datable}] provides attributes for normalization of elements that contain dates, times, or datable events.\hfil\\[-10pt]\begin{sansreflist}
    \item[@{\itshape calendar}]
  indicates the system or calendar to which the date represented by the content of this element belongs.
\end{sansreflist}  
\item [\textbf{att.datable.custom}] provides attributes for normalization of elements that contain datable events to a custom dating system (i.e. other than the Gregorian used by W3 and ISO).\hfil\\[-10pt]\begin{sansreflist}
    \item[@{\itshape when-custom}]
  supplies the value of a date or time in some custom standard form.
    \item[@{\itshape notBefore-custom}]
  specifies the earliest possible date for the event in some custom standard form.
    \item[@{\itshape notAfter-custom}]
  specifies the latest possible date for the event in some custom standard form.
    \item[@{\itshape from-custom}]
  indicates the starting point of the period in some custom standard form.
    \item[@{\itshape to-custom}]
  indicates the ending point of the period in some custom standard form.
    \item[@{\itshape datingMethod}]
  supplies a pointer to a \hyperref[TEI.calendar]{<calendar>} element or other means of interpreting the values of the custom dating attributes.
\end{sansreflist}  
\end{sansreflist}
 \par\bgroup\index{p=<p>|exampleindex}\index{hi=<hi>|exampleindex}\index{date=<date>|exampleindex}\index{calendar=@calendar!<date>|exampleindex}\exampleFont \begin{shaded}\noindent\mbox{}{<\textbf{p}>}The Poole by S. {<\textbf{hi}>}Giles{</\textbf{hi}>} Churchyarde was a large water in the yeare {<\textbf{date}\hspace*{1em}{calendar}="{\#julianEngland}">}1244{</\textbf{date}>}.{</\textbf{p}>}\end{shaded}\egroup\par \noindent  Here, the {\itshape calendar} attribute points to the \hyperref[TEI.calendar]{<calendar>} element in the header which defines and describes the calendar used.\par
The {\itshape calendar} attribute is used to specify the calendar used in the \textit{text content} of the dating element which bears it. However, just as we use, for instance, {\itshape when}, {\itshape notBefore}, {\itshape notAfter} etc. to provide more precise expressions of dates and times in a constrained and computable form, it is often necessary to express a date or a date-range from a non-Gregorian calendar in a more precise manner. The attributes whose names end in ‘-custom’ are provided for this purpose, and the {\itshape datingMethod} is used to identify the calendar used in the content of these attributes: \par\bgroup\index{head=<head>|exampleindex}\index{lb=<lb>|exampleindex}\index{date=<date>|exampleindex}\index{when-custom=@when-custom!<date>|exampleindex}\index{datingMethod=@datingMethod!<date>|exampleindex}\index{calendar=@calendar!<date>|exampleindex}\exampleFont \begin{shaded}\noindent\mbox{}{<\textbf{head}>} The Tryumphs of Peace.{<\textbf{lb}/>} That Celebrated the Solemnity of the right Honorable Sr Francis Iones Knight, at\mbox{}\newline 
 his Inauguration into the Maioraltie of London, on Monday being the {<\textbf{date}\hspace*{1em}{when-custom}="{1620-10-30}"\mbox{}\newline 
\hspace*{1em}\hspace*{1em}{datingMethod}="{\#julianEngland}"\hspace*{1em}{calendar}="{\#julianEngland}">} 30. of October, 1620. {</\textbf{date}>}\mbox{}\newline 
{</\textbf{head}>}\end{shaded}\egroup\par \noindent  Here, the {\itshape calendar} attribute specifies the calendar used in the text content of the \hyperref[TEI.date]{<date>} element, as before, whereas the {\itshape datingMethod} attribute signifies that the calendar used in the {\itshape when-custom} attribute is also Julian. The schema could be customized in order to constrain the content of custom attributes in a manner similar to the constraints provided on regular Gregorian dating attributes such as {\itshape when}, to enforce consistency in the use of non-Gregorian dates.\par
Custom dating attributes can be combined with any of the standard dating attributes in order to provide a standardized Gregorian version of a non-Gregorian date. We might enhance the preceding example with the addition of {\itshape when}, providing the Gregorian calendar equivalent of the Julian date: \par\bgroup\index{date=<date>|exampleindex}\index{when-custom=@when-custom!<date>|exampleindex}\index{when=@when!<date>|exampleindex}\index{datingMethod=@datingMethod!<date>|exampleindex}\index{calendar=@calendar!<date>|exampleindex}\exampleFont \begin{shaded}\noindent\mbox{}{<\textbf{date}\hspace*{1em}{when-custom}="{1620-10-30}"\mbox{}\newline 
\hspace*{1em}{when}="{1620-11-09}"\hspace*{1em}{datingMethod}="{\#julianEngland}"\mbox{}\newline 
\hspace*{1em}{calendar}="{\#julianEngland}">} 30. of\mbox{}\newline 
 October, 1620. {</\textbf{date}>}\end{shaded}\egroup\par 
\subsection[{Module for Names and Dates}]{Module for Names and Dates}\par
The module described in this chapter makes available the following components: \begin{description}

\item[{Module namesdates: Names and dates}]\hspace{1em}\hfill\linebreak
\mbox{}\\[-10pt] \begin{itemize}
\item {\itshape Elements defined}: \hyperref[TEI.addName]{addName} \hyperref[TEI.affiliation]{affiliation} \hyperref[TEI.age]{age} \hyperref[TEI.birth]{birth} \hyperref[TEI.bloc]{bloc} \hyperref[TEI.climate]{climate} \hyperref[TEI.country]{country} \hyperref[TEI.death]{death} \hyperref[TEI.district]{district} \hyperref[TEI.education]{education} \hyperref[TEI.event]{event} \hyperref[TEI.faith]{faith} \hyperref[TEI.floruit]{floruit} \hyperref[TEI.forename]{forename} \hyperref[TEI.genName]{genName} \hyperref[TEI.geo]{geo} \hyperref[TEI.geogFeat]{geogFeat} \hyperref[TEI.geogName]{geogName} \hyperref[TEI.langKnowledge]{langKnowledge} \hyperref[TEI.langKnown]{langKnown} \hyperref[TEI.listEvent]{listEvent} \hyperref[TEI.listNym]{listNym} \hyperref[TEI.listObject]{listObject} \hyperref[TEI.listOrg]{listOrg} \hyperref[TEI.listPerson]{listPerson} \hyperref[TEI.listPlace]{listPlace} \hyperref[TEI.listRelation]{listRelation} \hyperref[TEI.location]{location} \hyperref[TEI.nameLink]{nameLink} \hyperref[TEI.nationality]{nationality} \hyperref[TEI.nym]{nym} \hyperref[TEI.object]{object} \hyperref[TEI.objectIdentifier]{objectIdentifier} \hyperref[TEI.objectName]{objectName} \hyperref[TEI.occupation]{occupation} \hyperref[TEI.offset]{offset} \hyperref[TEI.org]{org} \hyperref[TEI.orgName]{orgName} \hyperref[TEI.persName]{persName} \hyperref[TEI.persPronouns]{persPronouns} \hyperref[TEI.person]{person} \hyperref[TEI.personGrp]{personGrp} \hyperref[TEI.persona]{persona} \hyperref[TEI.place]{place} \hyperref[TEI.placeName]{placeName} \hyperref[TEI.population]{population} \hyperref[TEI.region]{region} \hyperref[TEI.relation]{relation} \hyperref[TEI.residence]{residence} \hyperref[TEI.roleName]{roleName} \hyperref[TEI.settlement]{settlement} \hyperref[TEI.sex]{sex} \hyperref[TEI.socecStatus]{socecStatus} \hyperref[TEI.state]{state} \hyperref[TEI.surname]{surname} \hyperref[TEI.terrain]{terrain} \hyperref[TEI.trait]{trait}
\item {\itshape Classes defined}: \hyperref[TEI.att.datable.custom]{att.datable.custom} \hyperref[TEI.att.datable.iso]{att.datable.iso} \hyperref[TEI.model.persNamePart]{model.persNamePart}
\end{itemize} 
\end{description}  The selection and combination of modules to form a TEI schema is described in \textit{\hyperref[STIN]{1.2.\ Defining a TEI Schema}}.