
\section[{Performance Texts}]{Performance Texts}\label{DR}\par
This module is intended for use when encoding printed dramatic texts, screen plays or radio scripts, and written transcriptions of any other form of performance. \par
Section \textit{\hyperref[DRFAB]{7.1.\ Front and Back Matter }} discusses elements such as cast lists, which can appear only in the front or back matter of printed dramatic texts. Section \textit{\hyperref[DRBOD]{7.2.\ The Body of a Performance Text}} discusses the structural components of performance texts: these include major structural divisions such as acts and scenes (section \textit{\hyperref[DRDIV]{7.2.1.\ Major Structural Divisions}}); individual speeches (section \textit{\hyperref[DRSP]{7.2.2.\ Speeches and Speakers}}); groups of speeches (section \textit{\hyperref[DRSPG]{7.2.3.\ Grouped Speeches}}); stage directions (section \textit{\hyperref[DRSTA]{7.2.4.\ Stage Directions}}); and the elements making up individual speeches (section \textit{\hyperref[DRPAL]{7.2.5.\ Speech Contents}}). Section \textit{\hyperref[DREMB]{7.2.6.\ Embedded Structures}} discusses ways of encoding units which cross the simple hierarchic structure so far defined, such as embedded songs or masques. Finally, section \textit{\hyperref[DROTH]{7.3.\ Other Types of Performance Text}} discusses a small number of additional elements characteristic of screen plays and radio or television scripts, as well as some elements for representing technical stage directions such as lighting or blocking.\par
The default structure for dramatic texts is similar to that defined by chapter \textit{\hyperref[DS]{4.\ Default Text Structure}}, as further discussed in section \textit{\hyperref[DRDIV]{7.2.1.\ Major Structural Divisions}}.\par
Two element classes are used by this module. The \textsf{model.frontPart.drama} class supplies specialized elements which can appear only in the front or back matter of performance texts. The \textsf{model.stageLike} class supplies a set of elements for stage directions and similar items such as camera movements, which can occur between or within speeches.
\subsection[{Front and Back Matter }]{Front and Back Matter }\label{DRFAB}\par
In dramatic texts, as in all TEI-conformant documents, the header element is followed by a \hyperref[TEI.text]{<text>} element, which contains optional front and back matter, and either a \hyperref[TEI.body]{<body>} or else a \hyperref[TEI.group]{<group>} of nested \hyperref[TEI.text]{<text>} elements. For more information on these, see chapter \textit{\hyperref[DS]{4.\ Default Text Structure}}.\par
The \hyperref[TEI.front]{<front>} and \hyperref[TEI.back]{<back>} elements are most likely to be of use when encoding preliminary materials in published performance texts. When the module defined by this chapter is included in a schema, the following additional elements not generally found in other forms of text become available as part of the front or back matter: 
\begin{sansreflist}
  
\item [\textbf{<performance>}] (performance) contains a section of front or back matter describing how a dramatic piece is to be performed in general or how it was performed on some specific occasion.
\item [\textbf{<prologue>}] (prologue) contains the prologue to a drama, typically spoken by an actor out of character, possibly in association with a particular performance or venue.
\item [\textbf{<epilogue>}] (epilogue) contains the epilogue to a drama, typically spoken by an actor out of character, possibly in association with a particular performance or venue.
\item [\textbf{<set>}] (setting) contains a description of the setting, time, locale, appearance, etc., of the action of a play, typically found in the front matter of a printed performance text (not a stage direction).
\item [\textbf{<castList>}] (cast list) contains a single cast list or dramatis personae.
\end{sansreflist}
\par
Elements for encoding each of these specific kinds of front matter are discussed in the remainder of this section, in the order given above. In addition, the front matter of dramatic texts may include the same elements as that of any other kind of text, notably title pages and various kinds of text division, as discussed in section \textit{\hyperref[DSFRONT]{4.5.\ Front Matter}}. The encoder may choose to ignore the specialized elements discussed in this section and instead use constructions of the type <div type="performance"> or <div1 type="set">.\par
Most other material in the front matter of a performance text will be marked with the default text structure elements described in chapter \textit{\hyperref[DS]{4.\ Default Text Structure}}. For example, the title page, dedication, other commendatory material, preface, etc., in a printed text should be encoded using \hyperref[TEI.div]{<div>} or \hyperref[TEI.div1]{<div1>} elements, containing headings, paragraphs, and other core tags.
\subsubsection[{The Set Element}]{The Set Element}\label{DRSET}\par
A special form of note describing the setting of a dramatic text (that is, the time and place of its action) is sometimes found in the front matter. 
\begin{sansreflist}
  
\item [\textbf{<set>}] (setting) contains a description of the setting, time, locale, appearance, etc., of the action of a play, typically found in the front matter of a printed performance text (not a stage direction).
\end{sansreflist}
 Descriptions of the setting may also appear as initial stage directions in the body of the play, but such descriptions should be marked as stage directions, not \hyperref[TEI.set]{<set>}. The \hyperref[TEI.set]{<set>} element should be used only where the description forms part of the front matter, as in the following examples: \par\bgroup\index{front=<front>|exampleindex}\index{castList=<castList>|exampleindex}\index{castItem=<castItem>|exampleindex}\index{set=<set>|exampleindex}\index{p=<p>|exampleindex}\exampleFont \begin{shaded}\noindent\mbox{}{<\textbf{front}>}\mbox{}\newline 
\hspace*{1em}{<\textbf{castList}>}\mbox{}\newline 
\hspace*{1em}\hspace*{1em}{<\textbf{castItem}>} ... {</\textbf{castItem}>}\mbox{}\newline 
\hspace*{1em}{</\textbf{castList}>}\mbox{}\newline 
\hspace*{1em}{<\textbf{set}>}\mbox{}\newline 
\hspace*{1em}\hspace*{1em}{<\textbf{p}>}The action of the play is set in Chicago's\mbox{}\newline 
\hspace*{1em}\hspace*{1em}\hspace*{1em}\hspace*{1em} Southside, sometime between World War II and the\mbox{}\newline 
\hspace*{1em}\hspace*{1em}\hspace*{1em}\hspace*{1em} present.{</\textbf{p}>}\mbox{}\newline 
\hspace*{1em}{</\textbf{set}>}\mbox{}\newline 
{</\textbf{front}>}\end{shaded}\egroup\par \noindent  \par\bgroup\index{front=<front>|exampleindex}\index{titlePage=<titlePage>|exampleindex}\index{type=@type!<titlePage>|exampleindex}\index{docTitle=<docTitle>|exampleindex}\index{titlePart=<titlePart>|exampleindex}\index{div=<div>|exampleindex}\index{type=@type!<div>|exampleindex}\index{div=<div>|exampleindex}\index{type=@type!<div>|exampleindex}\index{div=<div>|exampleindex}\index{type=@type!<div>|exampleindex}\index{div=<div>|exampleindex}\index{type=@type!<div>|exampleindex}\index{head=<head>|exampleindex}\index{p=<p>|exampleindex}\index{div=<div>|exampleindex}\index{type=@type!<div>|exampleindex}\index{head=<head>|exampleindex}\index{castList=<castList>|exampleindex}\index{castItem=<castItem>|exampleindex}\index{set=<set>|exampleindex}\index{p=<p>|exampleindex}\index{performance=<performance>|exampleindex}\index{p=<p>|exampleindex}\exampleFont \begin{shaded}\noindent\mbox{}{<\textbf{front}>}\mbox{}\newline 
\hspace*{1em}{<\textbf{titlePage}\hspace*{1em}{type}="{half-title}">}\mbox{}\newline 
\hspace*{1em}\hspace*{1em}{<\textbf{docTitle}>}\mbox{}\newline 
\hspace*{1em}\hspace*{1em}\hspace*{1em}{<\textbf{titlePart}>}Peer Gynt{</\textbf{titlePart}>}\mbox{}\newline 
\hspace*{1em}\hspace*{1em}{</\textbf{docTitle}>}\mbox{}\newline 
\hspace*{1em}{</\textbf{titlePage}>}\mbox{}\newline 
\hspace*{1em}{<\textbf{div}\hspace*{1em}{type}="{copyright\textunderscore page}"/>}\mbox{}\newline 
\hspace*{1em}{<\textbf{div}\hspace*{1em}{type}="{Contents}"/>}\mbox{}\newline 
\hspace*{1em}{<\textbf{div}\hspace*{1em}{type}="{Introduction}"/>}\mbox{}\newline 
\hspace*{1em}{<\textbf{div}\hspace*{1em}{type}="{note}">}\mbox{}\newline 
\hspace*{1em}\hspace*{1em}{<\textbf{head}>}Note on the Translation{</\textbf{head}>}\mbox{}\newline 
\hspace*{1em}\hspace*{1em}{<\textbf{p}>} ... {</\textbf{p}>}\mbox{}\newline 
\hspace*{1em}{</\textbf{div}>}\mbox{}\newline 
\hspace*{1em}{<\textbf{div}\hspace*{1em}{type}="{Dramatis\textunderscore Personae}">}\mbox{}\newline 
\hspace*{1em}\hspace*{1em}{<\textbf{head}>}Characters{</\textbf{head}>}\mbox{}\newline 
\hspace*{1em}\hspace*{1em}{<\textbf{castList}>}\mbox{}\newline 
\hspace*{1em}\hspace*{1em}\hspace*{1em}{<\textbf{castItem}>}\mbox{}\newline 
\textit{<!-- ... -->}\mbox{}\newline 
\hspace*{1em}\hspace*{1em}\hspace*{1em}{</\textbf{castItem}>}\mbox{}\newline 
\hspace*{1em}\hspace*{1em}{</\textbf{castList}>}\mbox{}\newline 
\hspace*{1em}{</\textbf{div}>}\mbox{}\newline 
\hspace*{1em}{<\textbf{set}>}\mbox{}\newline 
\hspace*{1em}\hspace*{1em}{<\textbf{p}>}The action, which opens in the beginning of the nineteenth\mbox{}\newline 
\hspace*{1em}\hspace*{1em}\hspace*{1em}\hspace*{1em} century, and ends around the 1860s, takes place partly in\mbox{}\newline 
\hspace*{1em}\hspace*{1em}\hspace*{1em}\hspace*{1em} Gudbrandsdalen, and on the mountains around it, partly on the coast\mbox{}\newline 
\hspace*{1em}\hspace*{1em}\hspace*{1em}\hspace*{1em} of Morocco, in the desert of Sahara, in a madhouse at Cairo, at sea,\mbox{}\newline 
\hspace*{1em}\hspace*{1em}\hspace*{1em}\hspace*{1em} etc.{</\textbf{p}>}\mbox{}\newline 
\hspace*{1em}{</\textbf{set}>}\mbox{}\newline 
\hspace*{1em}{<\textbf{performance}>}\mbox{}\newline 
\hspace*{1em}\hspace*{1em}{<\textbf{p}>}\mbox{}\newline 
\textit{<!-- ... -->}\mbox{}\newline 
\hspace*{1em}\hspace*{1em}{</\textbf{p}>}\mbox{}\newline 
\hspace*{1em}{</\textbf{performance}>}\mbox{}\newline 
{</\textbf{front}>}\end{shaded}\egroup\par 
\subsubsection[{Prologues and Epilogues}]{Prologues and Epilogues}\label{DRPRO}\par
Many plays in the Western tradition include in their front matter a prologue, spoken by an actor, generally not in character. Similar speeches often also occur at the end of the play, as epilogues. The elements \hyperref[TEI.prologue]{<prologue>} and \hyperref[TEI.epilogue]{<epilogue>} are provided for the encoding of such features within the front or back matter, where appropriate. 
\begin{sansreflist}
  
\item [\textbf{<prologue>}] (prologue) contains the prologue to a drama, typically spoken by an actor out of character, possibly in association with a particular performance or venue.
\item [\textbf{<epilogue>}] (epilogue) contains the epilogue to a drama, typically spoken by an actor out of character, possibly in association with a particular performance or venue.
\end{sansreflist}
 A prologue may be encoded just like a distinct poem, as in the following example: \par\bgroup\index{front=<front>|exampleindex}\index{prologue=<prologue>|exampleindex}\index{head=<head>|exampleindex}\index{name=<name>|exampleindex}\index{l=<l>|exampleindex}\index{l=<l>|exampleindex}\index{l=<l>|exampleindex}\index{l=<l>|exampleindex}\index{l=<l>|exampleindex}\index{l=<l>|exampleindex}\index{castList=<castList>|exampleindex}\index{head=<head>|exampleindex}\index{castItem=<castItem>|exampleindex}\index{set=<set>|exampleindex}\index{head=<head>|exampleindex}\index{p=<p>|exampleindex}\exampleFont \begin{shaded}\noindent\mbox{}{<\textbf{front}>}\mbox{}\newline 
\hspace*{1em}{<\textbf{prologue}>}\mbox{}\newline 
\hspace*{1em}\hspace*{1em}{<\textbf{head}>}Prologue, spoken by {<\textbf{name}>}Mr. Hart{</\textbf{name}>}\mbox{}\newline 
\hspace*{1em}\hspace*{1em}{</\textbf{head}>}\mbox{}\newline 
\hspace*{1em}\hspace*{1em}{<\textbf{l}>}Poets like Cudgel'd Bullys, never do{</\textbf{l}>}\mbox{}\newline 
\hspace*{1em}\hspace*{1em}{<\textbf{l}>}At first, or second blow, submit to you;{</\textbf{l}>}\mbox{}\newline 
\hspace*{1em}\hspace*{1em}{<\textbf{l}>}But will provoke you still, and ne're have done,{</\textbf{l}>}\mbox{}\newline 
\hspace*{1em}\hspace*{1em}{<\textbf{l}>}Till you are weary first, with laying on:{</\textbf{l}>}\mbox{}\newline 
\hspace*{1em}\hspace*{1em}{<\textbf{l}>}We patiently you see, give up to you,{</\textbf{l}>}\mbox{}\newline 
\hspace*{1em}\hspace*{1em}{<\textbf{l}>}Our Poets, Virgins, nay our Matrons too.{</\textbf{l}>}\mbox{}\newline 
\hspace*{1em}{</\textbf{prologue}>}\mbox{}\newline 
\hspace*{1em}{<\textbf{castList}>}\mbox{}\newline 
\hspace*{1em}\hspace*{1em}{<\textbf{head}>}The Persons{</\textbf{head}>}\mbox{}\newline 
\hspace*{1em}\hspace*{1em}{<\textbf{castItem}>} ... {</\textbf{castItem}>}\mbox{}\newline 
\hspace*{1em}{</\textbf{castList}>}\mbox{}\newline 
\hspace*{1em}{<\textbf{set}>}\mbox{}\newline 
\hspace*{1em}\hspace*{1em}{<\textbf{head}>}The SCENE{</\textbf{head}>}\mbox{}\newline 
\hspace*{1em}\hspace*{1em}{<\textbf{p}>}London{</\textbf{p}>}\mbox{}\newline 
\hspace*{1em}{</\textbf{set}>}\mbox{}\newline 
{</\textbf{front}>}\end{shaded}\egroup\par \noindent  \par
A prologue or epilogue may also be encoded as a speech, using the \hyperref[TEI.sp]{<sp>} element described in section \textit{\hyperref[CODR]{3.13.2.\ Core Tags for Drama}}. This is particularly appropriate where stage directions, etc., are involved, as in the following example: \par\bgroup\index{epilogue=<epilogue>|exampleindex}\index{head=<head>|exampleindex}\index{name=<name>|exampleindex}\index{name=<name>|exampleindex}\index{sp=<sp>|exampleindex}\index{lg=<lg>|exampleindex}\index{type=@type!<lg>|exampleindex}\index{l=<l>|exampleindex}\index{l=<l>|exampleindex}\index{l=<l>|exampleindex}\index{l=<l>|exampleindex}\index{lg=<lg>|exampleindex}\index{type=@type!<lg>|exampleindex}\index{l=<l>|exampleindex}\index{l=<l>|exampleindex}\index{l=<l>|exampleindex}\index{stage=<stage>|exampleindex}\index{lg=<lg>|exampleindex}\index{type=@type!<lg>|exampleindex}\index{l=<l>|exampleindex}\exampleFont \begin{shaded}\noindent\mbox{}{<\textbf{epilogue}>}\mbox{}\newline 
\hspace*{1em}{<\textbf{head}>}Written by {<\textbf{name}>}Colley Cibber, Esq{</\textbf{name}>}\mbox{}\newline 
\hspace*{1em}\hspace*{1em} and spoken by {<\textbf{name}>}Mrs. Cibber{</\textbf{name}>}\mbox{}\newline 
\hspace*{1em}{</\textbf{head}>}\mbox{}\newline 
\hspace*{1em}{<\textbf{sp}>}\mbox{}\newline 
\hspace*{1em}\hspace*{1em}{<\textbf{lg}\hspace*{1em}{type}="{stanza}">}\mbox{}\newline 
\hspace*{1em}\hspace*{1em}\hspace*{1em}{<\textbf{l}>}Since Fate has robb'd me of the hapless Youth,{</\textbf{l}>}\mbox{}\newline 
\hspace*{1em}\hspace*{1em}\hspace*{1em}{<\textbf{l}>}For whom my heart had hoarded up its truth;{</\textbf{l}>}\mbox{}\newline 
\hspace*{1em}\hspace*{1em}\hspace*{1em}{<\textbf{l}>}By all the Laws of Love and Honour, now,{</\textbf{l}>}\mbox{}\newline 
\hspace*{1em}\hspace*{1em}\hspace*{1em}{<\textbf{l}>}I'm free again to chuse, — and one of you{</\textbf{l}>}\mbox{}\newline 
\hspace*{1em}\hspace*{1em}{</\textbf{lg}>}\mbox{}\newline 
\hspace*{1em}\hspace*{1em}{<\textbf{lg}\hspace*{1em}{type}="{stanza}">}\mbox{}\newline 
\hspace*{1em}\hspace*{1em}\hspace*{1em}{<\textbf{l}>}Suppose I search the sober Gallery; — No,{</\textbf{l}>}\mbox{}\newline 
\hspace*{1em}\hspace*{1em}\hspace*{1em}{<\textbf{l}>}There's none but Prentices — \& Cuckolds all a row:{</\textbf{l}>}\mbox{}\newline 
\hspace*{1em}\hspace*{1em}\hspace*{1em}{<\textbf{l}>}And these, I doubt, are those that make 'em so.{</\textbf{l}>}\mbox{}\newline 
\hspace*{1em}\hspace*{1em}{</\textbf{lg}>}\mbox{}\newline 
\hspace*{1em}\hspace*{1em}{<\textbf{stage}>}Pointing to the Boxes.{</\textbf{stage}>}\mbox{}\newline 
\hspace*{1em}\hspace*{1em}{<\textbf{lg}\hspace*{1em}{type}="{stanza}">}\mbox{}\newline 
\hspace*{1em}\hspace*{1em}\hspace*{1em}{<\textbf{l}>}'Tis very well, enjoy the jest:{</\textbf{l}>}\mbox{}\newline 
\hspace*{1em}\hspace*{1em}{</\textbf{lg}>}\mbox{}\newline 
\hspace*{1em}{</\textbf{sp}>}\mbox{}\newline 
{</\textbf{epilogue}>}\end{shaded}\egroup\par \noindent  \par
In cases where the prologue or epilogue is clearly a significant part of the dramatic action, it may be preferable to include it in the body of a text, rather than in the front or back matter. In such cases, the encoder (and theatrical tradition) will determine whether or not to regard it as a new scene or division, or simply the final speech in the play. In the First Folio version of Shakespeare's \textit{Tempest}, for example, Prospero's final speech is clearly marked off as a distinct textual unit by the headings and layout of the page, and might therefore be encoded as back matter: \par\bgroup\index{text=<text>|exampleindex}\index{body=<body>|exampleindex}\index{div1=<div1>|exampleindex}\index{type=@type!<div1>|exampleindex}\index{sp=<sp>|exampleindex}\index{l=<l>|exampleindex}\index{part=@part!<l>|exampleindex}\index{l=<l>|exampleindex}\index{l=<l>|exampleindex}\index{stage=<stage>|exampleindex}\index{back=<back>|exampleindex}\index{epilogue=<epilogue>|exampleindex}\index{head=<head>|exampleindex}\index{sp=<sp>|exampleindex}\index{l=<l>|exampleindex}\index{l=<l>|exampleindex}\index{l=<l>|exampleindex}\index{l=<l>|exampleindex}\index{stage=<stage>|exampleindex}\index{set=<set>|exampleindex}\index{p=<p>|exampleindex}\index{castList=<castList>|exampleindex}\index{head=<head>|exampleindex}\index{castItem=<castItem>|exampleindex}\index{castItem=<castItem>|exampleindex}\index{castItem=<castItem>|exampleindex}\index{trailer=<trailer>|exampleindex}\exampleFont \begin{shaded}\noindent\mbox{}{<\textbf{text}>}\mbox{}\newline 
\hspace*{1em}{<\textbf{body}>}\mbox{}\newline 
\hspace*{1em}\hspace*{1em}{<\textbf{div1}\hspace*{1em}{type}="{scene}">}\mbox{}\newline 
\hspace*{1em}\hspace*{1em}\hspace*{1em}{<\textbf{sp}>}\mbox{}\newline 
\hspace*{1em}\hspace*{1em}\hspace*{1em}\hspace*{1em}{<\textbf{l}\hspace*{1em}{part}="{Y}">}I'le deliver all,{</\textbf{l}>}\mbox{}\newline 
\hspace*{1em}\hspace*{1em}\hspace*{1em}\hspace*{1em}{<\textbf{l}>}And promise you calme Seas, auspicious gales,{</\textbf{l}>}\mbox{}\newline 
\hspace*{1em}\hspace*{1em}\hspace*{1em}\hspace*{1em}{<\textbf{l}>}Be free and fare thou well: please you, draw neere.{</\textbf{l}>}\mbox{}\newline 
\hspace*{1em}\hspace*{1em}\hspace*{1em}\hspace*{1em}{<\textbf{stage}>}Exeunt omnes.{</\textbf{stage}>}\mbox{}\newline 
\hspace*{1em}\hspace*{1em}\hspace*{1em}{</\textbf{sp}>}\mbox{}\newline 
\hspace*{1em}\hspace*{1em}{</\textbf{div1}>}\mbox{}\newline 
\hspace*{1em}{</\textbf{body}>}\mbox{}\newline 
\hspace*{1em}{<\textbf{back}>}\mbox{}\newline 
\hspace*{1em}\hspace*{1em}{<\textbf{epilogue}>}\mbox{}\newline 
\hspace*{1em}\hspace*{1em}\hspace*{1em}{<\textbf{head}>}Epilogue, spoken by Prospero.{</\textbf{head}>}\mbox{}\newline 
\hspace*{1em}\hspace*{1em}\hspace*{1em}{<\textbf{sp}>}\mbox{}\newline 
\hspace*{1em}\hspace*{1em}\hspace*{1em}\hspace*{1em}{<\textbf{l}>}Now my Charmes are all ore-throwne,{</\textbf{l}>}\mbox{}\newline 
\hspace*{1em}\hspace*{1em}\hspace*{1em}\hspace*{1em}{<\textbf{l}>}And what strength I have's mine owne{</\textbf{l}>}\mbox{}\newline 
\hspace*{1em}\hspace*{1em}\hspace*{1em}\hspace*{1em}{<\textbf{l}>}As you from crimes would pardon'd be,{</\textbf{l}>}\mbox{}\newline 
\hspace*{1em}\hspace*{1em}\hspace*{1em}\hspace*{1em}{<\textbf{l}>}Let your Indulgence set me free.{</\textbf{l}>}\mbox{}\newline 
\hspace*{1em}\hspace*{1em}\hspace*{1em}{</\textbf{sp}>}\mbox{}\newline 
\hspace*{1em}\hspace*{1em}\hspace*{1em}{<\textbf{stage}>}Exit{</\textbf{stage}>}\mbox{}\newline 
\hspace*{1em}\hspace*{1em}{</\textbf{epilogue}>}\mbox{}\newline 
\hspace*{1em}\hspace*{1em}{<\textbf{set}>}\mbox{}\newline 
\hspace*{1em}\hspace*{1em}\hspace*{1em}{<\textbf{p}>}The Scene, an un-inhabited Island.{</\textbf{p}>}\mbox{}\newline 
\hspace*{1em}\hspace*{1em}{</\textbf{set}>}\mbox{}\newline 
\hspace*{1em}\hspace*{1em}{<\textbf{castList}>}\mbox{}\newline 
\hspace*{1em}\hspace*{1em}\hspace*{1em}{<\textbf{head}>}Names of the Actors.{</\textbf{head}>}\mbox{}\newline 
\hspace*{1em}\hspace*{1em}\hspace*{1em}{<\textbf{castItem}>}Alonso, K. of Naples{</\textbf{castItem}>}\mbox{}\newline 
\hspace*{1em}\hspace*{1em}\hspace*{1em}{<\textbf{castItem}>}Sebastian, his Brother.{</\textbf{castItem}>}\mbox{}\newline 
\hspace*{1em}\hspace*{1em}\hspace*{1em}{<\textbf{castItem}>}Prospero, the right Duke of Millaine.{</\textbf{castItem}>}\mbox{}\newline 
\hspace*{1em}\hspace*{1em}{</\textbf{castList}>}\mbox{}\newline 
\hspace*{1em}\hspace*{1em}{<\textbf{trailer}>}FINIS{</\textbf{trailer}>}\mbox{}\newline 
\hspace*{1em}{</\textbf{back}>}\mbox{}\newline 
{</\textbf{text}>}\end{shaded}\egroup\par \noindent  \par
In many modern editions, the editors have chosen to regard Prospero's speech as a part of the preceding scene: \par\bgroup\index{sp=<sp>|exampleindex}\index{speaker=<speaker>|exampleindex}\index{l=<l>|exampleindex}\index{part=@part!<l>|exampleindex}\index{l=<l>|exampleindex}\index{l=<l>|exampleindex}\index{stage=<stage>|exampleindex}\index{type=@type!<stage>|exampleindex}\index{stage=<stage>|exampleindex}\index{type=@type!<stage>|exampleindex}\index{note=<note>|exampleindex}\index{place=@place!<note>|exampleindex}\index{l=<l>|exampleindex}\index{l=<l>|exampleindex}\index{l=<l>|exampleindex}\index{l=<l>|exampleindex}\index{stage=<stage>|exampleindex}\index{type=@type!<stage>|exampleindex}\exampleFont \begin{shaded}\noindent\mbox{}{<\textbf{sp}>}\mbox{}\newline 
\hspace*{1em}{<\textbf{speaker}>}Prospero{</\textbf{speaker}>}\mbox{}\newline 
\hspace*{1em}{<\textbf{l}\hspace*{1em}{part}="{Y}">}I'll deliver all,{</\textbf{l}>}\mbox{}\newline 
\hspace*{1em}{<\textbf{l}>}And promise you calm seas, auspicious gales,{</\textbf{l}>}\mbox{}\newline 
\hspace*{1em}{<\textbf{l}>}Be free and fare thou well. {<\textbf{stage}\hspace*{1em}{type}="{exit}">}Exit Ariel{</\textbf{stage}>}\mbox{}\newline 
\hspace*{1em}\hspace*{1em} Please you, draw near. {<\textbf{stage}\hspace*{1em}{type}="{exit}">}Exeunt all but Prospero{</\textbf{stage}>}\mbox{}\newline 
\hspace*{1em}\hspace*{1em}{<\textbf{note}\hspace*{1em}{place}="{margin}">}Epilogue{</\textbf{note}>}\mbox{}\newline 
\hspace*{1em}{</\textbf{l}>}\mbox{}\newline 
\hspace*{1em}{<\textbf{l}>}Now my charms are all o'erthrown,{</\textbf{l}>}\mbox{}\newline 
\hspace*{1em}{<\textbf{l}>}And what strength I have's mine own{</\textbf{l}>}\mbox{}\newline 
\hspace*{1em}{<\textbf{l}>}As you from crimes would pardoned be,{</\textbf{l}>}\mbox{}\newline 
\hspace*{1em}{<\textbf{l}>}Let your indulgence set me free.{</\textbf{l}>}\mbox{}\newline 
{</\textbf{sp}>}\mbox{}\newline 
{<\textbf{stage}\hspace*{1em}{type}="{mix}">}He awaits applause, then exit.{</\textbf{stage}>}\end{shaded}\egroup\par \noindent   
\subsubsection[{Records of Performances}]{Records of Performances}\label{DRPERF}\par
Performance texts are not only printed in books to be read, they are also performed. It is common practice therefore to include within the front matter of a printed dramatic text some brief account of particular performances, using the following element: 
\begin{sansreflist}
  
\item [\textbf{<performance>}] (performance) contains a section of front or back matter describing how a dramatic piece is to be performed in general or how it was performed on some specific occasion.
\end{sansreflist}
 The \hyperref[TEI.performance]{<performance>} element may be used to group any and all information relating to the actual performance of a play or screenplay, whether it specifies how the play should be performed in general or how it was performed in practice on some occasion.\par
Performance information may include complex structures such as cast lists, or paragraphs describing the date and location of a performance, details about the setting portrayed in the performance and so forth. (See the discussion of these specialized structures in section \textit{\hyperref[DRFAB]{7.1.\ Front and Back Matter }} above.) If information for more than one performance is being recorded, then more than one \hyperref[TEI.performance]{<performance>} element should be used, wherever possible.\par
Names of persons, places, and dates of particular significance within the performance record may be explicitly marked using the general purpose \hyperref[TEI.name]{<name>}, <rs type="place"> and \hyperref[TEI.date]{<date>} elements described in section \textit{\hyperref[CONADA]{3.6.4.\ Dates and Times}}. No particular elements for such features as stagehouses, directors, etc., are proposed at this time.\par
For example: \par\bgroup\index{performance=<performance>|exampleindex}\index{head=<head>|exampleindex}\index{p=<p>|exampleindex}\index{p=<p>|exampleindex}\index{castList=<castList>|exampleindex}\index{head=<head>|exampleindex}\index{note=<note>|exampleindex}\index{rend=@rend!<note>|exampleindex}\index{place=@place!<note>|exampleindex}\index{castItem=<castItem>|exampleindex}\index{role=<role>|exampleindex}\index{actor=<actor>|exampleindex}\index{castItem=<castItem>|exampleindex}\index{role=<role>|exampleindex}\index{actor=<actor>|exampleindex}\index{castItem=<castItem>|exampleindex}\index{role=<role>|exampleindex}\index{actor=<actor>|exampleindex}\index{castItem=<castItem>|exampleindex}\index{role=<role>|exampleindex}\index{actor=<actor>|exampleindex}\index{p=<p>|exampleindex}\index{name=<name>|exampleindex}\index{p=<p>|exampleindex}\index{name=<name>|exampleindex}\index{p=<p>|exampleindex}\index{name=<name>|exampleindex}\index{p=<p>|exampleindex}\index{name=<name>|exampleindex}\index{rend=@rend!<name>|exampleindex}\index{name=<name>|exampleindex}\index{rend=@rend!<name>|exampleindex}\index{rs=<rs>|exampleindex}\index{type=@type!<rs>|exampleindex}\index{date=<date>|exampleindex}\index{when=@when!<date>|exampleindex}\exampleFont \begin{shaded}\noindent\mbox{}{<\textbf{performance}>}\mbox{}\newline 
\hspace*{1em}{<\textbf{head}>}Death of a Salesman{</\textbf{head}>}\mbox{}\newline 
\hspace*{1em}{<\textbf{p}>}A New Play by Arthur Miller{</\textbf{p}>}\mbox{}\newline 
\hspace*{1em}{<\textbf{p}>}Staged by Elia Kazan{</\textbf{p}>}\mbox{}\newline 
\hspace*{1em}{<\textbf{castList}>}\mbox{}\newline 
\hspace*{1em}\hspace*{1em}{<\textbf{head}>}Cast{</\textbf{head}>}\mbox{}\newline 
\hspace*{1em}\hspace*{1em}{<\textbf{note}\hspace*{1em}{rend}="{small type flush left}"\mbox{}\newline 
\hspace*{1em}\hspace*{1em}\hspace*{1em}{place}="{inline}">}(in order of appearance){</\textbf{note}>}\mbox{}\newline 
\hspace*{1em}\hspace*{1em}{<\textbf{castItem}>}\mbox{}\newline 
\hspace*{1em}\hspace*{1em}\hspace*{1em}{<\textbf{role}>}Willy Loman{</\textbf{role}>}\mbox{}\newline 
\hspace*{1em}\hspace*{1em}\hspace*{1em}{<\textbf{actor}>}Lee J. Cobb{</\textbf{actor}>}\mbox{}\newline 
\hspace*{1em}\hspace*{1em}{</\textbf{castItem}>}\mbox{}\newline 
\hspace*{1em}\hspace*{1em}{<\textbf{castItem}>}\mbox{}\newline 
\hspace*{1em}\hspace*{1em}\hspace*{1em}{<\textbf{role}>}Linda{</\textbf{role}>}\mbox{}\newline 
\hspace*{1em}\hspace*{1em}\hspace*{1em}{<\textbf{actor}>}Mildred Dunnock{</\textbf{actor}>}\mbox{}\newline 
\hspace*{1em}\hspace*{1em}{</\textbf{castItem}>}\mbox{}\newline 
\hspace*{1em}\hspace*{1em}{<\textbf{castItem}>}\mbox{}\newline 
\hspace*{1em}\hspace*{1em}\hspace*{1em}{<\textbf{role}>}Biff{</\textbf{role}>}\mbox{}\newline 
\hspace*{1em}\hspace*{1em}\hspace*{1em}{<\textbf{actor}>}Arthur Kennedy{</\textbf{actor}>}\mbox{}\newline 
\hspace*{1em}\hspace*{1em}{</\textbf{castItem}>}\mbox{}\newline 
\hspace*{1em}\hspace*{1em}{<\textbf{castItem}>}\mbox{}\newline 
\hspace*{1em}\hspace*{1em}\hspace*{1em}{<\textbf{role}>}Happy{</\textbf{role}>}\mbox{}\newline 
\hspace*{1em}\hspace*{1em}\hspace*{1em}{<\textbf{actor}>}Cameron Mitchell{</\textbf{actor}>}\mbox{}\newline 
\hspace*{1em}\hspace*{1em}{</\textbf{castItem}>}\mbox{}\newline 
\textit{<!-- ... -->}\mbox{}\newline 
\hspace*{1em}{</\textbf{castList}>}\mbox{}\newline 
\hspace*{1em}{<\textbf{p}>}The setting and lighting were designed by\mbox{}\newline 
\hspace*{1em}{<\textbf{name}>}Jo Mielziner{</\textbf{name}>}.{</\textbf{p}>}\mbox{}\newline 
\hspace*{1em}{<\textbf{p}>}The incidental music was composed by {<\textbf{name}>}Alex North{</\textbf{name}>}.{</\textbf{p}>}\mbox{}\newline 
\hspace*{1em}{<\textbf{p}>}The costumes were designed by {<\textbf{name}>}Julia Sze{</\textbf{name}>}.{</\textbf{p}>}\mbox{}\newline 
\hspace*{1em}{<\textbf{p}>}Presented by {<\textbf{name}\hspace*{1em}{rend}="{unmarked}">}Kermit Bloomgarden{</\textbf{name}>}\mbox{}\newline 
\hspace*{1em}\hspace*{1em} and {<\textbf{name}\hspace*{1em}{rend}="{unmarked}">}Walter Fried{</\textbf{name}>} at the\mbox{}\newline 
\hspace*{1em}{<\textbf{rs}\hspace*{1em}{type}="{place}">}Morosco Theatre in New York{</\textbf{rs}>} on\mbox{}\newline 
\hspace*{1em}{<\textbf{date}\hspace*{1em}{when}="{1949-02-10}">}February 10, 1949{</\textbf{date}>}.{</\textbf{p}>}\mbox{}\newline 
{</\textbf{performance}>}\end{shaded}\egroup\par \noindent       Or: \par\bgroup\index{performance=<performance>|exampleindex}\index{p=<p>|exampleindex}\index{rs=<rs>|exampleindex}\index{type=@type!<rs>|exampleindex}\index{rs=<rs>|exampleindex}\index{type=@type!<rs>|exampleindex}\index{date=<date>|exampleindex}\index{name=<name>|exampleindex}\exampleFont \begin{shaded}\noindent\mbox{}{<\textbf{performance}>}\mbox{}\newline 
\hspace*{1em}{<\textbf{p}>}La Machine Infernale a été\mbox{}\newline 
\hspace*{1em}\hspace*{1em} représentée pour la première fois au\mbox{}\newline 
\hspace*{1em}{<\textbf{rs}\hspace*{1em}{type}="{place-theatre}">}théâtre Louis-Jouvet{</\textbf{rs}>}\mbox{}\newline 
\hspace*{1em}\hspace*{1em}{<\textbf{rs}\hspace*{1em}{type}="{place-theatre}">}(Comédie des\mbox{}\newline 
\hspace*{1em}\hspace*{1em}\hspace*{1em}\hspace*{1em} Champs-élysées){</\textbf{rs}>}\mbox{}\newline 
\hspace*{1em}\hspace*{1em}{<\textbf{date}>}le 10 avril 1934{</\textbf{date}>},\mbox{}\newline 
\hspace*{1em}\hspace*{1em} avec les décors et les costumes de\mbox{}\newline 
\hspace*{1em}{<\textbf{name}>}Christian Bérard.{</\textbf{name}>} ... {</\textbf{p}>}\mbox{}\newline 
{</\textbf{performance}>}\end{shaded}\egroup\par \noindent  
\subsubsection[{Cast Lists}]{Cast Lists}\label{DRCAST}\par
A \textit{cast list} is a specialized form of list, conventionally found at the start or end of a play, usually listing all the speaking and non-speaking roles in the play, often with additional description (‘Cataplasma, a maker of Periwigges and Attires’) or the name of an actor or actress (‘Old Lady Squeamish. Mrs Rutter’). Cast lists may be encoded with the general purpose \hyperref[TEI.list]{<list>} element described in section \textit{\hyperref[COLI]{3.8.\ Lists}}, but for more detailed work the following specialized elements are provided: 
\begin{sansreflist}
  
\item [\textbf{<castList>}] (cast list) contains a single cast list or dramatis personae.
\item [\textbf{<castGroup>}] (cast list grouping) groups one or more individual \hyperref[TEI.castItem]{<castItem>} elements within a cast list.
\item [\textbf{<castItem>}] (cast list item) contains a single entry within a cast list, describing either a single role or a list of non-speaking roles.\hfil\\[-10pt]\begin{sansreflist}
    \item[@{\itshape type}]
  characterizes the cast item.
\end{sansreflist}  
\end{sansreflist}
\par
A \hyperref[TEI.castItem]{<castItem>} element may contain any mixture of elements taken from the \textsf{model.castItemPart} class, members of which (when this module is included) are: 
\begin{sansreflist}
  
\item [\textbf{<role>}] (role) contains the name of a dramatic role, as given in a cast list.
\item [\textbf{<roleDesc>}] (role description) describes a character's role in a drama.
\item [\textbf{<actor>}] contains the name of an actor appearing within a cast list.
\end{sansreflist}
 Cast lists often have an internal structure of their own; it is quite usual to find, for example, nobility and commoners, or male and female roles, presented in different groups or sublists. Roles are also often grouped together by their function, for example: \begin{itemize}
\item Sons of Cato: \mbox{}\\[-10pt] \begin{itemize}
\item Portius
\item Marcus
\end{itemize} 
\end{itemize} \par
A cast list relating to a specific performance may be accompanied by notes about the time or place of that performance, indicating (for example) the name of the theatre where the play was first presented, the name of the producer or director, and so forth. When the cast list relates to a specific performance, it should be embedded within a \hyperref[TEI.performance]{<performance>} element (see section \textit{\hyperref[DRPERF]{7.1.3.\ Records of Performances}}), as in the following example: \par\bgroup\index{performance=<performance>|exampleindex}\index{p=<p>|exampleindex}\index{title=<title>|exampleindex}\index{date=<date>|exampleindex}\index{when=@when!<date>|exampleindex}\index{name=<name>|exampleindex}\index{name=<name>|exampleindex}\index{castList=<castList>|exampleindex}\index{castItem=<castItem>|exampleindex}\index{castItem=<castItem>|exampleindex}\index{castItem=<castItem>|exampleindex}\exampleFont \begin{shaded}\noindent\mbox{}{<\textbf{performance}>}\mbox{}\newline 
\hspace*{1em}{<\textbf{p}>}The first performance in Great Britain of {<\textbf{title}>}Waiting for\mbox{}\newline 
\hspace*{1em}\hspace*{1em}\hspace*{1em}\hspace*{1em} Godot{</\textbf{title}>} was given at the Arts Theatre, London, on\mbox{}\newline 
\hspace*{1em}{<\textbf{date}\hspace*{1em}{when}="{1955-08-03}">}3rd August 1955{</\textbf{date}>}. It was directed by\mbox{}\newline 
\hspace*{1em}{<\textbf{name}>}Peter Hall{</\textbf{name}>}, and the décor was by {<\textbf{name}>}Peter\mbox{}\newline 
\hspace*{1em}\hspace*{1em}\hspace*{1em}\hspace*{1em} Snow{</\textbf{name}>}. The cast was as follows:{</\textbf{p}>}\mbox{}\newline 
\hspace*{1em}{<\textbf{castList}>}\mbox{}\newline 
\hspace*{1em}\hspace*{1em}{<\textbf{castItem}>}Estragon: Peter Woodthorpe{</\textbf{castItem}>}\mbox{}\newline 
\hspace*{1em}\hspace*{1em}{<\textbf{castItem}>}Vladimir: Paul Daneman{</\textbf{castItem}>}\mbox{}\newline 
\hspace*{1em}\hspace*{1em}{<\textbf{castItem}>} ... {</\textbf{castItem}>}\mbox{}\newline 
\hspace*{1em}{</\textbf{castList}>}\mbox{}\newline 
{</\textbf{performance}>}\end{shaded}\egroup\par \noindent  \par
In this example, the \hyperref[TEI.castItem]{<castItem>} elements have no substructure. If desired, however, their components may be more finely distinguished using the elements \hyperref[TEI.role]{<role>}, \hyperref[TEI.roleDesc]{<roleDesc>}, and \hyperref[TEI.actor]{<actor>}. For example, the second cast item above might be encoded as follows: \par\bgroup\index{castItem=<castItem>|exampleindex}\index{role=<role>|exampleindex}\index{actor=<actor>|exampleindex}\exampleFont \begin{shaded}\noindent\mbox{}{<\textbf{castItem}>}\mbox{}\newline 
\hspace*{1em}{<\textbf{role}\hspace*{1em}{xml:id}="{vlad}">}Vladimir{</\textbf{role}>}:\mbox{}\newline 
{<\textbf{actor}>}Paul Daneman{</\textbf{actor}>}\mbox{}\newline 
{</\textbf{castItem}>}\end{shaded}\egroup\par \par
The {\itshape ref} attribute on \hyperref[TEI.actor]{<actor>} may be used to associate the name with information about the real-world person identified, as further discussed in section ND. In the previous example, we might associate the name of Paul Daneman with his entry in a widely used bibliography as follows : \par\bgroup\index{actor=<actor>|exampleindex}\index{ref=@ref!<actor>|exampleindex}\exampleFont \begin{shaded}\noindent\mbox{}{<\textbf{actor}\hspace*{1em}{ref}="{https://www.worldcat.org/identities/lccn-n84002994/}">}Paul Daneman{</\textbf{actor}>}\end{shaded}\egroup\par \par
The global {\itshape xml:id} attribute may be used to specify a unique identifier for the \hyperref[TEI.role]{<role>} element, where it is desired to link speeches within the text explicitly to the role, using the {\itshape who} attribute, as further discussed in section \textit{\hyperref[DRSP]{7.2.2.\ Speeches and Speakers}} below.\par
The occasionally lengthy descriptions of a role sometimes found in written play scripts may be marked using the \hyperref[TEI.roleDesc]{<roleDesc>} element, as in the following example: \par\bgroup\index{castItem=<castItem>|exampleindex}\index{role=<role>|exampleindex}\index{roleDesc=<roleDesc>|exampleindex}\index{actor=<actor>|exampleindex}\exampleFont \begin{shaded}\noindent\mbox{}{<\textbf{castItem}>}\mbox{}\newline 
\hspace*{1em}{<\textbf{role}>}Tom Thumb the Great{</\textbf{role}>}\mbox{}\newline 
\hspace*{1em}{<\textbf{roleDesc}>}a little hero with a great soul, something violent in his\mbox{}\newline 
\hspace*{1em}\hspace*{1em} temper, which is a little abated by his love for Huncamunca{</\textbf{roleDesc}>}\mbox{}\newline 
\hspace*{1em}{<\textbf{actor}>}Young Verhuyk{</\textbf{actor}>}\mbox{}\newline 
{</\textbf{castItem}>}\end{shaded}\egroup\par \noindent   For non-speaking or un-named roles, a \hyperref[TEI.castItem]{<castItem>} may contain a \hyperref[TEI.roleDesc]{<roleDesc>} without an accompanying \hyperref[TEI.role]{<role>}, for example \par\bgroup\index{castItem=<castItem>|exampleindex}\index{roleDesc=<roleDesc>|exampleindex}\exampleFont \begin{shaded}\noindent\mbox{}{<\textbf{castItem}>}\mbox{}\newline 
\hspace*{1em}{<\textbf{roleDesc}>}Costermonger{</\textbf{roleDesc}>}\mbox{}\newline 
{</\textbf{castItem}>}\end{shaded}\egroup\par \par
When a list of such minor roles is given together, the {\itshape type} attribute of the \hyperref[TEI.castItem]{<castItem>} should indicate that it contains more than one role, by taking a value such as list. The encoder may or may not elect to encode each separate constituent within such a composite \hyperref[TEI.castItem]{<castItem>}. Thus, either of the following is acceptable: \par\bgroup\index{castItem=<castItem>|exampleindex}\index{type=@type!<castItem>|exampleindex}\index{castItem=<castItem>|exampleindex}\index{type=@type!<castItem>|exampleindex}\index{roleDesc=<roleDesc>|exampleindex}\index{roleDesc=<roleDesc>|exampleindex}\index{roleDesc=<roleDesc>|exampleindex}\exampleFont \begin{shaded}\noindent\mbox{}{<\textbf{castItem}\hspace*{1em}{type}="{list}">}Constables, Drawer, Turnkey, etc.{</\textbf{castItem}>}\mbox{}\newline 
{<\textbf{castItem}\hspace*{1em}{type}="{list}">}\mbox{}\newline 
\hspace*{1em}{<\textbf{roleDesc}>}Constables,{</\textbf{roleDesc}>}\mbox{}\newline 
\hspace*{1em}{<\textbf{roleDesc}>}Drawer,{</\textbf{roleDesc}>}\mbox{}\newline 
\hspace*{1em}{<\textbf{roleDesc}>}Turnkey,{</\textbf{roleDesc}>}\mbox{}\newline 
 etc.\mbox{}\newline 
\mbox{}\newline 
{</\textbf{castItem}>}\end{shaded}\egroup\par \par
A group of cast items forming a distinct subdivision of a cast list may be marked as such by using the special purpose \hyperref[TEI.castGroup]{<castGroup>} element. The {\itshape rend} attribute may be used to indicate whether this grouping is indicated in the text by layout alone (i.e. the use of whitespace), by long braces or by some other means. A \hyperref[TEI.castGroup]{<castGroup>} may contain an optional heading (represented as usual by a \hyperref[TEI.head]{<head>} element) followed by a series of \hyperref[TEI.castItem]{<castItem>} elements: \par\bgroup\index{castGroup=<castGroup>|exampleindex}\index{rend=@rend!<castGroup>|exampleindex}\index{head=<head>|exampleindex}\index{castItem=<castItem>|exampleindex}\index{role=<role>|exampleindex}\index{actor=<actor>|exampleindex}\index{castItem=<castItem>|exampleindex}\index{role=<role>|exampleindex}\index{actor=<actor>|exampleindex}\exampleFont \begin{shaded}\noindent\mbox{}{<\textbf{castGroup}\hspace*{1em}{rend}="{braced}">}\mbox{}\newline 
\hspace*{1em}{<\textbf{head}>}friends of Mathias{</\textbf{head}>}\mbox{}\newline 
\hspace*{1em}{<\textbf{castItem}>}\mbox{}\newline 
\hspace*{1em}\hspace*{1em}{<\textbf{role}>}Walter{</\textbf{role}>}\mbox{}\newline 
\hspace*{1em}\hspace*{1em}{<\textbf{actor}>}Mr Frank Hall{</\textbf{actor}>}\mbox{}\newline 
\hspace*{1em}{</\textbf{castItem}>}\mbox{}\newline 
\hspace*{1em}{<\textbf{castItem}>}\mbox{}\newline 
\hspace*{1em}\hspace*{1em}{<\textbf{role}>}Hans{</\textbf{role}>}\mbox{}\newline 
\hspace*{1em}\hspace*{1em}{<\textbf{actor}>}Mr F.W. Irish{</\textbf{actor}>}\mbox{}\newline 
\hspace*{1em}{</\textbf{castItem}>}\mbox{}\newline 
{</\textbf{castGroup}>}\end{shaded}\egroup\par \noindent  \par
Alternatively, the encoder may prefer to regard the phrase ‘friends of Mathias’ as a role description, and encode the above example as follows: \par\bgroup\index{castGroup=<castGroup>|exampleindex}\index{rend=@rend!<castGroup>|exampleindex}\index{roleDesc=<roleDesc>|exampleindex}\index{castItem=<castItem>|exampleindex}\index{role=<role>|exampleindex}\index{actor=<actor>|exampleindex}\index{castItem=<castItem>|exampleindex}\index{role=<role>|exampleindex}\index{actor=<actor>|exampleindex}\exampleFont \begin{shaded}\noindent\mbox{}{<\textbf{castGroup}\hspace*{1em}{rend}="{braced}">}\mbox{}\newline 
\hspace*{1em}{<\textbf{roleDesc}>}friends of Mathias{</\textbf{roleDesc}>}\mbox{}\newline 
\hspace*{1em}{<\textbf{castItem}>}\mbox{}\newline 
\hspace*{1em}\hspace*{1em}{<\textbf{role}>}Walter{</\textbf{role}>}\mbox{}\newline 
\hspace*{1em}\hspace*{1em}{<\textbf{actor}>}Mr Frank Hall{</\textbf{actor}>}\mbox{}\newline 
\hspace*{1em}{</\textbf{castItem}>}\mbox{}\newline 
\hspace*{1em}{<\textbf{castItem}>}\mbox{}\newline 
\hspace*{1em}\hspace*{1em}{<\textbf{role}>}Hans{</\textbf{role}>}\mbox{}\newline 
\hspace*{1em}\hspace*{1em}{<\textbf{actor}>}Mr F.W. Irish{</\textbf{actor}>}\mbox{}\newline 
\hspace*{1em}{</\textbf{castItem}>}\mbox{}\newline 
{</\textbf{castGroup}>}\end{shaded}\egroup\par \par
This version has the advantage that all role descriptions are treated alike, rather than in some cases being treated as headings. On the other hand there are also cases, such as the following, where the role description does function more like a heading: \par\bgroup\index{castList=<castList>|exampleindex}\index{castGroup=<castGroup>|exampleindex}\index{head=<head>|exampleindex}\index{rend=@rend!<head>|exampleindex}\index{castItem=<castItem>|exampleindex}\index{role=<role>|exampleindex}\index{actor=<actor>|exampleindex}\index{castItem=<castItem>|exampleindex}\index{role=<role>|exampleindex}\index{actor=<actor>|exampleindex}\index{castItem=<castItem>|exampleindex}\index{role=<role>|exampleindex}\index{actor=<actor>|exampleindex}\index{castItem=<castItem>|exampleindex}\index{role=<role>|exampleindex}\index{actor=<actor>|exampleindex}\index{castItem=<castItem>|exampleindex}\index{role=<role>|exampleindex}\index{roleDesc=<roleDesc>|exampleindex}\index{actor=<actor>|exampleindex}\index{castGroup=<castGroup>|exampleindex}\index{head=<head>|exampleindex}\index{rend=@rend!<head>|exampleindex}\index{castItem=<castItem>|exampleindex}\index{role=<role>|exampleindex}\index{actor=<actor>|exampleindex}\index{castItem=<castItem>|exampleindex}\index{role=<role>|exampleindex}\index{actor=<actor>|exampleindex}\index{castItem=<castItem>|exampleindex}\index{role=<role>|exampleindex}\index{roleDesc=<roleDesc>|exampleindex}\index{actor=<actor>|exampleindex}\index{castItem=<castItem>|exampleindex}\index{role=<role>|exampleindex}\index{actor=<actor>|exampleindex}\index{castItem=<castItem>|exampleindex}\index{role=<role>|exampleindex}\index{roleDesc=<roleDesc>|exampleindex}\index{actor=<actor>|exampleindex}\exampleFont \begin{shaded}\noindent\mbox{}{<\textbf{castList}>}\mbox{}\newline 
\hspace*{1em}{<\textbf{castGroup}>}\mbox{}\newline 
\hspace*{1em}\hspace*{1em}{<\textbf{head}\hspace*{1em}{rend}="{braced}">}Mendicants{</\textbf{head}>}\mbox{}\newline 
\hspace*{1em}\hspace*{1em}{<\textbf{castItem}>}\mbox{}\newline 
\hspace*{1em}\hspace*{1em}\hspace*{1em}{<\textbf{role}>}Aafaa{</\textbf{role}>}\mbox{}\newline 
\hspace*{1em}\hspace*{1em}\hspace*{1em}{<\textbf{actor}>}Femi Johnson{</\textbf{actor}>}\mbox{}\newline 
\hspace*{1em}\hspace*{1em}{</\textbf{castItem}>}\mbox{}\newline 
\hspace*{1em}\hspace*{1em}{<\textbf{castItem}>}\mbox{}\newline 
\hspace*{1em}\hspace*{1em}\hspace*{1em}{<\textbf{role}>}Blindman{</\textbf{role}>}\mbox{}\newline 
\hspace*{1em}\hspace*{1em}\hspace*{1em}{<\textbf{actor}>}Femi Osofisan{</\textbf{actor}>}\mbox{}\newline 
\hspace*{1em}\hspace*{1em}{</\textbf{castItem}>}\mbox{}\newline 
\hspace*{1em}\hspace*{1em}{<\textbf{castItem}>}\mbox{}\newline 
\hspace*{1em}\hspace*{1em}\hspace*{1em}{<\textbf{role}>}Goyi{</\textbf{role}>}\mbox{}\newline 
\hspace*{1em}\hspace*{1em}\hspace*{1em}{<\textbf{actor}>}Wale Ogunyemi{</\textbf{actor}>}\mbox{}\newline 
\hspace*{1em}\hspace*{1em}{</\textbf{castItem}>}\mbox{}\newline 
\hspace*{1em}\hspace*{1em}{<\textbf{castItem}>}\mbox{}\newline 
\hspace*{1em}\hspace*{1em}\hspace*{1em}{<\textbf{role}>}Cripple{</\textbf{role}>}\mbox{}\newline 
\hspace*{1em}\hspace*{1em}\hspace*{1em}{<\textbf{actor}>}Tunji Oyelana{</\textbf{actor}>}\mbox{}\newline 
\hspace*{1em}\hspace*{1em}{</\textbf{castItem}>}\mbox{}\newline 
\hspace*{1em}{</\textbf{castGroup}>}\mbox{}\newline 
\hspace*{1em}{<\textbf{castItem}>}\mbox{}\newline 
\hspace*{1em}\hspace*{1em}{<\textbf{role}>}Si Bero{</\textbf{role}>}\mbox{}\newline 
\hspace*{1em}\hspace*{1em}{<\textbf{roleDesc}>}Sister to Dr Bero{</\textbf{roleDesc}>}\mbox{}\newline 
\hspace*{1em}\hspace*{1em}{<\textbf{actor}>}Deolo Adedoyin{</\textbf{actor}>}\mbox{}\newline 
\hspace*{1em}{</\textbf{castItem}>}\mbox{}\newline 
\hspace*{1em}{<\textbf{castGroup}>}\mbox{}\newline 
\hspace*{1em}\hspace*{1em}{<\textbf{head}\hspace*{1em}{rend}="{braced}">}Two old women{</\textbf{head}>}\mbox{}\newline 
\hspace*{1em}\hspace*{1em}{<\textbf{castItem}>}\mbox{}\newline 
\hspace*{1em}\hspace*{1em}\hspace*{1em}{<\textbf{role}>}Iya Agba{</\textbf{role}>}\mbox{}\newline 
\hspace*{1em}\hspace*{1em}\hspace*{1em}{<\textbf{actor}>}Nguba Agolia{</\textbf{actor}>}\mbox{}\newline 
\hspace*{1em}\hspace*{1em}{</\textbf{castItem}>}\mbox{}\newline 
\hspace*{1em}\hspace*{1em}{<\textbf{castItem}>}\mbox{}\newline 
\hspace*{1em}\hspace*{1em}\hspace*{1em}{<\textbf{role}>}Iya Mate{</\textbf{role}>}\mbox{}\newline 
\hspace*{1em}\hspace*{1em}\hspace*{1em}{<\textbf{actor}>}Bopo George{</\textbf{actor}>}\mbox{}\newline 
\hspace*{1em}\hspace*{1em}{</\textbf{castItem}>}\mbox{}\newline 
\hspace*{1em}{</\textbf{castGroup}>}\mbox{}\newline 
\hspace*{1em}{<\textbf{castItem}>}\mbox{}\newline 
\hspace*{1em}\hspace*{1em}{<\textbf{role}>}Dr Bero{</\textbf{role}>}\mbox{}\newline 
\hspace*{1em}\hspace*{1em}{<\textbf{roleDesc}>}Specialist{</\textbf{roleDesc}>}\mbox{}\newline 
\hspace*{1em}\hspace*{1em}{<\textbf{actor}>}Nat Okoro{</\textbf{actor}>}\mbox{}\newline 
\hspace*{1em}{</\textbf{castItem}>}\mbox{}\newline 
\hspace*{1em}{<\textbf{castItem}>}\mbox{}\newline 
\hspace*{1em}\hspace*{1em}{<\textbf{role}>}Priest{</\textbf{role}>}\mbox{}\newline 
\hspace*{1em}\hspace*{1em}{<\textbf{actor}>}Gbenga Sonuga{</\textbf{actor}>}\mbox{}\newline 
\hspace*{1em}{</\textbf{castItem}>}\mbox{}\newline 
\hspace*{1em}{<\textbf{castItem}>}\mbox{}\newline 
\hspace*{1em}\hspace*{1em}{<\textbf{role}>}The old man{</\textbf{role}>}\mbox{}\newline 
\hspace*{1em}\hspace*{1em}{<\textbf{roleDesc}>}Bero's father{</\textbf{roleDesc}>}\mbox{}\newline 
\hspace*{1em}\hspace*{1em}{<\textbf{actor}>}Dapo Adelugba{</\textbf{actor}>}\mbox{}\newline 
\hspace*{1em}{</\textbf{castItem}>}\mbox{}\newline 
{</\textbf{castList}>}\end{shaded}\egroup\par \noindent  
\subsection[{The Body of a Performance Text}]{The Body of a Performance Text}\label{DRBOD}\par
The body of a performance text may be divided into structural units, variously called acts, scenes, stasima, entr'actes, etc. All such formal divisions should be encoded using an appropriate text-division element (\hyperref[TEI.div]{<div>}, \hyperref[TEI.div1]{<div1>}, \hyperref[TEI.div2]{<div2>}, etc.), as further discussed in section \textit{\hyperref[DRDIV]{7.2.1.\ Major Structural Divisions}}. Whether divided up into such units or not, all performance texts consist of sequences of speeches (see \textit{\hyperref[DRSP]{7.2.2.\ Speeches and Speakers}}) and stage directions (see \textit{\hyperref[DRSTA]{7.2.4.\ Stage Directions}}). In musical performances, it is also common to identify groups of speeches which act as a single unit, sometimes called a \textit{number}; such units typically float within the structural hierarchy at the same level as speeches preceding or following them and cannot therefore be treated as text-divisions. (see \textit{\hyperref[DRSPG]{7.2.3.\ Grouped Speeches}}). Speeches will generally consist of a sequence of \textit{chunk}-level items: paragraphs, verse lines, stanzas, or (in case of uncertainty as to whether something is verse or prose) \hyperref[TEI.ab]{<ab>} elements (see \textit{\hyperref[DRPAL]{7.2.5.\ Speech Contents}}).\par
The boundaries of formal units such as verse lines or paragraphs do not always coincide with speech boundaries. Units such as songs may be discontinuous or shared among several speakers. As described below in section \textit{\hyperref[DREMB]{7.2.6.\ Embedded Structures}}, such fragmentation may be encoded in a relatively simple fashion using the linkage and aggregation mechanisms defined in chapter \textit{\hyperref[SA]{16.\ Linking, Segmentation, and Alignment}}.
\subsubsection[{Major Structural Divisions}]{Major Structural Divisions}\label{DRDIV}\par
Large divisions in drama such as acts, scenes, stasima, or entr'actes are indicated by numbered or unnumbered \hyperref[TEI.div]{<div>} elements, as described in section \textit{\hyperref[DSDIV]{4.1.\ Divisions of the Body}}. The {\itshape type} and {\itshape n} attributes may be used to define the type of division being marked, and to provide a name or number for it, as in the following example: \par\bgroup\index{body=<body>|exampleindex}\index{div1=<div1>|exampleindex}\index{type=@type!<div1>|exampleindex}\index{n=@n!<div1>|exampleindex}\index{head=<head>|exampleindex}\index{div1=<div1>|exampleindex}\index{type=@type!<div1>|exampleindex}\index{n=@n!<div1>|exampleindex}\index{head=<head>|exampleindex}\exampleFont \begin{shaded}\noindent\mbox{}{<\textbf{body}>}\mbox{}\newline 
\hspace*{1em}{<\textbf{div1}\hspace*{1em}{type}="{scene}"\hspace*{1em}{n}="{1}">}\mbox{}\newline 
\hspace*{1em}\hspace*{1em}{<\textbf{head}>}Night—Faust's Study (i){</\textbf{head}>}\mbox{}\newline 
\hspace*{1em}{</\textbf{div1}>}\mbox{}\newline 
\hspace*{1em}{<\textbf{div1}\hspace*{1em}{type}="{scene}"\hspace*{1em}{n}="{2}">}\mbox{}\newline 
\hspace*{1em}\hspace*{1em}{<\textbf{head}>}Outside the City Gate{</\textbf{head}>}\mbox{}\newline 
\hspace*{1em}{</\textbf{div1}>}\mbox{}\newline 
{</\textbf{body}>}\end{shaded}\egroup\par \noindent  \par
Where the largest divisions of a performance text are themselves subdivided, most obviously in the case of plays traditionally divided into acts and scenes, further nested text-division elements may be used, as in this example: \par\bgroup\index{body=<body>|exampleindex}\index{div1=<div1>|exampleindex}\index{type=@type!<div1>|exampleindex}\index{n=@n!<div1>|exampleindex}\index{head=<head>|exampleindex}\index{div2=<div2>|exampleindex}\index{type=@type!<div2>|exampleindex}\index{n=@n!<div2>|exampleindex}\index{stage=<stage>|exampleindex}\index{sp=<sp>|exampleindex}\index{speaker=<speaker>|exampleindex}\index{p=<p>|exampleindex}\index{div2=<div2>|exampleindex}\index{type=@type!<div2>|exampleindex}\index{n=@n!<div2>|exampleindex}\index{stage=<stage>|exampleindex}\index{div1=<div1>|exampleindex}\index{type=@type!<div1>|exampleindex}\index{n=@n!<div1>|exampleindex}\index{head=<head>|exampleindex}\index{div2=<div2>|exampleindex}\index{type=@type!<div2>|exampleindex}\index{n=@n!<div2>|exampleindex}\index{head=<head>|exampleindex}\index{div2=<div2>|exampleindex}\index{type=@type!<div2>|exampleindex}\index{n=@n!<div2>|exampleindex}\index{head=<head>|exampleindex}\exampleFont \begin{shaded}\noindent\mbox{}{<\textbf{body}>}\mbox{}\newline 
\hspace*{1em}{<\textbf{div1}\hspace*{1em}{type}="{act}"\hspace*{1em}{n}="{1}">}\mbox{}\newline 
\hspace*{1em}\hspace*{1em}{<\textbf{head}>}Act One{</\textbf{head}>}\mbox{}\newline 
\hspace*{1em}\hspace*{1em}{<\textbf{div2}\hspace*{1em}{type}="{scene}"\hspace*{1em}{n}="{1}">}\mbox{}\newline 
\hspace*{1em}\hspace*{1em}\hspace*{1em}{<\textbf{stage}>}Pa Ubu, Ma Ubu{</\textbf{stage}>}\mbox{}\newline 
\hspace*{1em}\hspace*{1em}\hspace*{1em}{<\textbf{sp}>}\mbox{}\newline 
\hspace*{1em}\hspace*{1em}\hspace*{1em}\hspace*{1em}{<\textbf{speaker}>}Pa Ubu{</\textbf{speaker}>}\mbox{}\newline 
\hspace*{1em}\hspace*{1em}\hspace*{1em}\hspace*{1em}{<\textbf{p}>}Pschitt!{</\textbf{p}>}\mbox{}\newline 
\hspace*{1em}\hspace*{1em}\hspace*{1em}{</\textbf{sp}>}\mbox{}\newline 
\hspace*{1em}\hspace*{1em}{</\textbf{div2}>}\mbox{}\newline 
\hspace*{1em}\hspace*{1em}{<\textbf{div2}\hspace*{1em}{type}="{scene}"\hspace*{1em}{n}="{2}">}\mbox{}\newline 
\hspace*{1em}\hspace*{1em}\hspace*{1em}{<\textbf{stage}>}A room in Pa Ubu's house, where a magnificent\mbox{}\newline 
\hspace*{1em}\hspace*{1em}\hspace*{1em}\hspace*{1em}\hspace*{1em}\hspace*{1em} collation is set out{</\textbf{stage}>}\mbox{}\newline 
\hspace*{1em}\hspace*{1em}{</\textbf{div2}>}\mbox{}\newline 
\hspace*{1em}{</\textbf{div1}>}\mbox{}\newline 
\hspace*{1em}{<\textbf{div1}\hspace*{1em}{type}="{act}"\hspace*{1em}{n}="{2}">}\mbox{}\newline 
\hspace*{1em}\hspace*{1em}{<\textbf{head}>}Act Two{</\textbf{head}>}\mbox{}\newline 
\hspace*{1em}\hspace*{1em}{<\textbf{div2}\hspace*{1em}{type}="{scene}"\hspace*{1em}{n}="{1}">}\mbox{}\newline 
\hspace*{1em}\hspace*{1em}\hspace*{1em}{<\textbf{head}>}Scene One{</\textbf{head}>}\mbox{}\newline 
\hspace*{1em}\hspace*{1em}{</\textbf{div2}>}\mbox{}\newline 
\hspace*{1em}\hspace*{1em}{<\textbf{div2}\hspace*{1em}{type}="{scene}"\hspace*{1em}{n}="{2}">}\mbox{}\newline 
\hspace*{1em}\hspace*{1em}\hspace*{1em}{<\textbf{head}>}Scene Two{</\textbf{head}>}\mbox{}\newline 
\hspace*{1em}\hspace*{1em}{</\textbf{div2}>}\mbox{}\newline 
\hspace*{1em}{</\textbf{div1}>}\mbox{}\newline 
{</\textbf{body}>}\end{shaded}\egroup\par \noindent  \par
In the example above, the \hyperref[TEI.div2]{<div2>} element has been used to represent the ‘French scene’ convention, (where the entrance of each new set of characters is marked as a distinct unit in the text) and the \hyperref[TEI.div1]{<div1>} element to represent the acts into which the play is divided. The elements chosen are determined only by the hierarchic position of these units in the text as a whole. If the text had no acts, but only scenes, then the scenes might be represented by \hyperref[TEI.div1]{<div1>} elements. Equally, if a play is divided only into ‘acts’, with no smaller subdivisions, then the \hyperref[TEI.div1]{<div1>} element might be used to represent acts. The {\itshape type} should be used, as above, to make explicit the name associated with a particular category of subdivision.\par
As an alternative to the use of numbered divisions, the encoder may represent all subdivisions with the same element, the unnumbered \hyperref[TEI.div]{<div>}. The second act in the above example would then be represented as follows: \par\bgroup\index{div=<div>|exampleindex}\index{type=@type!<div>|exampleindex}\index{n=@n!<div>|exampleindex}\index{head=<head>|exampleindex}\index{div=<div>|exampleindex}\index{type=@type!<div>|exampleindex}\index{n=@n!<div>|exampleindex}\index{head=<head>|exampleindex}\index{div=<div>|exampleindex}\index{type=@type!<div>|exampleindex}\index{n=@n!<div>|exampleindex}\index{head=<head>|exampleindex}\exampleFont \begin{shaded}\noindent\mbox{}{<\textbf{div}\hspace*{1em}{type}="{act}"\hspace*{1em}{n}="{2}">}\mbox{}\newline 
\hspace*{1em}{<\textbf{head}>}Act Two{</\textbf{head}>}\mbox{}\newline 
\hspace*{1em}{<\textbf{div}\hspace*{1em}{type}="{scene}"\hspace*{1em}{n}="{1}">}\mbox{}\newline 
\hspace*{1em}\hspace*{1em}{<\textbf{head}>}Scene One{</\textbf{head}>}\mbox{}\newline 
\hspace*{1em}{</\textbf{div}>}\mbox{}\newline 
\hspace*{1em}{<\textbf{div}\hspace*{1em}{type}="{scene}"\hspace*{1em}{n}="{2}">}\mbox{}\newline 
\hspace*{1em}\hspace*{1em}{<\textbf{head}>}Scene Two{</\textbf{head}>}\mbox{}\newline 
\hspace*{1em}{</\textbf{div}>}\mbox{}\newline 
{</\textbf{div}>}\end{shaded}\egroup\par \par
For further discussion of the use of numbered and unnumbered divisions, see section \textit{\hyperref[DSDIV]{4.1.\ Divisions of the Body}}.
\subsubsection[{Speeches and Speakers}]{Speeches and Speakers}\label{DRSP}\par
The following elements are used to identify speeches and speakers in a performance text: 
\begin{sansreflist}
  
\item [\textbf{<sp>}] (speech) contains an individual speech in a performance text, or a passage presented as such in a prose or verse text.
\item [\textbf{<speaker>}] contains a specialized form of heading or label, giving the name of one or more speakers in a dramatic text or fragment.
\end{sansreflist}
\par
As noted above, the structure of many performance texts may be analysed as multiply hierarchic: a scene of a verse play, for example, may be divided into speeches and, at the same time, into verse lines. The end of a line may or may not coincide with the end of a speech, and vice versa. Other structures, such as songs, may be discontinuous or split up over several speeches. For some purposes it will be appropriate to regard the verse-structure as the fundamental organizing principle of the text, and for others the speech structure; in some cases, the choice between the two may be arbitrary. The discussion in the remainder of this chapter assumes that it is the speech-based hierarchy which most prominently determines the structure of performance texts, but the same mechanisms could be employed to encode a view of a performance text in which individual speeches were entirely subordinate to the formal units of prose and verse. For more detailed discussion and examples of various treatments of this fundamental issue, refer to chapter \textit{\hyperref[NH]{20.\ Non-hierarchical Structures}}.\par
The {\itshape who} attribute and the \hyperref[TEI.speaker]{<speaker>} element are both used to indicate the speaker or speakers of a speech, but in rather different ways. The \hyperref[TEI.speaker]{<speaker>} element is used to encode the word or phrase actually used within the source text to indicate the speaker: it may contain any string or prefix, and may be thought of as a highly specialized form of stage direction. The {\itshape who} attribute however contains one or more pointer values, each of which indicates one or more other XML elements documenting the character to whom the speech is assigned. Typically, this attribute might point to a \hyperref[TEI.person]{<person>} element in the TEI header \textit{\hyperref[CCAHPA]{15.2.2.\ The Participant Description}}, to a \hyperref[TEI.role]{<role>} element in the cast list \textit{\hyperref[DRCAST]{7.1.4.\ Cast Lists}}, or even to some external source such as an online handbook of dramatic roles. The most usual case is that the pointer value supplied (prefixed by a sharp sign) corresponds with the value of an {\itshape xml:id} attribute, used elsewhere in the document to identify a particular element, as in the following examples: \par\bgroup\index{castList=<castList>|exampleindex}\index{castItem=<castItem>|exampleindex}\index{role=<role>|exampleindex}\index{castItem=<castItem>|exampleindex}\index{role=<role>|exampleindex}\index{sp=<sp>|exampleindex}\index{who=@who!<sp>|exampleindex}\index{speaker=<speaker>|exampleindex}\index{l=<l>|exampleindex}\index{sp=<sp>|exampleindex}\index{who=@who!<sp>|exampleindex}\index{speaker=<speaker>|exampleindex}\index{l=<l>|exampleindex}\index{sp=<sp>|exampleindex}\index{who=@who!<sp>|exampleindex}\index{speaker=<speaker>|exampleindex}\index{l=<l>|exampleindex}\index{l=<l>|exampleindex}\exampleFont \begin{shaded}\noindent\mbox{}{<\textbf{castList}>}\mbox{}\newline 
\hspace*{1em}{<\textbf{castItem}>}\mbox{}\newline 
\hspace*{1em}\hspace*{1em}{<\textbf{role}\hspace*{1em}{xml:id}="{menae}">}Menaechmus{</\textbf{role}>}\mbox{}\newline 
\hspace*{1em}{</\textbf{castItem}>}\mbox{}\newline 
\hspace*{1em}{<\textbf{castItem}>}\mbox{}\newline 
\hspace*{1em}\hspace*{1em}{<\textbf{role}\hspace*{1em}{xml:id}="{penic}">}Peniculus{</\textbf{role}>}\mbox{}\newline 
\hspace*{1em}{</\textbf{castItem}>}\mbox{}\newline 
{</\textbf{castList}>}\mbox{}\newline 
{<\textbf{sp}\hspace*{1em}{who}="{\#menae}">}\mbox{}\newline 
\hspace*{1em}{<\textbf{speaker}>}Menaechmus{</\textbf{speaker}>}\mbox{}\newline 
\hspace*{1em}{<\textbf{l}>}Responde, adulescens, quaeso, quid nomen tibist?{</\textbf{l}>}\mbox{}\newline 
{</\textbf{sp}>}\mbox{}\newline 
{<\textbf{sp}\hspace*{1em}{who}="{\#penic}">}\mbox{}\newline 
\hspace*{1em}{<\textbf{speaker}>}Peniculus{</\textbf{speaker}>}\mbox{}\newline 
\hspace*{1em}{<\textbf{l}>}Etiam derides, quasi nomen non noveris?{</\textbf{l}>}\mbox{}\newline 
{</\textbf{sp}>}\mbox{}\newline 
{<\textbf{sp}\hspace*{1em}{who}="{\#menae}">}\mbox{}\newline 
\hspace*{1em}{<\textbf{speaker}>}Menaechmus{</\textbf{speaker}>}\mbox{}\newline 
\hspace*{1em}{<\textbf{l}>}Non edepol ego te, quot sciam, umquam ante hunc diem{</\textbf{l}>}\mbox{}\newline 
\hspace*{1em}{<\textbf{l}>}Vidi neque novi; ...{</\textbf{l}>}\mbox{}\newline 
{</\textbf{sp}>}\end{shaded}\egroup\par \noindent  \par
If present, a \hyperref[TEI.speaker]{<speaker>} element may only appear as the first part of an \hyperref[TEI.sp]{<sp>} element. The distinction between the \hyperref[TEI.speaker]{<speaker>} element and the {\itshape who} attribute makes it possible to encode uniformly characters whose names are not indicated in a uniform fashion throughout the play, or characters who appear in disguise, as in the following examples: \par\bgroup\index{castList=<castList>|exampleindex}\index{castItem=<castItem>|exampleindex}\index{role=<role>|exampleindex}\index{sp=<sp>|exampleindex}\index{who=@who!<sp>|exampleindex}\index{speaker=<speaker>|exampleindex}\index{p=<p>|exampleindex}\exampleFont \begin{shaded}\noindent\mbox{}{<\textbf{castList}>}\mbox{}\newline 
\hspace*{1em}{<\textbf{castItem}>}\mbox{}\newline 
\hspace*{1em}\hspace*{1em}{<\textbf{role}\hspace*{1em}{xml:id}="{hh}">}Henry Higgins{</\textbf{role}>}\mbox{}\newline 
\hspace*{1em}{</\textbf{castItem}>}\mbox{}\newline 
{</\textbf{castList}>}\mbox{}\newline 
{<\textbf{sp}\hspace*{1em}{who}="{\#hh}">}\mbox{}\newline 
\hspace*{1em}{<\textbf{speaker}>}The Notetaker{</\textbf{speaker}>}\mbox{}\newline 
\hspace*{1em}{<\textbf{p}>} ... {</\textbf{p}>}\mbox{}\newline 
{</\textbf{sp}>}\end{shaded}\egroup\par \noindent  \par
If the speaker attributions are completely regular (and may thus be reconstructed mechanically from the values given for the {\itshape who} attribute), or are of no interest for the encoder of the text (as might be the case with editorially supplied attributions in older texts), then the \hyperref[TEI.speaker]{<speaker>} element need not be used; the former example above then might look like this: \par\bgroup\index{castList=<castList>|exampleindex}\index{castItem=<castItem>|exampleindex}\index{role=<role>|exampleindex}\index{castItem=<castItem>|exampleindex}\index{role=<role>|exampleindex}\index{sp=<sp>|exampleindex}\index{who=@who!<sp>|exampleindex}\index{l=<l>|exampleindex}\index{sp=<sp>|exampleindex}\index{who=@who!<sp>|exampleindex}\index{l=<l>|exampleindex}\index{sp=<sp>|exampleindex}\index{who=@who!<sp>|exampleindex}\index{l=<l>|exampleindex}\index{l=<l>|exampleindex}\exampleFont \begin{shaded}\noindent\mbox{}{<\textbf{castList}>}\mbox{}\newline 
\hspace*{1em}{<\textbf{castItem}>}\mbox{}\newline 
\hspace*{1em}\hspace*{1em}{<\textbf{role}\hspace*{1em}{xml:id}="{menaechmus}">}Menaechmus{</\textbf{role}>}\mbox{}\newline 
\hspace*{1em}{</\textbf{castItem}>}\mbox{}\newline 
\hspace*{1em}{<\textbf{castItem}>}\mbox{}\newline 
\hspace*{1em}\hspace*{1em}{<\textbf{role}\hspace*{1em}{xml:id}="{peniculus}">}Peniculus{</\textbf{role}>}\mbox{}\newline 
\hspace*{1em}{</\textbf{castItem}>}\mbox{}\newline 
{</\textbf{castList}>}\mbox{}\newline 
{<\textbf{sp}\hspace*{1em}{who}="{\#menaechmus}">}\mbox{}\newline 
\hspace*{1em}{<\textbf{l}>}Responde, adulescens, quaeso, quid nomen tibist?{</\textbf{l}>}\mbox{}\newline 
{</\textbf{sp}>}\mbox{}\newline 
{<\textbf{sp}\hspace*{1em}{who}="{\#peniculus}">}\mbox{}\newline 
\hspace*{1em}{<\textbf{l}>}Etiam derides, quasi nomen non noveris?{</\textbf{l}>}\mbox{}\newline 
{</\textbf{sp}>}\mbox{}\newline 
{<\textbf{sp}\hspace*{1em}{who}="{\#menaechmus}">}\mbox{}\newline 
\hspace*{1em}{<\textbf{l}>}Non edepol ego te, quot sciam, umquam ante hunc diem{</\textbf{l}>}\mbox{}\newline 
\hspace*{1em}{<\textbf{l}>}Vidi neque novi; ...{</\textbf{l}>}\mbox{}\newline 
{</\textbf{sp}>}\end{shaded}\egroup\par \noindent  \par
More than one identifier may be listed as value for the {\itshape who} attribute if the speech is spoken by more than one person, as in the following example: \par\bgroup\index{castList=<castList>|exampleindex}\index{castItem=<castItem>|exampleindex}\index{role=<role>|exampleindex}\index{castItem=<castItem>|exampleindex}\index{role=<role>|exampleindex}\index{stage=<stage>|exampleindex}\index{sp=<sp>|exampleindex}\index{who=@who!<sp>|exampleindex}\index{l=<l>|exampleindex}\index{l=<l>|exampleindex}\exampleFont \begin{shaded}\noindent\mbox{}{<\textbf{castList}>}\mbox{}\newline 
\hspace*{1em}{<\textbf{castItem}>}\mbox{}\newline 
\hspace*{1em}\hspace*{1em}{<\textbf{role}\hspace*{1em}{xml:id}="{nan}">}Nano{</\textbf{role}>}\mbox{}\newline 
\hspace*{1em}{</\textbf{castItem}>}\mbox{}\newline 
\hspace*{1em}{<\textbf{castItem}>}\mbox{}\newline 
\hspace*{1em}\hspace*{1em}{<\textbf{role}\hspace*{1em}{xml:id}="{cas}">}Castrone{</\textbf{role}>}\mbox{}\newline 
\hspace*{1em}{</\textbf{castItem}>}\mbox{}\newline 
{</\textbf{castList}>}\mbox{}\newline 
{<\textbf{stage}>}Nano and Castrone sing{</\textbf{stage}>}\mbox{}\newline 
{<\textbf{sp}\hspace*{1em}{who}="{\#nan \#cas}">}\mbox{}\newline 
\hspace*{1em}{<\textbf{l}>}Fools, they are the only nation{</\textbf{l}>}\mbox{}\newline 
\hspace*{1em}{<\textbf{l}>}Worth men's envy or admiration{</\textbf{l}>}\mbox{}\newline 
{</\textbf{sp}>}\end{shaded}\egroup\par \noindent  \par
In the event there is a speech that is assigned to a character that is not listed in the source cast list, a \hyperref[TEI.castList]{<castList>} may be encoded inside the \hyperref[TEI.standOff]{<standOff>} element to provide an element to which the {\itshape who} of \hyperref[TEI.sp]{<sp>} may point.\par
The \hyperref[TEI.sp]{<sp>} and \hyperref[TEI.speaker]{<speaker>} elements are both declared within the core module (see section \textit{\hyperref[CODV]{3.13.\ Passages of Verse or Drama}}).
\subsubsection[{Grouped Speeches}]{Grouped Speeches}\label{DRSPG}\par
This module makes available the following additional element for handling groups of speeches: 
\begin{sansreflist}
  
\item [\textbf{<spGrp>}] (speech group) contains a group of speeches or songs in a performance text presented in a source as constituting a single unit or ‘number’.
\end{sansreflist}
\par
The \hyperref[TEI.spGrp]{<spGrp>} element is intended for cases where the characters in a performance launch into something which might be regarded almost as a kind of separate structural division, typically associated with its own heading or numbering system, but which ‘floats’ in the text, at the same hierarchic level as speeches preceding or following it. Such units are often numbered, titled, and visually presented as distinct objects within the text. Here is a typical example from a well-known American musical comedy: \par\bgroup\index{spGrp=<spGrp>|exampleindex}\index{type=@type!<spGrp>|exampleindex}\index{n=@n!<spGrp>|exampleindex}\index{head=<head>|exampleindex}\index{sp=<sp>|exampleindex}\index{speaker=<speaker>|exampleindex}\index{lg=<lg>|exampleindex}\index{l=<l>|exampleindex}\index{l=<l>|exampleindex}\index{l=<l>|exampleindex}\index{l=<l>|exampleindex}\index{l=<l>|exampleindex}\index{sp=<sp>|exampleindex}\index{speaker=<speaker>|exampleindex}\index{lg=<lg>|exampleindex}\index{l=<l>|exampleindex}\index{l=<l>|exampleindex}\index{l=<l>|exampleindex}\index{l=<l>|exampleindex}\index{l=<l>|exampleindex}\index{sp=<sp>|exampleindex}\index{speaker=<speaker>|exampleindex}\index{lg=<lg>|exampleindex}\index{l=<l>|exampleindex}\index{l=<l>|exampleindex}\index{l=<l>|exampleindex}\index{l=<l>|exampleindex}\index{l=<l>|exampleindex}\index{l=<l>|exampleindex}\index{l=<l>|exampleindex}\index{l=<l>|exampleindex}\index{sp=<sp>|exampleindex}\index{speaker=<speaker>|exampleindex}\index{lg=<lg>|exampleindex}\index{l=<l>|exampleindex}\index{l=<l>|exampleindex}\exampleFont \begin{shaded}\noindent\mbox{}{<\textbf{spGrp}\hspace*{1em}{type}="{number}"\hspace*{1em}{n}="{3}">}\mbox{}\newline 
\hspace*{1em}{<\textbf{head}>}By Strauss : performed by Georges Guetary, Gene Kelly, and Oscar\mbox{}\newline 
\hspace*{1em}\hspace*{1em} Levant{</\textbf{head}>}\mbox{}\newline 
\hspace*{1em}{<\textbf{sp}>}\mbox{}\newline 
\hspace*{1em}\hspace*{1em}{<\textbf{speaker}>}HENRI BAUREL{</\textbf{speaker}>}\mbox{}\newline 
\hspace*{1em}\hspace*{1em}{<\textbf{lg}>}\mbox{}\newline 
\hspace*{1em}\hspace*{1em}\hspace*{1em}{<\textbf{l}>}The waltzes of Mittel Europa {</\textbf{l}>}\mbox{}\newline 
\hspace*{1em}\hspace*{1em}\hspace*{1em}{<\textbf{l}>}They charm you and warm you within {</\textbf{l}>}\mbox{}\newline 
\hspace*{1em}\hspace*{1em}\hspace*{1em}{<\textbf{l}>}While each day discloses {</\textbf{l}>}\mbox{}\newline 
\hspace*{1em}\hspace*{1em}\hspace*{1em}{<\textbf{l}>}What Broadway composes {</\textbf{l}>}\mbox{}\newline 
\hspace*{1em}\hspace*{1em}\hspace*{1em}{<\textbf{l}>}Is emptiness pounding on tin.{</\textbf{l}>}\mbox{}\newline 
\hspace*{1em}\hspace*{1em}{</\textbf{lg}>}\mbox{}\newline 
\hspace*{1em}{</\textbf{sp}>}\mbox{}\newline 
\hspace*{1em}{<\textbf{sp}\hspace*{1em}{xml:lang}="{de}">}\mbox{}\newline 
\hspace*{1em}\hspace*{1em}{<\textbf{speaker}>}JERRY MULLIGAN: ADAM COOK:{</\textbf{speaker}>}\mbox{}\newline 
\hspace*{1em}\hspace*{1em}{<\textbf{lg}>}\mbox{}\newline 
\hspace*{1em}\hspace*{1em}\hspace*{1em}{<\textbf{l}>}Mein Herr! {</\textbf{l}>}\mbox{}\newline 
\hspace*{1em}\hspace*{1em}\hspace*{1em}{<\textbf{l}>}Mein Herr!{</\textbf{l}>}\mbox{}\newline 
\hspace*{1em}\hspace*{1em}\hspace*{1em}{<\textbf{l}>}Bitte, bitte!{</\textbf{l}>}\mbox{}\newline 
\hspace*{1em}\hspace*{1em}\hspace*{1em}{<\textbf{l}>}Denke, danke!{</\textbf{l}>}\mbox{}\newline 
\hspace*{1em}\hspace*{1em}\hspace*{1em}{<\textbf{l}>}Aufwiedersehen! Aufwiedersehen!{</\textbf{l}>}\mbox{}\newline 
\hspace*{1em}\hspace*{1em}{</\textbf{lg}>}\mbox{}\newline 
\hspace*{1em}{</\textbf{sp}>}\mbox{}\newline 
\hspace*{1em}{<\textbf{sp}>}\mbox{}\newline 
\hspace*{1em}\hspace*{1em}{<\textbf{speaker}>}HENRI BAUREL:{</\textbf{speaker}>}\mbox{}\newline 
\hspace*{1em}\hspace*{1em}{<\textbf{lg}>}\mbox{}\newline 
\hspace*{1em}\hspace*{1em}\hspace*{1em}{<\textbf{l}>}How can I be civil {</\textbf{l}>}\mbox{}\newline 
\hspace*{1em}\hspace*{1em}\hspace*{1em}{<\textbf{l}>}When hearing this drivel? {</\textbf{l}>}\mbox{}\newline 
\hspace*{1em}\hspace*{1em}\hspace*{1em}{<\textbf{l}>}It's only for night-clubbing souses. {</\textbf{l}>}\mbox{}\newline 
\hspace*{1em}\hspace*{1em}\hspace*{1em}{<\textbf{l}>}Oh give me the free 'n easy {</\textbf{l}>}\mbox{}\newline 
\hspace*{1em}\hspace*{1em}\hspace*{1em}{<\textbf{l}>}Waltz that is Viennes-y {</\textbf{l}>}\mbox{}\newline 
\hspace*{1em}\hspace*{1em}\hspace*{1em}{<\textbf{l}>}And go tell the band{</\textbf{l}>}\mbox{}\newline 
\hspace*{1em}\hspace*{1em}\hspace*{1em}{<\textbf{l}>}If they want a hand{</\textbf{l}>}\mbox{}\newline 
\hspace*{1em}\hspace*{1em}\hspace*{1em}{<\textbf{l}>}The waltz must be Strauss's{</\textbf{l}>}\mbox{}\newline 
\hspace*{1em}\hspace*{1em}{</\textbf{lg}>}\mbox{}\newline 
\hspace*{1em}{</\textbf{sp}>}\mbox{}\newline 
\hspace*{1em}{<\textbf{sp}>}\mbox{}\newline 
\hspace*{1em}\hspace*{1em}{<\textbf{speaker}>}ALL{</\textbf{speaker}>}\mbox{}\newline 
\hspace*{1em}\hspace*{1em}{<\textbf{lg}>}\mbox{}\newline 
\hspace*{1em}\hspace*{1em}\hspace*{1em}{<\textbf{l}>}Ya ya ya {</\textbf{l}>}\mbox{}\newline 
\hspace*{1em}\hspace*{1em}\hspace*{1em}{<\textbf{l}>}Give me oom pah pah...{</\textbf{l}>}\mbox{}\newline 
\hspace*{1em}\hspace*{1em}{</\textbf{lg}>}\mbox{}\newline 
\hspace*{1em}{</\textbf{sp}>}\mbox{}\newline 
\textit{<!-- ... -->}\mbox{}\newline 
{</\textbf{spGrp}>}\end{shaded}\egroup\par 
\subsubsection[{Stage Directions}]{Stage Directions}\label{DRSTA}\par
Both between and within the speeches of a written performance text, it is normal practice to include a wide variety of descriptive directions to indicate non-verbal action. The following elements are provided to represent these: 
\begin{sansreflist}
  
\item [\textbf{<stage>}] (stage direction) contains any kind of stage direction within a dramatic text or fragment.\hfil\\[-10pt]\begin{sansreflist}
    \item[@{\itshape type}]
  indicates the kind of stage direction.
\end{sansreflist}  
\item [\textbf{<move>}] (movement) marks the actual movement of one or more characters.\hfil\\[-10pt]\begin{sansreflist}
    \item[@{\itshape type}]
  characterizes the movement, for example as an entrance or exit.
    \item[@{\itshape where}]
  specifies the direction of a stage movement.
    \item[@{\itshape perf}]
  (performance) identifies the performance or performances in which this movement occurred as specified by pointing to one or more \hyperref[TEI.performance]{<performance>} elements.
\end{sansreflist}  
\end{sansreflist}
\par
A satisfactory typology of stage directions is difficult to define. Certain basic types such as ‘entrance’, ‘exit’, ‘setting’, ‘delivery’, are easily identified. But the list is not a closed one, and it is not uncommon to mix types within a single direction. No closed set of values for the {\itshape type} attribute is therefore proposed at the present time, though some suggested values are indicated in the list below, which also indicates the range of possibilities. \par\bgroup\index{stage=<stage>|exampleindex}\index{type=@type!<stage>|exampleindex}\index{stage=<stage>|exampleindex}\index{type=@type!<stage>|exampleindex}\index{stage=<stage>|exampleindex}\index{type=@type!<stage>|exampleindex}\index{stage=<stage>|exampleindex}\index{type=@type!<stage>|exampleindex}\index{stage=<stage>|exampleindex}\index{type=@type!<stage>|exampleindex}\index{stage=<stage>|exampleindex}\index{type=@type!<stage>|exampleindex}\index{stage=<stage>|exampleindex}\index{type=@type!<stage>|exampleindex}\index{stage=<stage>|exampleindex}\index{type=@type!<stage>|exampleindex}\index{stage=<stage>|exampleindex}\index{type=@type!<stage>|exampleindex}\index{stage=<stage>|exampleindex}\index{type=@type!<stage>|exampleindex}\index{stage=<stage>|exampleindex}\index{type=@type!<stage>|exampleindex}\exampleFont \begin{shaded}\noindent\mbox{}{<\textbf{stage}\hspace*{1em}{type}="{setting}">}The throne descends.{</\textbf{stage}>}\mbox{}\newline 
{<\textbf{stage}\hspace*{1em}{type}="{setting}">}Music{</\textbf{stage}>}\mbox{}\newline 
{<\textbf{stage}\hspace*{1em}{type}="{entrance}">}Enter Husband as being thrown off his horse.{</\textbf{stage}>}\mbox{}\newline 
{<\textbf{stage}\hspace*{1em}{type}="{exit}">}Exit pursued by a bear.{</\textbf{stage}>}\mbox{}\newline 
{<\textbf{stage}\hspace*{1em}{type}="{business}">}He quickly takes the stone out.{</\textbf{stage}>}\mbox{}\newline 
{<\textbf{stage}\hspace*{1em}{type}="{delivery}">}To Lussurioso.{</\textbf{stage}>}\mbox{}\newline 
{<\textbf{stage}\hspace*{1em}{type}="{delivery}">}Aside.{</\textbf{stage}>}\mbox{}\newline 
{<\textbf{stage}\hspace*{1em}{type}="{delivery}">}Not knowing what to say.{</\textbf{stage}>}\mbox{}\newline 
{<\textbf{stage}\hspace*{1em}{type}="{costume}">}Disguised as Ansaldo.{</\textbf{stage}>}\mbox{}\newline 
{<\textbf{stage}\hspace*{1em}{type}="{location}">}At a window.{</\textbf{stage}>}\mbox{}\newline 
{<\textbf{stage}\hspace*{1em}{type}="{novelistic}">}Having had enough, and embarrassed\mbox{}\newline 
 for the family.{</\textbf{stage}>}\end{shaded}\egroup\par \par
The meaning of the values used for the {\itshape type} attribute on \hyperref[TEI.stage]{<stage>} elements may be defined within the \hyperref[TEI.tagUsage]{<tagUsage>} element of the TEI header (described in section \textit{\hyperref[HD57]{2.3.4.\ The Tagging Declaration}}). For example: \par\bgroup\index{tagUsage=<tagUsage>|exampleindex}\index{gi=@gi!<tagUsage>|exampleindex}\index{list=<list>|exampleindex}\index{type=@type!<list>|exampleindex}\index{label=<label>|exampleindex}\index{item=<item>|exampleindex}\index{label=<label>|exampleindex}\index{item=<item>|exampleindex}\index{label=<label>|exampleindex}\index{item=<item>|exampleindex}\index{label=<label>|exampleindex}\index{item=<item>|exampleindex}\index{label=<label>|exampleindex}\index{item=<item>|exampleindex}\exampleFont \begin{shaded}\noindent\mbox{}{<\textbf{tagUsage}\hspace*{1em}{gi}="{stage}">}This element is used for all stage directions,\mbox{}\newline 
 editorial or authorial. The type attribute on this element takes\mbox{}\newline 
 one or more of the following values:\mbox{}\newline 
{<\textbf{list}\hspace*{1em}{type}="{gloss}">}\mbox{}\newline 
\hspace*{1em}\hspace*{1em}{<\textbf{label}>}setting{</\textbf{label}>}\mbox{}\newline 
\hspace*{1em}\hspace*{1em}{<\textbf{item}>}describes the set{</\textbf{item}>}\mbox{}\newline 
\hspace*{1em}\hspace*{1em}{<\textbf{label}>}blocking{</\textbf{label}>}\mbox{}\newline 
\hspace*{1em}\hspace*{1em}{<\textbf{item}>}describes movement across stage, position, etc.{</\textbf{item}>}\mbox{}\newline 
\hspace*{1em}\hspace*{1em}{<\textbf{label}>}business{</\textbf{label}>}\mbox{}\newline 
\hspace*{1em}\hspace*{1em}{<\textbf{item}>}describes movement other than blocking{</\textbf{item}>}\mbox{}\newline 
\hspace*{1em}\hspace*{1em}{<\textbf{label}>}delivery{</\textbf{label}>}\mbox{}\newline 
\hspace*{1em}\hspace*{1em}{<\textbf{item}>}describes how the line is said{</\textbf{item}>}\mbox{}\newline 
\hspace*{1em}\hspace*{1em}{<\textbf{label}>}motivation{</\textbf{label}>}\mbox{}\newline 
\hspace*{1em}\hspace*{1em}{<\textbf{item}>}describes character's emotional state or through line{</\textbf{item}>}\mbox{}\newline 
\hspace*{1em}{</\textbf{list}>}\mbox{}\newline 
{</\textbf{tagUsage}>}\end{shaded}\egroup\par \par
This approach is purely documentary; in a real project it would generally be more effective to define the range of permitted values explicitly within the project's schema specification, using the techniques described in chapter \textit{\hyperref[MD]{23.3.\ Customization}}. For example, a specification like the following might be used to produce a schema in which the {\itshape type} attribute of the \hyperref[TEI.stage]{<stage>} element is permitted to take only the values listed above: \par\bgroup\index{schemaSpec=<schemaSpec>|exampleindex}\index{ident=@ident!<schemaSpec>|exampleindex}\index{moduleRef=<moduleRef>|exampleindex}\index{key=@key!<moduleRef>|exampleindex}\index{moduleRef=<moduleRef>|exampleindex}\index{key=@key!<moduleRef>|exampleindex}\index{moduleRef=<moduleRef>|exampleindex}\index{key=@key!<moduleRef>|exampleindex}\index{moduleRef=<moduleRef>|exampleindex}\index{key=@key!<moduleRef>|exampleindex}\index{moduleRef=<moduleRef>|exampleindex}\index{key=@key!<moduleRef>|exampleindex}\index{elementSpec=<elementSpec>|exampleindex}\index{ident=@ident!<elementSpec>|exampleindex}\index{mode=@mode!<elementSpec>|exampleindex}\index{attList=<attList>|exampleindex}\index{attDef=<attDef>|exampleindex}\index{ident=@ident!<attDef>|exampleindex}\index{mode=@mode!<attDef>|exampleindex}\index{valList=<valList>|exampleindex}\index{type=@type!<valList>|exampleindex}\index{valItem=<valItem>|exampleindex}\index{ident=@ident!<valItem>|exampleindex}\index{desc=<desc>|exampleindex}\index{valItem=<valItem>|exampleindex}\index{ident=@ident!<valItem>|exampleindex}\index{desc=<desc>|exampleindex}\index{valItem=<valItem>|exampleindex}\index{ident=@ident!<valItem>|exampleindex}\index{desc=<desc>|exampleindex}\index{valItem=<valItem>|exampleindex}\index{ident=@ident!<valItem>|exampleindex}\index{desc=<desc>|exampleindex}\index{valItem=<valItem>|exampleindex}\index{ident=@ident!<valItem>|exampleindex}\index{desc=<desc>|exampleindex}\exampleFont \begin{shaded}\noindent\mbox{}{<\textbf{schemaSpec}\hspace*{1em}{ident}="{myDrama}">}\mbox{}\newline 
\hspace*{1em}{<\textbf{moduleRef}\hspace*{1em}{key}="{core}"/>}\mbox{}\newline 
\hspace*{1em}{<\textbf{moduleRef}\hspace*{1em}{key}="{tei}"/>}\mbox{}\newline 
\hspace*{1em}{<\textbf{moduleRef}\hspace*{1em}{key}="{structure}"/>}\mbox{}\newline 
\hspace*{1em}{<\textbf{moduleRef}\hspace*{1em}{key}="{header}"/>}\mbox{}\newline 
\hspace*{1em}{<\textbf{moduleRef}\hspace*{1em}{key}="{drama}"/>}\mbox{}\newline 
\hspace*{1em}{<\textbf{elementSpec}\hspace*{1em}{ident}="{stage}"\hspace*{1em}{mode}="{change}">}\mbox{}\newline 
\hspace*{1em}\hspace*{1em}{<\textbf{attList}>}\mbox{}\newline 
\hspace*{1em}\hspace*{1em}\hspace*{1em}{<\textbf{attDef}\hspace*{1em}{ident}="{type}"\hspace*{1em}{mode}="{replace}">}\mbox{}\newline 
\hspace*{1em}\hspace*{1em}\hspace*{1em}\hspace*{1em}{<\textbf{valList}\hspace*{1em}{type}="{closed}">}\mbox{}\newline 
\hspace*{1em}\hspace*{1em}\hspace*{1em}\hspace*{1em}\hspace*{1em}{<\textbf{valItem}\hspace*{1em}{ident}="{setting}">}\mbox{}\newline 
\hspace*{1em}\hspace*{1em}\hspace*{1em}\hspace*{1em}\hspace*{1em}\hspace*{1em}{<\textbf{desc}>}describes the set{</\textbf{desc}>}\mbox{}\newline 
\hspace*{1em}\hspace*{1em}\hspace*{1em}\hspace*{1em}\hspace*{1em}{</\textbf{valItem}>}\mbox{}\newline 
\hspace*{1em}\hspace*{1em}\hspace*{1em}\hspace*{1em}\hspace*{1em}{<\textbf{valItem}\hspace*{1em}{ident}="{blocking}">}\mbox{}\newline 
\hspace*{1em}\hspace*{1em}\hspace*{1em}\hspace*{1em}\hspace*{1em}\hspace*{1em}{<\textbf{desc}>}describes movement across stage, position, etc.{</\textbf{desc}>}\mbox{}\newline 
\hspace*{1em}\hspace*{1em}\hspace*{1em}\hspace*{1em}\hspace*{1em}{</\textbf{valItem}>}\mbox{}\newline 
\hspace*{1em}\hspace*{1em}\hspace*{1em}\hspace*{1em}\hspace*{1em}{<\textbf{valItem}\hspace*{1em}{ident}="{business}">}\mbox{}\newline 
\hspace*{1em}\hspace*{1em}\hspace*{1em}\hspace*{1em}\hspace*{1em}\hspace*{1em}{<\textbf{desc}>}describes movement other than blocking{</\textbf{desc}>}\mbox{}\newline 
\hspace*{1em}\hspace*{1em}\hspace*{1em}\hspace*{1em}\hspace*{1em}{</\textbf{valItem}>}\mbox{}\newline 
\hspace*{1em}\hspace*{1em}\hspace*{1em}\hspace*{1em}\hspace*{1em}{<\textbf{valItem}\hspace*{1em}{ident}="{delivery}">}\mbox{}\newline 
\hspace*{1em}\hspace*{1em}\hspace*{1em}\hspace*{1em}\hspace*{1em}\hspace*{1em}{<\textbf{desc}>}describes how the line is said{</\textbf{desc}>}\mbox{}\newline 
\hspace*{1em}\hspace*{1em}\hspace*{1em}\hspace*{1em}\hspace*{1em}{</\textbf{valItem}>}\mbox{}\newline 
\hspace*{1em}\hspace*{1em}\hspace*{1em}\hspace*{1em}\hspace*{1em}{<\textbf{valItem}\hspace*{1em}{ident}="{motivation}">}\mbox{}\newline 
\hspace*{1em}\hspace*{1em}\hspace*{1em}\hspace*{1em}\hspace*{1em}\hspace*{1em}{<\textbf{desc}>}describes character's emotional state or through line{</\textbf{desc}>}\mbox{}\newline 
\hspace*{1em}\hspace*{1em}\hspace*{1em}\hspace*{1em}\hspace*{1em}{</\textbf{valItem}>}\mbox{}\newline 
\hspace*{1em}\hspace*{1em}\hspace*{1em}\hspace*{1em}{</\textbf{valList}>}\mbox{}\newline 
\hspace*{1em}\hspace*{1em}\hspace*{1em}{</\textbf{attDef}>}\mbox{}\newline 
\hspace*{1em}\hspace*{1em}{</\textbf{attList}>}\mbox{}\newline 
\hspace*{1em}{</\textbf{elementSpec}>}\mbox{}\newline 
{</\textbf{schemaSpec}>}\end{shaded}\egroup\par \par
The \hyperref[TEI.stage]{<stage>} element may appear both between and within \hyperref[TEI.sp]{<sp>} elements. It may contain a mixture of phrase level elements, possibly combined into paragraphs, as in the following example: \par\bgroup\index{div1=<div1>|exampleindex}\index{n=@n!<div1>|exampleindex}\index{type=@type!<div1>|exampleindex}\index{stage=<stage>|exampleindex}\index{type=@type!<stage>|exampleindex}\index{p=<p>|exampleindex}\index{p=<p>|exampleindex}\index{sp=<sp>|exampleindex}\index{speaker=<speaker>|exampleindex}\index{p=<p>|exampleindex}\index{stage=<stage>|exampleindex}\index{type=@type!<stage>|exampleindex}\exampleFont \begin{shaded}\noindent\mbox{}{<\textbf{div1}\hspace*{1em}{n}="{1}"\hspace*{1em}{type}="{act}">}\mbox{}\newline 
\hspace*{1em}{<\textbf{stage}\hspace*{1em}{type}="{setting}">}\mbox{}\newline 
\hspace*{1em}\hspace*{1em}{<\textbf{p}>}Scene. — A room furnished comfortably and\mbox{}\newline 
\hspace*{1em}\hspace*{1em}\hspace*{1em}\hspace*{1em} tastefully but not extravagantly ...\mbox{}\newline 
\hspace*{1em}\hspace*{1em}\hspace*{1em}\hspace*{1em} The floor is carpeted and a fire burns in the stove.\mbox{}\newline 
\hspace*{1em}\hspace*{1em}\hspace*{1em}\hspace*{1em} It is winter.{</\textbf{p}>}\mbox{}\newline 
\hspace*{1em}\hspace*{1em}{<\textbf{p}>}A bell rings in the hall; shortly afterwards the\mbox{}\newline 
\hspace*{1em}\hspace*{1em}\hspace*{1em}\hspace*{1em} door is heard to open. Enter NORA humming a tune ...{</\textbf{p}>}\mbox{}\newline 
\hspace*{1em}{</\textbf{stage}>}\mbox{}\newline 
\hspace*{1em}{<\textbf{sp}>}\mbox{}\newline 
\hspace*{1em}\hspace*{1em}{<\textbf{speaker}>}Nora{</\textbf{speaker}>}\mbox{}\newline 
\hspace*{1em}\hspace*{1em}{<\textbf{p}>}Hide the Christmas Tree carefully, Helen. Be sure the\mbox{}\newline 
\hspace*{1em}\hspace*{1em}\hspace*{1em}\hspace*{1em} children do not see it till this evening, when it is\mbox{}\newline 
\hspace*{1em}\hspace*{1em}\hspace*{1em}\hspace*{1em} dressed. {<\textbf{stage}\hspace*{1em}{type}="{delivery}">}To the PORTER taking\mbox{}\newline 
\hspace*{1em}\hspace*{1em}\hspace*{1em}\hspace*{1em}\hspace*{1em}\hspace*{1em} out her purse{</\textbf{stage}>} How much?{</\textbf{p}>}\mbox{}\newline 
\hspace*{1em}{</\textbf{sp}>}\mbox{}\newline 
{</\textbf{div1}>}\end{shaded}\egroup\par \noindent  \par
The \hyperref[TEI.stage]{<stage>} element may also be used in non-theatrical texts, to mark sound effects or musical effects, etc., as further discussed in section \textit{\hyperref[DROTH]{7.3.\ Other Types of Performance Text}}.\par
The \hyperref[TEI.move]{<move>} element is intended to help overcome the fact that the stage directions of a printed text may often not provide full information about either the intended or the actual movement of actors on stage. It may be used to keep track of entrances and exits in detail, so as to know which characters are on stage at which time. Its attributes permit a relatively formal specification for movements of characters, using user-defined codes to identify the characters involved (the {\itshape who} attribute), the direction of the movement ({\itshape type} attribute), and optionally which part of the stage is involved ({\itshape where} attribute). For stage-historical purposes, a {\itshape perf} attribute is also provided; this allows the recording of different \hyperref[TEI.move]{<move>} elements as taken in different performances of the same text.\par
The \hyperref[TEI.move]{<move>} element should be located at the position in the text where the move is presumed to take place. This will often coincide with a stage direction, as in the following simple example: \par\bgroup\index{castList=<castList>|exampleindex}\index{castItem=<castItem>|exampleindex}\index{role=<role>|exampleindex}\index{stage=<stage>|exampleindex}\index{type=@type!<stage>|exampleindex}\index{move=<move>|exampleindex}\index{who=@who!<move>|exampleindex}\index{type=@type!<move>|exampleindex}\exampleFont \begin{shaded}\noindent\mbox{}{<\textbf{castList}>}\mbox{}\newline 
\hspace*{1em}{<\textbf{castItem}>}\mbox{}\newline 
\hspace*{1em}\hspace*{1em}{<\textbf{role}\hspace*{1em}{xml:id}="{bella}">}Bellafront{</\textbf{role}>}\mbox{}\newline 
\hspace*{1em}{</\textbf{castItem}>}\mbox{}\newline 
{</\textbf{castList}>}\mbox{}\newline 
{<\textbf{stage}\hspace*{1em}{type}="{entrance}">}\mbox{}\newline 
\hspace*{1em}{<\textbf{move}\hspace*{1em}{who}="{\#bella}"\hspace*{1em}{type}="{enter}"/>}\mbox{}\newline 
 Enter Bellafront mad.\mbox{}\newline 
{</\textbf{stage}>}\end{shaded}\egroup\par \par
The \hyperref[TEI.move]{<move>} element can however appear independently of a stage direction, as in the following example: \par\bgroup\index{castList=<castList>|exampleindex}\index{castItem=<castItem>|exampleindex}\index{role=<role>|exampleindex}\index{castItem=<castItem>|exampleindex}\index{role=<role>|exampleindex}\index{sp=<sp>|exampleindex}\index{who=@who!<sp>|exampleindex}\index{speaker=<speaker>|exampleindex}\index{p=<p>|exampleindex}\index{move=<move>|exampleindex}\index{who=@who!<move>|exampleindex}\index{type=@type!<move>|exampleindex}\index{where=@where!<move>|exampleindex}\exampleFont \begin{shaded}\noindent\mbox{}{<\textbf{castList}>}\mbox{}\newline 
\hspace*{1em}{<\textbf{castItem}>}\mbox{}\newline 
\hspace*{1em}\hspace*{1em}{<\textbf{role}\hspace*{1em}{xml:id}="{lm}">}Lady Macbeth{</\textbf{role}>}\mbox{}\newline 
\hspace*{1em}{</\textbf{castItem}>}\mbox{}\newline 
\hspace*{1em}{<\textbf{castItem}>}\mbox{}\newline 
\hspace*{1em}\hspace*{1em}{<\textbf{role}\hspace*{1em}{xml:id}="{g1}">}First Gentleman{</\textbf{role}>}\mbox{}\newline 
\hspace*{1em}{</\textbf{castItem}>}\mbox{}\newline 
\textit{<!-- ... -->}\mbox{}\newline 
{</\textbf{castList}>}\mbox{}\newline 
{<\textbf{sp}\hspace*{1em}{who}="{\#g1}">}\mbox{}\newline 
\hspace*{1em}{<\textbf{speaker}>}Gent.{</\textbf{speaker}>}\mbox{}\newline 
\hspace*{1em}{<\textbf{p}>}Neither to you, nor any one; having no witness\mbox{}\newline 
\hspace*{1em}\hspace*{1em} to confirm my speech. {<\textbf{move}\hspace*{1em}{who}="{\#lm}"\hspace*{1em}{type}="{enter}"\hspace*{1em}{where}="{C}"/>}\mbox{}\newline 
\hspace*{1em}\hspace*{1em} Lo you! here she comes. This is her very guise; and,\mbox{}\newline 
\hspace*{1em}\hspace*{1em} upon my life, fast asleep.{</\textbf{p}>}\mbox{}\newline 
{</\textbf{sp}>}\end{shaded}\egroup\par 
\subsubsection[{Speech Contents}]{Speech Contents}\label{DRPAL}\par
The actual speeches of a dramatic text may be composed of running text, which must be formally organized into paragraphs, in the case of prose (see section \textit{\hyperref[COPA]{3.1.\ Paragraphs}}), verse lines or line groups in that of verse (see section \textit{\hyperref[CODV]{3.13.\ Passages of Verse or Drama}}), or \hyperref[TEI.seg]{<seg>} elements, in case of doubt as to whether the material should be treated as verse or prose. The following elements, all of which are defined in the core, are particularly useful when marking units of prose or verse within speeches: 
\begin{sansreflist}
  
\item [\textbf{<p>}] (paragraph) marks paragraphs in prose.
\item [\textbf{<lb>}] (line beginning) marks the beginning of a new (typographic) line in some edition or version of a text.
\item [\textbf{<l>}] (verse line) contains a single, possibly incomplete, line of verse.
\item [\textbf{<lg>}] (line group) contains one or more verse lines functioning as a formal unit, e.g. a stanza, refrain, verse paragraph, etc.
\end{sansreflist}
\par
Like other milestone elements, the element \hyperref[TEI.lb]{<lb>} additionally bears the attributes {\itshape ed} and {\itshape edRef}, from its membership in the class \textsf{att.edition}: 
\begin{sansreflist}
  
\item [\textbf{att.edition}] provides attributes identifying the source edition from which some encoded feature derives.\hfil\\[-10pt]\begin{sansreflist}
    \item[@{\itshape ed}]
  (edition) supplies a sigil or other arbitrary identifier for the source edition in which the associated feature (for example, a page, column, or line break) occurs at this point in the text.
    \item[@{\itshape edRef}]
  (edition reference) provides a pointer to the source edition in which the associated feature (for example, a page, column, or line break) occurs at this point in the text.
\end{sansreflist}  
\end{sansreflist}
\par
As a member of the classes \textsf{att.typed} and \textsf{att.divLike}, the \hyperref[TEI.lg]{<lg>} element also bears the following attributes: 
\begin{sansreflist}
  
\item [\textbf{att.typed}] provides attributes which can be used to classify or subclassify elements in any way.\hfil\\[-10pt]\begin{sansreflist}
    \item[@{\itshape type}]
  characterizes the element in some sense, using any convenient classification scheme or typology.
    \item[@{\itshape subtype}]
  (subtype) provides a sub-categorization of the element, if needed
\end{sansreflist}  
\item [\textbf{att.divLike}] provides attributes common to all elements which behave in the same way as divisions.\hfil\\[-10pt]\begin{sansreflist}
    \item[@{\itshape org}]
  (organization) specifies how the content of the division is organized.
    \item[@{\itshape sample}]
  indicates whether this division is a sample of the original source and if so, from which part.
\end{sansreflist}  
\item [\textbf{att.fragmentable}] provides an attribute for representing fragmentation of a structural element, typically as a consequence of some overlapping hierarchy.\hfil\\[-10pt]\begin{sansreflist}
    \item[@{\itshape part}]
  specifies whether or not its parent element is fragmented in some way, typically by some other overlapping structure: for example a speech which is divided between two or more verse stanzas, a paragraph which is split across a page division, a verse line which is divided between two speakers.
\end{sansreflist}  
\end{sansreflist}
\par
When the \textsf{verse} module is included in a schema, the elements \hyperref[TEI.l]{<l>} and \hyperref[TEI.lg]{<lg>} also gain additional attributes through their membership of the class \textsf{att.metrical}: 
\begin{sansreflist}
  
\item [\textbf{att.metrical}] defines a set of attributes which certain elements may use to represent metrical information.\hfil\\[-10pt]\begin{sansreflist}
    \item[@{\itshape met}]
  (metrical structure, conventional) contains a user-specified encoding for the conventional metrical structure of the element.
    \item[@{\itshape rhyme}]
  (rhyme scheme) specifies the rhyme scheme applicable to a group of verse lines.
\end{sansreflist}  
\end{sansreflist}
\par
In many texts, prose and verse may be inextricably mingled; particularly in earlier printed texts, prose may be printed as verse or verse as prose, or it may be impossible to distinguish the two. In cases of doubt, an encoder may prefer to tag the dubious material consistently as verse, to tag it all as prose, to follow the typography of the source text, or to use the neutral \hyperref[TEI.ab]{<ab>} element to contain the speech itself. When this question arises, the \hyperref[TEI.tagUsage]{<tagUsage>} element in the \hyperref[TEI.encodingDesc]{<encodingDesc>} element of the header may be used to record explicitly what policy has been adopted.\par
Even where they can reliably be distinguished, a single speech may frequently contain a mixture of prose (marked as \hyperref[TEI.p]{<p>}) and verse (marked as \hyperref[TEI.l]{<l>} or—if stanzaic—\hyperref[TEI.lg]{<lg>}).\par
The {\itshape part} attribute which both \hyperref[TEI.l]{<l>} and \hyperref[TEI.lg]{<lg>} elements inherit from the \textsf{att.fragmentable} class provides one simple way of indicating where the boundaries of a speech and of a verse line or line group do not coincide. The encoder may simply indicate that a line or line group is metrically incomplete by specifying the value Y or N, as in the following example: \par\bgroup\index{sp=<sp>|exampleindex}\index{speaker=<speaker>|exampleindex}\index{l=<l>|exampleindex}\index{part=@part!<l>|exampleindex}\index{l=<l>|exampleindex}\index{part=@part!<l>|exampleindex}\index{sp=<sp>|exampleindex}\index{speaker=<speaker>|exampleindex}\index{l=<l>|exampleindex}\index{part=@part!<l>|exampleindex}\index{sp=<sp>|exampleindex}\index{speaker=<speaker>|exampleindex}\index{l=<l>|exampleindex}\index{part=@part!<l>|exampleindex}\index{l=<l>|exampleindex}\index{part=@part!<l>|exampleindex}\index{sp=<sp>|exampleindex}\index{speaker=<speaker>|exampleindex}\index{l=<l>|exampleindex}\index{part=@part!<l>|exampleindex}\exampleFont \begin{shaded}\noindent\mbox{}{<\textbf{sp}>}\mbox{}\newline 
\hspace*{1em}{<\textbf{speaker}>}Face{</\textbf{speaker}>}\mbox{}\newline 
\hspace*{1em}{<\textbf{l}\hspace*{1em}{part}="{N}">}You most\mbox{}\newline 
\hspace*{1em}\hspace*{1em} notorious whelp, you insolent slave{</\textbf{l}>}\mbox{}\newline 
\hspace*{1em}{<\textbf{l}\hspace*{1em}{part}="{Y}">}Dare you do this?{</\textbf{l}>}\mbox{}\newline 
{</\textbf{sp}>}\mbox{}\newline 
{<\textbf{sp}>}\mbox{}\newline 
\hspace*{1em}{<\textbf{speaker}>}Subtle{</\textbf{speaker}>}\mbox{}\newline 
\hspace*{1em}{<\textbf{l}\hspace*{1em}{part}="{Y}">}Yes faith, yes faith.{</\textbf{l}>}\mbox{}\newline 
{</\textbf{sp}>}\mbox{}\newline 
{<\textbf{sp}>}\mbox{}\newline 
\hspace*{1em}{<\textbf{speaker}>}Face{</\textbf{speaker}>}\mbox{}\newline 
\hspace*{1em}{<\textbf{l}\hspace*{1em}{part}="{Y}">}Why! Who{</\textbf{l}>}\mbox{}\newline 
\hspace*{1em}{<\textbf{l}\hspace*{1em}{part}="{Y}">}Am I, my mongrel? Who am I?{</\textbf{l}>}\mbox{}\newline 
{</\textbf{sp}>}\mbox{}\newline 
{<\textbf{sp}>}\mbox{}\newline 
\hspace*{1em}{<\textbf{speaker}>}Subtle{</\textbf{speaker}>}\mbox{}\newline 
\hspace*{1em}{<\textbf{l}\hspace*{1em}{part}="{Y}">}I'll tell you,{</\textbf{l}>}\mbox{}\newline 
\textit{<!-- ... -->}\mbox{}\newline 
{</\textbf{sp}>}\end{shaded}\egroup\par \par
Alternatively, where the fragments of the line or line group are consecutive in the text (though possibly interrupted by stage directions), the values I (initial), M (medial), and F (final) may be used to indicate how metrical lines are constituted: \par\bgroup\index{sp=<sp>|exampleindex}\index{speaker=<speaker>|exampleindex}\index{l=<l>|exampleindex}\index{l=<l>|exampleindex}\index{part=@part!<l>|exampleindex}\index{sp=<sp>|exampleindex}\index{speaker=<speaker>|exampleindex}\index{l=<l>|exampleindex}\index{part=@part!<l>|exampleindex}\index{sp=<sp>|exampleindex}\index{speaker=<speaker>|exampleindex}\index{l=<l>|exampleindex}\index{part=@part!<l>|exampleindex}\index{l=<l>|exampleindex}\index{part=@part!<l>|exampleindex}\index{sp=<sp>|exampleindex}\index{speaker=<speaker>|exampleindex}\index{l=<l>|exampleindex}\index{part=@part!<l>|exampleindex}\exampleFont \begin{shaded}\noindent\mbox{}{<\textbf{sp}>}\mbox{}\newline 
\hspace*{1em}{<\textbf{speaker}>}Face{</\textbf{speaker}>}\mbox{}\newline 
\hspace*{1em}{<\textbf{l}>}You most\mbox{}\newline 
\hspace*{1em}\hspace*{1em} notorious whelp, you insolent slave{</\textbf{l}>}\mbox{}\newline 
\hspace*{1em}{<\textbf{l}\hspace*{1em}{part}="{I}">}Dare you do this?{</\textbf{l}>}\mbox{}\newline 
{</\textbf{sp}>}\mbox{}\newline 
{<\textbf{sp}>}\mbox{}\newline 
\hspace*{1em}{<\textbf{speaker}>}Subtle{</\textbf{speaker}>}\mbox{}\newline 
\hspace*{1em}{<\textbf{l}\hspace*{1em}{part}="{M}">}Yes faith, yes faith.{</\textbf{l}>}\mbox{}\newline 
{</\textbf{sp}>}\mbox{}\newline 
{<\textbf{sp}>}\mbox{}\newline 
\hspace*{1em}{<\textbf{speaker}>}Face{</\textbf{speaker}>}\mbox{}\newline 
\hspace*{1em}{<\textbf{l}\hspace*{1em}{part}="{F}">}Why! Who{</\textbf{l}>}\mbox{}\newline 
\hspace*{1em}{<\textbf{l}\hspace*{1em}{part}="{I}">}Am I, my mongrel? Who am I?{</\textbf{l}>}\mbox{}\newline 
{</\textbf{sp}>}\mbox{}\newline 
{<\textbf{sp}>}\mbox{}\newline 
\hspace*{1em}{<\textbf{speaker}>}Subtle{</\textbf{speaker}>}\mbox{}\newline 
\hspace*{1em}{<\textbf{l}\hspace*{1em}{part}="{F}">}I'll tell you,{</\textbf{l}>}\mbox{}\newline 
\textit{<!-- ... -->}\mbox{}\newline 
{</\textbf{sp}>}\end{shaded}\egroup\par \par
In dramatic texts, the \hyperref[TEI.lg]{<lg>} or line group element is most often of use for the encoding of songs and other stanzaic material. Line groups may be fragmented across speakers in the same way as individual lines, and the same set of attributes may be used to record this fact. The element \hyperref[TEI.spGrp]{<spGrp>} is provided in order to simplify the situation, very common in performances, where performance of a single entity, such as a song, is shared amongst several performers, as in the following example: \par\bgroup\index{spGrp=<spGrp>|exampleindex}\index{head=<head>|exampleindex}\index{sp=<sp>|exampleindex}\index{l=<l>|exampleindex}\index{l=<l>|exampleindex}\index{l=<l>|exampleindex}\index{sp=<sp>|exampleindex}\index{speaker=<speaker>|exampleindex}\index{l=<l>|exampleindex}\index{sp=<sp>|exampleindex}\index{speaker=<speaker>|exampleindex}\index{l=<l>|exampleindex}\exampleFont \begin{shaded}\noindent\mbox{}{<\textbf{spGrp}>}\mbox{}\newline 
\hspace*{1em}{<\textbf{head}>}Song — Sir Joseph{</\textbf{head}>}\mbox{}\newline 
\hspace*{1em}{<\textbf{sp}>}\mbox{}\newline 
\hspace*{1em}\hspace*{1em}{<\textbf{l}>}I am the monarch of the sea,{</\textbf{l}>}\mbox{}\newline 
\hspace*{1em}\hspace*{1em}{<\textbf{l}>}The ruler of the Queen's Navee.{</\textbf{l}>}\mbox{}\newline 
\hspace*{1em}\hspace*{1em}{<\textbf{l}>}Whose praise Great Britain loudly chants.{</\textbf{l}>}\mbox{}\newline 
\hspace*{1em}{</\textbf{sp}>}\mbox{}\newline 
\hspace*{1em}{<\textbf{sp}>}\mbox{}\newline 
\hspace*{1em}\hspace*{1em}{<\textbf{speaker}>}Cousin Hebe{</\textbf{speaker}>}\mbox{}\newline 
\hspace*{1em}\hspace*{1em}{<\textbf{l}>}And we are his sisters and his cousins and his aunts!{</\textbf{l}>}\mbox{}\newline 
\hspace*{1em}{</\textbf{sp}>}\mbox{}\newline 
\hspace*{1em}{<\textbf{sp}>}\mbox{}\newline 
\hspace*{1em}\hspace*{1em}{<\textbf{speaker}>}Rel.{</\textbf{speaker}>}\mbox{}\newline 
\hspace*{1em}\hspace*{1em}{<\textbf{l}>}And we are his sisters and his cousins and his aunts!{</\textbf{l}>}\mbox{}\newline 
\hspace*{1em}{</\textbf{sp}>}\mbox{}\newline 
\textit{<!-- ... -->}\mbox{}\newline 
{</\textbf{spGrp}>}\end{shaded}\egroup\par \par
This encoding however does not indicate that the three lines of Sir Joseph's song and the two lines following it together constitute a single verse stanza. This can be indicated by using the {\itshape part} attribute, as follows: \par\bgroup\index{spGrp=<spGrp>|exampleindex}\index{head=<head>|exampleindex}\index{sp=<sp>|exampleindex}\index{lg=<lg>|exampleindex}\index{part=@part!<lg>|exampleindex}\index{l=<l>|exampleindex}\index{l=<l>|exampleindex}\index{l=<l>|exampleindex}\index{sp=<sp>|exampleindex}\index{speaker=<speaker>|exampleindex}\index{lg=<lg>|exampleindex}\index{part=@part!<lg>|exampleindex}\index{l=<l>|exampleindex}\index{sp=<sp>|exampleindex}\index{speaker=<speaker>|exampleindex}\index{lg=<lg>|exampleindex}\index{part=@part!<lg>|exampleindex}\index{l=<l>|exampleindex}\exampleFont \begin{shaded}\noindent\mbox{}{<\textbf{spGrp}>}\mbox{}\newline 
\hspace*{1em}{<\textbf{head}>}Song — Sir Joseph{</\textbf{head}>}\mbox{}\newline 
\hspace*{1em}{<\textbf{sp}>}\mbox{}\newline 
\hspace*{1em}\hspace*{1em}{<\textbf{lg}\hspace*{1em}{part}="{I}">}\mbox{}\newline 
\hspace*{1em}\hspace*{1em}\hspace*{1em}{<\textbf{l}>}I am the monarch of the sea,{</\textbf{l}>}\mbox{}\newline 
\hspace*{1em}\hspace*{1em}\hspace*{1em}{<\textbf{l}>}The ruler of the Queen's Navee.{</\textbf{l}>}\mbox{}\newline 
\hspace*{1em}\hspace*{1em}\hspace*{1em}{<\textbf{l}>}Whose praise Great Britain loudly chants.{</\textbf{l}>}\mbox{}\newline 
\hspace*{1em}\hspace*{1em}{</\textbf{lg}>}\mbox{}\newline 
\hspace*{1em}{</\textbf{sp}>}\mbox{}\newline 
\hspace*{1em}{<\textbf{sp}>}\mbox{}\newline 
\hspace*{1em}\hspace*{1em}{<\textbf{speaker}>}Cousin Hebe{</\textbf{speaker}>}\mbox{}\newline 
\hspace*{1em}\hspace*{1em}{<\textbf{lg}\hspace*{1em}{part}="{M}">}\mbox{}\newline 
\hspace*{1em}\hspace*{1em}\hspace*{1em}{<\textbf{l}>}And we are his sisters and his cousins and his aunts!{</\textbf{l}>}\mbox{}\newline 
\hspace*{1em}\hspace*{1em}{</\textbf{lg}>}\mbox{}\newline 
\hspace*{1em}{</\textbf{sp}>}\mbox{}\newline 
\hspace*{1em}{<\textbf{sp}>}\mbox{}\newline 
\hspace*{1em}\hspace*{1em}{<\textbf{speaker}>}Rel.{</\textbf{speaker}>}\mbox{}\newline 
\hspace*{1em}\hspace*{1em}{<\textbf{lg}\hspace*{1em}{part}="{F}">}\mbox{}\newline 
\hspace*{1em}\hspace*{1em}\hspace*{1em}{<\textbf{l}>}And we are his sisters and his cousins and his aunts!{</\textbf{l}>}\mbox{}\newline 
\hspace*{1em}\hspace*{1em}{</\textbf{lg}>}\mbox{}\newline 
\hspace*{1em}{</\textbf{sp}>}\mbox{}\newline 
\textit{<!-- ... -->}\mbox{}\newline 
{</\textbf{spGrp}>}\end{shaded}\egroup\par 
\subsubsection[{Embedded Structures}]{Embedded Structures}\label{DREMB}\par
Although primarily composed of speeches, performance texts often contain other structural units such as songs or strophes which are shared among different speakers. More generally, complex nested structures of plays within plays, interpolated masques, or interludes are far from uncommon. In more modern material, comparably complex structural devices such as flashback or nested playback are equally frequent. In all kinds of performance material, it may be necessary to indicate several actions which are happening simultaneously.\par
A number of different devices are available within the TEI scheme to support these complexities in the general case. Texts may be composite or self-nesting (see section \textit{\hyperref[DSGRP]{4.3.1.\ Grouped Texts}}) and multiple hierarchies may be defined (see chapter \textit{\hyperref[NH]{20.\ Non-hierarchical Structures}}). The TEI encoding scheme provides a variety of linking mechanisms, which may be used to indicate temporal alignment and aggregation of fragmented structures. In this section we provide a few specific examples of the application of these techniques to performance texts: \begin{itemize}
\item the use of the \hyperref[TEI.floatingText]{<floatingText>} element
\item the use of the {\itshape part} attribute on fragmentary \hyperref[TEI.lg]{<lg>} elements
\item the use of the {\itshape next} and {\itshape prev} attributes on fragments of embedded structures to join them into a larger whole
\item the use of the \hyperref[TEI.join]{<join>} element to define a ‘virtual element’ composed of the fragments indicated
\end{itemize} \par
When the whole of a song appears within a single speech, it may require no special treatment if it is considered to form a part of the speech: \par\bgroup\index{sp=<sp>|exampleindex}\index{speaker=<speaker>|exampleindex}\index{stage=<stage>|exampleindex}\index{p=<p>|exampleindex}\index{stage=<stage>|exampleindex}\index{lg=<lg>|exampleindex}\index{type=@type!<lg>|exampleindex}\index{l=<l>|exampleindex}\index{l=<l>|exampleindex}\index{l=<l>|exampleindex}\index{l=<l>|exampleindex}\index{sp=<sp>|exampleindex}\index{speaker=<speaker>|exampleindex}\index{stage=<stage>|exampleindex}\index{p=<p>|exampleindex}\exampleFont \begin{shaded}\noindent\mbox{}{<\textbf{sp}>}\mbox{}\newline 
\hspace*{1em}{<\textbf{speaker}>}Kelly{</\textbf{speaker}>}\mbox{}\newline 
\hspace*{1em}{<\textbf{stage}>}(calmly).{</\textbf{stage}>}\mbox{}\newline 
\hspace*{1em}{<\textbf{p}>}Aha, so you've bad minds along with th' love of gain.\mbox{}\newline 
\hspace*{1em}\hspace*{1em} You thry to pin on others th' dirty decorations that\mbox{}\newline 
\hspace*{1em}\hspace*{1em} may be hangin' on your own coats.{</\textbf{p}>}\mbox{}\newline 
\hspace*{1em}{<\textbf{stage}>}(He points, one after the other at Conroy, Bull,\mbox{}\newline 
\hspace*{1em}\hspace*{1em} and Flagonson. Lilting){</\textbf{stage}>}\mbox{}\newline 
\hspace*{1em}{<\textbf{lg}\hspace*{1em}{type}="{song}">}\mbox{}\newline 
\hspace*{1em}\hspace*{1em}{<\textbf{l}>}Who were you with last night?{</\textbf{l}>}\mbox{}\newline 
\hspace*{1em}\hspace*{1em}{<\textbf{l}>}Who were you with last night?{</\textbf{l}>}\mbox{}\newline 
\hspace*{1em}\hspace*{1em}{<\textbf{l}>}Will you tell your missus when you go home{</\textbf{l}>}\mbox{}\newline 
\hspace*{1em}\hspace*{1em}{<\textbf{l}>}Who you were with last night?{</\textbf{l}>}\mbox{}\newline 
\hspace*{1em}{</\textbf{lg}>}\mbox{}\newline 
{</\textbf{sp}>}\mbox{}\newline 
{<\textbf{sp}>}\mbox{}\newline 
\hspace*{1em}{<\textbf{speaker}>}Flagonson{</\textbf{speaker}>}\mbox{}\newline 
\hspace*{1em}{<\textbf{stage}>}(in anguished indignation).{</\textbf{stage}>}\mbox{}\newline 
\hspace*{1em}{<\textbf{p}>}This is more than a hurt to us: this hits at the\mbox{}\newline 
\hspace*{1em}\hspace*{1em} decency of the whole nation!{</\textbf{p}>}\mbox{}\newline 
{</\textbf{sp}>}\end{shaded}\egroup\par \noindent  If however, the song is to be regarded as forming a distinct item, perhaps with its own front and back matter, it may be better to regard it as a floating text: \par\bgroup\index{sp=<sp>|exampleindex}\index{speaker=<speaker>|exampleindex}\index{stage=<stage>|exampleindex}\index{p=<p>|exampleindex}\index{stage=<stage>|exampleindex}\index{floatingText=<floatingText>|exampleindex}\index{front=<front>|exampleindex}\index{titlePart=<titlePart>|exampleindex}\index{body=<body>|exampleindex}\index{l=<l>|exampleindex}\index{l=<l>|exampleindex}\index{l=<l>|exampleindex}\index{l=<l>|exampleindex}\exampleFont \begin{shaded}\noindent\mbox{}{<\textbf{sp}>}\mbox{}\newline 
\hspace*{1em}{<\textbf{speaker}>}Kelly{</\textbf{speaker}>}\mbox{}\newline 
\hspace*{1em}{<\textbf{stage}>}(calmly).{</\textbf{stage}>}\mbox{}\newline 
\hspace*{1em}{<\textbf{p}>}Aha, so you've bad minds along with ...{</\textbf{p}>}\mbox{}\newline 
{</\textbf{sp}>}\mbox{}\newline 
{<\textbf{stage}>}(He points, one after the other at Conroy, Bull,\mbox{}\newline 
 and Flagonson. Lilting):{</\textbf{stage}>}\mbox{}\newline 
{<\textbf{floatingText}>}\mbox{}\newline 
\hspace*{1em}{<\textbf{front}>}\mbox{}\newline 
\hspace*{1em}\hspace*{1em}{<\textbf{titlePart}>}Kelly's Song{</\textbf{titlePart}>}\mbox{}\newline 
\hspace*{1em}{</\textbf{front}>}\mbox{}\newline 
\hspace*{1em}{<\textbf{body}>}\mbox{}\newline 
\hspace*{1em}\hspace*{1em}{<\textbf{l}>}Who were you with last night?{</\textbf{l}>}\mbox{}\newline 
\hspace*{1em}\hspace*{1em}{<\textbf{l}>}Who were you with last night?{</\textbf{l}>}\mbox{}\newline 
\hspace*{1em}\hspace*{1em}{<\textbf{l}>}Will you tell your missus when you go home{</\textbf{l}>}\mbox{}\newline 
\hspace*{1em}\hspace*{1em}{<\textbf{l}>}Who you were with last night?{</\textbf{l}>}\mbox{}\newline 
\hspace*{1em}{</\textbf{body}>}\mbox{}\newline 
{</\textbf{floatingText}>}\end{shaded}\egroup\par \par
When an embedded structure extends across more than one \hyperref[TEI.sp]{<sp>} element, each of its constituent parts must be regarded as a distinct fragment; the problem then facing the encoder is to reconstitute the interrupted whole in some way.\par
As already noted above, the \hyperref[TEI.spGrp]{<spGrp>} element may be used to group together consecutive speeches which are grouped together in some way, for example constituting a single song. Alternatively the {\itshape part} attribute, typically used to indicate that an \hyperref[TEI.l]{<l>} element contains a partial, not a complete, verse line, may also be used on the \hyperref[TEI.lg]{<lg>} element, to indicate that the line group is partial rather than complete, thus: \par\bgroup\index{sp=<sp>|exampleindex}\index{speaker=<speaker>|exampleindex}\index{stage=<stage>|exampleindex}\index{lg=<lg>|exampleindex}\index{type=@type!<lg>|exampleindex}\index{part=@part!<lg>|exampleindex}\index{l=<l>|exampleindex}\index{l=<l>|exampleindex}\index{sp=<sp>|exampleindex}\index{speaker=<speaker>|exampleindex}\index{stage=<stage>|exampleindex}\index{lg=<lg>|exampleindex}\index{type=@type!<lg>|exampleindex}\index{part=@part!<lg>|exampleindex}\index{l=<l>|exampleindex}\index{l=<l>|exampleindex}\exampleFont \begin{shaded}\noindent\mbox{}{<\textbf{sp}>}\mbox{}\newline 
\hspace*{1em}{<\textbf{speaker}>}Kelly{</\textbf{speaker}>}\mbox{}\newline 
\hspace*{1em}{<\textbf{stage}>}(wheeling quietly in his semi-dance,\mbox{}\newline 
\hspace*{1em}\hspace*{1em} as he goes out):{</\textbf{stage}>}\mbox{}\newline 
\hspace*{1em}{<\textbf{lg}\hspace*{1em}{type}="{stanza}"\hspace*{1em}{part}="{I}">}\mbox{}\newline 
\hspace*{1em}\hspace*{1em}{<\textbf{l}>}Goodbye to holy souls left here,{</\textbf{l}>}\mbox{}\newline 
\hspace*{1em}\hspace*{1em}{<\textbf{l}>}Goodbye to man an' fairy;{</\textbf{l}>}\mbox{}\newline 
\hspace*{1em}{</\textbf{lg}>}\mbox{}\newline 
{</\textbf{sp}>}\mbox{}\newline 
{<\textbf{sp}>}\mbox{}\newline 
\hspace*{1em}{<\textbf{speaker}>}Widda Machree{</\textbf{speaker}>}\mbox{}\newline 
\hspace*{1em}{<\textbf{stage}>}(wheeling quietly in her semi-dance,\mbox{}\newline 
\hspace*{1em}\hspace*{1em} as she goes out):{</\textbf{stage}>}\mbox{}\newline 
\hspace*{1em}{<\textbf{lg}\hspace*{1em}{type}="{stanza}"\hspace*{1em}{part}="{F}">}\mbox{}\newline 
\hspace*{1em}\hspace*{1em}{<\textbf{l}>}Goodbye to all of Leicester Square,{</\textbf{l}>}\mbox{}\newline 
\hspace*{1em}\hspace*{1em}{<\textbf{l}>}An' the long way to Tipperary.{</\textbf{l}>}\mbox{}\newline 
\hspace*{1em}{</\textbf{lg}>}\mbox{}\newline 
{</\textbf{sp}>}\end{shaded}\egroup\par \par
When the fragments of a song are separated by other intervening dialogue, or even when not, they may be linked together with the {\itshape next} and {\itshape prev} attributes defined in section \textit{\hyperref[SAAG]{16.7.\ Aggregation}}. For example, the line groups making up Ophelia's song might be encoded as follows: \par\bgroup\index{div1=<div1>|exampleindex}\index{n=@n!<div1>|exampleindex}\index{type=@type!<div1>|exampleindex}\index{div2=<div2>|exampleindex}\index{n=@n!<div2>|exampleindex}\index{type=@type!<div2>|exampleindex}\index{stage=<stage>|exampleindex}\index{stage=<stage>|exampleindex}\index{type=@type!<stage>|exampleindex}\index{sp=<sp>|exampleindex}\index{speaker=<speaker>|exampleindex}\index{p=<p>|exampleindex}\index{sp=<sp>|exampleindex}\index{speaker=<speaker>|exampleindex}\index{p=<p>|exampleindex}\index{sp=<sp>|exampleindex}\index{speaker=<speaker>|exampleindex}\index{stage=<stage>|exampleindex}\index{lg=<lg>|exampleindex}\index{next=@next!<lg>|exampleindex}\index{type=@type!<lg>|exampleindex}\index{part=@part!<lg>|exampleindex}\index{l=<l>|exampleindex}\index{l=<l>|exampleindex}\index{l=<l>|exampleindex}\index{l=<l>|exampleindex}\index{sp=<sp>|exampleindex}\index{speaker=<speaker>|exampleindex}\index{p=<p>|exampleindex}\index{sp=<sp>|exampleindex}\index{speaker=<speaker>|exampleindex}\index{p=<p>|exampleindex}\index{stage=<stage>|exampleindex}\index{lg=<lg>|exampleindex}\index{prev=@prev!<lg>|exampleindex}\index{type=@type!<lg>|exampleindex}\index{part=@part!<lg>|exampleindex}\index{l=<l>|exampleindex}\index{l=<l>|exampleindex}\index{l=<l>|exampleindex}\index{l=<l>|exampleindex}\index{p=<p>|exampleindex}\exampleFont \begin{shaded}\noindent\mbox{}{<\textbf{div1}\hspace*{1em}{n}="{4}"\hspace*{1em}{type}="{act}">}\mbox{}\newline 
\hspace*{1em}{<\textbf{div2}\hspace*{1em}{n}="{5}"\hspace*{1em}{type}="{scene}">}\mbox{}\newline 
\hspace*{1em}\hspace*{1em}{<\textbf{stage}>}Elsinore. A room in the Castle.{</\textbf{stage}>}\mbox{}\newline 
\hspace*{1em}\hspace*{1em}{<\textbf{stage}\hspace*{1em}{type}="{setting}">}Enter Ophelia, distracted.{</\textbf{stage}>}\mbox{}\newline 
\hspace*{1em}\hspace*{1em}{<\textbf{sp}>}\mbox{}\newline 
\hspace*{1em}\hspace*{1em}\hspace*{1em}{<\textbf{speaker}>}Ophelia{</\textbf{speaker}>}\mbox{}\newline 
\hspace*{1em}\hspace*{1em}\hspace*{1em}{<\textbf{p}>}Where is the beauteous Majesty of Denmark?{</\textbf{p}>}\mbox{}\newline 
\hspace*{1em}\hspace*{1em}{</\textbf{sp}>}\mbox{}\newline 
\hspace*{1em}\hspace*{1em}{<\textbf{sp}>}\mbox{}\newline 
\hspace*{1em}\hspace*{1em}\hspace*{1em}{<\textbf{speaker}>}Queen{</\textbf{speaker}>}\mbox{}\newline 
\hspace*{1em}\hspace*{1em}\hspace*{1em}{<\textbf{p}>}How now, Ophelia?{</\textbf{p}>}\mbox{}\newline 
\hspace*{1em}\hspace*{1em}{</\textbf{sp}>}\mbox{}\newline 
\hspace*{1em}\hspace*{1em}{<\textbf{sp}>}\mbox{}\newline 
\hspace*{1em}\hspace*{1em}\hspace*{1em}{<\textbf{speaker}>}Ophelia{</\textbf{speaker}>}\mbox{}\newline 
\hspace*{1em}\hspace*{1em}\hspace*{1em}{<\textbf{stage}>}Singing{</\textbf{stage}>}\mbox{}\newline 
\hspace*{1em}\hspace*{1em}\hspace*{1em}{<\textbf{lg}\hspace*{1em}{next}="{\#Tl2}"\hspace*{1em}{xml:id}="{Tl1}"\hspace*{1em}{type}="{song}"\mbox{}\newline 
\hspace*{1em}\hspace*{1em}\hspace*{1em}\hspace*{1em}{part}="{Y}">}\mbox{}\newline 
\hspace*{1em}\hspace*{1em}\hspace*{1em}\hspace*{1em}{<\textbf{l}>}How should I your true-love know{</\textbf{l}>}\mbox{}\newline 
\hspace*{1em}\hspace*{1em}\hspace*{1em}\hspace*{1em}{<\textbf{l}>}From another one?{</\textbf{l}>}\mbox{}\newline 
\hspace*{1em}\hspace*{1em}\hspace*{1em}\hspace*{1em}{<\textbf{l}>}By his cockle hat and staff{</\textbf{l}>}\mbox{}\newline 
\hspace*{1em}\hspace*{1em}\hspace*{1em}\hspace*{1em}{<\textbf{l}>}And his sandal shoon.{</\textbf{l}>}\mbox{}\newline 
\hspace*{1em}\hspace*{1em}\hspace*{1em}{</\textbf{lg}>}\mbox{}\newline 
\hspace*{1em}\hspace*{1em}{</\textbf{sp}>}\mbox{}\newline 
\hspace*{1em}\hspace*{1em}{<\textbf{sp}>}\mbox{}\newline 
\hspace*{1em}\hspace*{1em}\hspace*{1em}{<\textbf{speaker}>}Queen{</\textbf{speaker}>}\mbox{}\newline 
\hspace*{1em}\hspace*{1em}\hspace*{1em}{<\textbf{p}>}Alas, sweet lady, what imports this song?{</\textbf{p}>}\mbox{}\newline 
\hspace*{1em}\hspace*{1em}{</\textbf{sp}>}\mbox{}\newline 
\hspace*{1em}\hspace*{1em}{<\textbf{sp}>}\mbox{}\newline 
\hspace*{1em}\hspace*{1em}\hspace*{1em}{<\textbf{speaker}>}Ophelia{</\textbf{speaker}>}\mbox{}\newline 
\hspace*{1em}\hspace*{1em}\hspace*{1em}{<\textbf{p}>}Say you? Nay, pray you mark.{</\textbf{p}>}\mbox{}\newline 
\hspace*{1em}\hspace*{1em}\hspace*{1em}{<\textbf{stage}>}Sings{</\textbf{stage}>}\mbox{}\newline 
\hspace*{1em}\hspace*{1em}\hspace*{1em}{<\textbf{lg}\hspace*{1em}{prev}="{\#Tl1}"\hspace*{1em}{xml:id}="{Tl2}"\hspace*{1em}{type}="{song}"\mbox{}\newline 
\hspace*{1em}\hspace*{1em}\hspace*{1em}\hspace*{1em}{part}="{Y}">}\mbox{}\newline 
\hspace*{1em}\hspace*{1em}\hspace*{1em}\hspace*{1em}{<\textbf{l}>}He is dead and gone, lady,{</\textbf{l}>}\mbox{}\newline 
\hspace*{1em}\hspace*{1em}\hspace*{1em}\hspace*{1em}{<\textbf{l}>}He is dead and gone;{</\textbf{l}>}\mbox{}\newline 
\hspace*{1em}\hspace*{1em}\hspace*{1em}\hspace*{1em}{<\textbf{l}>}At his head a grass-green turf,{</\textbf{l}>}\mbox{}\newline 
\hspace*{1em}\hspace*{1em}\hspace*{1em}\hspace*{1em}{<\textbf{l}>}At his heels a stone.{</\textbf{l}>}\mbox{}\newline 
\hspace*{1em}\hspace*{1em}\hspace*{1em}{</\textbf{lg}>}\mbox{}\newline 
\hspace*{1em}\hspace*{1em}\hspace*{1em}{<\textbf{p}>}O, ho!{</\textbf{p}>}\mbox{}\newline 
\hspace*{1em}\hspace*{1em}{</\textbf{sp}>}\mbox{}\newline 
\hspace*{1em}{</\textbf{div2}>}\mbox{}\newline 
{</\textbf{div1}>}\end{shaded}\egroup\par \par
The {\itshape next} and {\itshape prev} attributes are discussed in section \textit{\hyperref[SAAG]{16.7.\ Aggregation}}: they form part of the module for alignment and linking; this module must therefore be included in a schema if they are to be used, as further discussed in section \textit{\hyperref[STIN]{1.2.\ Defining a TEI Schema}}.\par
The fragments of Ophelia's song might also be linked together using the \hyperref[TEI.join]{<join>} mechanism described in section \textit{\hyperref[SAAG]{16.7.\ Aggregation}}. The \hyperref[TEI.join]{<join>} element is specifically intended to encode the fact that several discontiguous elements of the text together form one ‘virtual’ element. Using this mechanism, the example might be encoded as follows: \par\bgroup\index{text=<text>|exampleindex}\index{body=<body>|exampleindex}\index{div1=<div1>|exampleindex}\index{n=@n!<div1>|exampleindex}\index{type=@type!<div1>|exampleindex}\index{div2=<div2>|exampleindex}\index{n=@n!<div2>|exampleindex}\index{type=@type!<div2>|exampleindex}\index{stage=<stage>|exampleindex}\index{type=@type!<stage>|exampleindex}\index{sp=<sp>|exampleindex}\index{speaker=<speaker>|exampleindex}\index{p=<p>|exampleindex}\index{sp=<sp>|exampleindex}\index{speaker=<speaker>|exampleindex}\index{stage=<stage>|exampleindex}\index{type=@type!<stage>|exampleindex}\index{lg=<lg>|exampleindex}\index{type=@type!<lg>|exampleindex}\index{part=@part!<lg>|exampleindex}\index{l=<l>|exampleindex}\index{l=<l>|exampleindex}\index{l=<l>|exampleindex}\index{l=<l>|exampleindex}\index{sp=<sp>|exampleindex}\index{speaker=<speaker>|exampleindex}\index{p=<p>|exampleindex}\index{sp=<sp>|exampleindex}\index{speaker=<speaker>|exampleindex}\index{p=<p>|exampleindex}\index{stage=<stage>|exampleindex}\index{type=@type!<stage>|exampleindex}\index{lg=<lg>|exampleindex}\index{type=@type!<lg>|exampleindex}\index{part=@part!<lg>|exampleindex}\index{l=<l>|exampleindex}\index{l=<l>|exampleindex}\index{l=<l>|exampleindex}\index{l=<l>|exampleindex}\index{p=<p>|exampleindex}\index{join=<join>|exampleindex}\index{type=@type!<join>|exampleindex}\index{target=@target!<join>|exampleindex}\exampleFont \begin{shaded}\noindent\mbox{}{<\textbf{text}>}\mbox{}\newline 
\hspace*{1em}{<\textbf{body}>}\mbox{}\newline 
\hspace*{1em}\hspace*{1em}{<\textbf{div1}\hspace*{1em}{n}="{4}"\hspace*{1em}{type}="{act}">}\mbox{}\newline 
\hspace*{1em}\hspace*{1em}\hspace*{1em}{<\textbf{div2}\hspace*{1em}{n}="{5}"\hspace*{1em}{type}="{scene}">}\mbox{}\newline 
\hspace*{1em}\hspace*{1em}\hspace*{1em}\hspace*{1em}{<\textbf{stage}\hspace*{1em}{type}="{setting}">}Elsinore. A room in the Castle.{</\textbf{stage}>}\mbox{}\newline 
\hspace*{1em}\hspace*{1em}\hspace*{1em}\hspace*{1em}{<\textbf{sp}>}\mbox{}\newline 
\hspace*{1em}\hspace*{1em}\hspace*{1em}\hspace*{1em}\hspace*{1em}{<\textbf{speaker}>}Queen{</\textbf{speaker}>}\mbox{}\newline 
\hspace*{1em}\hspace*{1em}\hspace*{1em}\hspace*{1em}\hspace*{1em}{<\textbf{p}>}How now, Ophelia?{</\textbf{p}>}\mbox{}\newline 
\hspace*{1em}\hspace*{1em}\hspace*{1em}\hspace*{1em}{</\textbf{sp}>}\mbox{}\newline 
\hspace*{1em}\hspace*{1em}\hspace*{1em}\hspace*{1em}{<\textbf{sp}>}\mbox{}\newline 
\hspace*{1em}\hspace*{1em}\hspace*{1em}\hspace*{1em}\hspace*{1em}{<\textbf{speaker}>}Ophelia{</\textbf{speaker}>}\mbox{}\newline 
\hspace*{1em}\hspace*{1em}\hspace*{1em}\hspace*{1em}\hspace*{1em}{<\textbf{stage}\hspace*{1em}{type}="{delivery}">}Singing{</\textbf{stage}>}\mbox{}\newline 
\hspace*{1em}\hspace*{1em}\hspace*{1em}\hspace*{1em}\hspace*{1em}{<\textbf{lg}\hspace*{1em}{xml:id}="{TL1}"\hspace*{1em}{type}="{song}"\hspace*{1em}{part}="{Y}">}\mbox{}\newline 
\hspace*{1em}\hspace*{1em}\hspace*{1em}\hspace*{1em}\hspace*{1em}\hspace*{1em}{<\textbf{l}>}How should I your true-love know{</\textbf{l}>}\mbox{}\newline 
\hspace*{1em}\hspace*{1em}\hspace*{1em}\hspace*{1em}\hspace*{1em}\hspace*{1em}{<\textbf{l}>}From another one?{</\textbf{l}>}\mbox{}\newline 
\hspace*{1em}\hspace*{1em}\hspace*{1em}\hspace*{1em}\hspace*{1em}\hspace*{1em}{<\textbf{l}>}By his cockle hat and staff{</\textbf{l}>}\mbox{}\newline 
\hspace*{1em}\hspace*{1em}\hspace*{1em}\hspace*{1em}\hspace*{1em}\hspace*{1em}{<\textbf{l}>}And his sandal shoon.{</\textbf{l}>}\mbox{}\newline 
\hspace*{1em}\hspace*{1em}\hspace*{1em}\hspace*{1em}\hspace*{1em}{</\textbf{lg}>}\mbox{}\newline 
\hspace*{1em}\hspace*{1em}\hspace*{1em}\hspace*{1em}{</\textbf{sp}>}\mbox{}\newline 
\hspace*{1em}\hspace*{1em}\hspace*{1em}\hspace*{1em}{<\textbf{sp}>}\mbox{}\newline 
\hspace*{1em}\hspace*{1em}\hspace*{1em}\hspace*{1em}\hspace*{1em}{<\textbf{speaker}>}Queen{</\textbf{speaker}>}\mbox{}\newline 
\hspace*{1em}\hspace*{1em}\hspace*{1em}\hspace*{1em}\hspace*{1em}{<\textbf{p}>}Alas, sweet lady, what imports this song?{</\textbf{p}>}\mbox{}\newline 
\hspace*{1em}\hspace*{1em}\hspace*{1em}\hspace*{1em}{</\textbf{sp}>}\mbox{}\newline 
\hspace*{1em}\hspace*{1em}\hspace*{1em}\hspace*{1em}{<\textbf{sp}>}\mbox{}\newline 
\hspace*{1em}\hspace*{1em}\hspace*{1em}\hspace*{1em}\hspace*{1em}{<\textbf{speaker}>}Ophelia{</\textbf{speaker}>}\mbox{}\newline 
\hspace*{1em}\hspace*{1em}\hspace*{1em}\hspace*{1em}\hspace*{1em}{<\textbf{p}>}Say you? Nay, pray you mark.{</\textbf{p}>}\mbox{}\newline 
\hspace*{1em}\hspace*{1em}\hspace*{1em}\hspace*{1em}\hspace*{1em}{<\textbf{stage}\hspace*{1em}{type}="{delivery}">}Sings{</\textbf{stage}>}\mbox{}\newline 
\hspace*{1em}\hspace*{1em}\hspace*{1em}\hspace*{1em}\hspace*{1em}{<\textbf{lg}\hspace*{1em}{xml:id}="{TL2}"\hspace*{1em}{type}="{song}"\hspace*{1em}{part}="{Y}">}\mbox{}\newline 
\hspace*{1em}\hspace*{1em}\hspace*{1em}\hspace*{1em}\hspace*{1em}\hspace*{1em}{<\textbf{l}>}He is dead and gone, lady,{</\textbf{l}>}\mbox{}\newline 
\hspace*{1em}\hspace*{1em}\hspace*{1em}\hspace*{1em}\hspace*{1em}\hspace*{1em}{<\textbf{l}>}He is dead and gone;{</\textbf{l}>}\mbox{}\newline 
\hspace*{1em}\hspace*{1em}\hspace*{1em}\hspace*{1em}\hspace*{1em}\hspace*{1em}{<\textbf{l}>}At his head a grass-green turf,{</\textbf{l}>}\mbox{}\newline 
\hspace*{1em}\hspace*{1em}\hspace*{1em}\hspace*{1em}\hspace*{1em}\hspace*{1em}{<\textbf{l}>}At his heels a stone.{</\textbf{l}>}\mbox{}\newline 
\hspace*{1em}\hspace*{1em}\hspace*{1em}\hspace*{1em}\hspace*{1em}{</\textbf{lg}>}\mbox{}\newline 
\hspace*{1em}\hspace*{1em}\hspace*{1em}\hspace*{1em}\hspace*{1em}{<\textbf{p}>}O, ho!{</\textbf{p}>}\mbox{}\newline 
\hspace*{1em}\hspace*{1em}\hspace*{1em}\hspace*{1em}\hspace*{1em}{<\textbf{join}\hspace*{1em}{type}="{lg}"\hspace*{1em}{target}="{\#TL1 \#TL2}"/>}\mbox{}\newline 
\hspace*{1em}\hspace*{1em}\hspace*{1em}\hspace*{1em}{</\textbf{sp}>}\mbox{}\newline 
\hspace*{1em}\hspace*{1em}\hspace*{1em}{</\textbf{div2}>}\mbox{}\newline 
\hspace*{1em}\hspace*{1em}{</\textbf{div1}>}\mbox{}\newline 
\hspace*{1em}{</\textbf{body}>}\mbox{}\newline 
{</\textbf{text}>}\end{shaded}\egroup\par \noindent  The location of the \hyperref[TEI.join]{<join>} element is not significant; here it has been placed shortly after the conclusion of the song, in order to have it close to the fragments it unifies.\par
Like the {\itshape next} and {\itshape prev} attributes, the \hyperref[TEI.join]{<join>} element requires the additional module for linking, which is selected as shown above.
\subsubsection[{Simultaneous Action}]{Simultaneous Action}\label{DRSIM}\par
In printed or written versions of performance texts, a variety of techniques may be used to indicate the temporal alignment of speeches or actions. Speeches may be printed vertically aligned on the page, or braced together; stage directions (e.g. ‘Speaking at the same time’) are also often used. In operatic or musical works in particular, the need to indicate timing and alignment of individual parts of a song may lead to very complex layout.\par
One simple method of indicating the temporal alignment of speeches or actions is to use the \hyperref[TEI.spGrp]{<spGrp>} element discussed in section \textit{\hyperref[DRSPG]{7.2.3.\ Grouped Speeches}}, with a {\itshape type} attribute to specify the reason for grouping, as in the following example: \par\bgroup\index{sp=<sp>|exampleindex}\index{speaker=<speaker>|exampleindex}\index{stage=<stage>|exampleindex}\index{type=@type!<stage>|exampleindex}\index{p=<p>|exampleindex}\index{stage=<stage>|exampleindex}\index{type=@type!<stage>|exampleindex}\index{spGrp=<spGrp>|exampleindex}\index{type=@type!<spGrp>|exampleindex}\index{rend=@rend!<spGrp>|exampleindex}\index{sp=<sp>|exampleindex}\index{speaker=<speaker>|exampleindex}\index{p=<p>|exampleindex}\index{sp=<sp>|exampleindex}\index{speaker=<speaker>|exampleindex}\index{p=<p>|exampleindex}\index{sp=<sp>|exampleindex}\index{speaker=<speaker>|exampleindex}\index{p=<p>|exampleindex}\index{sp=<sp>|exampleindex}\index{speaker=<speaker>|exampleindex}\index{p=<p>|exampleindex}\index{stage=<stage>|exampleindex}\index{type=@type!<stage>|exampleindex}\index{sp=<sp>|exampleindex}\index{speaker=<speaker>|exampleindex}\index{stage=<stage>|exampleindex}\index{type=@type!<stage>|exampleindex}\index{p=<p>|exampleindex}\exampleFont \begin{shaded}\noindent\mbox{}{<\textbf{sp}>}\mbox{}\newline 
\hspace*{1em}{<\textbf{speaker}>}Mangan{</\textbf{speaker}>}\mbox{}\newline 
\hspace*{1em}{<\textbf{stage}\hspace*{1em}{type}="{delivery}">}wildly{</\textbf{stage}>}\mbox{}\newline 
\hspace*{1em}{<\textbf{p}>}Look here: I'm going to take off all my clothes.{</\textbf{p}>}\mbox{}\newline 
\hspace*{1em}{<\textbf{stage}\hspace*{1em}{type}="{action}">}he begins tearing off his coat.{</\textbf{stage}>}\mbox{}\newline 
{</\textbf{sp}>}\mbox{}\newline 
{<\textbf{spGrp}\hspace*{1em}{type}="{simultaneous}"\hspace*{1em}{rend}="{braced}">}\mbox{}\newline 
\hspace*{1em}{<\textbf{sp}>}\mbox{}\newline 
\hspace*{1em}\hspace*{1em}{<\textbf{speaker}>}Lady Utterword{</\textbf{speaker}>}\mbox{}\newline 
\hspace*{1em}\hspace*{1em}{<\textbf{p}>}Mr Mangan!{</\textbf{p}>}\mbox{}\newline 
\hspace*{1em}{</\textbf{sp}>}\mbox{}\newline 
\hspace*{1em}{<\textbf{sp}>}\mbox{}\newline 
\hspace*{1em}\hspace*{1em}{<\textbf{speaker}>}Captain Shotover{</\textbf{speaker}>}\mbox{}\newline 
\hspace*{1em}\hspace*{1em}{<\textbf{p}>}Whats that?{</\textbf{p}>}\mbox{}\newline 
\hspace*{1em}{</\textbf{sp}>}\mbox{}\newline 
\hspace*{1em}{<\textbf{sp}>}\mbox{}\newline 
\hspace*{1em}\hspace*{1em}{<\textbf{speaker}>}Hector{</\textbf{speaker}>}\mbox{}\newline 
\hspace*{1em}\hspace*{1em}{<\textbf{p}>}Ha! ha! Do. Do.{</\textbf{p}>}\mbox{}\newline 
\hspace*{1em}{</\textbf{sp}>}\mbox{}\newline 
\hspace*{1em}{<\textbf{sp}>}\mbox{}\newline 
\hspace*{1em}\hspace*{1em}{<\textbf{speaker}>}Ellie{</\textbf{speaker}>}\mbox{}\newline 
\hspace*{1em}\hspace*{1em}{<\textbf{p}>}Please dont.{</\textbf{p}>}\mbox{}\newline 
\hspace*{1em}{</\textbf{sp}>}\mbox{}\newline 
\hspace*{1em}{<\textbf{stage}\hspace*{1em}{type}="{delivery}">}in consternation{</\textbf{stage}>}\mbox{}\newline 
{</\textbf{spGrp}>}\mbox{}\newline 
{<\textbf{sp}>}\mbox{}\newline 
\hspace*{1em}{<\textbf{speaker}>}Mrs. Hushabye{</\textbf{speaker}>}\mbox{}\newline 
\hspace*{1em}{<\textbf{stage}\hspace*{1em}{type}="{action}">}catching his arm and stopping him{</\textbf{stage}>}\mbox{}\newline 
\hspace*{1em}{<\textbf{p}>}Alfred: for shame! Are you mad?{</\textbf{p}>}\mbox{}\newline 
{</\textbf{sp}>}\end{shaded}\egroup\par \par
In the original, the stage direction ‘in consternation’ is printed opposite a brace grouping all four speeches, indicating that all four characters speak at once, and that the stage direction applies to all of them. Rather than attempting to represent the appearance of the source, this example encoding represents its presumed meaning: the \hyperref[TEI.stage]{<stage>} element is placed arbitrarily after the last relevant speech, and the four speeches with which it is to be associated are grouped by means of the \hyperref[TEI.spGrp]{<spGrp>} element. The {\itshape rend} attribute is used to specify the fact that the three speeches were grouped by the brace in the copy text. Producing a readable version of the text which simulates the original printed effect may however require more complex markup and processing.\par
More powerful and more precise mechanisms for temporal alignment are defined in chapter \textit{\hyperref[TS]{8.\ Transcriptions of Speech}}. These would be appropriate for encodings the focus of which is on the actual performance of a text rather than its structure or formal properties. The module described in that chapter includes a large number of other detailed proposals for the encoding of such features as voice quality, prosody, etc., which might be relevant to such a treatment of performance texts.
\subsection[{Other Types of Performance Text}]{Other Types of Performance Text}\label{DROTH}\par
Most of the elements and structures identified thus far are derived from traditional theatrical texts. Although other performance texts, such as screenplays or radio scripts, have not been discussed specifically, they can be encoded using the elements and structures listed above. Encoders may however find it convenient to use, as well, the additional specialized elements discussed in this section. For scripts containing very detailed technical information, the \hyperref[TEI.tech]{<tech>} element discussed in section \textit{\hyperref[DRTEC]{7.3.1.\ Technical Information}} may also be useful.\par
Like other texts, screenplays and television or radio scripts may be divided into text divisions marked with \hyperref[TEI.div]{<div>} or \hyperref[TEI.div1]{<div1>}, etc. Within units corresponding with the traditional ‘act’ and ‘scene’, further subdivisions or sequences may be identified, composed of individual ‘shots’, each associated with a single camera angle and setting. Shots and sequences should be encoded using an appropriate text-division element (i.e., a \hyperref[TEI.div3]{<div3>} element if numbered division elements are in use and the next largest unit is a \hyperref[TEI.div2]{<div2>}, or a \hyperref[TEI.div]{<div>} element if un-numbered divisions are in use) specifying sequence or shot as the value of the {\itshape type} attribute, as appropriate.\par
It is normal practice in screenplays and radio scripts to distinguish directions concerning camera angles, sound effects, etc., from other forms of stage direction. Such texts also generally include far more detailed specifications of what the audience actually sees: descriptions of actions and background, etc. Scripts derived from cinema and television productions may also include texts displayed as captions superimposed on the action. All of these may be encoded using the general purpose \hyperref[TEI.stage]{<stage>} element discussed in section \textit{\hyperref[DRSTA]{7.2.4.\ Stage Directions}}, and distinguished by means of its {\itshape type} attribute. Alternatively, or in addition, the following more specific elements may be used, where clear distinctions can be made: 
\begin{sansreflist}
  
\item [\textbf{<view>}] (view) describes the visual context of some part of a screen play in terms of what the spectator sees, generally independent of any dialogue.
\item [\textbf{<camera>}] (camera) describes a particular camera angle or viewpoint in a screen play.
\item [\textbf{<caption>}] (caption) contains the text of a caption or other text displayed as part of a film script or screenplay.
\item [\textbf{<sound>}] (sound) describes a sound effect or musical sequence specified within a screen play or radio script.\hfil\\[-10pt]\begin{sansreflist}
    \item[@{\itshape type}]
  categorizes the sound in some respect, e.g. as music, special effect, etc.
    \item[@{\itshape discrete}]
  indicates whether the sound overlaps the surrounding speeches or interrupts them.
\end{sansreflist}  
\end{sansreflist}
\par
Some examples of the use of these elements follow: \par\bgroup\index{camera=<camera>|exampleindex}\index{view=<view>|exampleindex}\exampleFont \begin{shaded}\noindent\mbox{}{<\textbf{camera}>}Angle on Olivia.{</\textbf{camera}>}\mbox{}\newline 
{<\textbf{view}>}Ryan's wife, standing nervously alone on the sidelines,\mbox{}\newline 
 biting her lip. She's scared and she shows it.{</\textbf{view}>}\end{shaded}\egroup\par \par
Where particular words or phrases within a direction are emphasized (by change of typeface or use of capital letters), an appropriate phrase-level element may be used to indicate the fact, as in the following examples, where certain words in the original are given in small capitals: \par\bgroup\index{view=<view>|exampleindex}\index{camera=<camera>|exampleindex}\index{hi=<hi>|exampleindex}\index{hi=<hi>|exampleindex}\exampleFont \begin{shaded}\noindent\mbox{}{<\textbf{view}>}George glances at the window--and freezes.\mbox{}\newline 
{<\textbf{camera}>}New angle--shock cut{</\textbf{camera}>} Out the window\mbox{}\newline 
 the body of a dead man suddenly slams into\mbox{}\newline 
{<\textbf{hi}>}frame{</\textbf{hi}>}. He dangles grotesquely,\mbox{}\newline 
 held up by his coat caught on a protruding bolt.\mbox{}\newline 
 George gasps. The train {<\textbf{hi}>}whistle{</\textbf{hi}>} screams.{</\textbf{view}>}\end{shaded}\egroup\par \noindent   \par\bgroup\index{view=<view>|exampleindex}\index{name=<name>|exampleindex}\index{sp=<sp>|exampleindex}\index{speaker=<speaker>|exampleindex}\index{p=<p>|exampleindex}\exampleFont \begin{shaded}\noindent\mbox{}{<\textbf{view}>}Ext. TV control van—Early morning.\mbox{}\newline 
 The {<\textbf{name}>}T.V. announcer{</\textbf{name}>} from the Ryan interview\mbox{}\newline 
 stands near the Control Van, the lake in b.g.{</\textbf{view}>}\mbox{}\newline 
{<\textbf{sp}>}\mbox{}\newline 
\hspace*{1em}{<\textbf{speaker}>}T.V. Announcer{</\textbf{speaker}>}\mbox{}\newline 
\hspace*{1em}{<\textbf{p}>}Several years ago, Jack Ryan was a highly\mbox{}\newline 
\hspace*{1em}\hspace*{1em} successful hydroplane racer ...{</\textbf{p}>}\mbox{}\newline 
{</\textbf{sp}>}\end{shaded}\egroup\par \noindent  \par
All of these elements, like other stage directions, can appear both within and between speeches. \par\bgroup\index{sp=<sp>|exampleindex}\index{speaker=<speaker>|exampleindex}\index{p=<p>|exampleindex}\index{view=<view>|exampleindex}\index{camera=<camera>|exampleindex}\exampleFont \begin{shaded}\noindent\mbox{}{<\textbf{sp}>}\mbox{}\newline 
\hspace*{1em}{<\textbf{speaker}>}TV Announcer VO{</\textbf{speaker}>}\mbox{}\newline 
\hspace*{1em}{<\textbf{p}>}Working with Ryan are his two coworkers—\mbox{}\newline 
\hspace*{1em}\hspace*{1em} Strut Bowman, the mechanical engineer—\mbox{}\newline 
\hspace*{1em}{<\textbf{view}>}\mbox{}\newline 
\hspace*{1em}\hspace*{1em}\hspace*{1em}{<\textbf{camera}>}Angle on Strut{</\textbf{camera}>}\mbox{}\newline 
\hspace*{1em}\hspace*{1em}\hspace*{1em}\hspace*{1em} standing in the tow boat, walkie-talkie in hand,\mbox{}\newline 
\hspace*{1em}\hspace*{1em}\hspace*{1em}\hspace*{1em} watching Ryan carefully.{</\textbf{view}>}\mbox{}\newline 
\hspace*{1em}\hspace*{1em} —and Roger Dalton, a rocket\mbox{}\newline 
\hspace*{1em}\hspace*{1em} systems analyst, and one of the scientists\mbox{}\newline 
\hspace*{1em}\hspace*{1em} from the Jet Propulsion Lab ...{</\textbf{p}>}\mbox{}\newline 
{</\textbf{sp}>}\end{shaded}\egroup\par \noindent  \par\bgroup\index{sp=<sp>|exampleindex}\index{speaker=<speaker>|exampleindex}\index{p=<p>|exampleindex}\index{sp=<sp>|exampleindex}\index{speaker=<speaker>|exampleindex}\index{p=<p>|exampleindex}\index{sound=<sound>|exampleindex}\index{sp=<sp>|exampleindex}\index{speaker=<speaker>|exampleindex}\index{p=<p>|exampleindex}\index{sp=<sp>|exampleindex}\index{speaker=<speaker>|exampleindex}\index{p=<p>|exampleindex}\exampleFont \begin{shaded}\noindent\mbox{}{<\textbf{sp}>}\mbox{}\newline 
\hspace*{1em}{<\textbf{speaker}>}Benjy{</\textbf{speaker}>}\mbox{}\newline 
\hspace*{1em}{<\textbf{p}>}Now to business.{</\textbf{p}>}\mbox{}\newline 
{</\textbf{sp}>}\mbox{}\newline 
{<\textbf{sp}>}\mbox{}\newline 
\hspace*{1em}{<\textbf{speaker}>}Ford and Zaphod{</\textbf{speaker}>}\mbox{}\newline 
\hspace*{1em}{<\textbf{p}>}To business.{</\textbf{p}>}\mbox{}\newline 
{</\textbf{sp}>}\mbox{}\newline 
{<\textbf{sound}>}Glasses clink.{</\textbf{sound}>}\mbox{}\newline 
{<\textbf{sp}>}\mbox{}\newline 
\hspace*{1em}{<\textbf{speaker}>}Benjy{</\textbf{speaker}>}\mbox{}\newline 
\hspace*{1em}{<\textbf{p}>}I beg your pardon?{</\textbf{p}>}\mbox{}\newline 
{</\textbf{sp}>}\mbox{}\newline 
{<\textbf{sp}>}\mbox{}\newline 
\hspace*{1em}{<\textbf{speaker}>}Ford{</\textbf{speaker}>}\mbox{}\newline 
\hspace*{1em}{<\textbf{p}>}I'm sorry, I thought you were proposing a toast.{</\textbf{p}>}\mbox{}\newline 
{</\textbf{sp}>}\end{shaded}\egroup\par \noindent   \par\bgroup\index{camera=<camera>|exampleindex}\index{caption=<caption>|exampleindex}\index{caption=<caption>|exampleindex}\index{sound=<sound>|exampleindex}\index{view=<view>|exampleindex}\index{sp=<sp>|exampleindex}\index{speaker=<speaker>|exampleindex}\index{p=<p>|exampleindex}\exampleFont \begin{shaded}\noindent\mbox{}{<\textbf{camera}>}Zoom in to overlay showing some stock film\mbox{}\newline 
 of hansom cabs galloping past.{</\textbf{camera}>}\mbox{}\newline 
{<\textbf{caption}>}London, 1895.{</\textbf{caption}>}\mbox{}\newline 
{<\textbf{caption}>}The residence of Mr Oscar Wilde.{</\textbf{caption}>}\mbox{}\newline 
{<\textbf{sound}>}Suitably classy music starts.{</\textbf{sound}>}\mbox{}\newline 
{<\textbf{view}>}Mix through to Wilde's drawing room. A crowd of suitably\mbox{}\newline 
 dressed folk are engaged in typically brilliant conversation,\mbox{}\newline 
 laughing affectedly and drinking champagne.{</\textbf{view}>}\mbox{}\newline 
{<\textbf{sp}>}\mbox{}\newline 
\hspace*{1em}{<\textbf{speaker}>}Prince of Wales{</\textbf{speaker}>}\mbox{}\newline 
\hspace*{1em}{<\textbf{p}>}My congratulations, Wilde. Your latest play is a great success.{</\textbf{p}>}\mbox{}\newline 
{</\textbf{sp}>}\end{shaded}\egroup\par \noindent  
\subsubsection[{Technical Information}]{Technical Information}\label{DRTEC}\par
Traditional stage scripts may contain additional technical information about such production-related factors as lighting, ‘blocking’ (that is, detailed notes on actors' movements), or props required at particular points. More technical information about intended production effects may also appear in published versions of screenplays or movie scripts. Where these are presented simply as marginal notes, they may be encoded using the general-purpose \hyperref[TEI.note]{<note>} element defined in section \textit{\hyperref[CONO]{3.9.\ Notes, Annotation, and Indexing}}. Alternatively, they may be formally distinguished from other stage directions by using the specialized \hyperref[TEI.tech]{<tech>} element: 
\begin{sansreflist}
  
\item [\textbf{<tech>}] (technical stage direction) describes a special-purpose stage direction that is not meant for the actors.\hfil\\[-10pt]\begin{sansreflist}
    \item[@{\itshape type}]
  categorizes the technical stage direction.
    \item[@{\itshape perf}]
  (performance) points to one or more \hyperref[TEI.performance]{<performance>} elements documenting the performance or performances to which this technical direction applies.
\end{sansreflist}  
\end{sansreflist}
\par
Like stage directions, \hyperref[TEI.tech]{<tech>} elements can appear anywhere within a speech or between speeches.
\subsection[{Module for Performance Texts}]{Module for Performance Texts}\par
The module described in this chapter makes available the following components: \begin{description}

\item[{Module drama: Performance texts}]\hspace{1em}\hfill\linebreak
\mbox{}\\[-10pt] \begin{itemize}
\item {\itshape Elements defined}: \hyperref[TEI.actor]{actor} \hyperref[TEI.camera]{camera} \hyperref[TEI.caption]{caption} \hyperref[TEI.castGroup]{castGroup} \hyperref[TEI.castItem]{castItem} \hyperref[TEI.castList]{castList} \hyperref[TEI.epilogue]{epilogue} \hyperref[TEI.move]{move} \hyperref[TEI.performance]{performance} \hyperref[TEI.prologue]{prologue} \hyperref[TEI.role]{role} \hyperref[TEI.roleDesc]{roleDesc} \hyperref[TEI.set]{set} \hyperref[TEI.sound]{sound} \hyperref[TEI.spGrp]{spGrp} \hyperref[TEI.tech]{tech} \hyperref[TEI.view]{view}
\end{itemize} 
\end{description}  The selection and combination of modules to form a TEI schema is described in \textit{\hyperref[STIN]{1.2.\ Defining a TEI Schema}}.